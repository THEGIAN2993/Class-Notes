\chapter{Introduction}\label{chapter:introduction}

\pagenumbering{arabic}

\section{Categories and Functors}\label{sec:categories-and-functors}
    \begin{definition}\label{def:class}
        A \textui{class} is a collection of sets (or sometimes other mathematical objects) that can be unambiguously defined by a property that all its members share.
    \end{definition}

    \begin{definition}\label{def:categories}
        A \textui{category} $\mathcal{C}$ consists of three ingredients: a class $\obj({\mathcal{C})}$ of \textui{objects}, a set of \textui{morphisms} $\Hom(A,B)$ for every ordered pair $(A,B)$ of objects, and \textui{composition} $\Hom{(A,B)} \times \Hom{(B,C)} \rightarrow \Hom{(A,C)}$, denoted by
            \begin{equation*}
            \begin{split}
                (f,g) \mapsto g\circ f,
            \end{split}
            \end{equation*}
        for every ordered tripled $A,B,C$ of objects. These ingredients are subject to the following axioms:
            \begin{enumerate}[label = (\arabic*)]
                \item The $\Hom{}$ sets are pairwise disjoint; i.e., each $f \in \Hom{(A,B)}$ has a unique \textui{domain} A and a unique \textui{target} B;
                \item for each object $A$, there is an \textui{identity morphism} $1_A \in \Hom{(A,A)}$ such that $f \circ 1_A = f$ and $1_B \circ f = f$ for all $f: A \rightarrow B$;
                \item composition is associative: given morphisms $A \xrightarrow{f} B \xrightarrow{g} C \xrightarrow{h} D$, then 
                    \begin{equation*}
                    \begin{split}
                        h\circ(g \circ f)=(h\circ g)\circ f.
                    \end{split}
                    \end{equation*}
            \end{enumerate}
    \end{definition}

    \begin{example}
        \begin{enumerate}[label = (\arabic*)]
            \item The category $\cSets$ has its objects as sets, morphisms as functions, and composition the usual composition of functions.
            \item The category $\cGroups$ has its objects as groups, morphisms as homomorphisms, and the usual composition of functions (homomorphisms are functions). One must verify that identity maps are homomorphisms and the composition of homomorphisms is also a homomorphism.
            \item An ordered set $X$ can be regarded as a category whose objects are elements of $X$ and whose $\Hom$ sets are:
                \begin{equation*}
                \begin{split}
                    \Hom(x,y) = \begin{cases} \emptyset, & x \leq y \\ \{\iota_y ^ x\}, & x \leq y \end{cases}.
                \end{split}
                \end{equation*}
            Note that $1_x = \iota_x^x$ by reflexivity, and composition follows from that fact that $\leq$ is transitive.
            \item If $X$ is a topological space with a topology $\cT$, then $\cT$ forms a category whose objects are its open sets and morphisms inclusion maps.
            \item The category $\cAb$ has its objects as abelian groups, its morphisms as homomorphisms, and composition as the usual composition of functions.
            \item The category $\cRings$ has its objects as rings, morphisms as ring homomorphisms, and composition as the usual composition of functions. We assume that all rings $R \in \obj(\cRings)$ are unital.
            \item The category $\cComRings$ has its objects as commutative rings, morphisms as ring homomorphisms, and composition as the usual composition of functions.
            \item The category $_R\cMod$ has its objects as left $R$-modules (where $R$ is a ring), its morphisms as $R$-module homomorphisms, and composition as the usual composition of functions. The $\Hom$ sets are denoted $\Hom_R(A,B)$. If $R = \bfZ$, then $_\bfZ\cMod = \cAb$, as abelian groups are $\bfZ$-modules. There is also a category of right $R$-modules denoted $\cMod_R$.
        \end{enumerate}
    \end{example}

    \begin{definition}
        A category $\cC$ is \textui{discrete} if its only morphisms are identity morphisms.
    \end{definition}

    \begin{definition}
        Let $\cC$ be a category. A category $\cS$ is a \textui{subcategory} of $\cC$ if:
            \begin{enumerate}[label = (\arabic*)]
                \item $\obj(\cS) \subseteq \obj(\cC)$;
                \item $\Hom_\cS(A,B) \subseteq \Hom_\cC(A,B)$
                \item If $f \in \Hom_\cS(A,B)$ and $g \in \Hom_S(B,C)$, then $g \circ f \in \Hom_\cS(A,C)$ is equal to $g \circ f \in \Hom_\cC(A,C)$.
                \item If $A \in \obj(\cS)$, then $1_A \in \Hom_S(A,A)$ is equal to $1_A \in \Hom_\cC(A,A)$.
            \end{enumerate}
        We say $\cS$ is a \textui{full subcategory} if, for all $A,B \in \obj(\cS)$, we have $\Hom_\cS(A,B) = \Hom_\cC(A,B)$.
    \end{definition}

    \begin{example}
        \phantom{a}
        \begin{enumerate}[label = (\arabic*)]
            \item The category of finite sets forms a full subcategory of $\cSets$.
            \item The category whose objects are sets and whose morphisms are bijections forms a non-full subcategory of $\cSets$.
        \end{enumerate}
    \end{example}
