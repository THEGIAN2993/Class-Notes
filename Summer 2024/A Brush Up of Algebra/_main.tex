%%%%%%%PACKAGES%%%%%%%
    \documentclass[11pt,twoside,openany]{memoir}
    \usepackage[T1]{fontenc}
    \usepackage[utf8]{inputenc}
    \usepackage{titlesec}
    \usepackage{anyfontsize}
    \usepackage{fancybox}
    \usepackage[dvipsnames,svgnames,x11names,hyperref]{xcolor}
    \usepackage{enumerate}
    \usepackage{comment}
    \usepackage{amsfonts}
    \usepackage{amsthm}
    \usepackage{amsmath}
    \usepackage{amssymb}
    \usepackage{hyperref}
    \usepackage{fullpage}
    \usepackage{bm}
    \usepackage{cprotect}
    \usepackage{calligra}
    \usepackage{emptypage}
    \usepackage{titleps}
    \usepackage{microtype}
    \usepackage{float}
    \usepackage{ocgx}
    \usepackage{appendix}
    \usepackage{graphicx}
    \usepackage{pdfcomment}
    \usepackage{enumitem}
    \usepackage{mathtools}
    \usepackage{tikz-cd}
    \usepackage{relsize}
    \usepackage[font=footnotesize,labelfont=bf]{caption}
    \usepackage{changepage}
    \usepackage{xcolor}
    \usepackage{ulem}
    \usepackage{marginnote}
        \newcommand*{\mnote}[1]{ % <----------
        \checkoddpage
        \ifoddpage
            \marginparmargin{left}
        \else
            \marginparmargin{right}
        \fi
            \marginnote{\tiny \textcolor{oorange}{#1}}
        }
    \usepackage{tgbonum}
    \usepackage{datetime}
        \newdateformat{specialdate}{\THEYEAR\ \monthname\ \THEDAY}
    \usepackage[margin=0.9in]{geometry}
        \setlength{\voffset}{-0.4in}
        \setlength{\headsep}{30pt}
    \usepackage{fancyhdr}
        \fancyhf{}
        \pagestyle{fancy}
        \cfoot{\footnotesize \thepage}
        \fancyhead[R]{\footnotesize \rightmark}
        \fancyhead[L]{\footnotesize \leftmark}
    \usepackage[T1]{fontenc}% http://ctan.org/pkg/fontenc
    \usepackage[outline]{contour}% http://ctan.org/pkg/contour
        \renewcommand{\arraystretch}{1.5}
        \contourlength{0.4pt}
        \contournumber{10}%
    \usepackage{letterspace}







    

%%%%%%%%%%%%%%%%%%%%%%
%%%%%%%%MACROS%%%%%%%%
%%%%%%%%%%%%%%%%%%%%%%
    %to make the correct symbol for Sha
%\newcommand\cyr{%
%\renewcommand\rmdefault{wncyr}%
%\renewcommand\sfdefault{wncyss}%
%\renewcommand\encodingdefault{OT2}%
%\normalfont \selectfont} \DeclareTextFontCommand{\textcyr}{\cyr}


\DeclareMathOperator{\ab}{ab}
\newcommand{\absgal}{\G_{\bbQ}}
\DeclareMathOperator{\ad}{ad}
\DeclareMathOperator{\adj}{adj}
\DeclareMathOperator{\alg}{alg}
\DeclareMathOperator{\Alt}{Alt}
\DeclareMathOperator{\Ann}{Ann}
\DeclareMathOperator{\arith}{arith}
\DeclareMathOperator{\Aut}{Aut}
\DeclareMathOperator{\Be}{B}
\DeclareMathOperator{\card}{card}
\DeclareMathOperator{\Char}{char}
\DeclareMathOperator{\csp}{csp}
\DeclareMathOperator{\codim}{codim}
\DeclareMathOperator{\coker}{coker}
\DeclareMathOperator{\coh}{H}
\DeclareMathOperator{\compl}{compl}
\DeclareMathOperator{\conj}{conj}
\DeclareMathOperator{\cont}{cont}
\DeclareMathOperator{\crys}{crys}
\DeclareMathOperator{\Crys}{Crys}
\DeclareMathOperator{\cusp}{cusp}
\DeclareMathOperator{\diag}{diag}
\DeclareMathOperator{\disc}{disc}
\DeclareMathOperator{\dR}{dR}
\DeclareMathOperator{\Eis}{Eis}
\DeclareMathOperator{\End}{End}
\DeclareMathOperator{\ev}{ev}
\DeclareMathOperator{\eval}{eval}
\DeclareMathOperator{\Eq}{Eq}
\DeclareMathOperator{\Ext}{Ext}
\DeclareMathOperator{\Fil}{Fil}
\DeclareMathOperator{\Fitt}{Fitt}
\DeclareMathOperator{\Frob}{Frob}
\DeclareMathOperator{\G}{G}
\DeclareMathOperator{\Gal}{Gal}
\DeclareMathOperator{\GL}{GL}
\DeclareMathOperator{\Gr}{Gr}
\DeclareMathOperator{\Graph}{Graph}
\DeclareMathOperator{\GSp}{GSp}
\DeclareMathOperator{\GUn}{GU}
\DeclareMathOperator{\Hom}{Hom}
\DeclareMathOperator{\id}{id}
\DeclareMathOperator{\Id}{Id}
\DeclareMathOperator{\Ik}{Ik}
\DeclareMathOperator{\IM}{Im}
\DeclareMathOperator{\Image}{im}
\DeclareMathOperator{\Ind}{Ind}
\DeclareMathOperator{\Inf}{inf}
\DeclareMathOperator{\Isom}{Isom}
\DeclareMathOperator{\J}{J}
\DeclareMathOperator{\Jac}{Jac}
\DeclareMathOperator{\lcm}{lcm}
\DeclareMathOperator{\length}{length}
\DeclareMathOperator{\Log}{Log}
\DeclareMathOperator{\M}{M}
\DeclareMathOperator{\Mat}{Mat}
\DeclareMathOperator{\N}{N}
\DeclareMathOperator{\Nm}{Nm}
\DeclareMathOperator{\NIk}{N-Ik}
\DeclareMathOperator{\NSK}{N-SK}
\DeclareMathOperator{\new}{new}
\DeclareMathOperator{\obj}{obj}
\DeclareMathOperator{\old}{old}
\DeclareMathOperator{\ord}{ord}
\DeclareMathOperator{\Or}{O}
\DeclareMathOperator{\PGL}{PGL}
\DeclareMathOperator{\PGSp}{PGSp}
\DeclareMathOperator{\rank}{rank}
\DeclareMathOperator{\Rel}{Rel}
\DeclareMathOperator{\Real}{Re}
\DeclareMathOperator{\RES}{res}
\DeclareMathOperator{\Res}{Res}
%\DeclareMathOperator{\Sha}{\textcyr{Sh}}
\DeclareMathOperator{\Sel}{Sel}
\DeclareMathOperator{\semi}{ss}
\DeclareMathOperator{\sgn}{sign}
\DeclareMathOperator{\SK}{SK}
\DeclareMathOperator{\SL}{SL}
\DeclareMathOperator{\SO}{SO}
\DeclareMathOperator{\Sp}{Sp}
\DeclareMathOperator{\Span}{span}
\DeclareMathOperator{\Spec}{Spec}
\DeclareMathOperator{\spin}{spin}
\DeclareMathOperator{\st}{st}
\DeclareMathOperator{\St}{St}
\DeclareMathOperator{\SUn}{SU}
\DeclareMathOperator{\supp}{supp}
\DeclareMathOperator{\Sup}{sup}
\DeclareMathOperator{\Sym}{Sym}
\DeclareMathOperator{\Tam}{Tam}
\DeclareMathOperator{\tors}{tors}
\DeclareMathOperator{\tr}{tr}
\DeclareMathOperator{\un}{un}
\DeclareMathOperator{\Un}{U}
\DeclareMathOperator{\val}{val}
\DeclareMathOperator{\vol}{vol}

\DeclareMathOperator{\Sets}{S \mkern1.04mu e \mkern1.04mu t \mkern1.04mu s}
    \newcommand{\cSets}{\scalebox{1.02}{\contour{black}{$\Sets$}}}
    
\DeclareMathOperator{\Groups}{G \mkern1.04mu r \mkern1.04mu o \mkern1.04mu u \mkern1.04mu p \mkern1.04mu s}
    \newcommand{\cGroups}{\scalebox{1.02}{\contour{black}{$\Groups$}}}

\DeclareMathOperator{\TTop}{T \mkern1.04mu o \mkern1.04mu p}
    \newcommand{\cTop}{\scalebox{1.02}{\contour{black}{$\TTop$}}}

\DeclareMathOperator{\Htp}{H \mkern1.04mu t \mkern1.04mu p}
    \newcommand{\cHtp}{\scalebox{1.02}{\contour{black}{$\Htp$}}}

\DeclareMathOperator{\Mod}{M \mkern1.04mu o \mkern1.04mu d}
    \newcommand{\cMod}{\scalebox{1.02}{\contour{black}{$\Mod$}}}

\DeclareMathOperator{\Ab}{A \mkern1.04mu b}
    \newcommand{\cAb}{\scalebox{1.02}{\contour{black}{$\Ab$}}}

\DeclareMathOperator{\Rings}{R \mkern1.04mu i \mkern1.04mu n \mkern1.04mu g \mkern1.04mu s}
    \newcommand{\cRings}{\scalebox{1.02}{\contour{black}{$\Rings$}}}

\DeclareMathOperator{\ComRings}{C \mkern1.04mu o \mkern1.04mu m \mkern1.04mu R \mkern1.04mu i \mkern1.04mu n \mkern1.04mu g \mkern1.04mu s}
    \newcommand{\cComRings}{\scalebox{1.05}{\contour{black}{$\ComRings$}}}

\DeclareMathOperator{\hHom}{H \mkern1.04mu o \mkern1.04mu m}
    \newcommand{\cHom}{\scalebox{1.02}{\contour{black}{$\hHom$}}}

         %  \item $\cGroups$
          %  \item $\cTop$
          %  \item $\cHtp$
          %  \item $\cMod$




\renewcommand{\k}{\kappa}
\newcommand{\Ff}{F_{f}}
\newcommand{\ts}{\,^{t}\!}


%Mathcal

\newcommand{\cA}{\mathcal{A}}
\newcommand{\cB}{\mathcal{B}}
\newcommand{\cC}{\mathcal{C}}
\newcommand{\cD}{\mathcal{D}}
\newcommand{\cE}{\mathcal{E}}
\newcommand{\cF}{\mathcal{F}}
\newcommand{\cG}{\mathcal{G}}
\newcommand{\cH}{\mathcal{H}}
\newcommand{\cI}{\mathcal{I}}
\newcommand{\cJ}{\mathcal{J}}
\newcommand{\cK}{\mathcal{K}}
\newcommand{\cL}{\mathcal{L}}
\newcommand{\cM}{\mathcal{M}}
\newcommand{\cN}{\mathcal{N}}
\newcommand{\cO}{\mathcal{O}}
\newcommand{\cP}{\mathcal{P}}
\newcommand{\cQ}{\mathcal{Q}}
\newcommand{\cR}{\mathcal{R}}
\newcommand{\cS}{\mathcal{S}}
\newcommand{\cT}{\mathcal{T}}
\newcommand{\cU}{\mathcal{U}}
\newcommand{\cV}{\mathcal{V}}
\newcommand{\cW}{\mathcal{W}}
\newcommand{\cX}{\mathcal{X}}
\newcommand{\cY}{\mathcal{Y}}
\newcommand{\cZ}{\mathcal{Z}}


%mathfrak (missing \fi)

\newcommand{\fa}{\mathfrak{a}}
\newcommand{\fA}{\mathfrak{A}}
\newcommand{\fb}{\mathfrak{b}}
\newcommand{\fB}{\mathfrak{B}}
\newcommand{\fc}{\mathfrak{c}}
\newcommand{\fC}{\mathfrak{C}}
\newcommand{\fd}{\mathfrak{d}}
\newcommand{\fD}{\mathfrak{D}}
\newcommand{\fe}{\mathfrak{e}}
\newcommand{\fE}{\mathfrak{E}}
\newcommand{\ff}{\mathfrak{f}}
\newcommand{\fF}{\mathfrak{F}}
\newcommand{\fg}{\mathfrak{g}}
\newcommand{\fG}{\mathfrak{G}}
\newcommand{\fh}{\mathfrak{h}}
\newcommand{\fH}{\mathfrak{H}}
\newcommand{\fI}{\mathfrak{I}}
\newcommand{\fj}{\mathfrak{j}}
\newcommand{\fJ}{\mathfrak{J}}
\newcommand{\fk}{\mathfrak{k}}
\newcommand{\fK}{\mathfrak{K}}
\newcommand{\fl}{\mathfrak{l}}
\newcommand{\fL}{\mathfrak{L}}
\newcommand{\fm}{\mathfrak{m}}
\newcommand{\fM}{\mathfrak{M}}
\newcommand{\fn}{\mathfrak{n}}
\newcommand{\fN}{\mathfrak{N}}
\newcommand{\fo}{\mathfrak{o}}
\newcommand{\fO}{\mathfrak{O}}
\newcommand{\fp}{\mathfrak{p}}
\newcommand{\fP}{\mathfrak{P}}
\newcommand{\fq}{\mathfrak{q}}
\newcommand{\fQ}{\mathfrak{Q}}
\newcommand{\fr}{\mathfrak{r}}
\newcommand{\fR}{\mathfrak{R}}
\newcommand{\fs}{\mathfrak{s}}
\newcommand{\fS}{\mathfrak{S}}
\newcommand{\ft}{\mathfrak{t}}
\newcommand{\fT}{\mathfrak{T}}
\newcommand{\fu}{\mathfrak{u}}
\newcommand{\fU}{\mathfrak{U}}
\newcommand{\fv}{\mathfrak{v}}
\newcommand{\fV}{\mathfrak{V}}
\newcommand{\fw}{\mathfrak{w}}
\newcommand{\fW}{\mathfrak{W}}
\newcommand{\fx}{\mathfrak{x}}
\newcommand{\fX}{\mathfrak{X}}
\newcommand{\fy}{\mathfrak{y}}
\newcommand{\fY}{\mathfrak{Y}}
\newcommand{\fz}{\mathfrak{z}}
\newcommand{\fZ}{\mathfrak{Z}}


%mathbf

\newcommand{\bfA}{\mathbf{A}}
\newcommand{\bfB}{\mathbf{B}}
\newcommand{\bfC}{\mathbf{C}}
\newcommand{\bfD}{\mathbf{D}}
\newcommand{\bfE}{\mathbf{E}}
\newcommand{\bfF}{\mathbf{F}}
\newcommand{\bfG}{\mathbf{G}}
\newcommand{\bfH}{\mathbf{H}}
\newcommand{\bfI}{\mathbf{I}}
\newcommand{\bfJ}{\mathbf{J}}
\newcommand{\bfK}{\mathbf{K}}
\newcommand{\bfL}{\mathbf{L}}
\newcommand{\bfM}{\mathbf{M}}
\newcommand{\bfN}{\mathbf{N}}
\newcommand{\bfO}{\mathbf{O}}
\newcommand{\bfP}{\mathbf{P}}
\newcommand{\bfQ}{\mathbf{Q}}
\newcommand{\bfR}{\mathbf{R}}
\newcommand{\bfS}{\mathbf{S}}
\newcommand{\bfT}{\mathbf{T}}
\newcommand{\bfU}{\mathbf{U}}
\newcommand{\bfV}{\mathbf{V}}
\newcommand{\bfW}{\mathbf{W}}
\newcommand{\bfX}{\mathbf{X}}
\newcommand{\bfY}{\mathbf{Y}}
\newcommand{\bfZ}{\mathbf{Z}}

\newcommand{\bfa}{\mathbf{a}}
\newcommand{\bfb}{\mathbf{b}}
\newcommand{\bfc}{\mathbf{c}}
\newcommand{\bfd}{\mathbf{d}}
\newcommand{\bfe}{\mathbf{e}}
\newcommand{\bff}{\mathbf{f}}
\newcommand{\bfg}{\mathbf{g}}
\newcommand{\bfh}{\mathbf{h}}
\newcommand{\bfi}{\mathbf{i}}
\newcommand{\bfj}{\mathbf{j}}
\newcommand{\bfk}{\mathbf{k}}
\newcommand{\bfl}{\mathbf{l}}
\newcommand{\bfm}{\mathbf{m}}
\newcommand{\bfn}{\mathbf{n}}
\newcommand{\bfo}{\mathbf{o}}
\newcommand{\bfp}{\mathbf{p}}
\newcommand{\bfq}{\mathbf{q}}
\newcommand{\bfr}{\mathbf{r}}
\newcommand{\bfs}{\mathbf{s}}
\newcommand{\bft}{\mathbf{t}}
\newcommand{\bfu}{\mathbf{u}}
\newcommand{\bfv}{\mathbf{v}}
\newcommand{\bfw}{\mathbf{w}}
\newcommand{\bfx}{\mathbf{x}}
\newcommand{\bfy}{\mathbf{y}}
\newcommand{\bfz}{\mathbf{z}}

%blackboard bold

\newcommand{\bbA}{\mathbb{A}}
\newcommand{\bbB}{\mathbb{B}}
\newcommand{\bbC}{\mathbb{C}}
\newcommand{\bbD}{\mathbb{D}}
\newcommand{\bbE}{\mathbb{E}}
\newcommand{\bbF}{\mathbb{F}}
\newcommand{\bbG}{\mathbb{G}}
\newcommand{\bbH}{\mathbb{H}}
\newcommand{\bbI}{\mathbb{I}}
\newcommand{\bbJ}{\mathbb{J}}
\newcommand{\bbK}{\mathbb{K}}
\newcommand{\bbL}{\mathbb{L}}
\newcommand{\bbM}{\mathbb{M}}
\newcommand{\bbN}{\mathbb{N}}
\newcommand{\bbO}{\mathbb{O}}
\newcommand{\bbP}{\mathbb{P}}
\newcommand{\bbQ}{\mathbb{Q}}
\newcommand{\bbR}{\mathbb{R}}
\newcommand{\bbS}{\mathbb{S}}
\newcommand{\bbT}{\mathbb{T}}
\newcommand{\bbU}{\mathbb{U}}
\newcommand{\bbV}{\mathbb{V}}
\newcommand{\bbW}{\mathbb{W}}
\newcommand{\bbX}{\mathbb{X}}
\newcommand{\bbY}{\mathbb{Y}}
\newcommand{\bbZ}{\mathbb{Z}}

\newcommand{\bmat}{\left( \begin{matrix}}
\newcommand{\emat}{\end{matrix} \right)}

\newcommand{\pmat}{\left( \begin{smallmatrix}}
\newcommand{\epmat}{\end{smallmatrix} \right)}

\newcommand{\lat}{\mathscr{L}}
\newcommand{\mat}[4]{\begin{pmatrix}{#1}&{#2}\\{#3}&{#4}\end{pmatrix}}
\newcommand{\ov}[1]{\overline{#1}}
\newcommand{\res}[1]{\underset{#1}{\RES}\,}
\newcommand{\up}{\upsilon}

\newcommand{\tac}{\textasteriskcentered}

%mahesh macros
\newcommand{\tm}{\textrm}

%Comments
\newcommand{\com}[1]{\vspace{5 mm}\par \noindent
\marginpar{\textsc{Comment}} \framebox{\begin{minipage}[c]{0.95
\textwidth} \tt #1 \end{minipage}}\vspace{5 mm}\par}

\newcommand{\Bmu}{\mbox{$\raisebox{-0.59ex}
  {$l$}\hspace{-0.18em}\mu\hspace{-0.88em}\raisebox{-0.98ex}{\scalebox{2}
  {$\color{white}.$}}\hspace{-0.416em}\raisebox{+0.88ex}
  {$\color{white}.$}\hspace{0.46em}$}{}}  %need graphicx and xcolor. this produces blackboard bold mu 

\newcommand{\hooktwoheadrightarrow}{%
  \hookrightarrow\mathrel{\mspace{-15mu}}\rightarrow
}

\makeatletter
\newcommand{\xhooktwoheadrightarrow}[2][]{%
  \lhook\joinrel
  \ext@arrow 0359\rightarrowfill@ {#1}{#2}%
  \mathrel{\mspace{-15mu}}\rightarrow
}
\makeatother

\renewcommand{\geq}{\geqslant}
    \renewcommand{\leq}{\leqslant}
    
    \newcommand{\bone}{\mathbf{1}}
    \newcommand{\sign}{\mathrm{sign}}
    \newcommand{\eps}{\varepsilon}
    \newcommand{\textui}[1]{\uline{\textit{#1}}}
    
    %\newcommand{\ov}{\overline}
    %\newcommand{\un}{\underline}
    \newcommand{\fin}{\mathrm{fin}}
    
    \newcommand{\chnum}{\titleformat
    {\chapter} % command
    [display] % shape
    {\centering} % format
    {\Huge \color{black} \shadowbox{\thechapter}} % label
    {-0.5em} % sep (space between the number and title)
    {\LARGE \color{black} \underline} % before-code
    }
    
    \newcommand{\chunnum}{\titleformat
    {\chapter} % command
    [display] % shape
    {} % format
    {} % label
    {0em} % sep
    { \begin{flushright} \begin{tabular}{r}  \Huge \color{black}
    } % before-code
    [
    \end{tabular} \end{flushright} \normalsize
    ] % after-code
    }

\newcommand{\littletaller}{\mathchoice{\vphantom{\big|}}{}{}{}}
\newcommand\restr[2]{{% we make the whole thing an ordinary symbol
  \left.\kern-\nulldelimiterspace % automatically resize the bar with \right
  #1 % the function
  \littletaller % pretend it's a little taller at normal size
  \right|_{#2} % this is the delimiter
  }}

\newcommand{\mtext}[1]{\hspace{6pt}\text{#1}\hspace{6pt}}

%This adds a "front cover" page.
%{\thispagestyle{empty}
%\vspace*{\fill}
%\begin{tabular}{l}
%\begin{tabular}{l}
%\includegraphics[scale=0.24]{oxy-logo.png}
%\end{tabular} \\
%\begin{tabular}{l}
%\Large \color{black} Module Theory, Linear Algebra, and Homological Algebra \\ \Large \color{black} Gianluca Crescenzo
%\end{tabular}
%\end{tabular}
%\newpage

    \newcommand{\TBC}{\textbf{TO BE CONTINUED}}
    \theoremstyle{plain}
    \newtheorem{theorem}{Theorem}[section]
    \newtheorem{proposition}[theorem]{Proposition}
    \newtheorem{corollary}[theorem]{Corollary}
    \newtheorem{lemma}[theorem]{Lemma}
    
    \theoremstyle{definition}
    \newtheorem{definition}{Definition}[section]
    \newtheorem{example}{Example}[section]
    \newtheorem{exercise}{Exercise}[chapter]
    \newtheorem{note}{Note}[section]
    
    \theoremstyle{remark}
    \newtheorem{remark}[theorem]{Remark}
    \newtheorem*{noproof}{Proof omitted}
    \numberwithin{equation}{section}
    
    \newenvironment{solution}[1]{\noindent \textbf{#1}:}{}
    
    \newcommand{\NN}{\mathbf{N}}
    \newcommand{\ZZ}{\mathbf{Z}}
    \newcommand{\QQ}{\mathbf{Q}}
    \newcommand{\RR}{\mathbf{R}}
    \newcommand{\CC}{\mathbf{C}}
    \newcommand{\HH}{\mathbf{H}}
    \newcommand{\KK}{\mathbf{K}}
    \newcommand{\FF}{\mathbf{F}}
    
    \newcommand{\bRR}{\overline{\RR}}
    \newcommand{\bRRp}{\overline{\RR}_{\geq 0}}
    
    \renewcommand{\geq}{\geqslant}
    \renewcommand{\leq}{\leqslant}
    
    \newcommand{\cA}{\mathcal{A}}
    \newcommand{\cB}{\mathcal{B}}
    \newcommand{\cC}{\mathcal{C}}
    \newcommand{\cF}{\mathcal{F}}
    \newcommand{\cL}{\mathcal{L}}
    \newcommand{\cM}{\mathcal{M}}
    \newcommand{\cN}{\mathcal{N}}
    \newcommand{\cU}{\mathcal{U}}
    \newcommand{\bone}{\mathbf{1}}
    \newcommand{\sign}{\mathrm{sign}}
    \newcommand{\eps}{\varepsilon}
    \newcommand{\textui}[1]{\uline{\textit{#1}}}
    
    %\newcommand{\ov}{\overline}
    %\newcommand{\un}{\underline}
    \newcommand{\fin}{\mathrm{fin}}
    
    \newcommand{\chnum}{\titleformat
    {\chapter} % command
    [display] % shape
    {} % format
    {} % label
    {-8em} % sep
    { \begin{flushright} \begin{tabular}{r} \fontsize{30}{20}\selectfont \color{black} \shadowbox{\thechapter} \\ \LARGE \color{black}
    } % before-code
    [
    \end{tabular} \end{flushright} \normalsize
    ] % after-code
    }
    
    \newcommand{\chunnum}{\titleformat
    {\chapter} % command
    [display] % shape
    {} % format
    {} % label
    {0em} % sep
    { \begin{flushright} \begin{tabular}{r}  \Huge \color{black}
    } % before-code
    [
    \end{tabular} \end{flushright} \normalsize
    ] % after-code
    }





    
%%%%%%%%%%%%%%%%%%%%%%
%%%%DOCUMENT SETUP%%%%
%%%%%%%%%%%%%%%%%%%%%%
    \setsecnumdepth{subsection}
    \definecolor{bluey}{RGB}{21, 80, 234}
    \definecolor{darkgreen}{rgb}{0, 0.5976, 0}
    \hypersetup{pdfauthor=Gianluca Crescenzo, pdftitle=Rotman's Notes, pdfstartview=FitH, colorlinks=true, linkcolor=darkgreen, citecolor=darkgreen}
    
    \begin{document}
    
    %This adds a "front cover" page.
    %{\thispagestyle{empty}
    %\vspace*{\fill}
    %\begin{tabular}{l}
    %\begin{tabular}{l}
    %\includegraphics[scale=0.24]{oxy-logo.png}
    %\end{tabular} \\
    %\begin{tabular}{l}
    %\Large \color{black} Module Theory, Linear Algebra, and Homological Algebra \\ \Large \color{black} Gianluca Crescenzo
    %\end{tabular}
    %\end{tabular}
    
    %\newpage
    \pagenumbering{roman}
    \tableofcontents
    
    \chunnum
    \vfill
    \specialdate
    Last update: \today
    \chnum

    \chapter{Euclidean Domains, PIDs, UFDs}\label{chapter:introduction}

\pagenumbering{arabic}

\section{Euclidean Domains}
    \begin{definition}
        Let $R$ be an integral domain. Any function $N:R \rightarrow \bfZ^+ \cup \{0\}$ with $N(0)=0$ is called a \textui{norm} on the integral domain $R$. If $N(a) > 0$ for $a \neq 0$ define $N$ to be a \textui{positive norm}.
    \end{definition}

    \begin{definition}
        The integral domain $R$ is said to be a \textui{Euclidean Domain} (or possess a \textui{Division Algorithm}) if there is a norm $N$ on $R$ such that for any two elements $a$ and $b$ of $R$ with $b \neq 0$  there exist elements $q$ and $r$ in $R$ with 
            \begin{equation*}
            \begin{split}
                a = qb + r \quad \text{with} \hspace{4pt} r=0 \hspace{4pt} \text{or} \hspace{4pt} N(r) < N(b).
            \end{split}
            \end{equation*}
        The element $q$ is called the \textui{quotient} and the element $r$ is called the $remainder$ of the division. 
    \end{definition}

    \begin{example}[Euclidean Algorithm]\label{example:euclidean-algorithm}
        Let $a$ and $b$ be any two elements of the Euclidean domain $R$. By successive "divisions" (these actually are divisions in the field of fractions of $R$) we can write
            \begin{equation*}
            \begin{split}
                a &= q_0 b + r_0 \\
                b &= q_1 r_0 + r_1 \\
                r_0 &= q_2 r_1 + r_2 \\
                & \vdots \\
                r_{n-2} &= q_{n} r_{n-1} + r_{n} \\
                r_{n-1} & = q_{n+1} r_{n}
            \end{split}
            \end{equation*}
        where $r_n$ is the last nonzero remainder. Such an $r_n$ exists since $N(b) > N(r_0) > N(r_1) > ... > N(r_n)$ is a decreasing sequence of nonnegative integers if the remainders are nonzero, and such a sequence cannot continue indefinitely. Note also that there is no guarentee that these elements are unique.
    \end{example}

    \begin{example}
        \phantom{a}
        \begin{enumerate}[label = (\arabic*)]
            \item Fields are trivial examples of Euclidean Domains where any norm will satisfy the defining condition (e.g., $N(a) = 0$ for all $a$). This is because for every $a,b$ with $b\neq 0$ we have $a = qb + 0$, where $q = ab^{-1}$.
            \item The integers $\bfZ$ are a Euclidean Domain with norm given by $N(a) = |a|$, the usual absolute value.
            \item If $F$ is a field, then the polynomial ring $F[x]$ is a Euclidean Domain with norm given by $N(p(x)) = \deg{p(x)}$. The Division Algorithm for polynomials is simply "long division" of polynomials. The proof is very similar to that for $\bfZ$ and is given in the next chapter. We will prove in Section~\ref{sec:2} that $R[x]$ is not a Euclidean Domain if $R$ is not a field.
        \end{enumerate}
    \end{example}

    \begin{proposition}\label{prop:1.1.1}
        Every ideal in a Euclidean Domain is principle. More precisely, if $I$ is any nonzero ideal in the Euclidean Domain $R$ then $I = (d)$, where $d$ is any nonzero element of $I$ of minimum norm.
    \end{proposition}
        \begin{proof}
            If $I$ is the zero ideal there is nothing to prove. Otherwise let $d \in I$ be any nonzero element of minimum norm (such a $d$ exists since the set $\{N(a) \mid a \in I\}$ has a minimum element by the well-ordering of $\bfZ$). Clearly $(d) \subseteq I$ since $d$ is an element of $I$. To show the reverse inclusion let $a \in I$ and use the Division Algorithm to write $a = qd + r$ with $r = 0$ or $N(r) < N(d)$. Then $r = a - qd$ and note that $a \in I$ and $qd \in I$, so $r$ is an element of $I$. By the minimality of the norm of $d$, it must be the case that $r = 0$. Hence $a = qd \in (d)$, showing $I \subseteq (d)$ which establishes the proposition that $I = (d)$.
        \end{proof}

    \begin{example}
        Let $R = \bfZ[x]$. Since the ideal $(2,x)$ is not principle, it follows that the ring $\bfZ[x]$ of polynomials with integer coefficients is not a Euclidean Domain.
    \end{example}

    \begin{definition}
        Let $R$ be a commutative ring and let $a,b \in R$ with $b \neq 0$.
        \begin{enumerate}[label = (\arabic*)]
            \item $a$ is said to be a \textui{multiple} of $b$ if there exists an element $x \in R$ with $a = bx$. In this case $b$ is said to \textui{divide} $a$ or be a \textui{divisor} of $a$, written $b \mid a$.
            \item A \textui{greatest common divisor} of $a$ and $b$ is a nonzero element $d$ such that 
                \begin{enumerate}[label = (\roman*)]
                    \item $d \mid a$ and $d \mid b$, and
                    \item if $d' \mid a$ and $d' \mid b$, then $d' \mid d$.
                \end{enumerate}
            A greatest common divisor of $a$ and $b$ will be denoted by $\gcd{(a,b)}$, or (abusing the notation) simply $(a,b)$.
        \end{enumerate}
    \end{definition}

    \begin{definition}
        If $I$ is the ideal of $R$ generated by $a$ and $b$ (that is, $I = (a,b)$), then $d$ is the greatest common divisor of $a$ and $b$ if
            \begin{enumerate}[label = (\roman*)]
                \item I is contained in the principial ideal $(d)$, and
                \item if $(d')$ is any principical ideal containing $I$ then $(d) \subseteq (d')$.
            \end{enumerate}
    \end{definition}

    \begin{proposition}
        If $a$ and $b$ are nonzero elements in the commutative ring $R$ such that the ideal generated by $a$ and $b$ is a principal ideal $(d)$, then $d$ is a a greatest common divisor of $a$ and $b$.
    \end{proposition}
        \begin{proof}
            This follows directly from the previous definition.
        \end{proof}
    
    \begin{proposition}
        Let $R$ be an integral domain. If two elements $d$ and $d'$ of $R$ generate the same principal ideal; i.e. $(d) = (d')$, then $d' = ud$ for some unit $u \in R$. In particular, if $d$ and $d'$ are both greatest common divisors of $a$ and $b$, then $d' = ud$ for some unit $u$.
    \end{proposition}
        \begin{proof}
            If either $d$ or $d'$ are $0$ then we are done. Assume $d$ and $d'$ are nonzero. Since $d \in (d')$ there is some $x \in R$ such that $d = xd'$. Since $d' \in (d)$ there is some $y \in R$ such that $d' = yd$. Thus $d = xyd$ and so $d(1-xy) = 0$. Since $d \neq 0$, it must be the case that $xy = 1$, that is, both $x$ and $y$ are units. This proves the first assertion.

            The second assertion follows from the first since any two greatest common divisors of $a$ and $b$ generate the same principle ideal (they divide eachother).
        \end{proof}
    \newpage
    \begin{theorem}
        Let $R$ be a Euclidean Domain and let $a$ and $b$ be nonzero elements of $R$. Let $d = r_n$ be the last nonzero remainder in the Euclidean Algorithm for $a$ and $b$ described in Example~\ref{example:euclidean-algorithm}. Then
            \begin{enumerate}[label = (\arabic*)]
                \item $d$ is the greatest common divisor of $a$ and $b$, and 
                \item the principal ideal $(d)$ is the ideal generated by $a$ and $b$. In particular, $d$ can be written as an $R$-linear combination of $a$ and $b$; i.e., there are elements $x$ and $y$ in $R$ such that
                    \begin{equation*}
                    \begin{split}
                        d = ax + by.
                    \end{split}
                    \end{equation*}
            \end{enumerate}
    \end{theorem}
        \begin{proof}
            By Proposition~\ref{prop:1.1.1}, the ideal generated by $a$ and $b$ is principal so $a,b$ do have a greatest common divisor, namely any element which generates the (principal) ideal $(a,b)$. Both parts of the theorem will follow once we show $d = r_n$ generates this ideal; i.e., once we show that
                \begin{enumerate}[label = (\roman*)]
                    \item $d \mid a$ and $d \mid b$ (which means $(a,b) \subseteq (d)$)
                    \item $d$ is an $R$-linear combination of $a$ and $b$ (which means $(d) \subseteq (a,b)$.)
                \end{enumerate}
            
            To prove that $d$ divides both $a$ and $b$, simply keep track of the divisibilities in the Euclidean Algorithm. Recall the following set of equations from Example~\ref{example:euclidean-algorithm}
            \begin{equation*}
                \begin{split}
                    a &= q_0 b + r_0 \quad\quad\quad\quad (0) \\
                    b &= q_1 r_0 + r_1 \quad\quad\quad\quad (1) \\
                    r_0 &= q_2 r_1 + r_2 \quad\quad\quad\quad (2) \\
                    & \vdots \\
                    r_{k-1} &= q_{k+1}r_k + r_{k+1} \quad\quad\quad\quad (k+1) \\
                    & \vdots \\
                    r_{n-2} &= q_{n} r_{n-1} + r_{n} \quad\quad\quad\quad (n) \\
                    r_{n-1} & = q_{n+1} r_{n} \quad\quad\quad\quad (n+1)
                \end{split}
                \end{equation*}
            We proceed with induction with $n$ as the base case. Equation $(n+1)$ gives $r_n \mid r_{n-1}$ and clearly $r_n \mid r_n$. Assume $r_n \mid r_{k+1}$ and $r_n \mid r_{k}$ as our inductive hypothesis. By Equation $(k+1)$ we see that $r_n$ divides both terms on the right hand side \textemdash hence $r_n \mid r_{k-1}$. From Equation (1) $r_n \mid b$ and from Equation (0) $r_n \mid a$, which establishes $(i)$.
        \end{proof}

    
    \end{document}
%%%%%%%%%%%%%%%%%%%%%%%%%%%%%%%%%%%%%%%%%%%%%%%%%%%%%%%%%%%%%%%%%%%%%%%%%