%%%%%%%PACKAGES%%%%%%%
\documentclass[10pt,twoside,openany]{memoir}


\usepackage[math]{anttor}
\usepackage[T1]{fontenc}


\usepackage{titlesec}
    \titlespacing*{\chapter}
    {0pt} % Left margin
    {*1} % Space before the chapter number (increase this value for more space)
    {0pt} % Space after the chapter number and before the title
\usepackage{anyfontsize}
\usepackage{fancybox}
\usepackage[dvipsnames,svgnames,x11names,hyperref]{xcolor}
\usepackage{enumerate}
\usepackage{comment}
\usepackage{amsfonts}
\usepackage{amsthm}
\usepackage{amsmath}
\usepackage{amssymb}
\usepackage{mathrsfs}
\usepackage{hyperref}
\usepackage{fullpage}
\usepackage{bm}
\usepackage{cprotect}
\usepackage{calligra}
\usepackage{emptypage}
\usepackage{titleps}
\usepackage{microtype}
\usepackage{float}
\usepackage{ocgx}
\usepackage{appendix}
\usepackage{graphicx}
\usepackage{pdfcomment}
\usepackage{enumitem}
\usepackage{mathtools}
\usepackage{tikz-cd}
\usepackage{relsize}
\usepackage[font=footnotesize,labelfont=bf]{caption}
\usepackage{changepage}
\usepackage{xcolor}
\usepackage{ulem}
\usepackage{pgfplots}
\usepackage{marginnote}
    \newcommand*{\mnote}[1]{ % <----------
    \checkoddpage
    \ifoddpage
        \marginparmargin{left}
    \else
        \marginparmargin{right}
    \fi
        \marginnote{\tiny \textcolor{oorange}{#1}}
    }
\usepackage{datetime}
    \newdateformat{specialdate}{\THEYEAR\ \monthname\ \THEDAY}
\usepackage[margin=0.9in]{geometry}
    \setlength{\voffset}{-0.4in}
    \setlength{\headsep}{30pt}
\usepackage{fancyhdr}
    \fancyhf{}
    \cfoot{\footnotesize \thepage}
    \fancyhead[R]{\footnotesize \rightmark}
    \fancyhead[L]{\footnotesize \leftmark}
\usepackage[outline]{contour}% http://ctan.org/pkg/contour
    \renewcommand{\arraystretch}{1.5}
    \contourlength{0.4pt}
    \contournumber{10}%
\usepackage{letterspace}
    \linespread{1.25}
\usepackage{thmtools}
    \declaretheoremstyle[
        spaceabove=6pt, spacebelow=6pt,
        headfont=\normalfont\bfseries,
        notefont=\mdseries, notebraces={(}{)},
        bodyfont=\normalfont,
        postheadspace=1em
        %qed=\qedsymbol
        ]{mystyle}

    \declaretheorem[
        numberwithin=section,
        %shaded={
        %    rulecolor={RGB}{255,250,177},
        %    rulewidth=5pt, 
        %    bgcolor={RGB}{255,250,177}}
    ]{theorem}

    \declaretheorem[
        sibling=theorem,
        numberwithin=section,
        %shaded={
        %    rulecolor=Lavender,
        %    rulewidth=5pt, 
        %    bgcolor=Lavender}
    ]{lemma}

    \declaretheorem[
        sibling=theorem,
        numberwithin=section,
        %shaded={
        %    rulecolor=Lavender,
        %    rulewidth=5pt, 
        %    bgcolor=Lavender}
    ]{proposition}

    \declaretheorem[
        sibling=theorem,
        numberwithin=section,
        %shaded={
        %    rulecolor=Lavender,
        %    rulewidth=5pt, 
        %    bgcolor=Lavender}
    ]{corollary}

    \declaretheorem[
        numberwithin=section,
        style=mystyle,
        %shaded={
        %    rulecolor={RGB}{233,233,233},
        %    rulewidth=5pt, 
        %    bgcolor={RGB}{233,233,233}}
    ]{example}

    \declaretheorem[
        numberwithin=section,
        style=mystyle,
        %shaded={
        %    rulecolor={RGB}{255,250,177},
        %    rulewidth=5pt, 
        %    bgcolor={RGB}{255,250,177}}
    ]{definition}

    \declaretheorem[
        %shaded={
        %    rulecolor=Lavender,
        %    rulewidth=5pt, 
        %    bgcolor=Lavender}
    ]{exercise}
\declaretheorem[numbered=unless unique,style=mystyle]{Note}
    




%%%%%%%%%%%%%%%%%%%%%%
%%%%%%%%MACROS%%%%%%%%
%%%%%%%%%%%%%%%%%%%%%%
%to make the correct symbol for Sha
%\newcommand\cyr{%
%\renewcommand\rmdefault{wncyr}%
%\renewcommand\sfdefault{wncyss}%
%\renewcommand\encodingdefault{OT2}%
%\normalfont \selectfont} \DeclareTextFontCommand{\textcyr}{\cyr}


\DeclareMathOperator{\ab}{ab}
\newcommand{\absgal}{\G_{\bbQ}}
\DeclareMathOperator{\ad}{ad}
\DeclareMathOperator{\adj}{adj}
\DeclareMathOperator{\alg}{alg}
\DeclareMathOperator{\Alt}{Alt}
\DeclareMathOperator{\Ann}{Ann}
\DeclareMathOperator{\arith}{arith}
\DeclareMathOperator{\Aut}{Aut}
\DeclareMathOperator{\Be}{B}
\DeclareMathOperator{\card}{card}
\DeclareMathOperator{\Char}{char}
\DeclareMathOperator{\csp}{csp}
\DeclareMathOperator{\codim}{codim}
\DeclareMathOperator{\coker}{coker}
\DeclareMathOperator{\coh}{H}
\DeclareMathOperator{\compl}{compl}
\DeclareMathOperator{\conj}{conj}
\DeclareMathOperator{\cont}{cont}
\DeclareMathOperator{\crys}{crys}
\DeclareMathOperator{\Crys}{Crys}
\DeclareMathOperator{\cusp}{cusp}
\DeclareMathOperator{\diag}{diag}
\DeclareMathOperator{\disc}{disc}
\DeclareMathOperator{\dR}{dR}
\DeclareMathOperator{\Eis}{Eis}
\DeclareMathOperator{\End}{End}
\DeclareMathOperator{\ev}{ev}
\DeclareMathOperator{\eval}{eval}
\DeclareMathOperator{\Eq}{Eq}
\DeclareMathOperator{\Ext}{Ext}
\DeclareMathOperator{\Fil}{Fil}
\DeclareMathOperator{\Fitt}{Fitt}
\DeclareMathOperator{\Frob}{Frob}
\DeclareMathOperator{\G}{G}
\DeclareMathOperator{\Gal}{Gal}
\DeclareMathOperator{\GL}{GL}
\DeclareMathOperator{\Gr}{Gr}
\DeclareMathOperator{\Graph}{Graph}
\DeclareMathOperator{\GSp}{GSp}
\DeclareMathOperator{\GUn}{GU}
\DeclareMathOperator{\Hom}{Hom}
\DeclareMathOperator{\id}{id}
\DeclareMathOperator{\Id}{Id}
\DeclareMathOperator{\Ik}{Ik}
\DeclareMathOperator{\IM}{Im}
\DeclareMathOperator{\Image}{im}
\DeclareMathOperator{\Ind}{Ind}
\DeclareMathOperator{\Inf}{inf}
\DeclareMathOperator{\Isom}{Isom}
\DeclareMathOperator{\J}{J}
\DeclareMathOperator{\Jac}{Jac}
\DeclareMathOperator{\lcm}{lcm}
\DeclareMathOperator{\length}{length}
\DeclareMathOperator{\Log}{Log}
\DeclareMathOperator{\M}{M}
\DeclareMathOperator{\Mat}{Mat}
\DeclareMathOperator{\N}{N}
\DeclareMathOperator{\Nm}{Nm}
\DeclareMathOperator{\NIk}{N-Ik}
\DeclareMathOperator{\NSK}{N-SK}
\DeclareMathOperator{\new}{new}
\DeclareMathOperator{\obj}{obj}
\DeclareMathOperator{\old}{old}
\DeclareMathOperator{\ord}{ord}
\DeclareMathOperator{\Or}{O}
\DeclareMathOperator{\PGL}{PGL}
\DeclareMathOperator{\PGSp}{PGSp}
\DeclareMathOperator{\rank}{rank}
\DeclareMathOperator{\Rel}{Rel}
\DeclareMathOperator{\Real}{Re}
\DeclareMathOperator{\RES}{res}
\DeclareMathOperator{\Res}{Res}
%\DeclareMathOperator{\Sha}{\textcyr{Sh}}
\DeclareMathOperator{\Sel}{Sel}
\DeclareMathOperator{\semi}{ss}
\DeclareMathOperator{\sgn}{sign}
\DeclareMathOperator{\SK}{SK}
\DeclareMathOperator{\SL}{SL}
\DeclareMathOperator{\SO}{SO}
\DeclareMathOperator{\Sp}{Sp}
\DeclareMathOperator{\Span}{span}
\DeclareMathOperator{\Spec}{Spec}
\DeclareMathOperator{\spin}{spin}
\DeclareMathOperator{\st}{st}
\DeclareMathOperator{\St}{St}
\DeclareMathOperator{\SUn}{SU}
\DeclareMathOperator{\supp}{supp}
\DeclareMathOperator{\Sup}{sup}
\DeclareMathOperator{\Sym}{Sym}
\DeclareMathOperator{\Tam}{Tam}
\DeclareMathOperator{\tors}{tors}
\DeclareMathOperator{\tr}{tr}
\DeclareMathOperator{\un}{un}
\DeclareMathOperator{\Un}{U}
\DeclareMathOperator{\val}{val}
\DeclareMathOperator{\vol}{vol}

\DeclareMathOperator{\Sets}{S \mkern1.04mu e \mkern1.04mu t \mkern1.04mu s}
    \newcommand{\cSets}{\scalebox{1.02}{\contour{black}{$\Sets$}}}
    
\DeclareMathOperator{\Groups}{G \mkern1.04mu r \mkern1.04mu o \mkern1.04mu u \mkern1.04mu p \mkern1.04mu s}
    \newcommand{\cGroups}{\scalebox{1.02}{\contour{black}{$\Groups$}}}

\DeclareMathOperator{\TTop}{T \mkern1.04mu o \mkern1.04mu p}
    \newcommand{\cTop}{\scalebox{1.02}{\contour{black}{$\TTop$}}}

\DeclareMathOperator{\Htp}{H \mkern1.04mu t \mkern1.04mu p}
    \newcommand{\cHtp}{\scalebox{1.02}{\contour{black}{$\Htp$}}}

\DeclareMathOperator{\Mod}{M \mkern1.04mu o \mkern1.04mu d}
    \newcommand{\cMod}{\scalebox{1.02}{\contour{black}{$\Mod$}}}

\DeclareMathOperator{\Ab}{A \mkern1.04mu b}
    \newcommand{\cAb}{\scalebox{1.02}{\contour{black}{$\Ab$}}}

\DeclareMathOperator{\Rings}{R \mkern1.04mu i \mkern1.04mu n \mkern1.04mu g \mkern1.04mu s}
    \newcommand{\cRings}{\scalebox{1.02}{\contour{black}{$\Rings$}}}

\DeclareMathOperator{\ComRings}{C \mkern1.04mu o \mkern1.04mu m \mkern1.04mu R \mkern1.04mu i \mkern1.04mu n \mkern1.04mu g \mkern1.04mu s}
    \newcommand{\cComRings}{\scalebox{1.05}{\contour{black}{$\ComRings$}}}

\DeclareMathOperator{\hHom}{H \mkern1.04mu o \mkern1.04mu m}
    \newcommand{\cHom}{\scalebox{1.02}{\contour{black}{$\hHom$}}}

         %  \item $\cGroups$
          %  \item $\cTop$
          %  \item $\cHtp$
          %  \item $\cMod$




\renewcommand{\k}{\kappa}
\newcommand{\Ff}{F_{f}}
\newcommand{\ts}{\,^{t}\!}


%Mathcal

\newcommand{\cA}{\mathcal{A}}
\newcommand{\cB}{\mathcal{B}}
\newcommand{\cC}{\mathcal{C}}
\newcommand{\cD}{\mathcal{D}}
\newcommand{\cE}{\mathcal{E}}
\newcommand{\cF}{\mathcal{F}}
\newcommand{\cG}{\mathcal{G}}
\newcommand{\cH}{\mathcal{H}}
\newcommand{\cI}{\mathcal{I}}
\newcommand{\cJ}{\mathcal{J}}
\newcommand{\cK}{\mathcal{K}}
\newcommand{\cL}{\mathcal{L}}
\newcommand{\cM}{\mathcal{M}}
\newcommand{\cN}{\mathcal{N}}
\newcommand{\cO}{\mathcal{O}}
\newcommand{\cP}{\mathcal{P}}
\newcommand{\cQ}{\mathcal{Q}}
\newcommand{\cR}{\mathcal{R}}
\newcommand{\cS}{\mathcal{S}}
\newcommand{\cT}{\mathcal{T}}
\newcommand{\cU}{\mathcal{U}}
\newcommand{\cV}{\mathcal{V}}
\newcommand{\cW}{\mathcal{W}}
\newcommand{\cX}{\mathcal{X}}
\newcommand{\cY}{\mathcal{Y}}
\newcommand{\cZ}{\mathcal{Z}}


%mathfrak (missing \fi)

\newcommand{\fa}{\mathfrak{a}}
\newcommand{\fA}{\mathfrak{A}}
\newcommand{\fb}{\mathfrak{b}}
\newcommand{\fB}{\mathfrak{B}}
\newcommand{\fc}{\mathfrak{c}}
\newcommand{\fC}{\mathfrak{C}}
\newcommand{\fd}{\mathfrak{d}}
\newcommand{\fD}{\mathfrak{D}}
\newcommand{\fe}{\mathfrak{e}}
\newcommand{\fE}{\mathfrak{E}}
\newcommand{\ff}{\mathfrak{f}}
\newcommand{\fF}{\mathfrak{F}}
\newcommand{\fg}{\mathfrak{g}}
\newcommand{\fG}{\mathfrak{G}}
\newcommand{\fh}{\mathfrak{h}}
\newcommand{\fH}{\mathfrak{H}}
\newcommand{\fI}{\mathfrak{I}}
\newcommand{\fj}{\mathfrak{j}}
\newcommand{\fJ}{\mathfrak{J}}
\newcommand{\fk}{\mathfrak{k}}
\newcommand{\fK}{\mathfrak{K}}
\newcommand{\fl}{\mathfrak{l}}
\newcommand{\fL}{\mathfrak{L}}
\newcommand{\fm}{\mathfrak{m}}
\newcommand{\fM}{\mathfrak{M}}
\newcommand{\fn}{\mathfrak{n}}
\newcommand{\fN}{\mathfrak{N}}
\newcommand{\fo}{\mathfrak{o}}
\newcommand{\fO}{\mathfrak{O}}
\newcommand{\fp}{\mathfrak{p}}
\newcommand{\fP}{\mathfrak{P}}
\newcommand{\fq}{\mathfrak{q}}
\newcommand{\fQ}{\mathfrak{Q}}
\newcommand{\fr}{\mathfrak{r}}
\newcommand{\fR}{\mathfrak{R}}
\newcommand{\fs}{\mathfrak{s}}
\newcommand{\fS}{\mathfrak{S}}
\newcommand{\ft}{\mathfrak{t}}
\newcommand{\fT}{\mathfrak{T}}
\newcommand{\fu}{\mathfrak{u}}
\newcommand{\fU}{\mathfrak{U}}
\newcommand{\fv}{\mathfrak{v}}
\newcommand{\fV}{\mathfrak{V}}
\newcommand{\fw}{\mathfrak{w}}
\newcommand{\fW}{\mathfrak{W}}
\newcommand{\fx}{\mathfrak{x}}
\newcommand{\fX}{\mathfrak{X}}
\newcommand{\fy}{\mathfrak{y}}
\newcommand{\fY}{\mathfrak{Y}}
\newcommand{\fz}{\mathfrak{z}}
\newcommand{\fZ}{\mathfrak{Z}}


%mathbf

\newcommand{\bfA}{\mathbf{A}}
\newcommand{\bfB}{\mathbf{B}}
\newcommand{\bfC}{\mathbf{C}}
\newcommand{\bfD}{\mathbf{D}}
\newcommand{\bfE}{\mathbf{E}}
\newcommand{\bfF}{\mathbf{F}}
\newcommand{\bfG}{\mathbf{G}}
\newcommand{\bfH}{\mathbf{H}}
\newcommand{\bfI}{\mathbf{I}}
\newcommand{\bfJ}{\mathbf{J}}
\newcommand{\bfK}{\mathbf{K}}
\newcommand{\bfL}{\mathbf{L}}
\newcommand{\bfM}{\mathbf{M}}
\newcommand{\bfN}{\mathbf{N}}
\newcommand{\bfO}{\mathbf{O}}
\newcommand{\bfP}{\mathbf{P}}
\newcommand{\bfQ}{\mathbf{Q}}
\newcommand{\bfR}{\mathbf{R}}
\newcommand{\bfS}{\mathbf{S}}
\newcommand{\bfT}{\mathbf{T}}
\newcommand{\bfU}{\mathbf{U}}
\newcommand{\bfV}{\mathbf{V}}
\newcommand{\bfW}{\mathbf{W}}
\newcommand{\bfX}{\mathbf{X}}
\newcommand{\bfY}{\mathbf{Y}}
\newcommand{\bfZ}{\mathbf{Z}}

\newcommand{\bfa}{\mathbf{a}}
\newcommand{\bfb}{\mathbf{b}}
\newcommand{\bfc}{\mathbf{c}}
\newcommand{\bfd}{\mathbf{d}}
\newcommand{\bfe}{\mathbf{e}}
\newcommand{\bff}{\mathbf{f}}
\newcommand{\bfg}{\mathbf{g}}
\newcommand{\bfh}{\mathbf{h}}
\newcommand{\bfi}{\mathbf{i}}
\newcommand{\bfj}{\mathbf{j}}
\newcommand{\bfk}{\mathbf{k}}
\newcommand{\bfl}{\mathbf{l}}
\newcommand{\bfm}{\mathbf{m}}
\newcommand{\bfn}{\mathbf{n}}
\newcommand{\bfo}{\mathbf{o}}
\newcommand{\bfp}{\mathbf{p}}
\newcommand{\bfq}{\mathbf{q}}
\newcommand{\bfr}{\mathbf{r}}
\newcommand{\bfs}{\mathbf{s}}
\newcommand{\bft}{\mathbf{t}}
\newcommand{\bfu}{\mathbf{u}}
\newcommand{\bfv}{\mathbf{v}}
\newcommand{\bfw}{\mathbf{w}}
\newcommand{\bfx}{\mathbf{x}}
\newcommand{\bfy}{\mathbf{y}}
\newcommand{\bfz}{\mathbf{z}}

%blackboard bold

\newcommand{\bbA}{\mathbb{A}}
\newcommand{\bbB}{\mathbb{B}}
\newcommand{\bbC}{\mathbb{C}}
\newcommand{\bbD}{\mathbb{D}}
\newcommand{\bbE}{\mathbb{E}}
\newcommand{\bbF}{\mathbb{F}}
\newcommand{\bbG}{\mathbb{G}}
\newcommand{\bbH}{\mathbb{H}}
\newcommand{\bbI}{\mathbb{I}}
\newcommand{\bbJ}{\mathbb{J}}
\newcommand{\bbK}{\mathbb{K}}
\newcommand{\bbL}{\mathbb{L}}
\newcommand{\bbM}{\mathbb{M}}
\newcommand{\bbN}{\mathbb{N}}
\newcommand{\bbO}{\mathbb{O}}
\newcommand{\bbP}{\mathbb{P}}
\newcommand{\bbQ}{\mathbb{Q}}
\newcommand{\bbR}{\mathbb{R}}
\newcommand{\bbS}{\mathbb{S}}
\newcommand{\bbT}{\mathbb{T}}
\newcommand{\bbU}{\mathbb{U}}
\newcommand{\bbV}{\mathbb{V}}
\newcommand{\bbW}{\mathbb{W}}
\newcommand{\bbX}{\mathbb{X}}
\newcommand{\bbY}{\mathbb{Y}}
\newcommand{\bbZ}{\mathbb{Z}}

\newcommand{\bmat}{\left( \begin{matrix}}
\newcommand{\emat}{\end{matrix} \right)}

\newcommand{\pmat}{\left( \begin{smallmatrix}}
\newcommand{\epmat}{\end{smallmatrix} \right)}

\newcommand{\lat}{\mathscr{L}}
\newcommand{\mat}[4]{\begin{pmatrix}{#1}&{#2}\\{#3}&{#4}\end{pmatrix}}
\newcommand{\ov}[1]{\overline{#1}}
\newcommand{\res}[1]{\underset{#1}{\RES}\,}
\newcommand{\up}{\upsilon}

\newcommand{\tac}{\textasteriskcentered}

%mahesh macros
\newcommand{\tm}{\textrm}

%Comments
\newcommand{\com}[1]{\vspace{5 mm}\par \noindent
\marginpar{\textsc{Comment}} \framebox{\begin{minipage}[c]{0.95
\textwidth} \tt #1 \end{minipage}}\vspace{5 mm}\par}

\newcommand{\Bmu}{\mbox{$\raisebox{-0.59ex}
  {$l$}\hspace{-0.18em}\mu\hspace{-0.88em}\raisebox{-0.98ex}{\scalebox{2}
  {$\color{white}.$}}\hspace{-0.416em}\raisebox{+0.88ex}
  {$\color{white}.$}\hspace{0.46em}$}{}}  %need graphicx and xcolor. this produces blackboard bold mu 

\newcommand{\hooktwoheadrightarrow}{%
  \hookrightarrow\mathrel{\mspace{-15mu}}\rightarrow
}

\makeatletter
\newcommand{\xhooktwoheadrightarrow}[2][]{%
  \lhook\joinrel
  \ext@arrow 0359\rightarrowfill@ {#1}{#2}%
  \mathrel{\mspace{-15mu}}\rightarrow
}
\makeatother

\renewcommand{\geq}{\geqslant}
    \renewcommand{\leq}{\leqslant}
    
    \newcommand{\bone}{\mathbf{1}}
    \newcommand{\sign}{\mathrm{sign}}
    \newcommand{\eps}{\varepsilon}
    \newcommand{\textui}[1]{\uline{\textit{#1}}}
    
    %\newcommand{\ov}{\overline}
    %\newcommand{\un}{\underline}
    \newcommand{\fin}{\mathrm{fin}}
    
    \newcommand{\chnum}{\titleformat
    {\chapter} % command
    [display] % shape
    {\centering} % format
    {\Huge \color{black} \shadowbox{\thechapter}} % label
    {-0.5em} % sep (space between the number and title)
    {\LARGE \color{black} \underline} % before-code
    }
    
    \newcommand{\chunnum}{\titleformat
    {\chapter} % command
    [display] % shape
    {} % format
    {} % label
    {0em} % sep
    { \begin{flushright} \begin{tabular}{r}  \Huge \color{black}
    } % before-code
    [
    \end{tabular} \end{flushright} \normalsize
    ] % after-code
    }

\newcommand{\littletaller}{\mathchoice{\vphantom{\big|}}{}{}{}}
\newcommand\restr[2]{{% we make the whole thing an ordinary symbol
  \left.\kern-\nulldelimiterspace % automatically resize the bar with \right
  #1 % the function
  \littletaller % pretend it's a little taller at normal size
  \right|_{#2} % this is the delimiter
  }}

\newcommand{\mtext}[1]{\hspace{6pt}\text{#1}\hspace{6pt}}

%This adds a "front cover" page.
%{\thispagestyle{empty}
%\vspace*{\fill}
%\begin{tabular}{l}
%\begin{tabular}{l}
%\includegraphics[scale=0.24]{oxy-logo.png}
%\end{tabular} \\
%\begin{tabular}{l}
%\Large \color{black} Module Theory, Linear Algebra, and Homological Algebra \\ \Large \color{black} Gianluca Crescenzo
%\end{tabular}
%\end{tabular}
%\newpage





%%%%%%%%%%%%%%%%%%%%%%
%%%%DOCUMENT SETUP%%%%
%%%%%%%%%%%%%%%%%%%%%%
\setsecnumdepth{subsection}
\definecolor{darkgreen}{rgb}{0, 0.5976, 0}
\hypersetup{pdfauthor=Gianluca Crescenzo, pdftitle=Real Analysis I Notes, pdfstartview=FitH, colorlinks=true, linkcolor=darkgreen, citecolor=darkgreen}


%This adds a "front cover" page.
%{\thispagestyle{empty}
%\vspace*{\fill}
%\begin{tabular}{l}
%\begin{tabular}{l}
%\includegraphics[scale=0.24]{oxy-logo.png}
%\end{tabular} \\
%\begin{tabular}{l}
%\Large \color{black} Module Theory, Linear Algebra, and Homological Algebra \\ \Large \color{black} Gianluca Crescenzo
%\end{tabular}
%\end{tabular}
\begin{document}
\begin{center}
    { \Large Math 310 \\[0.1in]Homework 3 \\[0.1in]
    Due: 9/27/2024}\\[.25in]
    { Name:} {\underline{Gianluca Crescenzo\hspace*{2in}}}\\[0.15in]
    \end{center}
    \vspace{4pt}
%%%%%%%%%%%%%%%%%%%%%%%%%%%%%%%%%%%%%%%%%%%%%%%%%%%%%%%%%%%%%%%%%%%%%%%%%%%%%%%%%%%%%%%%%%%%%%%%%%%%%%%%%%%%%%%   
    \begin{exercise}
        Find $\Sup{(A)}$ and $\Inf{(A)}$ where:
            \begin{enumerate}[label = (\arabic*)]
                \item $A_1 = \left\{1 - \frac{(-1)^n}{n} \mid n \in \bfN\right\}$.
                \item $A_2 = \left\{\frac{1}{n} - \frac{1}{m} \mid m,n \in \bfN\right\}$.
                \item $A_3 = \left\{ \frac{m}{n} \mid m,n \in \bfN,\hspace{2pt} m+n \leq 10\right\}$.
            \end{enumerate} 
    \end{exercise}
        {\color{red} \begin{proof}
            Claim: $\Inf{(A_1)} = \frac{1}{2}$. Note that $\frac{1}{2}$ is a lowerbound because $\frac{1}{2} \leq a$ for all $a \in A_1$. Let $t$ be a lowerbound of $A_1$. If $t \leq \frac{1}{2}$, then we are done. If $t>\frac{1}{2}$, then $t - \frac{1}{2} > 0$. By the Archimedean Property, there exists an element $n \in \bfN$ with $t - \frac{1}{2} > \frac{1}{n}$ This gives $t > \frac{1}{2} + \frac{1}{n}$, which is a contradiction because $\frac{1}{2}+\frac{1}{n} \in A_1$ for all positive natural numbers. Thus $\inf{(A_1)} = \frac{1}{2}$. Note that $2 \geq 1  + \left|  - \frac{(-1)^n}{n}\right| = 1 + \frac{(-1)^n}{n}$ for all $n \in \bfN$. Hence $2$ is an upper bound. Furthermore, since $2 \in A_1$, it must be the case that $\sup(A_1) = 2$.

            (2) 

            (3) Note that $\frac{1}{9} \leq \frac{m}{n} \leq \frac{9}{1}$ for all $m,n \in \bfN$, $m+n \leq 10$. So $\frac{1}{9}$ is a lower bound of $A_3$ and $\frac{9}{1}$ is an upper bound of $A_3$. Since $\frac{1}{9}, \frac{9}{1} \in A_3$, it must be the case that $\inf(A_3) = \frac{1}{9}$ and $\sup{(A_3)} = \frac{9}{1}$.


        \end{proof}}
%%%%%%%%%%%%%%%%%%%%%%%%%%%%%%%%%%%%%%%%  
    \begin{exercise}
        Suppose $u = \Sup{(A)}$ such that $u \not\in A$. Show that there is a strictly increasing sequence
            \begin{equation*}
            \begin{split}
                t_1 < t_2 < t_3 < ...
            \end{split}
            \end{equation*}
        with $t_n \in A$ and $t_n + \frac{1}{n} > u$ for all $n \geq 1$.
    \end{exercise}  
        {\color{red} \begin{proof}
            Note that for all $\epsilon > 0$, there exists an $a_\epsilon$ with $u - \epsilon < a_\epsilon$. Define:
                \begin{equation*}
                \begin{split}
                    t_1 &> u - 1 \\
                    t_2 &> \max \left\{t_1, u - \frac{1}{2}\right\} \\
                    t_3 &> \max \left\{t_2, u - \frac{1}{3}\right\} \\
                    &\vdots \\
                    t_n &> \max \left\{t_{n-1}, u - \frac{1}{n} \right\}.
                \end{split}
                \end{equation*}
            If $\max \left\{t_{n-1}, u - \frac{1}{n} \right\} = t_{n-1}$, then clearly $t_n > t_{n-1}$. If $\max \left\{t_{n-1}, u - \frac{1}{n} \right\} = u - \frac{1}{n}$, then $t_n > u - \frac{1}{n} > t_{n-1}$. This gives $t_n + \frac{1}{n} > u$ for all $n \geq 1$, and furthermore, we obtain a strictly increasing sequence:
                \begin{equation*}
                \begin{split}
                    t_1 < t_2 < t_3 < ... 
                \end{split}
                \end{equation*}
                \qedhere
        \end{proof}}
%%%%%%%%%%%%%%%%%%%%%%%%%%%%%%%%%%%%%%%%  
    \begin{exercise}
        If $m$ is a lower bound for $A \subseteq \bfR$, show that the following are equivalent:
            \begin{enumerate}[label = (\arabic*)]
                \item $m = \inf{(A)}$.
                \item For all $t>m$, there exists $a_t \in A$ with $a_t <  t$.
                \item For all $\epsilon > 0$ there exists $a_\epsilon \in A$ with $m+\epsilon > a_\epsilon$.
            \end{enumerate}
    \end{exercise}
        {\color{red} \begin{proof}
            Let $m = \Inf{(A)}$. Assuming $t > m$, suppose towards contradiction there does not exist an $a \in A$ with $a < t$. Then it must be the case that $m < t \leq a$ for all $t > m$. This is a contradiction, because $m$ is the greatest lower bound.

            Now assume for all $t > m$, there exists $a_t \in A$ with $a_t < t$. Given $\epsilon > 0$, pick $t = m + \epsilon$. Then by (2) there exists an $a_t$ with $m+ \epsilon > a_t$.

            Now assume for all $\epsilon >0$ there exists $a_\epsilon \in A$ with $m + \epsilon > a_\epsilon$. Given that $m$ is a lower bound for $A$, assume there exists another lower bound for $A$ with $l > m$. Pick $\epsilon = l - m$, then there exists an $a \in A$ with $m + (l - m) > a$. Simplifying yields $l > a$, which contradicts $l$ being a lower bound. Hence $\inf{(A)} = m$.
        \end{proof}}
%%%%%%%%%%%%%%%%%%%%%%%%%%%%%%%%%%%%%%%%  
    \begin{exercise}
        Let $A,B \subseteq \bfR$ be bounded subsets.
            \begin{enumerate}[label = (\arabic*)]
                \item Show that
                    \begin{equation*}
                    \begin{split}
                        \Sup{(A+B)} &= \Sup{(A)} + \Sup{(B)}, \\
                        \Inf(A+B) &= \Inf(A) + \Inf(B).
                    \end{split}
                    \end{equation*}
                \item If $t>0$, show that 
                    \begin{equation*}
                    \begin{split}
                        \Sup(tA) &= t \Sup(A), \\
                        \Inf(tA) &= t \Inf(A).
                    \end{split}
                    \end{equation*}
            \end{enumerate}
                {\color{red} \begin{proof}
                    (1) Define $\Sup{(A)} = u$ and $\Sup{(B)} = v$. Then for all $\epsilon > 0$, there exists $a_\epsilon \in A$, $b_\epsilon \in B$ with $u - \epsilon < a_\epsilon$ and $v - \epsilon < b_\epsilon$. Pick $\epsilon = \frac{\epsilon}{2}$. Then adding both inequalities gives $(u + v) - \epsilon < a_\epsilon + b_\epsilon \in A+B$. Hence $\Sup(A+B) = u + v = \Sup(A) + \Sup(B)$. Similarly, define $\Inf(A) = m$ and $\Inf(B) = n$. Then for all $\epsilon > 0$, there exists $a_\epsilon \in A$, $b_\epsilon \in B$ with $m + \epsilon > a_\epsilon$ and $n + \epsilon > b_\epsilon$. Pick $\epsilon = \frac{\epsilon}{2}$. Then adding both inequalities gives $(m+n)+\epsilon > a_\epsilon + b_\epsilon \in A+B$. Hence $\Inf(A+B) = m + n = \Inf(A) + \Inf(B)$.

                    (2) Let $\sup(A) = u$. Then $a \leq u $ for all $a \in A$. We have that $u - \epsilon < a$ for some $a \in A$. Pick $\epsilon = \frac{\epsilon}{t}$. Then $t u - \epsilon < ta$ for some $t a \in tA$. Hence $\sup(tA) = tu = t\sup(A)$. Similarly, let $\inf(A) = m$. Then $m \leq a$ for all $a \in A$. We have that $m+\epsilon > a$ for some $a \in A$. Pick $\epsilon = \frac{\epsilon}{t}$. Then $tm + \epsilon > ta$ for some $ta \in tA$. Hence $\inf(tA) = tm = t\inf(A)$.
                \end{proof}}
    \end{exercise}
%%%%%%%%%%%%%%%%%%%%%%%%%%%%%%%%%%%%%%%%  
    \begin{exercise}
        Let $I = (0,1)$ denote the open interval and consider the function
            \begin{equation*}
            \begin{split}
                F:I \times I \rightarrow \bfR \mtext{defined by} F(x,y) = 2x + y.
            \end{split}
            \end{equation*}
        Compute
            \begin{equation*}
            \begin{split}
                \sup_{y \in I} \left(\inf_{x \in I} F(x,y)\right),
            \end{split}
            \end{equation*}
        and
            \begin{equation*}
            \begin{split}
                \inf_{y \in I} \left(\sup_{x \in I} F(x,y)\right).
            \end{split}
            \end{equation*}
        Are they equal?
    \end{exercise}
        {\color{red} \begin{proof}
            Observe that:
                \begin{equation*}
                \begin{split}
                    \sup_{y \in I}\left(\inf_{x \in I}(2x+y)\right)
                    & = \sup_{y \in I}\left(2\inf_{x \in I}x+\inf_{x \in I}y\right) \\
                    & = \sup_{y \in I}y \\
                    & = 1,
                \end{split}
                \end{equation*}
                \begin{equation*}
                \begin{split}
                    \inf_{y \in I}\left(\sup_{x \in I}(2x+y)\right)
                    & = \inf_{y \in I} \left( \sup_{x \in I}2x + \sup_{x \in I}y\right) \\
                    & = \inf_{y \in I} \left( 2 + y\right) \\
                    & = \inf_{y \in I}2 + \inf_{y \in I}y \\
                    & = 2.
                \end{split}
                \end{equation*}
        \end{proof}}
%%%%%%%%%%%%%%%%%%%%%%%%%%%%%%%%%%%%%%%%
    \begin{exercise}
        Let $D$ be a nonempty set and consider the set of all bounded functions:
            \begin{equation*}
            \begin{split}
                \ell_\infty(D) := \{f \mid f:D \rightarrow \bfR \hspace{4pt}\text{is bounded}\}
            \end{split}
            \end{equation*}
        with point-wise addition and scalar multiplication. Show that
            \begin{equation*}
            \begin{split}
                d_u(f,g) := \sup_{x \in D}\left|f(x) - g(x)\right|
            \end{split}
            \end{equation*}
        defines a metric on $\ell_\infty(D)$. We call $d_u$ the \textbf{uniform metric}.
    \end{exercise}
        {\color{red} \begin{proof}
            Observe that:
                \begin{equation*}
                \begin{split}
                    d_u(f,g)
                    & = \sup_{x \in D}\left(\left|f(x) - g(x)\right|\right) \\
                    & = \sup_{x \in D}\left(\left|g(x) - f(x)\right|\right) \\
                    & = d_u(g,f).
                \end{split}
                \end{equation*}
            Thus $(\ell_\infty,d_u)$ is symmetric. We also have that:
                \begin{equation*}
                \begin{split}
                    d_u(f,h)
                    & = \sup_{x \in D}\left(\left|f(x) - h(x)\right|\right) \\
                    & = \sup_{x \in D}\left(\left|f(x) - g(x) + g(x) - h(x)\right|\right)\\
                    & \leq \sup_{x \in D}\left(\left|f(x)-g(x)\right| + \left|g(x) - h(x)\right|\right)\\
                    & = \sup_{x \in D}\left(\left|f(x)-g(x)\right|\right) + \sup_{x \in D}\left(\left|g(x)-h(x)\right|\right)\\
                    & = d(f,g) + d(g,h).
                \end{split}
                \end{equation*}
            Hence $(\ell_\infty,d_u)$ satisfies the triangle-inequality. Furthermore:
                \begin{equation*}
                \begin{split}
                    d_u(f,f)
                    & = \sup_{x \in D}\left(\left|f(x) - f(x)\right|\right) \\
                    & = \sup_{x \in D}0 \\
                    & = 0.
                \end{split}
                \end{equation*}
            Lastly $d_u(f,g)=0$ implies $\sup_{x \in D}\left(\left|f(x) - g(x)\right|\right) = 0$. By definition of the absolute value, $|f(x)-g(x)| \geq 0$, so it must be the case that $|f(x)-g(x)| = 0$. Hence $f(x) = g(x)$, establishing that $(\ell_\infty,d_u)$ forms a metric space.
        \end{proof}}
%%%%%%%%%%%%%%%%%%%%%%%%%%%%%%%%%%%%%%%%
    \begin{exercise}
        Let $f,g:D \rightarrow \bfR$ be bounded functions. Show that
            \begin{enumerate}[label = (\arabic*)]
                \item $\sup_{x\in D}(f+g)(x) \leq \sup_{x\in D}f(x) + \sup_{x\in D}g(x)$.
                \item $\inf_{x\in D}(f+g)(x) \geq \inf_{x\in D}f(x) + \inf_{x\in D}g(x)$.
                \item $\left|\sup_{x \in D}f(x) - \sup_{x \in D}g(x)\right| \leq \sup_{x\in D}\left|f(x) - g(x)\right|$.
            \end{enumerate}
    \end{exercise}
%%%%%%%%%%%%%%%%%%%%%%%%%%%%%%%%%%%%%%%%
    \begin{exercise}
        Find $\bigcap_{n = 1}^\infty I_n$ where
            \begin{enumerate}[label = (\arabic*)]
                \item $I_n = \left[0,\frac{1}{n}\right]$,
                \item $I_n = \left(0, \frac{1}{n}\right)$,
                \item $I_n = \left[n , \infty\right)$.
            \end{enumerate}
    \end{exercise}
        {\color{red} \begin{proof}
            (1) Note that $[0,\frac{1}{n}]$ is closed and bounded for all $n \geq 1$. Note that:
                \begin{equation*}
                \begin{split}
                    \inf \{\text{length}\left(\left[0,1/n\right]\right) \mid n \geq 1 \} 
                     &= \inf_{n \geq 1} \left(\frac{1}{n} - 0\right) 
                      = 0.
                \end{split}
                \end{equation*}
            By the Nested Interval Theorem:
                \begin{equation*}
                \begin{split}
                    \bigcap_{n=1}^\infty \left[0 , \frac{1}{n}\right] = \sup_{n \geq 1} 0 = \inf_{n \geq 1} \frac{1}{n} = 0.
                \end{split}
                \end{equation*}

            (2) Claim: $\bigcap_{n=1}^\infty \left(0, \frac{1}{n}\right) = \emptyset$. Suppose towards contradiction there exists $t \in \bigcap_{n=1}^\infty \left(0, \frac{1}{n}\right)$. Then $t \in \left(0 ,\frac{1}{n}\right)$ for all $n \geq 1$. So $t < \frac{1}{n}$ implies $\frac{1}{t} > n$ for all $n\geq 1$, meaning $\bfN$ is bounded above. This is a contradiction, hence $\bigcap_{n=1}^\infty \left(0, \frac{1}{n}\right) = \emptyset$.

            (3) Claim: $\bigcap_{n=1}^\infty \left[n, \infty\right) = \emptyset$. Suppose towards contradiction there exists $t \in \bigcap_{n=1}^\infty \left[n, \infty\right)$. Then $t \in [n,\infty)$ for all $n \geq 1$. So $t \geq n$ for all $n \geq 1$. Hence $\bfN$ is bounded above, which is a contradiction. Thus $\bigcap_{n=1}^\infty \left[n, \infty\right) = \emptyset$.
        \end{proof}}
%%%%%%%%%%%%%%%%%%%%%%%%%%%%%%%%%%%%%%%%
    \begin{exercise}
        If $x>0$, show that there is an $n \in \bfN$ with $\frac{1}{2^n} <x$.
    \end{exercise}
        {\color{red} \begin{proof}
            By the Archimedean Property 2, there exists $n \in \bfN$ such that $0 < \frac{1}{n} < x$. Claim: $\frac{1}{2^n} < \frac{1}{n}$. It suffices to show that $2^n > n$. Bernoulli's inequality gives $(1+1)^n \geq 1 + n$, hence $2^n > n$.
        \end{proof}}
%%%%%%%%%%%%%%%%%%%%%%%%%%%%%%%%%%%%%%%%
    \begin{exercise}
        The \textbf{Dyadic Rationals} are defined as 
            \begin{equation*}
            \begin{split}
                \bfD := \left\{\frac{m}{2^n} \mid m,n \in \bfZ\right\}.
            \end{split}
            \end{equation*}
        Show that $\bfD \subseteq \bfR$ is dense.
    \end{exercise}
        {\color{red} \begin{proof}
            Let $I = (a,b)$. Then $b-a > 0$. By Archimedean Property 2 there exists $n \in \bfN$ such that $b-a > \frac{1}{n}$. Exercise 9 gives that $b - a > \frac{1}{2^n}$ for some $n \in \bfZ$. This simplifies to $2^nb > 1 + 2^na$. Since $2^na \in \bfR$, there exists $m \in \bfZ$ with $m-1 \leq 2^na < m$, implying that $a < \frac{m}{2^n}$. Furthermore, we also have that $m \leq 1 + 2^n a < m+1$, and substituting for $2^n b$ gives $m < 2^n b$. So $\frac{m}{2^n} < b$, which means $\frac{m}{2^n} \in (a,b)$. Thus $I \cap D \neq \emptyset$.
        \end{proof}}
\end{document}
