%%%%%%%PACKAGES%%%%%%%
\documentclass[11pt,twoside,openany]{memoir}

\usepackage[p,osf]{scholax}
% T1 and textcomp are loaded by package. Change that here, if you want
% load sans and typewriter packages here, if needed
\usepackage{amsmath,amsthm}% must be loaded before newtxmath
% amssymb should not be loaded
\usepackage[scaled=1.075,ncf,vvarbb]{newtxmath}% need to scale up math package
% vvarbb selects the STIX version of blackboard bold.

\usepackage{titlesec}
        \titlespacing*{\chapter}
        {0pt} % Left margin
        {*1} % Space before the chapter number (increase this value for more space)
        {0pt} % Space after the chapter number and before the title
\usepackage{anyfontsize}
\usepackage{fancybox}
\usepackage[dvipsnames,svgnames,x11names,hyperref]{xcolor}
\usepackage{enumerate}
\usepackage{comment}
\usepackage{amsfonts}
\usepackage{mathrsfs}
\usepackage{fullpage}
\usepackage{bm}
\usepackage{cprotect}
\usepackage{calligra}
\usepackage{emptypage}
\usepackage{titleps}
\usepackage{microtype}
\usepackage{float}
\usepackage{ocgx}
\usepackage{appendix}
\usepackage{graphicx}
\usepackage{pdfcomment}
\usepackage{enumitem}
\usepackage{mathtools}
\usepackage{tikz-cd}
\usepackage{relsize}
\usepackage[font=footnotesize,labelfont=bf]{caption}
\usepackage{changepage}
\usepackage{xcolor}
\usepackage{ulem}
\usepackage{pgfplots}
\usepackage{marginnote}
        \newcommand*{\mnote}[1]{ % <----------
        \checkoddpage
        \ifoddpage
            \marginparmargin{left}
        \else
            \marginparmargin{right}
        \fi
            \marginnote{\tiny \textcolor{oorange}{#1}}
        }

\usepackage{datetime}
        \newdateformat{specialdate}{\THEYEAR\ \monthname\ \THEDAY}
\usepackage[margin=0.9in]{geometry}
        \setlength{\voffset}{-0.4in}
        \setlength{\headsep}{30pt}
\usepackage{fancyhdr}
        \fancyhf{}
        \pagestyle{fancy}
        \cfoot{\footnotesize \thepage}
        \fancyhead[R]{\tiny \rightmark}
        \fancyhead[L]{\tiny \leftmark}
\usepackage[T1]{fontenc}% http://ctan.org/pkg/fontenc
\usepackage[outline]{contour}% http://ctan.org/pkg/contour
        \renewcommand{\arraystretch}{1.5}
        \contourlength{0.4pt}
        \contournumber{10}%
\usepackage{letterspace}
        \linespread{1.1}
\usepackage{thmtools}
        \declaretheoremstyle[
            spaceabove=15pt,
            headfont=\normalfont\bfseries,
            notefont=\mdseries, notebraces={(}{)},
            bodyfont=\normalfont,
            postheadspace=0.5em
            %qed=\qedsymbol
            ]{defs}

        \declaretheoremstyle[
            spaceabove=15pt, % space above the theorem
            headfont=\normalfont\bfseries,
            bodyfont=\normalfont\itshape,
            postheadspace=0.5em
            ]{thmstyle}
        
        \declaretheorem[
            style=thmstyle,
            numberwithin=section
        ]{theorem}

        \declaretheorem[
            style=thmstyle,
            sibling=theorem,
        ]{proposition}

        \declaretheorem[
            style=thmstyle,
            sibling=theorem,
        ]{lemma}

        \declaretheorem[
            style=thmstyle,
            sibling=theorem,
        ]{corollary}

        \declaretheorem[
            numberwithin=section,
            style=defs,
        ]{example}

        \declaretheorem[
            numberwithin=section,
            style=defs,
        ]{definition}

        \declaretheorem[
            style=thmstyle,
            sibling=theorem,
            numberwithin=section,
        ]{exercise}

        \declaretheorem[
            numbered=unless unique,
            shaded={rulecolor=black,
        rulewidth=1pt, bgcolor={rgb}{1,1,1}}
        ]{axiom}

    \declaretheorem[numbered=unless unique,style=defs]{note}
    \declaretheorem[numbered=unless unique,style=defs]{question}
    \declaretheorem[numbered=no,style=remark]{answer}
    \declaretheorem[numbered=no,style=remark]{remark}
    \declaretheorem[numbered=no,style=remark]{solution}

\usepackage{hyperref}
        






%to make the correct symbol for Sha
%\newcommand\cyr{%
%\renewcommand\rmdefault{wncyr}%
%\renewcommand\sfdefault{wncyss}%
%\renewcommand\encodingdefault{OT2}%
%\normalfont \selectfont} \DeclareTextFontCommand{\textcyr}{\cyr}


\DeclareMathOperator{\ab}{ab}
\newcommand{\absgal}{\G_{\bbQ}}
\DeclareMathOperator{\ad}{ad}
\DeclareMathOperator{\adj}{adj}
\DeclareMathOperator{\alg}{alg}
\DeclareMathOperator{\Alt}{Alt}
\DeclareMathOperator{\Ann}{Ann}
\DeclareMathOperator{\arith}{arith}
\DeclareMathOperator{\Aut}{Aut}
\DeclareMathOperator{\Be}{B}
\DeclareMathOperator{\card}{card}
\DeclareMathOperator{\Char}{char}
\DeclareMathOperator{\csp}{csp}
\DeclareMathOperator{\codim}{codim}
\DeclareMathOperator{\coker}{coker}
\DeclareMathOperator{\coh}{H}
\DeclareMathOperator{\compl}{compl}
\DeclareMathOperator{\conj}{conj}
\DeclareMathOperator{\cont}{cont}
\DeclareMathOperator{\crys}{crys}
\DeclareMathOperator{\Crys}{Crys}
\DeclareMathOperator{\cusp}{cusp}
\DeclareMathOperator{\diag}{diag}
\DeclareMathOperator{\disc}{disc}
\DeclareMathOperator{\dR}{dR}
\DeclareMathOperator{\Eis}{Eis}
\DeclareMathOperator{\End}{End}
\DeclareMathOperator{\ev}{ev}
\DeclareMathOperator{\eval}{eval}
\DeclareMathOperator{\Eq}{Eq}
\DeclareMathOperator{\Ext}{Ext}
\DeclareMathOperator{\Fil}{Fil}
\DeclareMathOperator{\Fitt}{Fitt}
\DeclareMathOperator{\Frob}{Frob}
\DeclareMathOperator{\G}{G}
\DeclareMathOperator{\Gal}{Gal}
\DeclareMathOperator{\GL}{GL}
\DeclareMathOperator{\Gr}{Gr}
\DeclareMathOperator{\Graph}{Graph}
\DeclareMathOperator{\GSp}{GSp}
\DeclareMathOperator{\GUn}{GU}
\DeclareMathOperator{\Hom}{Hom}
\DeclareMathOperator{\id}{id}
\DeclareMathOperator{\Id}{Id}
\DeclareMathOperator{\Ik}{Ik}
\DeclareMathOperator{\IM}{Im}
\DeclareMathOperator{\Image}{im}
\DeclareMathOperator{\Ind}{Ind}
\DeclareMathOperator{\Inf}{inf}
\DeclareMathOperator{\Isom}{Isom}
\DeclareMathOperator{\J}{J}
\DeclareMathOperator{\Jac}{Jac}
\DeclareMathOperator{\lcm}{lcm}
\DeclareMathOperator{\length}{length}
\DeclareMathOperator{\Log}{Log}
\DeclareMathOperator{\M}{M}
\DeclareMathOperator{\Mat}{Mat}
\DeclareMathOperator{\N}{N}
\DeclareMathOperator{\Nm}{Nm}
\DeclareMathOperator{\NIk}{N-Ik}
\DeclareMathOperator{\NSK}{N-SK}
\DeclareMathOperator{\new}{new}
\DeclareMathOperator{\obj}{obj}
\DeclareMathOperator{\old}{old}
\DeclareMathOperator{\ord}{ord}
\DeclareMathOperator{\Or}{O}
\DeclareMathOperator{\PGL}{PGL}
\DeclareMathOperator{\PGSp}{PGSp}
\DeclareMathOperator{\rank}{rank}
\DeclareMathOperator{\Rel}{Rel}
\DeclareMathOperator{\Real}{Re}
\DeclareMathOperator{\RES}{res}
\DeclareMathOperator{\Res}{Res}
%\DeclareMathOperator{\Sha}{\textcyr{Sh}}
\DeclareMathOperator{\Sel}{Sel}
\DeclareMathOperator{\semi}{ss}
\DeclareMathOperator{\sgn}{sign}
\DeclareMathOperator{\SK}{SK}
\DeclareMathOperator{\SL}{SL}
\DeclareMathOperator{\SO}{SO}
\DeclareMathOperator{\Sp}{Sp}
\DeclareMathOperator{\Span}{span}
\DeclareMathOperator{\Spec}{Spec}
\DeclareMathOperator{\spin}{spin}
\DeclareMathOperator{\st}{st}
\DeclareMathOperator{\St}{St}
\DeclareMathOperator{\SUn}{SU}
\DeclareMathOperator{\supp}{supp}
\DeclareMathOperator{\Sup}{sup}
\DeclareMathOperator{\Sym}{Sym}
\DeclareMathOperator{\Tam}{Tam}
\DeclareMathOperator{\tors}{tors}
\DeclareMathOperator{\tr}{tr}
\DeclareMathOperator{\un}{un}
\DeclareMathOperator{\Un}{U}
\DeclareMathOperator{\val}{val}
\DeclareMathOperator{\vol}{vol}

\DeclareMathOperator{\Sets}{S \mkern1.04mu e \mkern1.04mu t \mkern1.04mu s}
    \newcommand{\cSets}{\scalebox{1.02}{\contour{black}{$\Sets$}}}
    
\DeclareMathOperator{\Groups}{G \mkern1.04mu r \mkern1.04mu o \mkern1.04mu u \mkern1.04mu p \mkern1.04mu s}
    \newcommand{\cGroups}{\scalebox{1.02}{\contour{black}{$\Groups$}}}

\DeclareMathOperator{\TTop}{T \mkern1.04mu o \mkern1.04mu p}
    \newcommand{\cTop}{\scalebox{1.02}{\contour{black}{$\TTop$}}}

\DeclareMathOperator{\Htp}{H \mkern1.04mu t \mkern1.04mu p}
    \newcommand{\cHtp}{\scalebox{1.02}{\contour{black}{$\Htp$}}}

\DeclareMathOperator{\Mod}{M \mkern1.04mu o \mkern1.04mu d}
    \newcommand{\cMod}{\scalebox{1.02}{\contour{black}{$\Mod$}}}

\DeclareMathOperator{\Ab}{A \mkern1.04mu b}
    \newcommand{\cAb}{\scalebox{1.02}{\contour{black}{$\Ab$}}}

\DeclareMathOperator{\Rings}{R \mkern1.04mu i \mkern1.04mu n \mkern1.04mu g \mkern1.04mu s}
    \newcommand{\cRings}{\scalebox{1.02}{\contour{black}{$\Rings$}}}

\DeclareMathOperator{\ComRings}{C \mkern1.04mu o \mkern1.04mu m \mkern1.04mu R \mkern1.04mu i \mkern1.04mu n \mkern1.04mu g \mkern1.04mu s}
    \newcommand{\cComRings}{\scalebox{1.05}{\contour{black}{$\ComRings$}}}

\DeclareMathOperator{\hHom}{H \mkern1.04mu o \mkern1.04mu m}
    \newcommand{\cHom}{\scalebox{1.02}{\contour{black}{$\hHom$}}}

         %  \item $\cGroups$
          %  \item $\cTop$
          %  \item $\cHtp$
          %  \item $\cMod$




\renewcommand{\k}{\kappa}
\newcommand{\Ff}{F_{f}}
\newcommand{\ts}{\,^{t}\!}


%Mathcal

\newcommand{\cA}{\mathcal{A}}
\newcommand{\cB}{\mathcal{B}}
\newcommand{\cC}{\mathcal{C}}
\newcommand{\cD}{\mathcal{D}}
\newcommand{\cE}{\mathcal{E}}
\newcommand{\cF}{\mathcal{F}}
\newcommand{\cG}{\mathcal{G}}
\newcommand{\cH}{\mathcal{H}}
\newcommand{\cI}{\mathcal{I}}
\newcommand{\cJ}{\mathcal{J}}
\newcommand{\cK}{\mathcal{K}}
\newcommand{\cL}{\mathcal{L}}
\newcommand{\cM}{\mathcal{M}}
\newcommand{\cN}{\mathcal{N}}
\newcommand{\cO}{\mathcal{O}}
\newcommand{\cP}{\mathcal{P}}
\newcommand{\cQ}{\mathcal{Q}}
\newcommand{\cR}{\mathcal{R}}
\newcommand{\cS}{\mathcal{S}}
\newcommand{\cT}{\mathcal{T}}
\newcommand{\cU}{\mathcal{U}}
\newcommand{\cV}{\mathcal{V}}
\newcommand{\cW}{\mathcal{W}}
\newcommand{\cX}{\mathcal{X}}
\newcommand{\cY}{\mathcal{Y}}
\newcommand{\cZ}{\mathcal{Z}}


%mathfrak (missing \fi)

\newcommand{\fa}{\mathfrak{a}}
\newcommand{\fA}{\mathfrak{A}}
\newcommand{\fb}{\mathfrak{b}}
\newcommand{\fB}{\mathfrak{B}}
\newcommand{\fc}{\mathfrak{c}}
\newcommand{\fC}{\mathfrak{C}}
\newcommand{\fd}{\mathfrak{d}}
\newcommand{\fD}{\mathfrak{D}}
\newcommand{\fe}{\mathfrak{e}}
\newcommand{\fE}{\mathfrak{E}}
\newcommand{\ff}{\mathfrak{f}}
\newcommand{\fF}{\mathfrak{F}}
\newcommand{\fg}{\mathfrak{g}}
\newcommand{\fG}{\mathfrak{G}}
\newcommand{\fh}{\mathfrak{h}}
\newcommand{\fH}{\mathfrak{H}}
\newcommand{\fI}{\mathfrak{I}}
\newcommand{\fj}{\mathfrak{j}}
\newcommand{\fJ}{\mathfrak{J}}
\newcommand{\fk}{\mathfrak{k}}
\newcommand{\fK}{\mathfrak{K}}
\newcommand{\fl}{\mathfrak{l}}
\newcommand{\fL}{\mathfrak{L}}
\newcommand{\fm}{\mathfrak{m}}
\newcommand{\fM}{\mathfrak{M}}
\newcommand{\fn}{\mathfrak{n}}
\newcommand{\fN}{\mathfrak{N}}
\newcommand{\fo}{\mathfrak{o}}
\newcommand{\fO}{\mathfrak{O}}
\newcommand{\fp}{\mathfrak{p}}
\newcommand{\fP}{\mathfrak{P}}
\newcommand{\fq}{\mathfrak{q}}
\newcommand{\fQ}{\mathfrak{Q}}
\newcommand{\fr}{\mathfrak{r}}
\newcommand{\fR}{\mathfrak{R}}
\newcommand{\fs}{\mathfrak{s}}
\newcommand{\fS}{\mathfrak{S}}
\newcommand{\ft}{\mathfrak{t}}
\newcommand{\fT}{\mathfrak{T}}
\newcommand{\fu}{\mathfrak{u}}
\newcommand{\fU}{\mathfrak{U}}
\newcommand{\fv}{\mathfrak{v}}
\newcommand{\fV}{\mathfrak{V}}
\newcommand{\fw}{\mathfrak{w}}
\newcommand{\fW}{\mathfrak{W}}
\newcommand{\fx}{\mathfrak{x}}
\newcommand{\fX}{\mathfrak{X}}
\newcommand{\fy}{\mathfrak{y}}
\newcommand{\fY}{\mathfrak{Y}}
\newcommand{\fz}{\mathfrak{z}}
\newcommand{\fZ}{\mathfrak{Z}}


%mathbf

\newcommand{\bfA}{\mathbf{A}}
\newcommand{\bfB}{\mathbf{B}}
\newcommand{\bfC}{\mathbf{C}}
\newcommand{\bfD}{\mathbf{D}}
\newcommand{\bfE}{\mathbf{E}}
\newcommand{\bfF}{\mathbf{F}}
\newcommand{\bfG}{\mathbf{G}}
\newcommand{\bfH}{\mathbf{H}}
\newcommand{\bfI}{\mathbf{I}}
\newcommand{\bfJ}{\mathbf{J}}
\newcommand{\bfK}{\mathbf{K}}
\newcommand{\bfL}{\mathbf{L}}
\newcommand{\bfM}{\mathbf{M}}
\newcommand{\bfN}{\mathbf{N}}
\newcommand{\bfO}{\mathbf{O}}
\newcommand{\bfP}{\mathbf{P}}
\newcommand{\bfQ}{\mathbf{Q}}
\newcommand{\bfR}{\mathbf{R}}
\newcommand{\bfS}{\mathbf{S}}
\newcommand{\bfT}{\mathbf{T}}
\newcommand{\bfU}{\mathbf{U}}
\newcommand{\bfV}{\mathbf{V}}
\newcommand{\bfW}{\mathbf{W}}
\newcommand{\bfX}{\mathbf{X}}
\newcommand{\bfY}{\mathbf{Y}}
\newcommand{\bfZ}{\mathbf{Z}}

\newcommand{\bfa}{\mathbf{a}}
\newcommand{\bfb}{\mathbf{b}}
\newcommand{\bfc}{\mathbf{c}}
\newcommand{\bfd}{\mathbf{d}}
\newcommand{\bfe}{\mathbf{e}}
\newcommand{\bff}{\mathbf{f}}
\newcommand{\bfg}{\mathbf{g}}
\newcommand{\bfh}{\mathbf{h}}
\newcommand{\bfi}{\mathbf{i}}
\newcommand{\bfj}{\mathbf{j}}
\newcommand{\bfk}{\mathbf{k}}
\newcommand{\bfl}{\mathbf{l}}
\newcommand{\bfm}{\mathbf{m}}
\newcommand{\bfn}{\mathbf{n}}
\newcommand{\bfo}{\mathbf{o}}
\newcommand{\bfp}{\mathbf{p}}
\newcommand{\bfq}{\mathbf{q}}
\newcommand{\bfr}{\mathbf{r}}
\newcommand{\bfs}{\mathbf{s}}
\newcommand{\bft}{\mathbf{t}}
\newcommand{\bfu}{\mathbf{u}}
\newcommand{\bfv}{\mathbf{v}}
\newcommand{\bfw}{\mathbf{w}}
\newcommand{\bfx}{\mathbf{x}}
\newcommand{\bfy}{\mathbf{y}}
\newcommand{\bfz}{\mathbf{z}}

%blackboard bold

\newcommand{\bbA}{\mathbb{A}}
\newcommand{\bbB}{\mathbb{B}}
\newcommand{\bbC}{\mathbb{C}}
\newcommand{\bbD}{\mathbb{D}}
\newcommand{\bbE}{\mathbb{E}}
\newcommand{\bbF}{\mathbb{F}}
\newcommand{\bbG}{\mathbb{G}}
\newcommand{\bbH}{\mathbb{H}}
\newcommand{\bbI}{\mathbb{I}}
\newcommand{\bbJ}{\mathbb{J}}
\newcommand{\bbK}{\mathbb{K}}
\newcommand{\bbL}{\mathbb{L}}
\newcommand{\bbM}{\mathbb{M}}
\newcommand{\bbN}{\mathbb{N}}
\newcommand{\bbO}{\mathbb{O}}
\newcommand{\bbP}{\mathbb{P}}
\newcommand{\bbQ}{\mathbb{Q}}
\newcommand{\bbR}{\mathbb{R}}
\newcommand{\bbS}{\mathbb{S}}
\newcommand{\bbT}{\mathbb{T}}
\newcommand{\bbU}{\mathbb{U}}
\newcommand{\bbV}{\mathbb{V}}
\newcommand{\bbW}{\mathbb{W}}
\newcommand{\bbX}{\mathbb{X}}
\newcommand{\bbY}{\mathbb{Y}}
\newcommand{\bbZ}{\mathbb{Z}}

\newcommand{\bmat}{\left( \begin{matrix}}
\newcommand{\emat}{\end{matrix} \right)}

\newcommand{\pmat}{\left( \begin{smallmatrix}}
\newcommand{\epmat}{\end{smallmatrix} \right)}

\newcommand{\lat}{\mathscr{L}}
\newcommand{\mat}[4]{\begin{pmatrix}{#1}&{#2}\\{#3}&{#4}\end{pmatrix}}
\newcommand{\ov}[1]{\overline{#1}}
\newcommand{\res}[1]{\underset{#1}{\RES}\,}
\newcommand{\up}{\upsilon}

\newcommand{\tac}{\textasteriskcentered}

%mahesh macros
\newcommand{\tm}{\textrm}

%Comments
\newcommand{\com}[1]{\vspace{5 mm}\par \noindent
\marginpar{\textsc{Comment}} \framebox{\begin{minipage}[c]{0.95
\textwidth} \tt #1 \end{minipage}}\vspace{5 mm}\par}

\newcommand{\Bmu}{\mbox{$\raisebox{-0.59ex}
  {$l$}\hspace{-0.18em}\mu\hspace{-0.88em}\raisebox{-0.98ex}{\scalebox{2}
  {$\color{white}.$}}\hspace{-0.416em}\raisebox{+0.88ex}
  {$\color{white}.$}\hspace{0.46em}$}{}}  %need graphicx and xcolor. this produces blackboard bold mu 

\newcommand{\hooktwoheadrightarrow}{%
  \hookrightarrow\mathrel{\mspace{-15mu}}\rightarrow
}

\makeatletter
\newcommand{\xhooktwoheadrightarrow}[2][]{%
  \lhook\joinrel
  \ext@arrow 0359\rightarrowfill@ {#1}{#2}%
  \mathrel{\mspace{-15mu}}\rightarrow
}
\makeatother

\renewcommand{\geq}{\geqslant}
    \renewcommand{\leq}{\leqslant}
    
    \newcommand{\bone}{\mathbf{1}}
    \newcommand{\sign}{\mathrm{sign}}
    \newcommand{\eps}{\varepsilon}
    \newcommand{\textui}[1]{\uline{\textit{#1}}}
    
    %\newcommand{\ov}{\overline}
    %\newcommand{\un}{\underline}
    \newcommand{\fin}{\mathrm{fin}}
    
    \newcommand{\chnum}{\titleformat
    {\chapter} % command
    [display] % shape
    {\centering} % format
    {\Huge \color{black} \shadowbox{\thechapter}} % label
    {-0.5em} % sep (space between the number and title)
    {\LARGE \color{black} \underline} % before-code
    }
    
    \newcommand{\chunnum}{\titleformat
    {\chapter} % command
    [display] % shape
    {} % format
    {} % label
    {0em} % sep
    { \begin{flushright} \begin{tabular}{r}  \Huge \color{black}
    } % before-code
    [
    \end{tabular} \end{flushright} \normalsize
    ] % after-code
    }

\newcommand{\littletaller}{\mathchoice{\vphantom{\big|}}{}{}{}}
\newcommand\restr[2]{{% we make the whole thing an ordinary symbol
  \left.\kern-\nulldelimiterspace % automatically resize the bar with \right
  #1 % the function
  \littletaller % pretend it's a little taller at normal size
  \right|_{#2} % this is the delimiter
  }}

\newcommand{\mtext}[1]{\hspace{6pt}\text{#1}\hspace{6pt}}

%This adds a "front cover" page.
%{\thispagestyle{empty}
%\vspace*{\fill}
%\begin{tabular}{l}
%\begin{tabular}{l}
%\includegraphics[scale=0.24]{oxy-logo.png}
%\end{tabular} \\
%\begin{tabular}{l}
%\Large \color{black} Module Theory, Linear Algebra, and Homological Algebra \\ \Large \color{black} Gianluca Crescenzo
%\end{tabular}
%\end{tabular}
%\newpage


\setsecnumdepth{subsection}
\definecolor{darkgreen}{rgb}{0, 0.5976, 0}
\hypersetup{pdfauthor=Gianluca Crescenzo, pdftitle=Real Analysis I Notes, pdfstartview=FitH, colorlinks=true, linkcolor=darkgreen, citecolor=darkgreen}

\begin{document}

\pagenumbering{roman}
\tableofcontents

\chunnum
\vfill
\specialdate
Last update: \today
\chnum

\chapter{Orderings and Functions}\label{chapter:orderings-and-functions}

\pagenumbering{arabic}
\vspace{12pt}

\section{Basic Notation}
    \begin{definition}
        \phantom{a}
        \begin{enumerate}[label = (\arabic*)]
            \item The \textui{natural numbers} are defined as $\bfN = \{1,2,3,...\}$,
            \item The \textui{positive integers} are defined as $\bfN_0 = \bfZ^{+} = \{0, 1, 2, 3, ...\}$,
            \item The \textui{integers} are defined as $\bfZ = \{0, \pm 1, \pm 2, \pm 3,... \}$,
            \item The \textui{rational numbers} are defined as $\bfQ = \{\frac{a}{b} \mid a,b \in \bfZ, b \neq 0\}$,
            \item The \textui{real numbers} are "defined" (we will get more into this later) as the set $(-\infty, \infty)$,
            \item The \textui{complex numbers} are defined as $\bfC = \{a + bi \mid a,b \in \bfR, i^2 = -1\}$.
        \end{enumerate}
    \end{definition}
    \begin{example}
        Note that $\sqrt{2}, \pi, e \not\in \bfQ$, as they cannot be expressed as fractions.
    \end{example}

    \begin{definition}
        Let $A$ and $B$ be sets. The \textui{cartesian product} is defined as $A \times B = \{(a,b) \mid a \in A, b \in B\}$.
    \end{definition}

    \begin{definition}
        A \textui{relation} from $A$ to $B$ is a subset $R \subseteq A \times B$. Typically, when one says "a relation on $A$" that means a relation from $A$ to $A$; i.e., $R \subseteq A \times A$.
    \end{definition}

    \begin{definition}
        Let $A$ be a set and $R$ a relation on $A$. Then $R$ is:
            \begin{enumerate}[label = (\arabic*)]
                \item \textui{reflexive} if $(a,a) \in R$ for all $a \in A$,
                \item \textui{transitive} if $(a,b),(b,c) \in R$ implies $(a,c) \in R$,
                \item \textui{symmetric} if $(a,b) \in R$ implies $(b,a) \in R$, and
                \item \textui{antisymmetric} if $(a,b),(b,a) \in R$ implies $a = b$.
            \end{enumerate}
    \end{definition}
    
\section{Orderings}
    \begin{definition}
        Let $A$ be a set. An \textui{ordering} of $A$ is a relation $R$ on $A$ that is reflexive, transitive, and antisymmetric. If this is the case, we write $(a,b) \in R$ as $a \leq_R b$. If $A$ is an ordered set we write it as the ordered pair $(A,\leq_R)$ (or just $A$ if the ordering is obvious by context).
    \end{definition}

    \begin{example}\label{example:orderings}
        \phantom{a}
        \begin{enumerate}[label = (\arabic*)]
            \item Let $m,n \in \bfZ$. The \textui{algebraic ordering} $\leq_a$ is defined as follows: $m \leq_a n$ if and only if there exists an element $k \in \bfN_0$ with $m + k = n$.
            \item The set of natural numbers $\bfN$ equipped with the relation of divisibility form an ordering. Let $m,n \in \bfN$. Then $m \leq_d n$ if and only if $m \mid n$.
            \item Let $S$ be any set. The subsets of $S$ (i.e., elements of its power set) equipped with the relation of inclusion form an ordering. Let $A,B \in \cP(S)$. Then $A \leq_{\cP(S)} B$ if and only if $A \subseteq B$.
            \item The set of rational numbers $\bfQ$ form an algebraic ordering as follows: if $\frac{a}{b},\frac{c}{d} \in \bfQ$, then $\frac{a}{b} \leq_a \frac{c}{d}$ if and only if $ad \leq_a bc$ (in $\bfZ$).
        \end{enumerate}
    \end{example}

    \begin{definition}
        An ordered set $(A, \leq_R)$ is \textui{total} (or \textui{linear}) if for all $a,b \in A$ we have that $a \leq_R b$ or $b \leq_R a$.
    \end{definition}

    \begin{example}
        The ordered sets $(\bfZ,\leq_a)$ and $(\bfQ,\leq_a)$ are total orderings, whereas $(\bfN,\leq_d)$ and $(\cP(S),\leq_{\cP(S)})$ are not total orderings.
    \end{example}

    \begin{definition}
        Let $(X,\leq)$ be an ordered set. Let $A \subseteq X$.
        \begin{enumerate}[label = (\arabic*)]
            \item $A$ is called \textui{bounded above} if there exists an element $u \in X$ with $a \leq u$ for all $a \in A$. Such a $u$ (not necessarily unique) is called an \textui{upperbound} for $A$.
            \item $A$ is called \textui{bounded below} if there exists an element $v \in X$ with $v \leq a$ for all $a \in A$. Such a $v$ (not necessarily unique) is called a \textui{lowerbound} for $A$.
            \item If $A$ admits an upperbound $u$ with $u \in A$, then $u$ is called \textui{the greatest element of $A$}.
            \item If $A$ admits a lowerbound $v$ with $v \in A$, then $v$ is called \textui{the least element of $A$}.
            \item Let $A$ be bounded above. The \textui{set of upperbounds of $A$} is defined as \newline $\cU_A = \{u \in X \mid \text{$u$ is an upperbound of $A$}\}$. If $l$ is the least element of $\cU_A$, we write $l = \sup{(A)}$ and call it \textui{the supremum of $A$}.
            \item Let $A$ be bounded below. The \textui{set of lowerbounds of $A$} is defined as \newline $\mathscr{L}_A = \{v \in X \mid \text{$v$ is a lowerbound of $A$}\}$. If $g$ is the greatest element of $\mathscr{L}_A$, we write $g = \inf{(A)}$ and call it \textui{the infimum of $A$}.
            \item A \textui{maximal element of $A$} is an element $m \in A$ such that if $a \geq m$, then $a = m$ (not necessarily unique).
            \item A \textui{minimal element of $A$} is an element $n \in A$ such that if $a \leq n$, then $a = n$ (not necessarily unique).
            \item If $(A,\leq)$ is a total ordering, then $A$ is called a \textui{chain}.
        \end{enumerate}
    \end{definition}

    \begin{proposition}
        Let $(X , \leq)$ be an ordered set and $A \subseteq X$.
        \begin{enumerate}[label = (\arabic*)]
            \item If $A$ admits a greatest element, then it is unique, %i made a small change 
            \item If $A$ admits a least element, then it is unique,
            \item If $A$ admits a least upper bound, then it is unique,
            \item If $A$ admits a greatest lower bound, then it is unique.
        \end{enumerate}
    \end{proposition}
        \begin{proof}
            Suppose $u,u'$ are greatest elements of $A$, then $u,u' \in A$. Hence $u \leq u'$ and $u' \leq u$. By antisymmetry, $u = u'$, meaning the greatest element is unique. The proof for least elements being unique is identical, which establishes (1) and (2). 

            Note that $\cU_A  \subseteq X$. By definition the least element of $\cU_A$ is defined to be the supremum of $A$, and since least elements are unique the supremum of $A$ must be unique.  Similarly, $\mathscr{L}_A \subseteq X$. By definition the greatest element of $\mathscr{L}_A$ is defined to be the infimum of $A$, and since greatest elements are unique the infimum of $A$ must be unique. This establishes (3) and (4).
        \end{proof}

    \begin{lemma}[Zorn's Lemma]\label{lemma:zorns}
        Let $X$ be an ordered set with the property that every chain has an upperbound. Then $X$ contains a maximal element.
    \end{lemma}

    \begin{example}
        Considered the ordered set $(\bfN,\leq_d)$ and the subset $A = \{4,7,12,28,35\}$.
            \begin{itemize}
                \item $A$ is bounded above with $4\times7\times12\times28\times35$ as an upperbound.
                \item The supremum of $A$ is $\lcm{(4,7,12,28,35)}$. 
                \item There does not exist a greatest element.
                \item $12,28$, and $35$ are maximal elements (no other element in $A$ divides them).
            \end{itemize}
    \end{example}

    \begin{definition}
        Let $(X,\leq)$ be an ordered set and $A \subseteq X$. If $A$ is bounded above and below, then we say $A$ is \textui{bounded}. 
    \end{definition}

    \begin{definition}
        Let $(X,\leq)$ be an ordered set. Then $(X,\leq)$ is \textui{complete} if, for every bounded set $A \subseteq X$, $\sup{(A)}$ and $\inf{(A)}$ exist.
    \end{definition}

\section{Functions}\label{sec:functions}
    \begin{definition}
        Let $X$ and $Y$ be sets. A \textui{function} from $X$ to $Y$ is a relation $f \subseteq X \times Y$ such that for all $x \in X$, there exists a unique $y_x \in Y$ with $(x,y_x) \in f$.
            \begin{enumerate}[label = (\arabic*)]
                \item The set $X$ is the \textui{domain} of $f$.
                \item The set $Y$ is the \textui{codomain} of $f$.
                \item The \textui{image} of $f$ is defined as $f(X) = \{f(x) \mid x \in X \} \subseteq Y$ (also sometimes denoted $\Image{(f)}$).
                \item The \textui{preimage} of $f$ is defined as $f^{-1}(Y) = \{x \in X \mid f(x) \in Y\} \subseteq X$.
                \item The \textui{graph} of $f$ is defined as $\Graph{(f)} = \{(x,f(x)) \mid x \in X \} \subseteq X \times Y$.
            \end{enumerate}
        If $f$ is a function, we denote it by $f:X \rightarrow Y$ or $X \xrightarrow{f} Y$.
    \end{definition}

    \begin{example}\label{example:examples-of-functions}
        Let $X$ be a set.
        \begin{enumerate}[label = (\arabic*)]
            \item The \textit{identity map} $\id_X:X \rightarrow X$ is defined by $\id_X(x) = x$.
            \item If $X \subseteq Y$, the \textit{inclusion map} $\iota: X \rightarrow Y$ is defined by $\iota(x) = x$.
            \item If $A \subseteq X$ is a set, the \textit{characteristic function} (or \textit{step function}) $\mathbf{1}_A : X \rightarrow \bfR$  is defined by
                \begin{equation*}
                    \mathbf{1}_A(x) = 
                \begin{cases}
                    1, & x \in A \\
                    0, & x \not\in A.
                \end{cases}
                \end{equation*}
        \end{enumerate}
    \end{example}

    \begin{definition}
        Given $f,g : X \rightarrow \bfR$ and $\alpha \in \bfR$, the \textui{pointwise operations} on $f$ and $g$ are:
            \begin{itemize}
                \item $(f\pm g)(x) = f(x) \pm g(x)$,
                \item $(\alpha f)(x) = \alpha f(x)$,
                \item $(fg)(x) = f(x)g(x)$,
                \item $(f/g)(x) = f(x)/g(x)$.
            \end{itemize}
    \end{definition}

    \begin{definition}
        Let $f:X \rightarrow Y$ and $g: Y \rightarrow Z$ be maps between sets. The \textui{composition} of $f$ and $g$ is denoted $g \circ f :X \rightarrow Z$.
            \begin{center}
                \begin{tikzcd}
                    X \arrow[r, "f"] \arrow[rr, "g \circ f"', bend right] & Y \arrow[r, "g"] & Z
                    \end{tikzcd}
            \end{center}
    \end{definition}

    \begin{definition}
        Let $f:X \rightarrow Y$ be a map between sets.
            \begin{enumerate}[label = (\arabic*)]
                \item $f$ is \textui{left-invertible} if there exists a map $g:Y \rightarrow X$ with $g \circ f = \id_X$.
                \item $f$ is \textui{right-invertible} if there exists a map $h:Y \rightarrow X$ with $f \circ h = \id_Y$.
                \item $f$ is \textui{invertible} if there exists a map $k:Y \rightarrow X$ with $k\circ f = \id_X$ and $f \circ k = \id_Y$.
            \end{enumerate}
    \end{definition}
    
    \begin{example}
        The \textit{shift function} is a map $s: \bfN \rightarrow \bfN$ defined by $s(n) = n+1$. Note that this function is left-invertible: define $g: \bfN \rightarrow \bfN$  by
            \begin{equation*}
                g(n) = 
            \begin{cases}
                n-1, & n\geq 2 \\
                n_0, & n = 1,
            \end{cases}
            \end{equation*}
        where $n_0$ is an arbitrary natural number, then $g \circ s = \id_\bfN$. 

        Suppose that $s$ has a right inverse, that is, there exists a function $h:\bfN \rightarrow \bfN$ such that $s \circ h = \id_\bfN$. Observe that:
            \begin{equation*}
            \begin{split}
                (s\circ h)(1) = s(h(1)) = h(1) + 1 = 1.
            \end{split}
            \end{equation*}
        It must be the case that $h(1) = 0$, which is a contradiction. Hence $s$ is not right-invertible.
    \end{example}

    \begin{example}
        The function $g$ defined above is right invertible, but not left invertible. 
    \end{example}

    \begin{proposition}
        Let $f:X \rightarrow Y$ be a map between sets. The following are equivalent:
            \begin{enumerate}[label = (\arabic*)]
                \item $f$ is invertible,
                \item $f$ is right-invertible and left-invertible.
            \end{enumerate}
    \end{proposition}
        \begin{proof}
            Clearly (1) implies (2). Assume $f$ to be left and right-invertible. Then there exists maps $h,g:Y \rightarrow X$ with  $g \circ f = \id_X$ and $f \circ h = \id_Y$. Observe that:
                \begin{equation*}
                \begin{split}
                    h &= \id_X \circ h\\
                     &= (g \circ f) \circ h \\
                      &= g \circ (f \circ h) \\
                       &= g \circ \id_Y \\
                       &= g,
                \end{split}
                \end{equation*}
            establishing the proposition.
        \end{proof}
    
    \begin{definition}
        Let $f:X \rightarrow Y$ be a map between sets.
            \begin{enumerate}[label = (\arabic*)]
                \item $f$ is \textui{injective} if $f(x_1) = f(x_2)$ implies $x_1 = x_2$,
                \item $f$ is \textui{surjective} if $\Image{(f)} = Y$, and 
                \item $f$ is \textui{bijective} if it is injective and surjective.
            \end{enumerate}
    \end{definition}

    \begin{proposition}
        Let $f:X \rightarrow Y$ be a map between sets.
            \begin{enumerate}
                \item $f$ is injective if and only if $f$ is left-invertible.
                \item $f$ is surjective if and only if $f$ is right-invertible.
                \item $f$ is bijective if and only if $f$ is invertible. 
            \end{enumerate}
    \end{proposition}
        \begin{proof}
            (1) {\color{red} Do the forward direction yourself!} Now assume $f: X \rightarrow Y$ is injective. Define $g:Y \rightarrow X$ by
                \begin{equation*}
                    g(y) = 
                \begin{cases}
                    x_0, & y \not\in \Image{(f)} \\
                    x_y, & y \in \Image{(f)},
                \end{cases}
                \end{equation*}
            where $x_y$ is the unique element in $x$ mapping to $y$; i.e., $f(x_y) = y$. By our construction, $(g \circ f)(x) = x$ for all $x \in X$. 

            (2) {\color{red} Do the forward direction yourself!} Now assume $f:X \rightarrow Y$ is onto. Note that the preimage of $f$ is nonempty, so we can define $h:Y \rightarrow X$ by $h(y) = x_y$, where $x_y \in f^{-1}(X)$. By our construction $(f \circ h)(y) = f(x_y) = y$ for all $y \in Y$.

            (3) {\color{red} Do this yourself its easy!}
        \end{proof}

        \begin{corollary}
            Let $A,B$ be sets. There exists an injection $A \hookrightarrow B$ if and only if there exists a surjection $B \twoheadrightarrow A$.
        \end{corollary}
            \begin{proof}
                If $f:A \rightarrow B$ is injective, then $f$ is left invertible, that is, there exists a function $g:B \rightarrow A$ with $g \circ f = \id_A$. But this means $g$ is right invertible, so $g$ is onto. The other direction follows identically.
            \end{proof}
    
    \section{Cardinality}
        \begin{definition}
            Let $A,B$ be sets. Then $\card(A) = \card(B)$ if there exists a bijection $A \hooktwoheadrightarrow B$.
        \end{definition}

        \begin{example}\label{example:examples-of-cardiality}
            \phantom{a}
            \begin{enumerate}[label = (\arabic*)]
                \item Define $f:\bfN_0 \rightarrow \bfN$ by $f(n)= n+1$. This is a bijection, hence $\card(\bfN_0) = \card(\bfN)$.
                \item Let $[a,b]$ and $[c,d]$ be intervals with $a<b$ and $c<d$. Define $f:[a,b] \rightarrow [c,d]$ by $f(x) = (\frac{d-c}{b-a})(x-a)+c$. 
                    \begin{center}
                        \begin{tikzpicture}
                            \begin{axis}[
                                axis lines = middle,
                                xmin=-1, xmax=10, % x-axis range
                                ymin=-1, ymax=10, % y-axis range
                                xtick={3, 7}, % positions of the ticks
                                xticklabels={a, b}, % labels for the x-axis ticks
                                ytick={4, 6}, % positions of the ticks
                                yticklabels={c, d}, % labels for the y-axis ticks
                            ]
                            % Plot the line from (a, c) to (b, d)
                            \addplot [
                                color=blue,
                                mark=*, % open circle marker
                                mark options={fill=blue}, % set marker style
                            ]
                            coordinates {(3, 4) (7, 6)};
                            \addlegendentry{$f(x)$}
                            
                            \end{axis}
                        \end{tikzpicture}
                    \end{center}
                This is a bijection, hence $\card([a,b]) = \card([c,d])$. The result is the same had the intervals been open.
                \item Recall that $\tan:(-\frac{\pi}{2},\frac{\pi}{2}) \rightarrow \bfR$ is a bijection. Consider the maps $(0,1) \xhooktwoheadrightarrow{g} (-\frac{\pi}{2},\frac{\pi}{2}) \xhooktwoheadrightarrow{\tan} \bfR$. Since $g$ and $\tan$ are bijective, $\tan \circ g$ is bijective, hence $\card((0,1)) = \card(\bfR)$.
            \end{enumerate}
        \end{example}

        \begin{definition}
            A set $A$ is called \textui{finite} if there exists an $N \in \bfN$ such that $\card(A) = \card(\{1,...,N\})$. If not, then $A$ is called \textui{infinite}.
        \end{definition}

        \begin{proposition}
            Given $m,n \in \bfN$, $m \neq n$, then $\card(\{1,...,m\}) \neq \card(\{1,...,n\})$.
        \end{proposition}
            \begin{proof}
                Without loss of generality, let $m > n$. By way of contradiction, if there exists a bijection $f: \{1,...,m\} \rightarrow \{1,...,n\}$, then there exists $i,j \in \{1,...,m\}$ with $i \neq j$ and $f(i) = f(j)$. This is a contradiction ($f$ is not injective).
            \end{proof}

        \begin{proposition}
            $\bfN$ is infinite.
        \end{proposition}
            \begin{proof}
                Suppose towards contradiction there exists an $N \in \bfN$ and a bijection $f:\bfN \rightarrow \{1,...,N\}$. Note that the inclusion map $\iota:\{1,...,N,N+1\} \rightarrow \bfN$ is injective. Now consider the maps $\{1,...,N,N+1\} \xrightarrow{\iota} \bfN \xrightarrow{f} \{1,...,N\}$. Then $f \circ \iota : \{1,....,N,N+1\} \rightarrow \{1,...,N\}$. But by the previous example this cannot be true, thus $\bfN$ is infinite.
            \end{proof}

        \begin{exercise}\label{exercise:infinite-implies-injection}
            If $A$ is infinite, there exists an injection $\bfN \hookrightarrow A$.
        \end{exercise}
            \begin{proof}
                Let $\pi:\bfN \rightarrow A$ be a map. Let $a_1 \in A$. Define $\pi(1) = a_1$. Since $A$ is infinite, $A - \{a_1\}$ is also infinite. Pick $a_2 \in A$ and let $\pi(2) = a_2$. Inductively, we have that an injection $\bfN \hookrightarrow A$.
            \end{proof}

        \begin{example}\label{example:more-cardinality-examples}
            \phantom{a}
            \begin{enumerate}[label = (\arabic*)]
                \item Define $k:\bfZ \rightarrow \bfN$ by $k(n) = (-1)^{n-1} \left\lfloor \frac{n}{2} \right \rfloor$. This is a bijection, hence $\card(\bfZ) = \card(\bfN)$.
                \item Let $X$ be any set. Recall that the \textit{power set of $X$} is defined as $\mathcal{P}(X) = \{A \mid A \subseteq X\}$. Define $2^x = \{f \mid f:X \rightarrow \{0,1\}\}$. Let $A \subseteq X$. Define $\varphi:\mathcal{P}(X) \rightarrow 2^X$ by $\varphi(A) = \mathbf{1}_A$, where $\mathbf{1}_A$ is the \textit{characteristc function} defined in Example~\ref{example:examples-of-functions}. Note that $\varphi(A) = \varphi(B)$ if and only if $\mathbf{1}_A = \mathbf{1}_B$. Recall that functions are equal if and only if $\mathbf{1}_A (x) = \mathbf{1}_B (x)$ for all $x \in X$. $x \in A$ if and only if $\mathbf{1}_A (x) = 1$ if and only if $\mathbf{1}_B(x) = 1$, giving $x \in B$. Thus $A = B$ which means $\varphi$ is injective. Now let $f \in 2^X$. Let $A = \{x \in X \mid f(x) =1\}$. Then $\mathbf{1}_A = f$. Thus $\varphi$ is bijective and so $\card(\mathcal{P}(X)) = \card(2^X)$.
            \end{enumerate}
        \end{example}

    \begin{exercise}\label{exercise:power-set-2n}
        Show that $\card(\mathcal{P}(\{1,...,N\})) = 2^N$.
    \end{exercise}
        \begin{proof}
            {\color{red} do this}
        \end{proof}

    \begin{theorem}[Cantor's Diagonal Argument]\label{thm:cantors-diagonal} $\card(\bfN) < \card((0,1))$.
    \end{theorem}
        \begin{proof}
            Recall that every $\sigma \in (0,1)$ has a decimal expansion $\sigma = 0.\sigma_1 \sigma_2 ... = \sum_{k = 1}^\infty \frac{\sigma_k}{10^k}$, where $\sigma_j \in \{0,1,2,...,9\}$ which does not terminate in $9$'s. By way of contradiction, suppose there exists a surjection $r:\bfN \rightarrow (0,1)$ defined by $r(n) = 0.\sigma_1(n) \sigma_2(n) \sigma_3(n)...$, where $\sigma_j(n) \in \{0,1,2,...,9\}$ is the $j^\text{th}$ digit in the decimal expansion.

            Consider the map $\tau: \bfN \rightarrow \{0,1,...,9\}$ defined by:
                \begin{equation*}
                    \tau(n)=
                \begin{cases}
                    3,&\sigma_n(n) = 2\\
                    2,&\sigma_n(n)=3,
                \end{cases}
                \end{equation*}
            
            and let $t = 0.\tau(1)\tau(2)\tau(3)...$\hspace{5pt}Observe the following:
                \begin{equation*}
                \begin{split}
                    r(1) &= 0.\sigma_1(1) \sigma_2 (1) \sigma_3 (1) \sigma_4(1)... \\
                    r(2) &= 0.\sigma_1(2) \sigma_2 (2) \sigma_3 (2) \sigma_4 (2)... \\
                    r(3) &= 0.\sigma_1(3) \sigma_2 (3) \sigma_3 (3) \sigma_4 (3)... \\
                    r(4) &= 0.\sigma_1(4) \sigma_2 (4) \sigma_3 (4) \sigma_4 (4)... \\
                    &\vdots \\
                    r(n) &= 0.\sigma_1(n) \sigma_2 (n) \sigma_3 (n) \sigma_4 (n)\hspace{4pt}... \hspace{4pt}\sigma_n(n).
                \end{split}
                \end{equation*}
            Since $r$ is surjective, there is an $m \in \bfN$ with $r(m) = t$. It follows that:
                \begin{equation*}
                \begin{split}
                    r(m) &= 0.\sigma_1(m)\sigma_2(m)\sigma_3(m)...\sigma_m(m)... \\
                    & = 0.\tau(1)\tau(2)\tau(3)...\tau(m)...
                \end{split}
                \end{equation*}
            which implies that $\sigma_m(m) = \tau(m)$. But recall how we defined $\tau(n)$ \textemdash if $\sigma_m(m) = 2$, then $\tau(2) = 3$ and if $\sigma_m(m) \neq 2$, then $\tau(2) = 2$. This is a contradiction, hence there does not exist a surjection $\bfN \xrightarrow{r} (0,1)$.
        \end{proof}
    
    \begin{corollary}
        $\card(\bfN) \neq \card(\bfR)$
    \end{corollary}
        \begin{proof}
            It follows from Example~\ref{example:examples-of-cardiality} that $\card(\bfN) < \card((0,1)) = \card(\bfR)$.
        \end{proof}

    \begin{definition}
        Let $A$ and $B$ be sets.
        \begin{enumerate}[label = (\arabic*)]
            \item We write $\card(A) \leq \card(B)$ if there exists an injection $A \hookrightarrow B$.
            \item We write $\card(A) < \card(B)$ if $\card(A) \leq \card(B)$ and $\card(A) \neq \card(B)$
        \end{enumerate}
    \end{definition}

    \begin{example}
        \phantom{a}
        \begin{enumerate}[label = (\arabic*)]
            \item If $A \subseteq B$, then the inclusion map $\iota: A \rightarrow B$ gives $\card(A) \leq \card(B)$.
            \item If $m > n$, then $\card\{1,...,n\} < \card\{1,...,m\}$ 
        \end{enumerate}
    \end{example}

    \begin{proposition}\label{prop:power-set-bigger}
        Let $A$ be a set. Then $\card(A) < \card(\cP(A))$.
    \end{proposition}
        \begin{proof}
            Define $f:A \rightarrow \cP(A)$ by $a \mapsto \{a\}$. This is clearly an injective map. Now suppose towards contradiction that there exists a surjection $g:A \rightarrow \cP(A)$ defined by $a \mapsto g(a)$. Then $g(a) \subseteq A$ (by the definition of a power set).

            Let $S = \{a \in A \mid a \not\in g(a) \}$. Then $S \subseteq A$. Since $g$ is onto, there exists an element $x \in A$ with $g(x) = S$. Case 1: $x \in S$. This implies that $x \not\in g(x)$. But $g(x) = S$, so $x \not\in S$, a contradiction. Case 2: $x \not\in S$. This implies that $x \not\in g(x)$. But by definition this means $x \in S$, a contradiction. Since we have exhausted all the necessary cases, it must be that there does not exist a surjection from $A \rightarrow \cP(A)$. Hence $\card(A) < \card(\cP(A))$.
        \end{proof}
    
    \begin{lemma}
        Let $A$ and $B$ be sets. The following are equivalent:
            \begin{enumerate}[label = (\arabic*)]
                \item $\card(A) \leq \card(B)$;
                \item there exists an injection $A \hookrightarrow B$;
                \item there exists a surjection $B \twoheadrightarrow A$.
            \end{enumerate}
    \end{lemma}

    \begin{example}\label{example:n-z-q}
        \phantom{a}
        \begin{enumerate}[label = (\arabic*)]
            \item Define $\bfN \times \bfZ \rightarrow \bfQ$ by $(n,m) \mapsto \frac{m}{n}$. This is surjective, so $\card(\bfQ) \leq \card(\bfN \times \bfZ)$.
            \item Define $\bfN \times \bfN \rightarrow \bfN$ by $(m,n) \mapsto 2^m \cdot 3^n$. Then $g$ is injective by the fundamental theorem of arithmetic. So $\card(\bfN \times \bfN) \leq \card(\bfN)$.
            \item Recall from Example~\ref{example:more-cardinality-examples} that $k: \bfN \rightarrow \bfZ$ defined by $k(n) = (-1)^{n-1} \left\lfloor \frac{n}{2} \right \rfloor$ is a bijection. Define $K: \bfZ \times \bfN \rightarrow \bfN \times \bfN$ by $(m,n) \mapsto (k^{-1}(m),n)$. This is a bijection, so $\card(\bfZ \times \bfN) = \card(\bfN \times \bfN)$.
            \item From the previous examples, we've established that:
                \begin{equation*}
                \begin{split}
                    \card(\bfN) \leq \card(\bfQ) \leq \card(\bfZ \times \bfN) = \card(\bfN \times \bfN) \leq \card(\bfN)
                \end{split}
                \end{equation*}
        \end{enumerate}
    \end{example}

    \begin{theorem}\label{thm:cardinal-orderings}
        Let $\fN$ denote the class of cardinals. The pair $(\mathfrak{N},\leq)$ forms a total ordering \textemdash where $\leq$ is defined by $\card(A) \leq \card(B)$ if and only if $A \hookrightarrow B$. In particular, if $A,B,C$ are sets with $\card(A),\card(B),\card(C) \in \obj(\fN)$, then we have the following:
        \begin{enumerate}[label = (\arabic*)]
            \item $\card(A) \leq \card(A)$ (reflexive).
            \item If $\card(A) \leq \card(B) \leq \card(C)$, then $\card(A) \leq \card(C)$ (transitive).
            \item If $\card(A) \leq \card(B)$ and $\card(B) \leq \card(A)$, then $\card(A) = \card(B)$ (antisymmetric).
            \item  Either $\card(A) \leq \card(B)$ or $\card(B) \leq \card(A)$ (total).
        \end{enumerate}
    \end{theorem}
        \begin{proof}
            (1) and (2) follow by simply applying definitions. Note that any set bijects into itself, hence $A \hooktwoheadrightarrow A$ implies $A \hookrightarrow A$, establishing $\card(A) \leq \card(A)$. Similarly, if there are bijections $A\hooktwoheadrightarrow B \hooktwoheadrightarrow C$, then clearly there is a bijection $A \hooktwoheadrightarrow C$. Hence $\card(A) = \card(C)$.

            (3) (Cantor-Shr\"{o}der-Bernstein Theorem) We have injections $A \xhookrightarrow{f}$ and $B \xhookrightarrow{g} A$. Let:
                \begin{equation*}
                \begin{split}
                    A_0 &= \Image{(g)}^\complement \\
                    A_1 &= (g \circ f)(A_0) \\
                    A_2 & = (g \circ f)(A_1) \\
                    &\vdots \\
                    A_n & = (g \circ f)(A_{n-1}).
                \end{split}
                \end{equation*}
            Note that $A_1 \cap A_0 = \emptyset$ because $A_1 \subseteq \Image{(g)}$ and $A_0 = \Image{(g)}^\complement$. We similarly have that $A_2 \cap A_0 = \emptyset$. Claim: $A_1 \cap A_2$. {\color{red} finish this}
            
            (4) Let $A \rightarrow B$ be a map. Let $\cF = \{(D,f) \mid D \subseteq A, f:D \hookrightarrow B, \hspace{4pt}\text{$f$ is injective}\}$. Note that $\cF \neq \emptyset$ because $(\emptyset, k) \in \cF$ for some map $k$. Define an ordering on $\cF$ as follows: $(D,f) \leq_\cF (E,g)$ if and only if $D \subseteq E$ and $\restr{g}{D} = f$. Then $\cF$ admits an upperbound of $A$. By \nameref{lemma:zorns}, there exists a maximal element $(M,h) \in \cF$. Suppose towards contradiction there are elements $a \in A$, $a \not\in M$ and $b \in B$, $b \not\in h(M)$. Consider the map:
                \begin{equation*}
                    h': M \cup \{a\} \rightarrow B \mtext{defined by} \begin{cases}
                        h'(M) = h(M) \\
                        h'(a) = b
                    \end{cases}.
                \end{equation*}
            This set is clearly injective, and furthermore we have that $(M,h) \leq (M \cup \{a\},h')$. This is a contradiction, hence $M = A$ or $h(M) = B$. If $M=A$, then the injection $A \xhookrightarrow{h} B$ implies $\card(A) \leq \card(B)$. If $h(M) = B$, then the map $B \hookrightarrow M \xhookrightarrow{\iota} A$ implies $\card(B) \leq \card(A)$. 
        \end{proof}

    \begin{corollary}
        $\card(\bfQ) = \card(\bfN)$.
    \end{corollary}
        \begin{proof}
            This follows directly from Example~\ref{example:n-z-q} and Theorem~\ref{thm:cardinal-orderings}
        \end{proof}

    \begin{definition}
        A set $A$ is \textui{countable} if $\card(A) \leq \card(\bfN)$. Equivalently, there exists an injection $A \hookrightarrow \bfN$ and a surjection $\bfN \twoheadrightarrow A$. If $A$ is countable and infinite, $A$ is called \textui{denumerable} (or more commonly referred to as \textui{countably infinity}).
    \end{definition}

    \begin{definition}
        We say $\card(\bfN) = \card(\bfZ) = \card(\bfQ) := \aleph_0$, called \textui{aleph naught}. We also define $\card(\bfR) = \mathfrak{c}$, called the \textui{continuum}.
    \end{definition}

    \begin{example}
        By Theorem~\ref{thm:cantors-diagonal}, $\aleph_0 < \mathfrak{c}$.
    \end{example}

    \begin{corollary}
        There does not exist an infinite set $A$ with $\card(A) < \aleph_0$. In particular, if $A$ is infinite and countable, then $\card(A) = \aleph_0$.
    \end{corollary}
        \begin{proof}
            By Exercise~\ref{exercise:infinite-implies-injection}, $\card(\bfN) \leq \card(A)$, and by definition (since $A$ is countable), $\card(A) \leq \card(\bfN)$. So by Theorem~\ref{thm:cardinal-orderings}, $\card(A) = \card(\bfN) = \aleph_0$.
        \end{proof}

    \begin{example}
        $\card(\cP(\bfN)) > \card(\bfN) = \aleph_0$.
    \end{example}

    \begin{proposition}
        The countable union of countable sets is countable. More precisely, if $A_i$ is countable for all $i \in \bfN$, then $\bigcup_{i = 1}^\infty A_i$ is countable.
    \end{proposition}
        \begin{proof}
            By definition, there exist surjections $\pi_i : \bfN \rightarrow A_i$. Define $\pi: \bfN \times \bfN \rightarrow \bigcup_{i = 1}^\infty A_i$ by $\pi(i,j) = \pi_i (j)$. Claim: $\pi$ is onto. Let $x \in \bigcup_{i = 1}^\infty A_i$, then there exists an $i_0$ with $x \in A_{i_0}$. Since $\pi_{i_0}$ is onto, there exists a $j_0 \in \bfN$ with $\pi_{i_0}(j_0) = x$. So $\pi(i_0,j_0) = x$, establishing that $\pi$ is surjective as well. Therefore $\card(\bigcup_{i = 1}^\infty A_i) \leq \card(\bfN \times \bfN) = \card(\bfN)$.
        \end{proof}
    
    \begin{lemma}\label{lemma:1}
        $\card([0,1]) \leq \card(2^\bfN)$.
    \end{lemma}
        \begin{proof}
            Recall that every $\sigma \in [0,1]$ has a binary expansion $\sigma = \sum_{k = 1}^\infty \frac{\sigma_k}{2^k}$, where $\sigma_k \in \{0,1\}$. Consider the map $\varphi: 2^\bfN \rightarrow [0,1]$ defined by $\varphi(f) = \sum_{k = 1}^\infty \frac{f(k)}{2^k}$. Letting $f(k) = \sigma_k$ gives $\varphi$ is surjective.
        \end{proof}
    
    \begin{lemma}\label{lemma:2}
        $\card(\bfR) = \card([0,1])$.
    \end{lemma}
        \begin{proof}
            By inclusion $[0,1] \xhookrightarrow{\iota} \bfR$, which implies that $\card([0,1]) \leq \card(\bfR)$. Recall that \newline$\bfR \xhooktwoheadrightarrow{\tan} (0,1) \xhookrightarrow{\iota} [0,1]$, which implies that $\card(\bfR) \leq \card([0,1])$. Then Theorem~\ref{thm:cardinal-orderings} gives the desired result.
        \end{proof}

    \begin{lemma}\label{lemma:3}
        $\card(2^\bfN) \leq \card([0,1])$.
    \end{lemma}
        \begin{proof}
            Consider the map $\lambda:2^\bfN \rightarrow [0,1]$ defined by $\lambda(f) = \sum_{k = 1}^\infty  \frac{f(k)}{3^k}$. Claim: $\lambda$ is injective. Let $f,g \in 2^\bfN$ with $f\neq g$. Let $k_0$ be the \textit{smallest point $k$ where $f$ and $g$ are different}. So in particular:
                \begin{equation*}
                \begin{split}
                    f(1) &= g(1) \\
                    f(2) & = g(2) \\
                    &\vdots \\
                    f(k_0 - 1) &= g(k_0 - 1) \\
                    f(k_0) &\neq g(k_0).
                \end{split}
                \end{equation*}
            Let:
                \begin{equation*}
                \begin{split}
                    t_1 &= \sum_{k > k_0} \frac{f(k)}{3^k} \quad {\text{\tiny sum past $k_0$}}\\
                    t_2 &= \sum_{k > k_0} \frac{g(k)}{3^k} \quad {\text{\tiny sum past $k_0$}}\\
                    s_1 &= \sum_{k = 1}^{k_0 - 1} \frac{f(k)}{3^k} \quad {\text{\tiny sum before $k_0$}}\\
                    s_1 &= \sum_{k = 1}^{k_0 - 1} \frac{g(k)}{3^k} \quad {\text{\tiny sum before $k_0$}}
                \end{split}
                \end{equation*}
            We have that:
                \begin{equation*}
                \begin{split}
                    \lambda(f) &= s_1 + \frac{f(k_0)}{3^{k_0}} + t_1 \\
                    \lambda(g) &= s_2 + \frac{g(k_0)}{3^{k_0}} + t_2
                \end{split}
                \end{equation*}
            Because $f$ and $g$ differ at $k_0$, without loss of generality let $f(k_0) = 0$ and $g(k_0) = 1$. Then $\lambda(g) - \lambda(f) = \frac{1}{3^{k_0}} + t_2 - t_1$. Observe that:
                \begin{equation*}
                \begin{split}
                    |t_2 - t_2|
                    & = \left| \sum_{k > k_0} \frac{g(k)-f(k)}{3^k}\right| \\
                    & \leq \sum_{k > k_0} \frac{|g(k)-f(k)|}{3^k} \quad \quad \text{\tiny By triangle inequality}\\
                    & \leq \sum_{k > k_0} \frac{1}{3^k} \quad \quad \quad \quad \quad \quad \hspace{5.3pt}\text{\tiny By comparison test}\\
                    & = \frac{1}{3^{k_0 + 1}}\sum_{k \geq 0}\frac{1}{3^k} \\
                    & = \frac{1}{3^{k_0 + 1}} \cdot \frac{1}{1-\frac{1}{3}} \\
                    & = \frac{3}{2 \cdot 3^{k_0 + 1}} \\
                    & = \frac{1}{2\cdot 3^{k_0}} \\
                    & < \frac{1}{3^{k_0}}.
                \end{split}
                \end{equation*}
            Since $|t_2 - t_2| < \frac{1}{3^{k_0}}$, $\lambda(g) - \lambda(f) \neq 0$, establishing $\lambda$ as an injection. Thus $\card(2^\bfN) \leq \card([0,1])$.
        \end{proof}
    
    \begin{theorem}
        $\card(2^\bfN) = \card(\cP(\bfN)) = \card(\bfR)$.
    \end{theorem}
        \begin{proof}
            This follows from Lemma~\ref{lemma:1}, Lemma~\ref{lemma:2}, and Lemma~\ref{lemma:3}.
        \end{proof}
    
\chapter{Ordered Fields}\label{chapter:ordered Fields}
\vspace{12pt}

\section{Ordering of $\mathbb{Z}$}
    \begin{definition}
        Define $\bfZ^+  = \{n \in \bfZ \mid n \geq_a 0\}$, where $\geq_a$ is the \textit{algebraic ordering} from Example~\ref{example:orderings}. We call $\bfZ^+$ the \textui{cone of positive integers}, and they admit the following axioms:
        \begin{enumerate}[label = (\arabic*)]
            \item If $m,n \in \bfZ^+$, then $m+n \in \bfZ^+$ and $mn \in \bfZ^+$.
            \item For all $m \in \bfZ$, $m \in \bfZ^+$ or $-m \in \bfZ^+$.
            \item If $m \in \bfZ^+$ and $-m \in \bfZ^+$, then $m=0$. 
        \end{enumerate}
    \end{definition}
    
    \begin{proposition}[Properties of $\leq_a$]
        \phantom{a}
        \begin{enumerate}[label = (\arabic*)]
            \item $m \leq_a n$ if and only if $n - m \in \bfZ^+$.
            \item If $m \leq_a n$ and $p \leq_a q$, then $m + p \leq_a n + q$.
            \item If $m \leq_a n$ and $p \in \bfZ^+$, then $pm \leq_a pn$.
            \item If $m \leq_a n$ then $-n \leq_a -m$.
            \item $(\bfZ,\leq_a)$ forms a total ordering.
            \item If $m >_a 0$ and $mn >_a 0$, then $n >_a 0$.
            \item If $m >_a 0$ and $mn \geq_a mp$, then $n \geq_a p$.
        \end{enumerate}
    \end{proposition}
        \begin{proof}
            (5) Let $m,n \in \bfZ$, since $\bfZ$ is closed under subtraction $m - n \in \bfZ$. So either $m-n \in \bfZ^+$ or $n - m \in \bfZ^+$. Then by (1) $n \leq_a m$ or $m \leq_a n$. Thus $(\bfZ,\leq_a)$ is a total ordering.

            (6) We have $mn >_a 0 $ with $m >_a 0$. If $n = 0$, we are done. So now assume $n \neq 0$. Then either $n \in \bfZ^+$ or $-n \in \bfZ^+$. If $-n \in \bfZ^+$, then $m(-n) = -(mn) \in \bfZ^+$. But we had assumed $mn >_a 0$; i.e., $mn \in \bfZ^+$, hence it must be the case that $mn = 0$, a contradiction. Therefore it must be that $n \in \bfZ^+$.
        \end{proof}

\section{Ordering of $\mathbb{Q}$}
    \begin{proposition}
        Define $Q := \bfZ \times \bfN$. Show that $\sim$ forms an equivalence relation, where $(a,b) \sim (c,d)$ if and only if $ad = bc$.
    \end{proposition}
        \begin{proof}
            {\color{red} I dont wanna do this} %i've made a small change!
        \end{proof}
    
    \begin{definition}
        The set of equivalence classes of $Q$ is $\bfQ = Q/\sim = \{ [(a,b)] \mid (a,b) \in Q\}$. We call this set the \textui{rational numbers}, and denote the equivalence classes $[(a,b)]$ as $\frac{a}{b}$.
    \end{definition}

    \begin{proposition}
        The operations 
            \begin{equation*}
            \begin{split}
                +:\bfQ \times \bfQ &\rightarrow \bfQ \hspace{4pt}\text{defined by}\hspace{4pt} [(a,b)] + [(c,d)] = [(ad+bc,bd)]\\
                \cdot:\bfQ \times \bfQ &\rightarrow \bfQ \hspace{4pt}\text{defined by}\hspace{4pt} [(a,b)]\cdot[(c,d)] = [(ac,bd)]\\
            \end{split}
            \end{equation*}
        are well-defined. Furthermore, $(\bfQ,+,\cdot)$ forms a field.
    \end{proposition}
        \begin{proof}
            {\color{red} I dont wana}
        \end{proof}
    
    \begin{lemma}\label{lemma:order-embedding-z-q}
        There is an injective map $\bfZ \xhookrightarrow{j} \bfQ$ defined by $j(n) = \frac{n}{1}$ satisfying the properties
            \begin{equation*}
            \begin{split}
                j(n+m) &= j(n) + j(m)\\
                j(nm) &= j(n)j(m).
            \end{split}
            \end{equation*}
    \end{lemma}
        \begin{proof}
            Note that $j(n) = j(m)$ if and only if $\frac{n}{1} + \frac{m}{1}$. By definition this is equivalent to $n = m$, hence $j$ is injective.
            
            Observe that $j(n+m) = \frac{n+m}{1} = \frac{n}{1} + \frac{m}{1} = j(n) + j(m)$ and $j(nm) = \frac{nm}{1} = \frac{n}{1}\cdot\frac{m}{1} = j(n)j(m)$.
        \end{proof}
    
    \begin{theorem}\label{thm:q-total-ordering}
        $(\bfQ,\leq_Q)$ is a total ordering, where $\leq_Q$ is a well-defined ordering defined by $\frac{a}{b} \leq_Q \frac{c}{d}$ if and only if $ad \leq_a bc$ in $(\bfZ,\leq_a)$. Furthermore, the map $j:\bfZ \hookrightarrow \bfQ$ is \textit{order preserving}, that is, if $n \leq_a m$ in $(\bfZ, \leq_a)$, then $j(n) \leq_Q j(m)$ in $(\bfQ,\leq_Q)$.
    \end{theorem}
        \begin{proof}
            {\color{red} i REALLY dont}
        \end{proof}
    
    \begin{definition}
        Define $\bfQ_+ := \{q \in \bfQ \mid q \geq_Q 0\}$ as the \textui{cone of positive rationals}, and they admit the following axioms:
            \begin{enumerate}[label = (\arabic*)]
                \item If $q_1,q_2 \in \bfQ^+$, then $q_1+q_2 \in \bfZ^+$ and $q_1 q_2 \in \bfZ^+$.
                \item For all $q \in \bfQ$, $q \in \bfQ^+$ or $-q \in \bfQ^+$.
                \item If $q \in \bfQ^+$ and $-q \in \bfQ^+$, then $q=0$.
                \item $q_1 \leq_Q q_2$ if and only if $q_2 - q_1 \in \bfQ^+$.
            \end{enumerate}
    \end{definition}

    \begin{proposition}
        Let $r,s,t,u \in \bfQ$
        \begin{enumerate}[label = (\arabic*)]
            \item If $r \leq_Q s$ and $t \leq_Q u$, then $r+ t \leq_Q s+ u$.
            \item If $r \leq_Q s$ and $t \geq_Q 0$, then $tr \leq_Q ts$.
        \end{enumerate}
    \end{proposition}
        \begin{proof}
            {\color{red} do this shi later}
        \end{proof}
    
\section{Rings and Fields}
    \begin{definition}
        A \textui{ring} is a non-empty set $R$ equipped with two binary operations: 
            \begin{equation*}
            \begin{split}
                R \times R &\xrightarrow{a} R \hspace{4pt}\text{defined by}\hspace{4pt} a(r,s) = r+s\\
                R \times R &\xrightarrow{m} R \hspace{4pt}\text{defined by}\hspace{4pt} m(r,s) = rs,
            \end{split}
            \end{equation*}
        such that they admit the following axioms:
            \begin{enumerate}[label = (\arabic*)]
                \item $R$ is an \textit{abelian group} under addition:
                    \begin{enumerate}[label = (\roman*)]
                        \item $r+(s+t) = (r+s) + t$ for all $r,s,t \in R$,
                        \item there exists an element $0_R \in R$ with $r + 0_R = r = 0_R = r$ for all $r \in R$,
                        \item For all $r \in R$ there exists an $s \in R$ such that $r+s = 0_R = s + r$ (such an $s$ is unique, and is denoted $-r$),
                        \item $r+s = s+r$ for all $r,s \in R$.
                    \end{enumerate}
                \item $r(st) = (rs)t$ for all $r,s,t \in R$,
                \item $(r+s)t = rt + rs$ and $r(s + t)= rs + rt$ for all $r,s,t \in R$.
            \end{enumerate}
        If $R$ contains an element $1_R$ such that $1_R r = r = r 1_R$, then we say $R$ is \textui{unital}. If $rs = sr$ for all $r,s \in R$, then we say $R$ is \textui{commutative}. If $R$ is a unital ring such that $1_R \neq 0_R$ \textit{and} for all $r \in R$ there exists an $s \in R$ such that $rs = 1_R = sr$ (such an $s$ is unique, and denoted $r^{-1}$), then we say $R$ is a \textui{division ring}.
    \end{definition}

    \begin{definition}
        A \textui{field} is a commutative division ring.
    \end{definition}

    \begin{example}
        \phantom{a}
        \begin{enumerate}[label = (\arabic*)]
            \item $\bfQ$ is a field.
            \item $\bfZ/p \bfZ$ is a field.
            \item $\bfC_\bfQ = \{r+si \mid r,s \in \bfQ , i^2 = -1\}$ with addition and multiplication defined by
                \begin{equation*}
                \begin{split}
                    (r+si) + (t+ui) := (r+t) + (s+u)i \\
                    (r+si)(t+ui) := (rt-su) + (ru+st)i
                \end{split}
                \end{equation*}
            is a field. We call this set the \textit{complex rationals}. 
        \end{enumerate}
    \end{example}

    \begin{definition}
        An \textui{ordered field} is a field $F$ equipped with a total ordering $\leq_F$ such that:
            \begin{enumerate}[label = (\arabic*)]
                \item If $x \leq_F y$ and $u \leq_F v$, then $x+u \leq_F y+v$.
                \item If $x \leq_F y$ and $z \geq_F 0$, then $xz \leq_F zy$.
            \end{enumerate}
        We similarly define $F^+ = \{x \in F \mid x \geq_F 0\}$ as the \textui{cone of positive elements}.
    \end{definition}

    \begin{proposition}
        Let $(F,\leq_F)$ be an ordered field.
        \begin{enumerate}[label = (\arabic*)]
            \item If $x,y \in F^+$, then $x+y \in F^+$ and $xy \in F^+$.
            \item If $x \in F$, then $-x \in F^+$ or $x \in F^+$.
            \item If $x,-x \in F^+$, then $x = 0$.
        \end{enumerate}
    \end{proposition}
        {\color{red} \begin{proof}
            need to do
        \end{proof}}

    \begin{example}
        \phantom{a}
        \begin{enumerate}[label = (\arabic*)]
            \item $\bfQ$ is an ordered field.
            \item Is $\bfC_\bfQ$ an ordered field?
        \end{enumerate}
    \end{example}

    \begin{proposition}
        Let $(F,\leq)$ be an ordered field with $1_F \neq 0_F$.
            \begin{enumerate}[label = (\arabic*)]
                \item For all $a \in F$, $a^2 \in F$.
                \item $0,1 \in F^+$.
                \item If $n \in \bfN$, then $n\cdot 1_F := \underbrace{1_F + 1_F + ... + 1_F}_{n\mtext{times}}$, implying $n \cdot 1_F \in F^+$.
                \item If $x \in F^+$ and $x \neq 0$, then $x^{-1} \in F^+$.
                \item If $xy \in F^+$ and $xy \neq 0$, then $x,y \in F^+$ or $-x,-y \in F^+$.
                \item If $0 < x \leq y$, then $y^{-1} \leq x^{-1}$.
                \item If $x \leq y$, then $-y \leq -x$.
                \item If $x \geq 1_F$, then $x^2 \geq x$.
                \item If $x \leq 1_F$, then $x^2 \leq x$.
            \end{enumerate}
    \end{proposition}
        \begin{proof}
            (1) If $a \in F^+$, then $a \cdot a  = a^2 \in F^+$. If $-a \in F^+$, then $(-a) \cdot (-a) = a^2 \in F^+$.

            (2) From part (1) we have that $0 = 0 \cdot 0 \in F^+$. Similarly, $1 = 1 \cdot 1 \in F^+$ and $(-1) \cdot (-1) \in F^+$

            (3) Since $F^+$ is closed under addition, we can inductively show that $n \cdot 1 = 1 + 1 + ... + 1 \in F^+$.

            (4) Suppose towards contradiction $x^{-1} \not\in F^+$. Then $-(x^{-1}) \in F^+$, so $(-(x^{-1}))\cdot x = -1(x^{-1}\cdot x) = -1 \in F^+$. But $-1,1 \in F^+$ implies $1 = 0$, a contradiction. Thus $x^{-1} \in F^+$.

            (6) $y \geq x > 0$ implies $x,y \in F^+$. So $x^{-1},y^{-1} \in F^+$. Then $y^{-1}xx^{-1} \leq y^{-1}yx^{-1}$, and simplifying yields $y^{-1} \leq x^{-1}$.
            {\color{red} finish the rest (i'm not going to)}
        \end{proof}

\chapter{The Real Numbers}\label{chapter:the real numbers}
\vspace{12pt}

\section{The Completion of $\bbQ$}
    \begin{definition}
        A \textui{Dedekind cut} is a nonempty subset $D$ of $\bfQ$ with the following properties:
            \begin{enumerate}[label = (\arabic*)]
                \item $D \neq \bfQ$;
                \item If $b \in D$, then $a \in D$ for all $a \in \bfQ$ with $a < b$;
                \item D does not contain a largest element.
            \end{enumerate}
    \end{definition}

    \begin{example}
        The following examples are Dedekind cuts:
        \begin{enumerate}[label = (\arabic*)]
            \item $\{a \in \bfQ \mid a < 3\}$ (the set of all rational numbers less than 3).
            \item $\{a \in \bfQ \mid a < 0 \hspace{4pt}\text{or}\hspace{4pt} a^2 < 2\}$ (the set of all rational numbers less than $\sqrt{2}$).
            \item $\{a \in \bfQ \mid a < 1 + \frac{1}{1!} + \frac{1}{2!} + ... + \frac{1}{n!} \hspace{4pt}\text{for some}\hspace{4pt} n \in \bfZ^+\} $ (the set of all rational numbers less than $e$).
        \end{enumerate}
    \end{example}

    \begin{definition}
        Let $C$ and $D$ be Dedekind cuts. 
    \end{definition}
    {\color{red} will probably not finish this}

\section{Ordering of $\bbR$}
    \begin{axiom}
        $\bfR$ is an ordered field.
    \end{axiom}

    \begin{proposition}
        $\bfQ^+ \subseteq \bfR^+$.
    \end{proposition}
        \begin{proof}
            If $x \in \bfQ^+$, then $x = \frac{p}{q}$ with $p \in \bfZ^+$ and $q \in \bfN$. Write $p = \underbrace{1 + 1 + ... + 1}_{\mtext{$p$ times}}$, then $p \in \bfR^+$. Similarly, write $q = \underbrace{1 + 1 + ... + 1}_{\mtext{$q$ times}}$. Then $q \in \bfR^+$, which implies that $q^{-1} \in \bfR^+$. Hence $\frac{p}{q} \in \bfR^+$, establishing $\bfQ^+ \subseteq \bfR^+$.
        \end{proof}

    \begin{proposition}
        The maps $Z \xhookrightarrow{j} \bfQ \xhookrightarrow{i} \bfR$ are order embeddings (defined in Lemma~\ref{lemma:order-embedding-z-q} and Theorem~\ref{thm:q-total-ordering}).
    \end{proposition}
        \begin{proof}
            Suppose $i(q_1) \leq_Q i(g_2)$. Then $q_1 \leq_\bfR q_2$, hence $q_2 - q_1 \in \bfR^+$. Now If $q_2 - q_2 \in \bfQ^+$, then $q_2 - q_1 \in \bfR^+$. Hence $q_1 \leq_\bfR q_2$. {\color{red} wtf is this saying?}
        \end{proof}

    \begin{proposition}\label{prop:average}
        Let $a,b \in \bfR$. If $a \leq b$ (or $a < b$), then $a \leq \frac{1}{2}(a+b) \leq b$ (or $a < \frac{1}{2}(a+b) < b$).
    \end{proposition}
        \begin{proof}
            By the order axioms, $a \leq b$ implies $a + a \leq a + b$. So $2a \leq a + b$, which is equivalent to $a \leq \frac{1}{2}(a+b)$. Similarly, $a + b \leq b + b$, which similarly gives $\frac{1}{2}(a+b) \leq b$, establishing the proposition.
        \end{proof}

    \begin{corollary}\label{cor:after-average}
        Given $b>0$, we have $0 < \frac{1}{2}b < b$.
    \end{corollary}
        \begin{proof}
            From Proposition~\ref{prop:average}, setting $a = 0$ yields the desired result.
        \end{proof}

    \begin{proposition}
        Suppose $a \in \bfR$. For all $\epsilon > 0$, if $0 \leq a \leq \epsilon$, then $a = 0$.
    \end{proposition}
        \begin{proof}
            If $a = 0$ we are done. If $a > 0$, by Corollary~\ref{cor:after-average} $0 \leq \frac{1}{2}a \leq a$. Pick $\epsilon = \frac{1}{2}a$, then $a \leq \frac{1}{2}a$, a contradiction. Thus $a = 0$.
        \end{proof}

    \begin{definition}
        Let $a_1,a_2,...,a_n > 0$. The \textui{arithmetic mean} is $\frac{1}{2} \left(\sum_{j=1}^n a_j\right)$. The \textui{geometric mean} is $\left(\prod_{j = 1}^n a_j\right)^{\frac{1}{n}}$.
    \end{definition}

    \begin{proposition}[AM-GM Inequality]\label{prop:am-gm}
        For all $a_1, a_2, ... ,a_n \geq 0$, then $\left(\prod_{j = 1}^n a_j\right)^{\frac{1}{n}} \leq \frac{1}{2} \left(\sum_{j=1}^n a_j\right)$.
    \end{proposition}
        \begin{proof}
            We will only prove the $n = 2$ case. Consider the fact that $(a_1  - a_2)^2 \geq 0$, and expanding gives $a_1^2 - 2a_1a_2 + a_2 ^2$. So $2a_1 a_2 \leq a_1^2 + a_2^2$. Adding $2a_1a_2$ to both sides yields $4a_1a_2 \leq a_1^2 + 2a_1a_2 + a_2^2$, which is equivalent to $4a_1a_2 \leq (a_1 + a_2)^2$. Then simplifying yields the desired result of $(a_1 a_2)^{\frac{1}{2}} \leq \frac{1}{2}(a_1 + a_2)$.
        \end{proof}
    
    \begin{proposition}[Bernoulli's Inequality]\label{prop:bernoulli}
        If $x > -1$, then $(1+x)^n \geq 1+nx$ for all $n \in \bfN_0$.
    \end{proposition}
        \begin{proof}
            We proceed with induction with base case $n=0$ and $n=1$; these hold by inspection. Assume the inequality holds true for $n = k$. For $n=k+1$:
                \begin{equation*}
                \begin{split}
                    (1+x)^{k+1}
                    & = (1+x)^k(1+x) \\
                    & \geq (1+nx)(1+x)\footnotemark \\
                    & = 1 + (n+1)x + nx^2 \\
                    & \geq 1 +(n+1)x.
                \end{split}
                \end{equation*}
            \footnotetext{Because order is preserved under multiplication by positive elements.}
        \end{proof}

    \begin{proposition}[Cauchy-Schwartz Inequality]\label{prop:cauchy-schwartz}
        Let $a_1,...,a_n$, $b_1,...,b_n \in \bfR^n$. Then:
            \begin{equation*}
            \begin{split}
                \left|\sum_{j=1}^n a_jb_j\right| \leq \left(\sum_{j=1}^n a_j^2\right)^\frac{1}{2} \left(\sum_{j=1}^n b_j^2\right)^\frac{1}{2}.
            \end{split}
            \end{equation*}
    \end{proposition}
        \begin{proof}
            Consider the map $F:\bfR^n \rightarrow \bfR^n$ defined by $F(t) = \sum_{j=1}^n(a_j - b_j t)^2$. Note that $\sum_{j=1}^n(a_j - b_j t)^2 \geq 0$. Observe that:
                \begin{equation*}
                \begin{split}
                    \sum_{j=1}^n(a_j - b_j t)^2
                    & = \sum_{j=1}^n(a_j^2 - 2a_jb_j t + b_j^2t^2) \\
                    & = \sum_{j=1}^n a_j^2 - \sum_{j=1}^n2a_jb_j t + \sum_{j=1}^n b_j^2 t^2.
                \end{split}
                \end{equation*}
            This is a quadratic equation, and since $F(t) \geq 0 $, the discriminant will be less than or equal to 0. Hence:
                \begin{equation*}
                \begin{split}
                    \Delta = \left(\sum_{j=1}^n2a_jb_j\right)^2 - 4 \left(\sum_{j=1}^n b_j^2\right) \left(\sum_{j=1}^na_j^2\right) \leq 0.
                \end{split}
                \end{equation*}
            Simplifying gives:
                \begin{equation*}
                \begin{split}
                    \left(\sum_{j=1}^n2a_jb_j\right)^2 \leq 4\left(\sum_{j=1}^n b_j^2\right) \left(\sum_{j=1}^na_j^2\right).
                \end{split}
                \end{equation*}
            Pulling $2$ out from the left-hand side, dividing both sides by $4$, and then square-rooting gives the desired result.
        \end{proof}
    
    \begin{question}
        When do we have equality?
    \end{question} 
        \begin{answer}
            When $\Delta = 0$, there exists a $t_0 \in \bfR$ with $F(t_0) = 0$. So $\sum_{j=1}^n(a_j - b_j t_0) = 0$ implies $a_j - b_j t_0 = 0$ for all $j$. Hence there is equality only when $a_j = b_j t_0$ for all $j$.
        \end{answer}

    \begin{proposition}[Triangle Inequality]\label{prop:triangle-ineq}
        Let $a_1,...,a_n$, $b_1,...,b_n \in \bfR^n$. Then:
            \begin{equation*}
            \begin{split}
                \left(\sum_{j=1}^n (a_j+b_j)^2 \right)^\frac{1}{2} \leq \left(\sum_{j=1}^n a_j^2\right)^\frac{1}{2} + \left(\sum_{j=1}^n b_j^2 \right)^\frac{1}{2}.
            \end{split}
            \end{equation*}
    \end{proposition}
        \begin{proof}
            Observe that:
                \begin{equation*}
                \begin{split}
                    \sum_{j=1}^n(a_j + b_j)^2
                    & = \sum_{j=1}^n(a_j^2 + 2a_jb_j + b_j^2) \\
                    & = \sum_{j=1}^n a_j^2 + \sum_{j=1}^n2a_jb_j +\sum_{j=1}^nb_j^2 \\
                    & \leq \sum_{j=1}^n a_j^2 + 2\left(\sum_{j=1}^na_j^2\right)^\frac{1}{2}\left(\sum_{j=1}^nb_j^2\right)^\frac{1}{2} + \sum_{j=1}^n b_j^2. \\
                    & = \left(\left(\sum_{j=1}^na_j^2\right)^\frac{1}{2} + \left(\sum_{j=1}^n b_j^2\right)^\frac{1}{2}\right)^2.
                \end{split}
                \end{equation*}
            Squaring both sides gives the desired result.
        \end{proof}

\section{Metrics and Norms on $\bbR^n$}
    \begin{definition}
        The \textui{absolute value} is a function $|\cdot|: \bfR \rightarrow \bfR$ defined by:
            \begin{equation*}
                |x|=
            \begin{cases}
                x, & x \in \bfR^+ \\
                -x, & -x \in \bfR^+.
            \end{cases}
            \end{equation*}
    \end{definition}

    \begin{proposition}
        Let $a,b \in \bfR$ and $\delta > 0$.
        \begin{enumerate}[label = (\arabic*)]
            \item $|ab| = |a||b|$.
            \item $|a|^2 = |a^2|$.
            \item $|-a| = |a|$.
            \item $|a| \in \bfR+$.
            \item $-|a| \leq a \leq |a|$.
            \item $|a| \leq \delta$ if and only if $-\delta \leq a \leq \delta$.
            \item $|a+b| \leq |a| + |b|$.
            \item $|a-b| \leq |a| + |b|$.
            \item $\left||a| - |b|\right| \leq |a - b|$.
        \end{enumerate}
    \end{proposition}
        \begin{proof}
            {\color{red} do later}
        \end{proof}

    \begin{lemma}
        $\pm x \leq \delta$ if and only if $|x| \leq \delta$.
    \end{lemma}
        \begin{proof}
            {\color{red} do lter}
        \end{proof}

    \begin{lemma}
        $A \subseteq \bfR$ is bounded if and only if there exists an $r>0$ such that $|a| < r$ for all $a \in A$.
    \end{lemma}
        \begin{proof}
            Suppose $A \subseteq \bfR$ is bounded. Then there exists an $l,u \in \bfR$ with $l \leq a \leq u$ for all $a \in A$. We have that:
                \begin{equation*}
                \begin{split}
                    -|l| \leq l \leq a \leq u \leq |u|.
                \end{split}
                \end{equation*}
            Let $r = \max{\left\{|l|, |u|\right\}} \geq 0$. So $-r \leq |l| \leq a \leq |u| \leq r$. Thus $|a| \leq r$.

            Conversely, suppose there exists an $r > 0$ with $|a| \leq r$ for all $a \in A$. Then $-r \leq  a \leq r$ for all $a \in A$, hence $A$ is bounded.
        \end{proof}
    
    \begin{definition}
        A function $f:D \rightarrow \bfR$ is \textui{bounded} if $\Image{(f)} \subseteq \bfR$ is a bounded subset. Equivalently, there exists a $c >0$ such that $|f(x)| < c$ for all $x \in D$.
    \end{definition}

    \begin{example}
        Consider the function $f:[3,7] \rightarrow \bfR$ defined by $f(x) = \frac{x^2 + 2x + 1}{x-1}$. Since $3 \leq x \leq 7$, observe that:
            \begin{equation*}
            \begin{split}
                |x^2 + 2x + 1| &\leq |x^2| + |2x| + 1 \\
                & = |x|^2 + 2|x| + 1 \quad{\text{\tiny Evaluate at 7}}\\
                & = 64
            \end{split}
            \end{equation*}
        Likewise, $3 \leq x \leq 7$ implies $|x-1| \geq 2$, hence $\frac{1}{|x-1|} \leq \frac{1}{2}$. Together, we have that:
            \begin{equation*}
            \begin{split}
                \left|\frac{x^2+2x+1}{x-1}\right| \leq \frac{64}{2} = 32.
            \end{split}
            \end{equation*}
    \end{example}
    
    \begin{definition}
        Let $s,t \in \bfR$. We define the \textui{distance} between $s$ and $t$ as $d(s,t) = |s-t|$.
    \end{definition}

    \begin{definition}
        Let $X$ be a nonempty set equipped with a map $d:X \times X \rightarrow \bfR^+$. We say $(X,d)$ is a \textui{semi-metric} if for all $x,y,z \in X$,
            \begin{enumerate}[label = (\arabic*)]
                \item $d(x,y) = d(y,x)$,
                \item $d(x,z) \leq d(x,y) + d(y,z)$, and
                \item $d(x,x) = 0$.
            \end{enumerate}
        We say $(X,d)$ is a \textui{metric space} if it satisfies the additional axiom:
            \begin{enumerate}[label = (\arabic*)]
                \addtocounter{enumi}{3}
                \item $d(x,y) = 0$ implies $x = y$.
            \end{enumerate}
    \end{definition}

    \begin{proposition}
        \phantom{a}
        \begin{enumerate}[label = (\arabic*)]
            \item $\left(\bfR , d_1(s,t) = |s-t|\right)$ is a metric space.
            \item $\left(\bfR^n, d_1(\vec{x},\vec{y}) = \sum_{j=1}^n |y_j - x_j|\right)$ is a metric space.
            \item $\left(\bfR^n, d_\infty(\vec{x},\vec{y}) = \max_{j=1}^n \left\{|y_j - x_j\right\})\right)$ is a metric space.
            \item $\left(\bfR^n, d_2(\vec{x},\vec{y}) = \left(\sum_{j=1}^n |y_j - x_j|^2\right)^\frac{1}{2}\right)$ is a metric space.
            \item $\left(\bfR^n, d_p(\vec{x},\vec{y}) = \left(\sum_{j=1}^n |y_j - x_j|^p\right)^\frac{1}{p}\right)$ for some $p \in \bfQ$ is a metric space.
        \end{enumerate}
    \end{proposition}
        \begin{proof}
            (1) We have $d(s,t) = |s-t| = |t-s| = d(t,s)$. Similarly, $d(s,r) = |s-r| = |s-t + t -r| \leq |s-t| + |t-r| = d(s,t) + d(t,r)$. Clearly $d(s,s) = |s-s| = 0$. Lastly, if $d(s,t) =0$, then $|s-t| = 0$, which is equivalent to $s-t = 0$; i.e., $s=t$. Thus $(\bfR,d_1)$ is a metric space.

            (4) Axioms 2 and 3 of metric spaces are clearly satisfied. If $d_2(\vec{x},\vec{y}) = 0$ then $|y_j - x_j|^2 = 0$ for all $j$. Hence $y_j-x_j = 0$; i.e., $y_j = x_j$ for all $j$, establishing axiom 4. Observe that:
                \begin{equation*}
                \begin{split}
                    d_2(\vec{x},\vec{z})
                    & = \left(\sum_{j=1}^n |z_j - x_j|^2\right)^\frac{1}{2}\\
                    & = \left(\sum_{j=1}^n |z_j - y_j + y_j - x_j|^2\right)^\frac{1}{2} \\
                    & = \left(\sum_{j=1}^n (z_j - y_j + y_j -  x_j)^2\right)^\frac{1}{2} \\
                    & \leq \left(\sum_{j=1}^n (z_j - y_j)^2\right)^\frac{1}{2} + \left(\sum_{j=1}^n (y_j - x_j)^2\right)^\frac{1}{2} \\
                    & = \left(\sum_{j=1}^n |z_j - y_j|^2\right)^\frac{1}{2} + \left(\sum_{j=1}^n |y_j - x_j|^2\right)^\frac{1}{2} \\
                    & = d_2(\vec{x},\vec{y}) + d_2(\vec{y},\vec{z}).
                \end{split}
                \end{equation*}
            Thus $(\bfR^n , d_2)$ is a metric space.
        \end{proof}

    \begin{definition}
        Let $(X,d)$ be a metric space.
            \begin{enumerate}[label = (\arabic*)]
                \item The \textui{open ball} centered at $x_0$ with radius $\delta > 0$ is $U(x_0,\delta) = \{y \in X \mid d(y,x_0) < \delta\}$.
                \item The \textui{closed ball} centered at $x_0$ with radius $\delta > 0$ is $B(x_0,\delta) = \{y \in X \mid d(y,x_0) \leq \delta\}$.
                \item A subset $A\subseteq X$ is called \textui{open} if for all $a \in A$, there exists a $\delta >0$ such that $U(a,\delta) \subseteq A$.
                \item A subset $C \subseteq X$ is called \textui{closed} if $\compl(C) = X \setminus C$ is open.
            \end{enumerate}
    \end{definition}

    \begin{example}
        Consider $X = \bfR$ and $d(s,t) = |s-t|$. Observe that:
            \begin{equation*}
            \begin{split}
                U(t,\delta) 
                & = \{s \in \bfR \mid d(s,t) < \delta\} \\
                & = \{s \in \bfR \mid |s-t| < \delta\} \\
                & = \{s \in \bfR \mid -\delta <s-t < \delta\} \\
                & = \{s \in \bfR \mid -\delta + t <s < \delta + t\}\\
                & = (t - \delta , t + \delta).
            \end{split}
            \end{equation*}
        It follows similarly that $B(t, \delta) = [t- \delta,t+\delta]$.
    \end{example}

    \begin{proposition}
        If $I$ is an open interval, then $I$ is open.
    \end{proposition}
        \begin{proof}
            Let $I = (a,b)$. Let $x \in I$. Let $\delta_x = \min{\left\{x-a,b-x\right\}} > 0$. Now let $t \in V_\delta(x)$. Then $t \in (x - \delta, x+\delta)$. Case 1: $\min{\left\{x-a,b-x\right\}} = x-a$. Then $x-(x-a) < t < x + x-a$,
                {\color{red} idk how to do this}
        \end{proof}
    

\chapter{Supremum, Infimum, and Completeness}\label{chapter:sup-inf-compl}
\vspace{12pt}

\section{Supremum and Infimum}

    \begin{theorem}\label{thm:supremum-property}
        Let $\emptyset \neq A \subseteq \bfR$. Let $u$ be an upperbound for $A$. The following are equivalent:
            \begin{enumerate}[label = (\arabic*)]
                \item $u = \Sup{(A)}$.
                \item If $t<u$, then there exists an $a_t \in A$ with $t < a_t$.
                \item For all $\epsilon > 0$, there exists an $a_\epsilon \in A$ such that $u - \epsilon < a_\epsilon$.
            \end{enumerate}
    \end{theorem}
        \begin{proof}
            $\left[(1)\implies(2)\right]$ Assume $u = \Sup{(A)}$. Let $t < u$. Suppose towards contradiction there does not exist and $a \in A$ with $a > t$. Then $a \leq t$ for all $a \in A$. But this implies $t$ is an upperbound of $A$ less than $u$, which is a contradiction because $u$ is the least upper bound. $\left[(2)\implies(3)\right]$ Given $\epsilon > 0$, let $t = u -\epsilon$. Then applying (2) gives the desired result. $\left[(3)\implies(1)\right]$ We know $u$ is an upperbound of $A$, we aim to show that it is the least upperbound. Let $v$ be an upperbound for $A$ with $ v < u$. Pick $\epsilon  = u - v > 0$. By (3), there exists an $a_\epsilon \in A$ such that $u - (u - v)  < a_\epsilon$. So $v < a_\epsilon$, which is a contradiction {\tiny( $v$ is an upperbound, how can it be smaller than an element of $A$?)}. 
        \end{proof}

    \begin{example}
        Claim: $\Sup([0,1)) = 1$. If $s \in [0,1)$, by definition $s<1$, so $1$ is an upper bound for $[0,1)$. Given $t<1$, set $\delta = 1-t > 0$. Then $0 < \frac{\delta}{2} < \delta$ {\color{red} this is not trivial, have to show $\delta - \delta/2$ is positive}. This gives:
            \begin{equation*}
            \begin{split}
                t < t+ \frac{\delta}{2} < t + \delta = 1.
            \end{split}
            \end{equation*}
        Pick $a_t = t + \frac{\delta}{2}$. By (2) of Theorem~\ref{thm:supremum-property}, $a_t \in [0,1)$, hence $1 = \Sup([0,1))$.
    \end{example}

    \begin{proposition}
        Let $A,B \subseteq \bfR$ and $a \leq b$ for all $a \in A$ and $b \in B$. Then $\Sup(A) \leq \Inf(B)$.
    \end{proposition}
        \begin{proof}
            Fix a point $b_0 \in B$. Then $a \leq b_0$ for all $a \in A$. Then $b_0$ is an upperbound for $A$. This gives $u := \Sup(A) \leq b_0$. But since $b_0$ was arbitrary, we have $u \leq b$ for all $b \in B$. So $u$ is a lower bound for $B$, therefore $u \leq \Inf(B)$.
        \end{proof}

    \begin{axiom}[Completeness of $\bfR$]
        Given any nonempty subset $A \subseteq \bfR$ which is bounded above, $\Sup(A)$ exists.
    \end{axiom}

    \begin{lemma}
        For $A \subseteq \bfR$ which is bounded below, $\sup(-A) = -\inf(A)$.
    \end{lemma}
        \begin{proof}
            If $A$ is bounded below, then $-A$ is bounded above. Then $\sup(-A)$ exists, define it as $u$. So for all $a \in A$, $-a \leq u$. Hence $-u$ is a lower bound for $A$. Suppose $v$ is another lower bound for $A$. Then $v \leq a$ for all $a \in A$. So $-v \geq -a$ for all $a \in A$. Thus $-v$ is an upper bound of $-A$. Therefore, since $u$ is the least upper bound, $-v \geq u$; i.e., $-u \geq v$. Thus $-u = \inf(A)$. 
        \end{proof}

    \begin{axiom}[Well-Ordering Princple]\label{axiom:wop}
        Every nonempty subset $A \subseteq \bfN$ contains a least element.
    \end{axiom}

    \begin{proposition}[Arcimedean Property 1]\label{prop:arch-1}
        If $x \in \bfR$, then there exists $n_x \in \bfN$ with $x < n_x$.
    \end{proposition}
        \begin{proof}
            Suppose not. That is, suppose $n \leq x$ for all $n \in \bfN$. Then $x$ is an upper bound for $\bfN$. Thus $\sup(A) := u$ exists. From part (3) of Theorem~\ref{thm:supremum-property}, take $\epsilon = 1$. Then there exists an $n \in \bfN$ such that $u - 1 < n$. So $u < n  + 1 \in \bfN$, which is a contradiction. 
        \end{proof}

    \begin{proposition}[Archimedean Property 2]\label{prop:arch-2}
        If $t > 0$, there exists $n_t \in \bfN$ with $\frac{1}{n_t} < t$.
    \end{proposition}
        \begin{proof}
            From \nameref{prop:arch-1}, pick $x = \frac{1}{t}$.
        \end{proof}

    \begin{corollary}
        Given $t>0$, there exists $m \in \bfN$ with $\frac{1}{2^m} < t$.
    \end{corollary}
        \begin{proof}
            By \nameref{prop:arch-2} there exists an $n \in \bfN$ with $\frac{1}{n} < t$. Claim: $\frac{1}{2^n} < \frac{1}{n}$. It suffices to show that $2^n > n$. Proposition~\ref{prop:power-set-bigger} gives $\card(\{1,2,...,n\}) < \card\left({\cP\left(\{1,2,...,n\}\right)}\right)$. Then Exercise~\ref{exercise:power-set-2n} gives:
                \begin{equation*}
                \begin{split}
                    n = \card(\{1,2,...,n\}) < \card\left({\cP\left(\{1,2,...,n\}\right)}\right) = 2^n.
                \end{split}
                \end{equation*}
            Alternatively, \nameref{prop:bernoulli} gives $(1+1)^n \geq 1 + n$. Hence $2^n > n$.
        \end{proof}

    \begin{example}
        \phantom{a}
        \begin{enumerate}[label = (\arabic*)]
            \item Claim: $\Inf{\left\{\frac{1}{n} \mid n \in N\right\}} = 0$. Note that $0$ is indeed a lower bound because $0 < \frac{1}{n}$ for all $n \in \bfN$. Suppose $t$ is another lower bound. If $t \leq 0$, then we are done. If $t > 0$, by the Archimedean Property there exists an $n_t \in \bfN$ such that $\frac{1}{n_t} < t$, which is a contradiction {\tiny (because we asserted that $t$ is a lower bound, and $\frac{1}{n_t} \in \Inf{\{\frac{1}{n} \mid n \in N\}}$)}. Thus $\Inf{\left\{\frac{1}{n} \mid n \in N\right\}} = 0$.
            \item Claim: $\Inf{\left\{\frac{1}{2^m} \mid m \in N\right\}} = 0$. This follows from the above example and previous corollary.
        \end{enumerate}
    \end{example}

    \begin{corollary}\label{cor:natural-density}
        Let $x \in \bfR$, Then there exists $n_x \in \bfZ$ with $n_x - 1 \leq x < n_x$.
    \end{corollary}
        \begin{proof}
            Case 1: $x \geq 0$. Let $S_x = \{n \in \bfN \mid x < n\}$. By \nameref{prop:arch-1} $S_x \neq 0$. By the \nameref{axiom:wop}, there exists a least element in this set, call it $n_x$. Since $n_x \in S_x$, it must be the case that $x < n_x$. But since $n_x$ is the least element, $n_x - 1 \not\in S_x$. Since $S_x$ is the set of all natural numbers with lower bound $x$, $n_x - 1$ is not bounded below by $x$. Whence $n_x - 1 \leq x$.

            Case 2: $x <0$. Define $S_{-x} = \{n \in \bfN \mid n < -x\}$. As a consequence of the \nameref{axiom:wop}, any subset of the integers which is bounded above admits a greatest element, define it to be $n_{-x} \in \bfZ$. Then $n_{-x} + 1 \not\in S_{-x}$, hence $n_{-x} < -x \leq n_{-x} + 1$. This establishes $-n_{-x} - 1 \leq x < -n_{-x}$.
        \end{proof}

    \begin{definition}
        Let $I$ be an open interval. A subset $D \subseteq \bfR$ is \textui{dense} if $I \cap D \neq \emptyset$.
    \end{definition}

    \begin{theorem}\label{thm:density-of-q}
        $\bfQ \subseteq \bfR$ is dense.
    \end{theorem}
        \begin{proof}
            Let $I$ be an open interval. Then there exists $a,b \in \bfR$ with $(a,b) \subseteq I$. We have that $b - a > 0$. By \nameref{prop:arch-2} there exists $n \in \bfN$ with $\frac{1}{n} < b-a$. So $1 + na < nb$. By Corollary~\ref{cor:natural-density}, there exists $m \in \bfZ$ with $m-1 \leq na < m$. Equivalently, we have that $a < \frac{m}{n}$. We also have that $m \leq na + 1 < nb$, which yields $\frac{m}{n} < b$. Thus $\frac{m}{n} \in (a,b) \cap \bfQ$.
        \end{proof}

    \begin{corollary}
        $\bfR \setminus \bfQ \subseteq \bfR$ is dense.
    \end{corollary}
        \begin{proof}
            Let $a < b$. Consider $a' = a\sqrt{2}$ and $b' = b\sqrt{2}$. Then $a' < b'$. By Theorem~\ref{thm:density-of-q}, there exists a $q \in \bfQ$ with $a' < q < b'$. Thus $a < \frac{q}{\sqrt{2}} < b$. Since $\frac{q}{\sqrt{2}} \not\in \bfQ$, the corollary is established.

            Alternatively, observe the following picture:
                \begin{center}
                    \phantom{a}\\
                    \begin{tikzpicture}
                        % Draw the number line with arrows on both sides
                        \draw[thick, <->] (-3,0) -- (6,0) node[right] {$\mathbf{R}$};
                    
                        % Mark points a and b
                        \filldraw (0,-0.23) node[below] {$a$};
                        \filldraw (3,-0.23) node[below] {$b$};
                    
                        % Add parentheses around the interval (a, b)
                        \node at (0, 0) {$($};
                        \node at (3, 0) {$)$};
                        \node at (1.5, -0.5) {$\mathlarger{\uparrow}$};
                    
                    \end{tikzpicture}
                \end{center}
            If there is not an irrational number between $(a,b)$, then $(a,b) \subseteq \bfQ$, which is a contradiction.
        \end{proof}

    \begin{theorem}
        There exists a unique positive number $x$ with $x^2 = 2$.
    \end{theorem}
        \begin{proof}
            Consider the set $S = \{t \in \bfR \mid t > 0, t^2 < 2 \}$. Note that $S \neq 0$ because $1 \in S$. If $t \geq 2$, then $t^2 \geq 2t > 4$, meaning it would not be an element of $S$. So $S$ is bounded above by $2$. Hence there exists $u := \sup(S)$.
                \begin{center}
                    \begin{tikzpicture}
                        \draw[thick] (0.3,0) -- (2.3,0);
                        \node at (2.33, 0) {$/\,$};
                        \node at (2.62, 0) {$/\,$};
                        \draw[thick] (2.6,0) -- (4.6,0);
                    \end{tikzpicture}
                \end{center}
            
            \vspace{-10pt}
            {\smaller Scratchwork: Assume $u^2 < 2$. Find a sufficiently small $n$ so that $(u + \frac{1}{n})^2 \in S$; i.e.,   $(u + \frac{1}{n})^2 < 2$. Solving for $n$ yields:
                \begin{gather*}
                    u^2 + \frac{2u}{n} + \frac{1}{n^2} < 2 \\
                    \iff \\
                    \frac{2u}{n} + \frac{1}{n^2} < 2 - u^2 \\
                    \iff \\
                    \frac{1}{n} \left(2u + \frac{1}{n}\right) < 2-u^2\\
                    \iff \\
                    \frac{1}{n} \left(2u + 1\right) < 2-u^2 \\
                    \iff \\
                    \frac{1}{n} < \frac{2-u^2}{2u+1} \in \bfR^+\setminus\{0\}
                \end{gather*}}
                \vspace{-20pt}
                \begin{center}
                    \begin{tikzpicture}
                        \draw[thick] (0.3,0) -- (2.3,0);
                        \node at (2.33, 0) {$/\,$};
                        \node at (2.62, 0) {$/\,$};
                        \draw[thick] (2.6,0) -- (4.6,0);
                    \end{tikzpicture}
                \end{center}
            

            If $u^2 < 2$, then $\frac{2-u^2}{2u+1} > 0$. By \nameref{prop:arch-2}, there exists an $n\in \bfN$ with $\frac{1}{n} < \frac{2-u^2}{2u+1}$. Simplifying yields $(u+\frac{1}{n})^2 < 2$, or equivalently $u + \frac{1}{n} \in S$, which is a contradiction. It must be the case that $u^2 \geq 2$; i.e., $u^2 - 2 \geq 0$. Now since $u = \sup(S)$, for all $m \in \bfN$, there exists $t_m \in S$ with $u - \frac{1}{m} < t_m$. We have that $(u - \frac{1}{m})^2 < t_m^2 < 2$. This simplifies to $u^2 - 2 < \frac{2u}{m} - \frac{1}{m^2} < \frac{2u}{m}$, or equivalently $\frac{u^2 - 2}{2u} < \frac{1}{m}$. But if $\frac{u^2 - 2}{2u} < \frac{1}{m}$ for all $m \in \bfN$, it must be that $\frac{u^2 - 2}{2u} = 0$, hence $u^2 = 2$.

            Lastly we show that $u^2$ is unique. Suppose $u^2 = 2 = v^2$. Since $u,v \geq 0$, $(u^2 - v^2) = 0$. Then $(u-v)(u+v) = 0$. If $u+v = 0$, then $u = 0$ and $v = 0$, which is a contradiction. So $u-v = 0$ implies $u = v$.
        \end{proof}

        \begin{remark}
            Picking $2$ was completely arbitrary, we could have showed $x^2 = a$ for any $a \geq 0$.
        \end{remark}

        \begin{remark}
            Using the same argument, we have that for all $a > 0$, there exists a unique $b > 0$ with $b^2 = a$. So we have a map:
                \begin{equation*}
                    \bfR^+ \xrightarrow{\sqrt{}} \bfR^+,
                \end{equation*}
            where $\sqrt{x}$ is the unique positive number with $(\sqrt{x})^2 = x$.
        \end{remark}

        \begin{remark}
            We could have similarly defined $S$ as:
                \begin{equation*}
                \begin{split}
                    S' = \{t \in \bfQ \mid t>0, t^2 < 2\},
                \end{split}
                \end{equation*}
            and the proof would not have changed. However, $\sup(S') = \sqrt{2} \not\in \bfQ$, meaning $\bfQ$ is \textit{not} complete.
        \end{remark}

\section{Nested Intervals}
    \begin{axiom}\label{axiom:4}
        Given any interval $I$, if $x,y \in I$ with $x<y$, then $[x,y] \in I$.
    \end{axiom}
    \begin{theorem}
        Let $S \subseteq \bfR$ be any subset containing at least two points. If $S$ satisfies Axiom~\ref{axiom:4}, then $S$ is an interval.
    \end{theorem}
        \begin{proof}
            We proceed with cases. Case 1: $S$ is bounded. Write $a = \inf(S)$ and $b = \sup(S)$. Therefore $S \subseteq [a,b]$. If we show $(a,b) \subseteq S$, then it follows that $S = (a,b]$, or $[a,b)$, or $(a,b)$ or $[a,b]$. We must use that $S$ satisfies Axiom 4 and $a = \inf(S)$ and $b = \sup(S)$. Let $x \in (a,b)$. Since $x > a$, there exists and $s_1 \in S$ with $s_1 < x$. Since $x < b$, there exists an $s_2 \in S$ with $x< s_2$. Thus $s_1,s_2 \in S$ and $s_1 < s_2$. By Axiom 4 $[s_1 ,s_2] \subseteq S$. But $x \in [s_1,s_2]$ implies $x \in S$. Thus $(a,b) \subseteq S$.

            Case 2: $S$ is bounded above {\color{red} do this}.

            Case 3: $S$ is bounded below {\color{red} need to do}.
        \end{proof}

    \begin{definition}
        A sequence of intervals $(I_n)_{n\geq 1}$ is said to be $\textui{nested}$ if $I_1 \supseteq I_2 \supseteq I_3 \supseteq ...$.
    \end{definition}

    \begin{proposition}
        $\bigcap_{n\geq 1}\left[0,\frac{1}{n}\right) = \{0 \}$. 
    \end{proposition}
        \begin{proof}
            Note that $0 \in \left[0,\frac{1}{n}\right)$ for all $n\geq 1$. So $0 \in \bigcap_{n\geq 1}\left[0,\frac{1}{n}\right)$. Let $a \in \bigcap_{n\geq 1}\left[0,\frac{1}{n}\right)$. Then $0 \leq a < \frac{1}{n}$ for all $n \geq 1$. Hence $a =0$.
        \end{proof}

    \begin{proposition}
        $\bigcap_{n\geq 1}\left[n,\infty\right) = \emptyset$. 
    \end{proposition}
        \begin{proof}
            Suppose towards contradiction there exists a $t \in \bigcap_{n\geq 1}\left[n,\infty\right) = \emptyset$. Then $t \in [n,\infty)$ for all $n\geq 1$. So $t \geq n$ for all $n \geq 1$. Hence $\bfN$ is bounded above, which is a contradiction.
        \end{proof}
    
    \begin{theorem}[Nested Intervals]\label{thm:nested-intervals}
        Let $(I_n)_{n \geq 1}$ be a sequence of closed and bounded nested intervals. Then $\bigcap_{n \geq 1}I_n \neq \emptyset$. Furthermore, if $\inf\left\{\text{length}(I_n)\mid n \geq 1\right\} = 0$, then $\bigcap_{n \geq 1}I_n = \{\xi\}$.
    \end{theorem}
        \begin{proof}
            Let $I_n = [a_n,b_n]$. Note that:
                \begin{equation*}
                \begin{split}
                    a_1 \leq a_2 \leq a_3 \leq ... \\
                    b_1 \geq b_2 \geq b_3 \geq ...
                \end{split}
                \end{equation*}
            We have that $a_1 \leq a_n \leq b_n \leq b_1$ for all $n \geq 1$. So the set $\{a_n \mid n \geq 1\}$ is bounded above, and similarly $\{b_n \mid n \geq 1 \}$ is bounded below. Let
                \begin{equation*}
                \begin{split}
                    \xi &= \sup_{n \geq 1}\left\{a_n\right\} \\
                    \eta &= \inf_{n \geq 1}\left\{b_n\right\}.
                \end{split}
                \end{equation*}
            Claim: $\xi \leq b_n$ for all $n \geq 1$. Assume towards contradiction $\xi > b_m$ for some $m \geq 1$. Since $\xi = \sup_{n \geq 1} \left\{a_n\right\}$, there exists an $a_k$ with $b_m < a_k \leq \xi$. If $k \geq m$, then $b_m < a_k \leq b_k \leq b_m$, which is a contradiction. If $k < m$, then $a_k \leq a_m \leq b_m < a_k$, which is a contradiction.

            Claim: $a_n \leq \xi$ for all $n \geq 1$. Then $\xi \leq \eta$ since $\sup_{n \geq 1} \left\{a_n\right\} = \xi$. We have $[\xi, \eta] \subseteq [a_n,b_n]$ for all $n \in \bfN$. Let $x \in [\xi, \eta]$. Then:
                \begin{equation*}
                \begin{split}
                    a_n \leq \xi \leq x \leq \eta \leq b_n,
                \end{split}
                \end{equation*}
            hence $x \in [a_n,b_n]$; i.e., $[\xi ,\eta] \subseteq [a_n,b_n]$ for all $n \geq 1$. Thus $[[\xi ,\eta] \subseteq \bigcap_{n \geq 1}[a_n,b_n]]$. Conversely, let $t \in [a_n,b_n]$ for all $n \geq 1$. Then $a_n \leq t \leq b_n$. This implies $t$ is both an upper bound for $\left\{a_n\right\}_{n\geq 1}$ and a lower bound for $\left\{a_b\right\}_{n\geq 1}$. Hence $\xi \leq t \leq eta$, implying $t \in [\xi, \eta]$. This establishes $[\xi , \eta] = \bigcap_{n \geq 1}[a_n,b_n]$.

            Now suppose $\inf\left\{\text{length}(I_n)\mid n \geq 1\right\} = 0$. Then:
                \begin{equation*}
                \begin{split}
                    0
                    & = \inf_{n \geq 1}(b_n - a_n) \\
                    & = \inf_{n \geq 1}b_n - \inf_{n \geq 1}a_n \\
                    & = \eta - \xi.
                \end{split}
                \end{equation*}
            Hence $\xi = \eta$, which establishes the theorem.

            Alternatively, had we assumed $\xi \neq \eta$, then $\eta - \xi > 0$. So there exists an $m$ such that $b_m - a_m < \eta - \xi$, which is a contradiction since $[\xi , \eta] \subseteq [a_m,b_m]$.
        \end{proof}

    \begin{corollary}
        $[0,1]$ is uncountable.
    \end{corollary}
        \begin{proof}
            By way of contradiction, suppose $[0,1] = \{t_1,t_2,t_3,...\}$. Consider the following picture:
            \begin{center}
                \phantom{a}\\
                \begin{tikzpicture}
                    % Draw the number line with arrows on both sides
                    \draw[thick, ] (-3,0) -- (6,0) node[right] {$\mathbf{R}$};
                
                    % Mark points a and b
                    \filldraw (1,-0.23) node[below] {$a_1$};
                    \filldraw (4.5,-0.23) node[below] {$b_1$};

                    \filldraw (-3.2,-0.23) node[below] {$a_0 = 0$};
                    \filldraw (5.8,-0.23) node[below] {$b_0 = 1$};

                    \filldraw (-3.2,0) node[right] {$|$};
                    \filldraw (5.8,0) node[right] {$|$};

                    \filldraw (-0.6,0) node[right] {$|$};
                    \filldraw (-0.4,-0.23) node[below] {$t_1$};
                    
                
                    % Add parentheses around the interval (a, b)
                    \node at (1, 0) {$[$};
                    \node at (4.5, 0) {$]$};
                
                \end{tikzpicture}
            \end{center}
            Find $[a_1,b_1] \subseteq [0,1]$ with $t_1 \not\in [a_1,b_1]$. Find $[a_2,b_2] \subseteq [a_1,b_1]$ with $t_2 \not\in [a_2,b_2]$. Inductively, find $[a_n,b_n] \subseteq [a_{n-1},b_{n-1}]$ with $t_n \not\in [a_n,b_n]$. Thus $[a_n,b_n]$ is nested. Now let $\xi \in \bigcap_{n \geq 1}[a_n,b_n]$. Then $\xi \in [0,1]$. But $\xi \neq t_n$ for all $n$, which is a contradiction.
        \end{proof}


\chapter{Sequences}\label{chapter:sequences}
\vspace{12pt}

\section{Basic Definitions and Examples}
    \begin{definition}
        A \textui{sequence} in a metric space $X$ is a map $x: \bfN \rightarrow X$. We often write $x = (x_n)_{n \geq 1} = (x_1,x_2,x_3,...)$, where $x_n = x(n)$. If $X = \bfR$, we call $x$ a \textui{real sequence}.
    \end{definition}

    \begin{example}[Sequences Defined Explictly]
        \phantom{a}
        \begin{enumerate}[label = (\arabic*)]
            \item A constant sequence: $x_n = t$, $ (x_n)_{n \geq 1} = (t,t,t,t,...)$.
            \item Sequences defined by a function: $d_n = (1+\frac{1}{n})^n$.
            \item Geometric sequences\footnote{These are called geometric because the ratio between each $x_n$ is constant: $x_{n+1}/x_n = b^{n+1}/b^n = b$.}: fix $b \in \bfR$, $x_n = b^n$. Then $(x_n)_{n \geq 1} = (1,b,b^2,b^3,...)$.
        \end{enumerate}
    \end{example}

    \begin{example}[Sequences Defined Recursively]
        \phantom{a}
        \begin{enumerate}[label = (\arabic*)]
            \item Let $a_1 = 1$, $a_{n+1} = 2a_n + 1$. Then $(a_n)_{n \geq 1} = (1,3,7,15,...)$.
            \item Let $f_1 = 1$, $f_2 = 1$, $f_{n+1} = f_n + f_{n-1}$. Then $(f_n)_{n=1}^\infty = (1,1,2,3,5,8,...)$. This is the \textit{Fibonacci sequence}.
            \item Let $X$ be a metric space and $f:X \rightarrow X$ be an endomorphism. Fix $x_0 \in X$. Then define:
                \begin{equation*}
                \begin{split}
                    x_1 &= f(x_0) \\
                    x_2 &= f(x_1) \\
                    &\vdots \\
                    x_n &= f(x_{n-1}).
                \end{split}
                \end{equation*}
        \end{enumerate}
    \end{example}

    \begin{example}[New Sequences from Old]
        \phantom{a}
        \begin{enumerate}[label = (\arabic*)]
            \item Let $(a_n)_{n\geq 1}$ and $(b_n)_{n \geq 1}$ be sequences. Then define:
                \begin{equation*}
                \begin{split}
                    (a_n)_{n\geq 1} \pm (b_n)_{n\geq 1} &= (a_n+b_n)_{n\geq 1}, \\
                    t(a_n)_{n\geq 1} &= (ta_n)_{n\geq 1}, \\
                    (a_n)_{n\geq 1}\cdot (b_n)_{n\geq 1} &= (a_n\cdot b_n)_{n\geq 1}. \\
                \end{split}
                \end{equation*}
            If $(b_n)_{n\geq 1} \neq 0$ for all $n$, then:
                \begin{equation*}
                \begin{split}
                    \frac{(a_n)_{n\geq 1}}{(b_n)_{n\geq 1}} & = \left(\frac{a_b}{b_n}\right)_{n\geq 1}.
                \end{split}
                \end{equation*}

            \item Given $(x_n)_{n \geq 1}$ and $k \in \bfN$, consider $(x_{n+k})_{n=0}^\infty = (x_k,x_{k+1},x_{k+1},...)$. This is called a \textit{shift} or the \textit{$k^{\text{th}}$ tail} of $(x_n)_{n \geq 1}$.
            
            \item If $(a_n)_{n \geq 1}$ is a sequence, $a_n \neq 0$ for all $n$, consider:
                \begin{equation*}
                \begin{split}
                    r_n = \frac{a_{n+1}}{a_n}.
                \end{split}
                \end{equation*}
            So $(r_n)_{n \geq 1} = \left(\frac{a_2}{a_1},\frac{a_3}{a_2},\frac{a_4}{a_3},...\right)$. These are called sequences of \textit{ratios}.

            \item Given a real sequence $(x_k)_{k=1}^\infty$, consider the sequence $(s_n)_{n=1}^\infty$ where:
                \begin{equation*}
                \begin{split}
                    s_1 &= x_1 \\
                    s_2 &= x_1 + x_2 = s_1 + x_2 \\
                    s_3 &= x_1 + x_2 + x_3 = s_2 + x_3 \\
                    &\vdots \\
                    s_n &= \sum_{k = 1}^n x_k = s_{n-1} + x_k.
                \end{split}
                \end{equation*}
            We call these \textit{$n^{\text{th}}$ partial sums}. An example of these are geometric sequences and telescoping sequences.
        \end{enumerate}
    \end{example}

\section{Convergence}
    \begin{definition}
        Let $(x_n)_{n\geq 1}$ be a sequence.
        \begin{enumerate}[label = (\arabic*)]
            \item $x_n$ is \textui{increasing} if $x_1 \leq x_2 \leq x_3 \leq ...$
            \item $x_n$ is \textui{decreasing} if $x_1 \geq x_2 \geq x_3 \geq ...$
            \item $x_n$ is \textui{strictly increasing} if $x_1 < x_2 < x_3 < ...$
            \item $x_n$ is \textui{strictly decreasing} if $x_1 > x_2 > x_3 > ...$
        \end{enumerate}
    \end{definition}

    \begin{note}
        A sequence is said to \textit{eventually} have a certain property, if it does not have the said property across all its ordered instances, but will after some instances have passed.
    \end{note}

    \begin{note}
        $x_n$ is \textui{monotone} if it is either increasing or decreasing, strictly increasing, or strictly decreasing.
    \end{note}

    \begin{definition}
        A sequence $(x_n)_n$ in a metric space $X$ \textui{converges} to $x \in X$ if:
            \begin{equation*}
            \begin{split}
                (\forall \epsilon > 0)(\exists N_\epsilon \in \bfN) \mtext{s.t.} n \geq N_\epsilon \implies d(x_n,x) < \epsilon.\footnotemark
            \end{split}
            \end{equation*}
        If no such $x$ exists, the sequence is \textui{divergent}. If $(x_n)_n$ converges to $x$, we write $(x_n)_n \xrightarrow{n \rightarrow \infty} x$ or $\lim_{n \rightarrow \infty}x_n = x$.
        \footnotetext{I try not to use first-order logic symbols but this will be one of the few exceptions.}
    \end{definition}

    \begin{example}
        Let $X = \bfR$. Then from the above definition, write $d(x_n,x) = |x_n - x|$. Recall that this is equivalent to $x_n \in V_\epsilon(x)$. We can visually represent convergence as follows:

        \begin{center}
            \begin{tikzpicture}
                \begin{axis}[
                    axis lines = middle,
                    xmin=0, xmax=11, % x-axis starts at 0 and extends to 10
                    ymin=-1, ymax=10, % y-axis range
                    xtick={1,2,3,4,5,6,7,8,9,10}, % add ticks 1 to 8 on the x-axis
                    xticklabels={1,2,3,4,5,6,7,8,9,10}, % labels for the x-axis ticks
                    ytick={4, 5, 6}, % positions for the y-axis ticks
                    yticklabels={$x-\epsilon$, $x$, $x+\epsilon$}, % labels for the y-axis ticks
                ]

                
                % Add dashed horizontal lines at x+\epsilon and x-\epsilon
                \addplot[dashed, color=red] coordinates {(0, 4) (11, 4)}; % dashed line at x+\epsilon
                \addplot[dashed, color=red] coordinates {(0, 6) (11, 6)}; % dashed line at x-\epsilon

                \addplot[only marks,mark=*,mark options={fill=green},color=green]coordinates {(1, 7)};
                \addplot[only marks,mark=*,mark options={fill=green},color=green]coordinates {(2, 2.5)};
                \addplot[only marks,mark=*,mark options={fill=green},color=green]coordinates {(3, 6.5)};
                \addplot[only marks,mark=*,mark options={fill=green},color=green]coordinates {(4, 3)};
                \addplot[only marks,mark=*,mark options={fill=green},color=green]coordinates {(5, 6.2)};
                \addplot[only marks,mark=*,mark options={fill=green},color=green]coordinates {(6, 4)};
                \addplot[only marks,mark=*,mark options={fill=green},color=green]coordinates {(7, 5.8)};
                \addplot[only marks,mark=*,mark options={fill=green},color=green]coordinates {(8, 4.4)};
                \addplot[only marks,mark=*,mark options={fill=green},color=green]coordinates {(9, 5.3)};
                \addplot[only marks,mark=*,mark options={fill=green},color=green]coordinates {(10, 5)};
        
                \end{axis}
            \end{tikzpicture}
        \end{center}
        If the sequence is convergent it will eventually be contained between the two dashed lines.
    \end{example}

    \begin{example}
        Prove that $\left(\frac{1}{n}\right)_{n\geq 1} \rightarrow 0$.
    \end{example}
    \begin{solution}
        This is a test for pushing to Github! This is another test
    \end{solution}



\end{document}