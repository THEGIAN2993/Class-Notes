\chapter{Sequences}\label{chapter:sequences}
\vspace{12pt}

\section{Basic Definitions and Examples}
    \begin{definition}
        A \textui{sequence} in a metric space $X$ is a map $x:\bfN \rightarrow X$. We often write $x = (x_n)_n = (x_1,x_2,...)$, where $x_n = x(n)$. If $X = \bfR$, we call $x$ a \textui{real sequence}.
    \end{definition}

    \begin{example}
        \phantom{a}
        \begin{enumerate}[label = (\arabic*)]
            \item Sequences defined explicitly:
                \begin{enumerate}[label = (\roman*)]
                    \item Constant sequences: $x_n = t$, $(x_n)_n  = (t,t,t,...)$
                    \item Sequences defined by a function: $d_n = \left(1 + \frac{1}{n}\right)^n$.
                    \item Geometric sequences: fix $b \in \bfR$, then $(b^n)_n = (1,b,b^2,...)$.    
                \end{enumerate}
            \item Sequences defined recursively:
                \begin{enumerate}[label = (\roman*)]
                    \item Let $a_1 = 1$, $a_{n+1} = 2a_n + 1$. Then $(a_n)_n = (1,3,7,15,...)$.
                    \item Let $f_1 = 1$, $f_2 = 1$, $f_{n+1} = f_n + f_{n-1}$. Then $(f_n)_n = (1,1,2,3,5,8,...)$. This is the \textit{Fibonacci sequence}.
                    \item Let $X$ be a metric space and $f:X \rightarrow X$ be an endomorphism. Fix $x_0 \in X$. Then define:
                        \begin{equation*}
                        \begin{split}
                            x_1 &= f(x_0) \\
                            x_2 &= f(x_1) \\
                            &\vdots \\
                            x_n &= f(x_{n-1}).
                        \end{split}
                        \end{equation*}  
                \end{enumerate}
            \item New sequences from old:
                \begin{enumerate}[label = (\roman*)]
                    \item Let $(a_n)_n$ and $(b_n)_n$ be sequences. Define:
                        \begin{equation*}
                        \begin{split}
                            (a_n)_n + (b_n)_n &= (a_n + b_n)_n \\
                            t(a_n)_n &= (ta_n)_n \\
                            (a_n)_n \cdot (a_n)_n &= (a_n b_n)_n \\
                            \frac{(a_n)_n}{(b_n)_n} &= \left(\frac{a_n}{b_n}\right)_n, \hspace{4pt} (b_n)_n \neq 0 \mtext{for all $n$}.
                        \end{split}
                        \end{equation*}
                    \item Given $(x_n)_n$ and $k \in \bfN$, consider $(x_{n+k})_n = (x_k,x_{k+1},...)$. This is called a \textit{shift} or the \textit{$k^\text{th}$ tail} of $(x_n)_n$.
                    \item If $(a_n)_n$ is a sequence, $a_n \neq 0$ for all $n$, consider:
                        \begin{equation*}
                        \begin{split}
                            r_n = \frac{a_{n+1}}{a_n}.
                        \end{split}
                        \end{equation*}
                    So $(r_n)_n = \left(\frac{a_2}{a_1},\frac{a_3}{a_2},\frac{a_4}{a_3},...\right)$. These are called \textit{sequences of ratios}.
                    \item Given a real sequence $(x_k)_k$, consider the sequence $(s_n)_n$ where:
                        \begin{equation*}
                        \begin{split}
                            s_n = \sum_{k=1}^n x_k = s_{n-1} + x_k.
                        \end{split}
                        \end{equation*}
                    We call these \textit{$n^\text{th}$ partial sums}. An example of these are geometric sequences and telescoping sequences.
                \end{enumerate}
        \end{enumerate}
    \end{example}

    \begin{example}
        Let $F$ be a field. The set $F^\bfN = \{x\mid x:\bfN \rightarrow F\}$ is the set of all $F$-sequences. This forms an $F$-vector space under componentwise addition and scalar multiplication.
    \end{example}

    \begin{definition}
        Let $(x_n)_n$ be a sequence.
            \begin{enumerate}[label = (\arabic*)]
                \item $x_n$ is \textui{increasing} if $x_1 \leq x_2 \leq x_3 \leq ...$
                \item $x_n$ is \textui{decreasing} if $x_1 \geq x_2 \geq x_3 \geq ...$
                \item $x_n$ is \textui{strictly increasing} if $x_1 < x_2 < x_3 < ...$
                \item $x_n$ is \textui{strictly decreasing} if $x_1 > x_2 > x_3 > ...$
            \end{enumerate}
    \end{definition}

    \begin{definition}
        A sequence is said to \textui{eventually} have a certain property if it does not have the said property across all its ordered instances, but will after some instances have passed.
    \end{definition}

    \begin{definition}
        A sequence $(x_n)_n$ is \textui{monotone} if it is either increasing, decreasing, strictly increasing, or strictly decreasing.
    \end{definition}

\section{Convergence}
    \begin{definition}
        Let $(x_n)_n$ be a sequence in a metric space $X$.
        \begin{enumerate}[label = (\arabic*)]
            \item $(x_n)_n$ \textui{converges} to $x \in X$ if:
                \begin{equation*}
                \begin{split}
                    (\forall \epsilon > 0)(\exists N_\epsilon \in \bfN)\ni(\forall n \in \bfN)(n \geq N_\epsilon \implies d(x_n,x) < \epsilon)).
                \end{split}
                \end{equation*}
            We denote this as $(x_n)_n \rightarrow x$ or $\lim_{n \rightarrow \infty} x_n = x$.

            \item $(x_n)_n$ \textui{does not exist} if:
                \begin{equation*}
                \begin{split}
                    (\exists \epsilon_0 > 0)(\forall N \in \bfN)\ni (\exists n \in \bfN)(n \geq N \hspace{2pt}\wedge\hspace{2pt} d(x_n,n)\geq \epsilon_0).
                \end{split}
                \end{equation*}
            We abbreviate this as D.N.E.

            \item $(x_n)_n$ \textui{diverges properly} to $+\infty$ if:
                \begin{equation*}
                \begin{split}
                    (\forall M > 0)(\exists N_M \in \bfN) \ni (\forall n \in \bfN)(n \geq N_M \implies x_n \geq M).
                \end{split}
                \end{equation*}
            We write $(x_n)_n \rightarrow +\infty$.

            \item $(x_n)_n$ \textui{diverges properly} to $-\infty$ if:
                \begin{equation*}
                \begin{split}
                    (\forall M < 0)(\exists N_M \in \bfN) \ni (\forall n \geq N_M)(x_n \leq M).
                \end{split}
                \end{equation*}
        \end{enumerate}      
    \end{definition} 

    \begin{example}
        Let $(x_n)_n$ be a real sequence. Then $d(x_n,x) < \epsilon \iff |x_n - x| < \epsilon \iff x_n \in V_\epsilon(x)$. We can visually represent a sequence as follows:
            \begin{center}
                \begin{tikzpicture}
                    \begin{axis}[
                        axis lines = middle,
                        xmin=0, xmax=11, % x-axis starts at 0 and extends to 10
                        ymin=-1, ymax=10, % y-axis range
                        xtick={1,2,3,4,5,6,7,8,9,10}, % add ticks 1 to 8 on the x-axis
                        xticklabels={}, % labels for the x-axis ticks
                        xlabel={$\bfN$},
                        ytick={4, 5, 6}, % positions for the y-axis ticks
                        yticklabels={$x-\epsilon$, $x$, $x+\epsilon$}, % labels for the y-axis ticks
                        ylabel={$\bfR$},
                    ]

                    
                    % Add dashed horizontal lines at x+\epsilon and x-\epsilon
                    \addplot[dashed, color=red] coordinates {(0, 4) (11, 4)}; % dashed line at x+\epsilon
                    \addplot[dashed, color=red] coordinates {(0, 6) (11, 6)}; % dashed line at x-\epsilon

                    \addplot[only marks,mark=*,mark options={fill=blue},color=blue]coordinates {(1, 7)};
                    \addplot[only marks,mark=*,mark options={fill=blue},color=blue]coordinates {(2, 2.5)};
                    \addplot[only marks,mark=*,mark options={fill=blue},color=blue]coordinates {(3, 6.5)};
                    \addplot[only marks,mark=*,mark options={fill=blue},color=blue]coordinates {(4, 3)};
                    \addplot[only marks,mark=*,mark options={fill=blue},color=blue]coordinates {(5, 6.2)};
                    \addplot[only marks,mark=*,mark options={fill=blue},color=blue]coordinates {(6, 4)};
                    \addplot[only marks,mark=*,mark options={fill=blue},color=blue]coordinates {(7, 5.8)};
                    \addplot[only marks,mark=*,mark options={fill=blue},color=blue]coordinates {(8, 4.4)};
                    \addplot[only marks,mark=*,mark options={fill=blue},color=blue]coordinates {(9, 5.3)};
                    \addplot[only marks,mark=*,mark options={fill=blue},color=blue]coordinates {(10, 5)};

                    \end{axis}
                \end{tikzpicture}
            \end{center}
        If a sequence is convergent it will eventually be contained between the two dashed lines.
    \end{example}

    \begin{example}
        \phantom{a}
        \begin{enumerate}[label = (\arabic*)]
            \item Prove $\left(\frac{1}{n}\right)_n \rightarrow 0$.
                \begin{solution}
                    Let $\epsilon > 0$. Find $N_\epsilon \in \bfN$ so that $\frac{1}{N_\epsilon} < \epsilon$. If $n \geq N_\epsilon$, then $\frac{1}{n} \leq \frac{1}{N_\epsilon} < \epsilon$. Hence $\frac{1}{n} = \left|\frac{1}{n} - 0\right| < \epsilon$.
                \end{solution}
            
            \item Prove $\left(\frac{5n-1}{3-n}\right)_{n=4}^\infty \rightarrow -5$.
                \begin{solution}
                    Note that:
                        \begin{equation*}
                        \begin{split}
                            |x_n - x| = \left|\frac{5n-1}{3-n}  + 5\right| = \frac{14}{|3-n|} = \frac{14}{n-3}.
                        \end{split}
                        \end{equation*}
                    Let $\epsilon > 0$. Find $N_\epsilon \in \bfN$ such that $N_\epsilon > \frac{14}{\epsilon} = 3$. If $n \geq N_\epsilon$, then $n > \frac{14}{\epsilon} + 3$ gives:
                        \begin{equation*}
                        \begin{split}
                            n-3 > \frac{14}{\epsilon} \implies \frac{14}{n-3} < \epsilon \implies |x_n - x| < \epsilon.
                        \end{split}
                        \end{equation*}
                \end{solution}
        \end{enumerate}
    \end{example}

    \begin{proposition}
        Let $(X,d)$ be a metric space. Then $(x_n)_n \rightarrow x$ if and only if $(d(x_n,x))_n \rightarrow 0$.
    \end{proposition}
        \begin{proof}
            Suppose $(x_n)_n \rightarrow x$. Given $\epsilon > 0$, there exists $N \in \bfN$ such that $n \geq N$ implies $d(x_n,x) < \epsilon$. This is equivalent to $|d(x_n,x) - 0| < \epsilon$. The converse follows identically.
        \end{proof}

    \begin{theorem}
        Let $(\epsilon_n)_n \rightarrow 0$ and $(x_n)_n$ be real sequences and $x \in \bfR$. If for some $c>0$ and $N \in \bfN$ we have:
            \begin{equation*}
            \begin{split}
                |x_n - x| \leq c|\epsilon_n| \hspace{6pt}\mtext{for all $n \in \bfN$ such that $n \geq N$,}
            \end{split}
            \end{equation*}
        then $(x_n)_n \rightarrow x$.
    \end{theorem}
        {\color{red} \begin{proof}
            Let $\epsilon > 0$ be given. Since $(\epsilon_n)_n \rightarrow 0$ it follows there exists a natural number $K$ such that if $n \geq K$ then
                \begin{equation*}
                \begin{split}
                    |a_n| = |a_n - 0| < \frac{\epsilon}{c}.
                \end{split}
                \end{equation*}
            If both $n \geq K$ and $n\geq N$, then
                \begin{equation*}
                \begin{split}
                    |x_n - x| \leq c|\epsilon_n| < \epsilon.
                \end{split}
                \end{equation*}
            Thus $(x_n)_n \rightarrow x$.
        \end{proof}}
    
    \begin{example}
        \phantom{a}
        \begin{enumerate}[label = (\arabic*)]
            \item Prove $\left(\frac{\sin(n^2 - 1)}{n^2 + 3}\right)_n \rightarrow 0$.
                \begin{solution}
                    Note that:
                        \begin{equation*}
                        \begin{split}
                            \left|\frac{\sin(n^2 - 1)}{n^2 + 3} - 0 \right|
                            = \frac{|\sin(n^2 - 1)|}{n^2 + 3} 
                            \leq \frac{1}{n^2 + 3} 
                            \leq \frac{1}{n^2} 
                            \leq \frac{1}{n}.
                        \end{split}
                        \end{equation*}
                    Since $\left(\frac{1}{n}\right)_n \rightarrow 0$, we have $\left(\frac{\sin(n^2 - 1)}{n^2 + 3}\right)_n \rightarrow 0$.
                    \end{solution}
            
            \item Prove $\left(\frac{1}{2^n}\right)_n \rightarrow 0$.
                \begin{solution}
                    Note that:
                        \begin{equation*}
                        \begin{split}
                            \left|\frac{1}{2^n} - 0\right| \leq \frac{1}{n}.
                        \end{split}
                        \end{equation*}
                    Since $\left(\frac{1}{n}\right)_n \rightarrow 0$, we have $\left(\frac{1}{2^n}\right)_n \rightarrow 0$.
                \end{solution}

            \item Prove $\left(\frac{1}{n} - \frac{1}{n+1}\right)_n \rightarrow 0$.
                \begin{solution}
                    Note that:
                        \begin{equation*}
                        \begin{split}
                            \left|\frac{1}{n} - \frac{1}{n+1} - 0\right| \leq \frac{1}{n}.
                        \end{split}
                        \end{equation*}
                        Since $\left(\frac{1}{n}\right)_n \rightarrow 0$, we have $\left(\frac{1}{n} - \frac{1}{n+1}\right)_n \rightarrow 0$.
                \end{solution}
        \end{enumerate}
    \end{example}

    \begin{proposition}
        Let $k\geq 1$ be fixed. Given a sequence $(x_n)_n$ in a metric space $(X,d)$, $(x_n)_n \rightarrow x$ if and only if $(x_{k+n})_n \rightarrow x$.
    \end{proposition}
        \begin{proof}
            $(\Rightarrow)$ Suppose $(x_n)_n \rightarrow x$. Let $\epsilon > 0$. We know there exists $N_\epsilon \in \bfN$ with $n\geq N_\epsilon$ implying $d(x_n,x) < \epsilon$. But if $n \geq N_\epsilon$, then $n+k \geq N_\epsilon$. Hence $d(x_{n+k},x) < \epsilon$.

            $(\Leftarrow)$ Conversely, assume that $(x_{n+k}) \rightarrow 0$. Let $\epsilon > 0$. We know there exists $N_\epsilon \in \bfN$ such that $n \geq N_\epsilon$ implies $d(x_{n+k},x) < \epsilon$. Consider $M = N_\epsilon + k$. Then if $n \geq M$, we have $n \geq N_\epsilon + k$, or equivalently $n- k \geq N_\epsilon$. Hence $d(x_{(n-k)+k},x) = d(x_n,x) < \epsilon$.
        \end{proof}

    \begin{proposition}
        If $(x_n)_n$ is a real sequence with $\left(\left|\frac{x_{n+1}}{x_n}\right|\right)\rightarrow L < 1$, then $(x_n)_n \rightarrow 0$.
    \end{proposition}
        \begin{proof}
            Since $L<1$, let $\rho$ be an number satisfying $L < \rho < 1$. Pick $\epsilon = \rho - L$ Since $\left(\left|\frac{x_{n+1}}{x_n}\right|\right)\rightarrow L$, there exists $N_\epsilon \in \bfN$ such that $n\geq N_\epsilon$ implies $\left|\frac{x_{n+1}}{x_n}\right| \in V_\epsilon(L)$, or equivalently $L - \epsilon < \frac{|x_{n+1}|}{|x_n|} < L + \epsilon$. Then $\frac{|x_{n+1}|}{|x_n|} < \rho$, which gives $|x_{n+1}| < \rho |x_n|$. Observe that:
                \begin{equation*}
                \begin{split}
                    |x_{N+1}| &< \rho|x_N|\\
                    |x_{N+2}| &< \rho|x_{N+1}| = \rho^2|x_{N}| \\
                    |x_{N+3}| &< \rho|x_{N+2}| = \rho^3|x_{N}| \\
                    &\vdots \\
                    \mtext{Inductively,}\hspace{6pt}|x_{N+n}| &= \rho^n|x_N|.
                \end{split}
                \end{equation*}
            Since $(\rho^n)_n \rightarrow 0$ (and taking $c = |x_N|$), we have that $(x_{N+n})_n \rightarrow 0$. Thus $(x_n)_n \rightarrow 0$.
        \end{proof}

    \begin{remark}
        Consider $(n)_n \rightarrow +\infty$. Then $\left(\frac{n+1}{n}\right)_n \rightarrow 1$. Now consider $\left(\frac{1}{n}\right)_n \rightarrow 0$. Then $\left(\frac{n}{n+1}\right) \rightarrow 1$. We gain no information if $L = 1$.
    \end{remark}

    \begin{example}\label{example:sequence-dne}
        \phantom{a}
        \begin{enumerate}[label = (\arabic*)]
            \item Prove $((-1)^n)_n$ does not exist.
                \begin{solution}
                    Suppose  $((-1)^n)_n \rightarrow x$. We want to find some $\epsilon_0 > 0$ such that for all $N \in \bfN$, we can find an $n \in \bfN$ satisfying:
                        \begin{equation*}
                        \begin{split}
                            n \geq N \mtext{and} |x_n - x| = |(-1)^n - x| \geq \epsilon_0.
                        \end{split}
                        \end{equation*}
                \end{solution}
                Pick $\epsilon_0 = \max\left\{|x-1|,|x+1|\right\}$. Let $N \in \bfN$. Set $n = 2N$. This gives:
                    \begin{equation*}
                    \begin{split}
                        (-1)^{2N} &= 1 \\
                        (-1)^{2N+1} &= -1
                    \end{split}
                    \end{equation*}
                So we have $n \geq N$ and: 
                    \begin{equation*}
                    \begin{split}
                        |(-1)^{2N} - x| = |1 - x| \geq \epsilon_0 \hspace{4pt}\mtext{or}\hspace{4pt} |(-1)^{2N+1} - x| = |1+x| \geq \epsilon_0.
                    \end{split}
                    \end{equation*}

            \item Prove $(\sin(n))_n$ does not exist.
                {\color{red} \begin{solution}
                    
                \end{solution}}
        \end{enumerate}
    \end{example}

    \begin{proposition}
        Let $(X,d)$ be a metric space. A sequence $(x_n)_n$ can have at most one limit.
    \end{proposition}
        \begin{proof}
            Suppose $(x_n)_n \rightarrow L_1$ and $(x_n)_n \rightarrow L_2$. Set $\epsilon = \frac{|L_1 - L_2|}{2}$. Then $V_\epsilon(L_1) \cap V_\epsilon(L_2) = \emptyset$. Since $(x_n)_n \rightarrow L_1$, there exists $N_1 \in \bfN$ such that $n \geq N_1$ implies $x_n \in V_\epsilon(L_1)$. Likewise, since $(x_n)_n \rightarrow L_2$, there exists $N_2 \in \bfN$ such that $n \geq N_2$ implies $x_n \in V_\epsilon(L_2)$. Pick $N = \max\{N_1,N_2\}$. Then $x_N \in V_\epsilon(L_1)\cap V_\epsilon(L_2)$, which is a contradiction. 
        \end{proof}

    \begin{lemma}\label{lemma:blarf}
        If $(x_n)_n \rightarrow x$, then $(|x_n|)_n \rightarrow |x|$.
    \end{lemma}
        \begin{proof}
            Since $(x_n)_n \rightarrow x$, then there exists $N \in \bfN$ such that $n\geq \bfN$ implies $|x_n - x | < \epsilon$. The triangle inequality gives:
                \begin{equation*}
                \begin{split}
                    ||x_n| - |x|| \leq |x_n - x| < \epsilon,
                \end{split}
                \end{equation*}
            hence $(|x_n|)_n \rightarrow |x|$. Note that the converse does not hold in general, as:
                \begin{equation*}
                \begin{split}
                    (|(-1)^n|)_n \rightarrow 1 \mtext{while} ((-1)^n)_n \mtext{does not exist.} \qedhere
                \end{split}
                \end{equation*}
        \end{proof}

    \begin{lemma}
        Let $(t_n)_n$ be a sequence in $(X,d)$. $(t_n)_n \rightarrow 0$ if and only if $(|t_n|)_n \rightarrow 0$.
    \end{lemma}
        {\color{red} \begin{proof}
            $(\Rightarrow)$ The forward direction follows from Lemma~\ref{lemma:blarf}.
            $(\Leftarrow)$ Suppose $(|t_n|)_n \rightarrow 0$. We have that:
                \begin{equation*}
                \begin{split}
                    ||t_n| - 0| \leq 
                \end{split}
                \end{equation*}
        \end{proof}}

    \begin{lemma}
        If $(x_n)_n \rightarrow x \in \bfR$ with $x_n \geq 0$, then $(\sqrt{x_n})_n \rightarrow \sqrt{x}$.
    \end{lemma}
        \begin{proof}
            Case 1: $x = 0$. Let $\epsilon>0$ be given. Since $(x_n)_n \rightarrow 0$, there exists $N \in \bfN$ such that $n \geq N$ implies $0 \leq x_n = |x_n -0| < \epsilon^2$. Hence $0 \leq \sqrt{x_n} < \epsilon$. Since $\epsilon > 0$, was arbitrary, $(\sqrt{x_n})_n \rightarrow 0$.

            Case 2: $x > 0$. Then $\sqrt{x} > 0$, and:
                \begin{equation*}
                \begin{split}
                    |\sqrt{x_n} - \sqrt{x}| = \left|(\sqrt{x_n}-\sqrt{x})\frac{\sqrt{x_n}+\sqrt{x}}{\sqrt{x_n}+\sqrt{x}}\right| = \left|\frac{x_n - x}{\sqrt{x_n} + \sqrt{x}}\right| \leq \left(\frac{1}{\sqrt{x}}\right)|x_n - x|.
                \end{split}
                \end{equation*}
            Hence the convergence $(\sqrt{x_n})_n$ is a consequence of $(x_n)_n \rightarrow x$.
        \end{proof}

    \begin{example}
        \phantom{a}
        \begin{enumerate}[label = (\arabic*)]
            \item Prove $(\sqrt{n})_n \rightarrow +\infty$.
                \begin{solution}
                    Let $M > 0$ be given. Find $N_M$ so that $N_M = \lceil M^2 \rceil$. Hence $N_M \geq M^2$. Then $n \geq M$ implies $n \geq M^2$, or equivalently $\sqrt{n} \geq M$.
                \end{solution}
            
            \item Prove $(n -\sqrt{n})_n \rightarrow +\infty$.
                \begin{solution}
                    Write $(n -\sqrt{n})_n = (n)_n\stackrel{\rightarrow 1}{\cancel{(1-\frac{1}{\sqrt{n}})_n}} = (n)_n$. Since $(n)_n$ trivially converges to $+\infty$, we have $(n -\sqrt{n})_n \rightarrow +\infty$.
                \end{solution}
            
            \item Prove:
                \begin{equation*}
                \begin{split}
                    (b^n)_{n=0}^\infty \rightarrow \begin{cases} 0, & |b| < 1 \\ 1,& b = 1 \\ +\infty, & b > 1 \\ \text{D.N.E.},& b \leq -1\end{cases}
                \end{split}
                \end{equation*}

                \begin{solution}
                    Cases $b=0$ and $b=1$ are trivial. We showed case $b = -1$ in Example~\ref{example:sequence-dne}.

                    Case 1: $0 < b < 1$. Then $b < 1$ implies $\frac{1}{b} > 1$. We have $\frac{1}{b} = 1 + a$ for some $a>0$, now observe that:
                        \begin{equation*}
                        \begin{split}
                            \left(\frac{1}{b}\right)^n = (1+a)^n \geq 1 + na.
                        \end{split}
                        \end{equation*}
                    This gives:
                        \begin{equation*}
                        \begin{split}
                            |b^n - 0| \leq \frac{1}{1+na} \leq \frac{1}{na} = \left(\frac{1}{a}\right) \frac{1}{n}.
                        \end{split}
                        \end{equation*}
                    Since $\left(\frac{1}{n}\right)_n \rightarrow 0$, we have $(b^n)_n \rightarrow 0$.

                    Case 2: $-1 < b < 0$. Since $(|b^n|)_n = (|b|^n)_n$, case 1 gives $(b^n)_n \rightarrow 0$ when $-1<b<0$.

                    Case 3: $b >1$. Then $b = 1+a$ for some $a > 0$. We have:
                        \begin{equation*}
                        \begin{split}
                            b^n = (1+a)^n \geq 1+na \geq na.
                        \end{split}
                        \end{equation*}
                    Let $M>0$ be given. Pick $N_M = \frac{\lceil M \rceil}{a}$. Then $N_M \geq \frac{M}{a}$. If $n \geq N_M$, then $n \geq \frac{M}{a}$, which simplifies to $na \geq M$. Hence $b^n \geq na \geq M$ gives $(b^n)_n \rightarrow +\infty$.

                    Case 4: $b < 1$. We prove that $(b^n)_n$ does not exist by contradiction. Suppose $(b_n)_n \rightarrow L$ for some $L \in \bfR$. Then $(|b_n|)_n \rightarrow |L|$. But this is a contradiction via the $b>1$ case. Now if $(b^n)_n \rightarrow +\infty$, there exists $N_1 \in \bfN$ such that $n\geq N_1$ implies $b^n \geq 1$. But for $n$ odd, $b^n <0$, which is a contradiction. Assuming $(b^n)_n \rightarrow -\infty$ leads to a similar contradiction, establishing the proof.
                \end{solution}
        \end{enumerate}
    \end{example}

    \begin{example}
        \phantom{a}
        \begin{enumerate}[label = (\arabic*)]
            \item Prove if $c > 0$, $(c^\frac{1}{n})_n \rightarrow 1$.
                \begin{solution}
                    If $c=1$, then clearly $(1^\frac{1}{n})_n \rightarrow 1$. Suppose $c > 1$, then $c^\frac{1}{n} > 1$. Write $c^\frac{1}{n} = 1 + a_n$, where $a_n > 0$ for all $n \in \bfN$. We have:
                        \begin{equation*}
                        \begin{split}
                            c = (c^\frac{1}{n})^n = (1+a_n)^n \geq 1 + n a_n \geq na_n.
                        \end{split}
                        \end{equation*}
                    So $0 < na_n \leq c$, giving $a_n \leq \frac{c}{n}$. We have:
                        \begin{equation*}
                        \begin{split}
                            |c^\frac{1}{n} - 1| = a_n \leq \frac{c}{n}.
                        \end{split}
                        \end{equation*}
                    Since $\left(\frac{1}{n}\right)_n \rightarrow 0$, $(c^\frac{1}{n})_n \rightarrow 1$. Now suppose $0<c<1$, then $c^\frac{1}{n} < 1$. Write $c^\frac{1}{n} = 1 + (-a_n)$ with $-1 < -a_n < 0$ for all $n$. Then:
                        \begin{equation*}
                        \begin{split}
                            c = (c^\frac{1}{n})^n = (1 + (-a_n))^n \geq 1 + n(-a_n) \geq n (-a_n).
                        \end{split}
                        \end{equation*}
                    So $n(-a_n) \leq c$, giving $-a_n \leq \frac{c}{n}$. We have:
                        \begin{equation*}
                        \begin{split}
                            |c^\frac{1}{n} - 1| = -a_n \leq \frac{c}{n}.
                        \end{split}
                        \end{equation*}
                    Since $\left(\frac{1}{n}\right)_n \rightarrow 0$, $(c^\frac{1}{n})_n \rightarrow 1$.
                \end{solution}

            \item Prove $(n^\frac{1}{n})_n \rightarrow 1$.
                \begin{proof}
                    Note that $n^\frac{1}{n} > 1$ for all $n> 1$. Write $n^\frac{1}{n} = 1 + a_n$. Then:
                        \begin{equation*}
                        \begin{split}
                            n = (1+a_n)^n = \sum_{k=0}^n {n \choose k}a_n^k \geq {n \choose 0} + {n \choose 2}a_n^2 = 1 + \frac{n(n-1)}{2}a_n^2.
                        \end{split}
                        \end{equation*}
                    We have:
                        \begin{equation*}
                        \begin{split}
                            n-1 \geq \frac{n(n-1)}{2}a_n^2,
                        \end{split}
                        \end{equation*}
                    which simplifies to:
                        \begin{equation*}
                        \begin{split}
                            \frac{2}{n} \geq a_n^2.
                        \end{split}
                        \end{equation*}
                    Hence $a_n \leq \sqrt{2} \hspace{3pt} \displaystyle{\frac{1}{n}}$, thus by our lemma $(a_n)_n^\infty \rightarrow 0$. Therefore:
                        \begin{equation*}
                        \begin{split}
                            |n^\frac{1}{n} - 1| = d_n,
                        \end{split}
                        \end{equation*}
                    establishing that $(n^\frac{1}{n})_n \rightarrow 1$.
                \end{proof}
        \end{enumerate}
    \end{example}

    \begin{proposition}\label{prop:conv-seq-bounded}
        A convergent sequence is bounded.
    \end{proposition}
        \begin{proof}
            Suppose $(x_n)_n \rightarrow x$. Since $(x_n)_n$ is convergent, we know for all $\epsilon > 0$ that $|x_n - x| < \epsilon$. Pick $\epsilon = 1$. Eventually the entire sequence will be contained in $V_1(x)$. More formally, there exists $N_1 \in \bfN$ such that $n \geq N_1$ implies $x_n \in V_1(x)$. Define:
                \begin{equation*}
                \begin{split}
                    c = \max \left\{|x_1|,|x_2|,...,|x_N|,|x-1|,|x+1|\right\}.
                \end{split}
                \end{equation*}
            If $n \leq N$, then $|x_n| \leq c$. If $n \geq N_1$, then $x-1 < x_n < x+1$; i.e., $|x_n| \leq c$.
        \end{proof}

        \begin{theorem}
            Let $x_n$, $y_n$, $z_n$ be convergent sequences with $(x_n)_n \rightarrow x$, $(y_n)_n \rightarrow y$, and $(z_n)_n \rightarrow z$ and $t \in \bfR$. Moreover, let $z_n \neq 0$ for all $n$ and $z \neq 0$. We have:
                \begin{enumerate}[label = (\arabic*)]
                    \item $(x_n \pm y_n)_n \rightarrow x \pm y$. 
                    \item $(tx_n)_n \rightarrow tx$.
                    \item $(x_n y_n)_n \rightarrow xy$.
                    \item $\left(\frac{1}{z_n}\right)_n \rightarrow \frac{1}{z}$.
                    \item $\left(\frac{x_n}{z_n}\right)_n \rightarrow \frac{x}{z}$.
                \end{enumerate}
        \end{theorem}
            \begin{proof}
                (3) We have:
                    \begin{equation*}
                    \begin{split}
                        |x_n y_n - xy|
                        & = |x_n y_n - xy_n + x y_n - xy| \\
                        & = \left|(x_n - x)y_n + x(y_n - y)\right|\\
                        & \leq |(x_n - x)y_n |+| x(y_n - y)| \\
                        & = |x_n - x||y_n| + |x||y_n - y|.
                    \end{split}
                    \end{equation*}
                Since $y_n$ is convergent, it is bounded. So there exists a $c > 0$ with $|y_n| \leq c$ for all $n \geq 1$. Hence:
                    \begin{equation*}
                    \begin{split}
                        |x_n - x||y_n| + |x||y_n - y|
                        & \leq \stackrel{\rightarrow 0}{\cancel{|x_n -x|}}c + |x|\stackrel{\rightarrow 0}{\cancel{|y_n - y|}.}
                    \end{split}
                    \end{equation*}
                Thus $(|x_n y_n - xy|)_n \rightarrow 0$, which implies $(x_n y_n)_n \rightarrow xy$.
    
                (4) We have:
                    \begin{equation*}
                    \begin{split}
                        \left|\frac{1}{z_n} - \frac{1}{z}\right| = \frac{|z- z_n|}{|z||z_n|}.
                    \end{split}
                    \end{equation*}
    
                Since $z \neq 0$, it won't be "near" zero. We have the following picture:
                    \begin{center}
                    \begin{tikzpicture}
                    % Draw the number line
                    \draw[<->] (-3,0) -- (3,0); 
                    
                    % Tick for x
                    \draw (0,0.1) -- (0,-0.1) node[below] {$0$}; 
                    \draw (1,0.1) -- (1,-0.1) node[below] {$z$}; 
                    
                    % Parenthesis ")" at x+1
                    \node at (1.5,0)  {$)$};
                    \node at (0.5,0)  {$($};
                    \end{tikzpicture}
                    \end{center}
                Let $\delta = \frac{|z|}{2} > 0$. There exists $N \in \bfN$ such that $n \geq N$ implies $z_n \in V_\delta(z)$. We have:
                    \begin{equation*}
                    \begin{split}
                        &\phantom{\implies} z - \delta < z_n < z + \delta \\
                        &\implies z - \frac{|z|}{2} < z_n \\
                        & \implies \frac{|z|}{2} < |z_n|.
                    \end{split}
                    \end{equation*}
                Since $|z_n| \geq \frac{|z|}{2}$, we have $\frac{1}{|z_n|} < \frac{2}{|z|}$. So for $n \geq N$, 
                    \begin{equation*}
                    \begin{split}
                        \left|\frac{1}{z_n} - \frac{1}{z}\right| = \frac{|z- z_n|}{|z||z_n|} \leq \frac{2}{|z|^2}|z - z_n|.
                    \end{split}
                    \end{equation*}
                Thus $\left(\frac{1}{z_n}\right)_n \rightarrow \frac{1}{z}$.
            \end{proof}
    
        \begin{theorem}\label{thm:ordering-of-sequences}
            Suppose $(x_n) \rightarrow x$ and $(y_n)_n \rightarrow y$ with $x_n \leq y_n$ for all $n$. Then $x \leq y$.
        \end{theorem}
            \begin{proof}
                We have that $(y_n - x_n)_n \rightarrow y-x$, and $y_n - x_n \geq 0$ for all $n$. Thus $y - x \geq 0$.
            \end{proof}
    
        \begin{corollary}
            If $(x_n)_n \rightarrow x$ and $a \leq x_n \leq b$, then $a \leq x \leq b$.
        \end{corollary}
            \begin{proof}
                Taking $(y_n)_n = (a,a,a,...)$ and $(y_n)_n = (b,b,b,...)$ gives the desired result.
            \end{proof}
    
        \begin{theorem}[Squeeze Theorem]
            Let $(x_n)_n$, $(y_n)_n$, and $(z_n)_n$ be sequences with $(x_n)_n \leq (y_n)_n \leq (z_n)_n$ for all $n \geq 1$. If $\lim x_n = \lim z_n = L$, then $(y_n)_n \rightarrow L$.
        \end{theorem}
            \begin{proof}
                Let $\epsilon > 0$. There exists $N_1 \in \bfN$ such that $n \geq N_1 $ implies $x_n \in V_\epsilon(L)$. Likewise, there exists $N_2 \in \bfN$ such that $n \geq N_2$ implies $z_n \in V_\epsilon(L)$. So for $n \geq \max \left\{N_1,N_2\right\} := N$, both $x_n,z_n \in V_\epsilon(L)$. We have:
                    \begin{equation*}
                    \begin{split}
                        L -\epsilon < x_n \leq y_n < z_n \leq L + \epsilon.
                    \end{split}
                    \end{equation*}
                Thus $y_n \in V_\epsilon(L)$ for $n \geq N$.
            \end{proof}
        
    \begin{theorem}[Monotone Convergence Theorem]\label{thm:monotome-conv-thm}
        Let $(x_n)_n$ be a monotone sequence. $(x_n)_n$ is convergent if and only if $(x_n)_n$ is bounded. Moreover,
            \begin{enumerate}[label = (\alph*)]
                \item If $(x_n)_n$ is increasing and bounded above, $\lim x_n = \sup \left\{x_n \mid n \in \bfN\right\}$.
                \item If $(x_n)_n$ is decreasing and bounded below, $\lim x_n = \inf \left\{x_n \mid n \in \bfN\right\}$.
            \end{enumerate}
    \end{theorem}
        \begin{proof}
            $(\Rightarrow)$ We showed this direction in Proposition~\ref{prop:conv-seq-bounded}. $(\Leftarrow)$ (a) Suppose $(x_n)_n$ is bounded above and increasing. Let $u = \sup\{x_n \mid n \in \bfN\}$. Given $\epsilon > 0$, there exists $N \in \bfN$ such that $u - \epsilon < x_N$. But for $n \geq N$, $u - \epsilon < x_N \leq x_n \leq u < u + \epsilon$. Hence $x_n \in V_\epsilon(u)$, establishing that $(x_n)_n \rightarrow u$.

            (b) Consider $y_n = -x_n$, we get $y_n$ is increasing and bounded above. By $(a)$, we get:
                \begin{equation*}
                \begin{split}
                    \lim y_n = \sup\{y_n \mid n \in \bfN\}
                    & \implies -\lim x_n = \sup\{-x_n \mid n \in \bfN\} \\
                    & \implies -\lim x_n = -\inf\{x_n \mid n \in \bfN\} \\
                    & \implies \lim x_n = \inf\{x_n \mid n \in \bfN\}. \qedhere
                \end{split}
                \end{equation*}
        \end{proof}
    
    \begin{example}
        \phantom{a}
        \begin{enumerate}[label = (\arabic*)]
            \item Consider the recursively defined sequence $x_1 = 8$, $x_{n+1} = \frac{1}{2}x_n + 2$. We will show by induction that it is bounded below by $4$. Clearly $x_1 = 8 \geq 4$. Now assume $x_k \geq 4$. Then:
                \begin{equation*}
                \begin{split}
                    x_{k+1} &= \frac{1}{2}x_n + 2 \\
                    & \geq \frac{1}{2}(4) + 2 \\
                    & = 4.
                \end{split}
                \end{equation*}
            Therefore $(x_n)_n$ is bounded below by 4. Now observe that:
                \begin{equation*}
                \begin{split}
                    x_{n+1} \leq x_n 
                    &\iff \frac{1}{2}x_n + 2 \leq x_n \\
                    &\iff 4 \leq x_n.
                \end{split}
                \end{equation*}
            Hence $(x_n)_n$ is decreasing. By the \nameref{thm:monotome-conv-thm}, $(x_n)_n \rightarrow L$. Now observe that:
                \begin{equation*}
                \begin{split}
                    (x_{n+1})_n = \left(\frac{1}{2}x_n + 2\right)_n 
                    &\iff L = \frac{1}{2}L + 2 \\
                    &\iff L = 4.
                \end{split}
                \end{equation*}

            \item Let $x_n = \sum_{k=1}^n \frac{1}{k^2} = 1 + \frac{1}{4} + \frac{1}{9}$. We will show that this sequence converges. Clearly $x_n \leq x_{n+1}$, so it is increasing. We will use the fact that $k^2 \geq k(k-1)$ as follows:
                \begin{equation*}
                \begin{split}
                    x_n 
                    &= \sum_{k=1}^n \frac{1}{k^2}\\
                    &= 1 + \sum_{k=2}^n \frac{1}{k^2} \\
                    &\leq 1 + \sum_{k=2}^n \frac{1}{k(k-1)} \\
                    & = 1 + \sum_{k=2}^n \left(\frac{1}{k-1} - \frac{1}{k}\right) \\
                    & = 1 + \left[\left(1 - \frac{1}{2}\right) + \left(\frac{1}{2} - \frac{1}{3}\right) + \left(\frac{1}{3}- \frac{1}{4}\right) + ... + \left(\frac{1}{n-1} - \frac{1}{n}\right)\right] \\
                    & = 1 + 1 -\frac{1}{n} \\
                    & = 2 - \frac{1}{n} \\
                    & \leq 2.
                \end{split}
                \end{equation*}
            So $(x_n)_n$ is increasing and bounded above, hence it has a limit.

            \item Given $a > 0$, we will find a sequence $(x_n)_n$ which converges to $\sqrt{a}$. Consider the recursively defined sequence $x_1 = 1$, $x_{n+1} = \frac{1}{2}\left(x_n + \frac{a}{x_n}\right)$. Claim: $x_n^2 \geq a$ for all $n \geq 2$. Note that:
                \begin{equation*}
                \begin{split}
                    2x_{n+1} = x_n + \frac{a}{x_n}
                    & \implies 2x_{n+1}x_n = x_n^2 + a \\
                    & \implies 0 = x_n^2 - 2x_{n+1}x_n = a.
                \end{split}
                \end{equation*}
            This polynomial has a real root, hence $\Delta \geq 0$. We get:
                \begin{equation*}
                \begin{split}
                    \Delta = 4x_{n+1}^2 -4a \geq 0 
                    &\implies x_{n+1} ^2\geq a \\
                \end{split}
                \end{equation*}
            We will now show that $(x_n)_n$ is eventually decreasing. Observe that:
                \begin{equation*}
                \begin{split}
                    x_n \geq x_{n+1}
                    & \iff x_n \geq \frac{1}{2}\left(x_n + \frac{a}{x_n}\right) \\
                    & \iff 2x_n \geq x_n + \frac{a}{x_n} \\
                    & \iff x_n \geq \frac{a}{x_n} \\
                    & \iff x_n^2 \geq a.
                \end{split}
                \end{equation*}
            By the \nameref{thm:monotome-conv-thm}, $(x_n)_n \rightarrow L$. We have:
                \begin{equation*}
                \begin{split}
                    x_{n+1} = \frac{1}{2}\left(x_n + \frac{a}{x_n}\right)
                    & \implies L = \frac{1}{2}\left(L = \frac{a}{L}\right) \\
                    & \implies L^2 = a \\
                    & \implies L = \sqrt{a}.
                \end{split}
                \end{equation*}
        \end{enumerate}        
    \end{example}

    \begin{example}[Euler's Number]
        {\color{red} I will do this later.}
    \end{example}

    \begin{proposition}
        If $(x_n)_n$ is increasing and unbounded, then $(x_n)_n$ diverges properly to $+\infty$. 
    \end{proposition}
        \begin{proof}
            Let $M$ be arbitrarily big. Since $(x_n)_n$ is unbounded, there exists $N \in \bfN$ with $x_n > M$. Hence if $n\geq M$, $x_n \geq x_N > M$ because $(x_n)_n$ is increasing.
        \end{proof}

    \begin{example}
        We will show that $h_n = \sum_{k=1}^n \frac{1}{k}$ diverges properly to $+\infty$. {\color{red} do this later}
    \end{example}

\section{Subsequences}
    \begin{definition}
        A \textui{natural sequence} is a strictly increasing sequence of natural numbers: $(n_k)_{k=1}^\infty$ with $n_k \in \bfN$, $n_1 < n_2 < ...$
    \end{definition}

    \begin{example}
        \phantom{a}
        \begin{enumerate}[label = (\arabic*)]
            \item $(2k+1)_k = (3,5,7,...)$ 
            \item $(k^2)_k = (1,4,9,...)$
        \end{enumerate}
    \end{example}

    \begin{exercise}
        Given a natural sequence $(n_k)_k$, prove $n_k \geq k$.
    \end{exercise}

    \begin{definition}
        Let $(x_n)_n$ be a sequence. A \textui{subsequence} of $(x_n)_n$ is a sequence $(x_{n_k})_{k=1}^\infty$ where $(n_k)_k$ is a natural sequence. Formally, a subsequence is a composition of maps:
            \begin{equation*}
            \begin{split}
                \bfN \underset{k \mapsto n_k}{\rightarrow} \bfN \underset{n_k \mapsto x_{n_k}}{\rightarrow} X.
            \end{split}
            \end{equation*}
    \end{definition}

    \begin{example}
        \phantom{a}
        \begin{enumerate}[label = (\arabic*)]
            \item Consider $(x_n)_n \rightarrow \frac{1}{n}$. Let $n_k = 2_k$. Then $(x_{n_k})_k = \left(\frac{1}{2k}\right)_k = \left(\frac{1}{2},\frac{1}{4},\frac{1}{6},...\right)$.
            \item Conider $(x_n)_n = (-1)^n$ Then $(x_{2k})_k = (1,1,1,...)$ and $(x_{2k+1})_k = (-1,-1,-1,...)$
        \end{enumerate}
    \end{example}

    \begin{proposition}
        Suppose $(x_n)_n \rightarrow x$. For any subsequence $(x_{n_k})_k$, we have $(x_{n_k})_k \rightarrow x$.
    \end{proposition}
        \begin{proof}
            Let $\epsilon > 0$. There exists $N \in \bfN$ such that $n \geq N$ implies $|x_n - x| < \epsilon$. Take $K = N$. Then $k \geq K$ implies $k \geq N$. But by Exercise~\ref{exercise:}, $n_k \geq k \geq N$. Hence $|x_{n_k} - x | < \epsilon$.
        \end{proof}

    \begin{example}
        We give an alternate proof of $(b^n)_n \rightarrow 0$ for $0 < b < 1$. Clearly $b^{n+1} < b^n$ if and only if $b < 1$. So $b^n$ is decreasing and bounded below by 0. By the \nameref{thm:monotome-conv-thm}, $(b^n)_n \rightarrow L$ for some $L$. But we also have that $(b^{2k})_k \rightarrow L$. So we have:
            \begin{equation*}
            \begin{split}
                (b^{2k})_k = (b^k)_k^2
                & \iff L = L^2 \\
                & \iff L(1-L) = 0.
            \end{split}
            \end{equation*}
        Since $L \neq 1$, it must be that $L = 0$.
    \end{example}

    \begin{proposition}
        Let $(x_n)_n$ be a sequence. Then $(x_n)_n \not\rightarrow x$ if and only if there exists an $\epsilon_0 > 0$ and subsequence $(x_{n_k})_k$ such that $d(x_{n_k},x) > \epsilon_0$.
    \end{proposition}
        \begin{proof}
            $(\Leftarrow)$ If $(x_n)_n \rightarrow x$, then any subsequence $(x_{n_k})_k$ converges to $x$.

            $(\Rightarrow)$ Since $(x_n)_n \not\rightarrow x$, we have:
                \begin{equation*}
                \begin{split}
                    (\exists \epsilon_0 > 0)(\forall N \in \bfN)(\exists n \geq N) \ni (x_n \not\in V_{\epsilon_0}(x)).
                \end{split}
                \end{equation*}
            With this $\epsilon_0$, we will construct our subsequence $x_{n_k}$. Note that:
                \begin{equation*}
                \begin{split}
                    N = 1 &\implies (\exists n_1 \geq 1) \ni (x_{n_1} \not\in V_{\epsilon_0}(x)) \\
                    N = n_1 + 1 & \implies (\exists n_2 \geq n_1) \ni (x_{n_2} \not\in V_{\epsilon_0}(x)) \\
                    N = n_2 + 1 & \implies (\exists n_3 \geq n_2) \ni (x_{n_3} \not\in V_{\epsilon_0}(x)) \\
                    &\vdots \\
                    \mtext{Inductively,} N = n_k + 1 & \implies (\exists n_{k+1} \geq n_k) \ni (x_{k+1} \not\in V_{\epsilon_0}(x))
                \end{split}
                \end{equation*}
            Thus $(x_{n_k})_k$ is a subsequence with $x_{n_k} \not\in V_{\epsilon_0}(x)$, so $|x_{n_k} - x | \geq \epsilon_0$ for all $k=1,2,3,...$
        \end{proof}

    \begin{definition}
        If $(x_n)_n$ is a sequence of real numbers, a \textui{peak} of the sequence is a term $x_m$ satisfying $x_m \geq x_n$ for all $n \geq m$.
    \end{definition}

    \begin{proposition}\label{prop:monotone-subsequence}
        Let $(x_n)_n$ be a real sequence. There is a subsequence that is monotone.
    \end{proposition}
        \begin{proof}
            Case 1: There are infinitely many peaks. Let $x_{n_1},x_{n_2},x_{n_3}...$ be an enumeration of peaks. Then $(x_{n_k})_k$ is decreasing by definition.

            Case 2: There are finitely many peaks. Let $x_{m_1},x_{m_2},...,x_{m_r}$ be the peaks of our sequence where $m_1 < m_2 < ... < m_r$. Let $n_1 = m_r + 1$. Since $x_{n_1}$ is not a peak, there exists $n_2 > n_1$ such that $x_{n_2} > x_{n_1}$. Since $x_{n_2}$ is not a peak, there exists $n_3 > n_2$ such that $x_{n_3} > x_{n_2}$. Inductively, we obtain a sequence $(x_{n_k})_k = (x_{n_1},x_{n_2},x_{n_3},...)$ with $x_{n_k} < x_{n_{k+1}}$.
        \end{proof}

    \begin{theorem}[Bolzano-Weierstass Theorem]
        If $(x_n)_n$ is a real sequence that is bounded, it admits a convergent subsequence.
    \end{theorem}
        \begin{proof}
            By Proposition~\ref{prop:monotone-subsequence}, there exists a subsequence $(x_{n_k})$ which is monotone and bounded. By the \nameref{thm:monotome-conv-thm}, $(x_{n_k})_k$ converges.
        \end{proof}

\section{Limit Inferior and Limit Superior}
    \begin{definition}
        Let $X = (x_n)_n$ be a fixed bounded sequence who's limit may not exist. Then
            \begin{equation*}
            \begin{split}
                \overline{X} = \{t \in \bfR \mid t = \lim_{k \rightarrow \infty}x_{n_k}, \hspace{-3pt}\mtext{$x_{n_k}$ some subsequence} \hspace{-5pt}\}
            \end{split}
            \end{equation*}
        is the set containing all \textui{subsequential limits} (or \textui{limit points}) of $X$.
    \end{definition}

    \begin{example}
        Let $X = ((-1)^n)_n$. Then $\overline{X} = \{-1,1\}$.
    \end{example}

    \iffalse
    \begin{example}
        \phantom{a}
        \begin{enumerate}[label = (\arabic*)]
            \item Let $X = ((-1)^n)_n$. Then $\overline{X} = \{-1,1\}$.
            \item Recall that $\bfQ \cap [0,1]$ is countable, hence we can enumarate $(r_n)_{n=1}^\infty = R$. Claim: $\overline{R} = [0,1]$...f.
        \end{enumerate}
    \end{example}
    \fi

    \begin{example}
        Fix a bounded sequence $(x_n)_n$. Let
            \begin{equation*}
            \begin{split}
                u_1 &= \sup_{n \geq 1}(x_n), \\
                l_1 &= \inf_{n \geq 1}(x_n).
            \end{split}
            \end{equation*}
        If a subsequence $(x_{n_k})_k \rightarrow x$, we know $x \in [l_1,u_1]$ because $l_1 \leq x \leq u_1$. Hence $l_1 \leq x_{n_k} \leq u_1$. Now let
            \begin{equation*}
            \begin{split}
                u_2 &= \sup_{n \geq 2}(x_n), \\
                l_2 &= \inf_{n \geq 1}(x_n).
            \end{split}
            \end{equation*}
        We have $u_2 \leq u_1$ (we know $u_1$ is an upper bound for all $n \geq 2$, hence $u_2$ must be the least upper bound) and $l_1 \leq l_2$. Similarly, if $(x_{n_k})_k \rightarrow x$ for some subsequence, then $x \in [l_2, u_2]$ because $l_2 \leq x_{n_k} \leq u_2$ for $k$ large enough. Inductively:
            \begin{equation*}
            \begin{split}
                u_m = \sup_{n \geq m} x_n, \\
                l_m = \inf_{n \geq m } x_n.
            \end{split}
            \end{equation*}
        We get:
            \begin{equation*}
            \begin{split}
                l_1 \leq l_2 \leq ... \leq l_m \leq u_m \leq ... \leq u_2 \leq u_1 .
            \end{split}
            \end{equation*}
        This holds for all $m \geq 1$. Let $I_m = [l_m, u_m]$. Then $(I_m)_m$ is a sequence of closed and bounded nested intervals. So
            \begin{equation*}
            \begin{split}
                \bigcap_{m\geq 1} I_m = [l,u]
            \end{split}
            \end{equation*}
        where
            \begin{equation*}
            \begin{split}
                l &= \sup_{m \geq 1}l_m = \sup_{m \geq 1}\left(\inf_{n \geq m}x_n\right), \\
                u &= \inf_{m \geq 1}u_m  = \inf_{m \geq 1}\left(\sup_{n \geq m}x_n\right). 
            \end{split}
            \end{equation*}
        Note that:
            \begin{equation*}
            \begin{split}
                \sup_{m \geq 1}l_m = \lim_{m \rightarrow \infty} l_m \\
                \inf_{m \geq 1}u_m = \lim_{m \rightarrow \infty} u_m.
            \end{split}
            \end{equation*}
        This follows from the \nameref{thm:monotome-conv-thm}, as $(l_m)_m$ is an increasing sequence bounded above and $(u_m)_m$ is a decreasing sequence bounded below.
    \end{example}

    \begin{definition}
        Let $(x_n)_n$ be a bounded sequence.
            \begin{enumerate}[label = (\arabic*)]
                \item $\displaystyle l = \lim_{m \rightarrow \infty} l_m = \lim_{m \rightarrow \infty}\left(\inf_{n \geq m}x_n\right) := \liminf x_n$.
                \item $\displaystyle u = \lim_{m \rightarrow \infty} u_m = \lim_{m \rightarrow \infty}\left(\sup_{n \geq m}x_n\right) := \limsup x_n$.
            \end{enumerate}
    \end{definition}

    \begin{proposition}
        Let $X = (x_n)_n$ be a bounded sequence with $l = \liminf x_n$ and $u = \limsup x_n$. If $x \in X$, then $x \in [l,u]$. We have:
            \begin{equation*}
            \begin{split}
                l_{n_k} = \inf_{n \geq n_k}x_n \leq x_{n_k}.
            \end{split}
            \end{equation*}
        Taking the limit as $k\rightarrow \infty$ yields $l \leq x$. Similarly, we have:
            \begin{equation*}
            \begin{split}
                u_{n_k} = \sup_{n \geq n_k}x_n \geq x_{n_k}.
            \end{split}
            \end{equation*}
        Taking the limit as $k \rightarrow \infty$ yields $x \leq u$. Thus $x \in [l,u]$.
    \end{proposition}

    \begin{question}
        Does $\overline{X} = [l,u]$?
    \end{question}
        \begin{answer}
            No. Take for example $x_n = (-1)^n$. Then $u = 1$ and $l = -1$ But $\overline{X} = \{-1,1\} \subset [-1,1]$.
        \end{answer}

    \begin{proposition}
        Let $(x_n)_n=X$ be a bounded sequence with $u_m = \sup_{n \geq m} x_n$. We have a strictly decreasing sequence $u_1 \geq u_2 \geq ...$ There exists a subsequence $(x_{n_k})_k \rightarrow u$. There exists a subsequence $(x_{n_k})_k \rightarrow l$. Equivalently, $u,l \in \overline{X}$.
    \end{proposition}
        \begin{proof}
            Recall that $u_m = \sup_{n \geq m}x_n$. By the supremum property:
                \begin{equation*}
                \begin{split}
                    \exists n_1 \in \bfN \mtext{with} &u_1-1 < x_{n_1} \leq u_1, \\
                    \exists n_2 \in \bfN \mtext{with} n_2 > n_1 + 1 > n_1 \mtext{and} &u_{n_1 + 1} - \frac{1}{2} < x_{n_2} \leq u_{n_1 + 1}, \\
                    \exists n_3 \in \bfN \mtext{with} n_3 \geq n_2 + 1 > n_2 \mtext{and}& u_{n_2 +1} - \frac{1}{3} < x_{n_3} \leq u_{n_2 + 1}.
                \end{split}
                \end{equation*}
            Inductively:
                \begin{equation*}
                \begin{split}
                    u_{n_{k-1} + 1} - \frac{1}{k} < x_{n_k} \leq u_{n_{k-1} + 1}.
                \end{split}
                \end{equation*}
            Let $m_k = n_{k-1} + 1$. We can rewrite the above equation as:
                \begin{equation*}
                \begin{split}
                    u_{m_k} - \frac{1}{k} < x_{n_k} \leq u_{m_k}.
                \end{split}
                \end{equation*}
            Letting $k\rightarrow \infty$ gives:
                \begin{equation*}
                    \lim_{k \rightarrow \infty} u_{m_k} - \frac{1}{k} < \lim_{k \rightarrow \infty} x_{n_k} \leq \lim_{k \rightarrow \infty}u_{m_k}
                    \vspace{3pt}
                \end{equation*}
                \begin{equation*}
                    \iff
                \end{equation*}
                \begin{equation*}
                    u < \lim_{k \rightarrow \infty} x_{n_k} \leq u.
                \end{equation*}
            By the squeeze theorem, $(x_{n_k})_k \rightarrow u$.
        \end{proof}

    \begin{proposition}
        Let $(x_n)_n$ be bounded.
            \begin{enumerate}[label = (\arabic*)]
                \item $\liminf x_n \leq \limsup x_n$.
                \item $(x_n)_n \rightarrow x$\hspace{2pt} if and only if \hspace{2pt}$\liminf x_n = \limsup x_n = x$.
            \end{enumerate}
    \end{proposition}
        \begin{proof}
            We have $l_m \leq u_m$ for all $m \geq 1$ by the previous motivating example. Letting $m \rightarrow \infty$ gives $l \leq u$.

            {\color{red} I don't know the second part}
        \end{proof}

\section{Cauchy Sequences}
    \begin{definition}
        A sequence $(x_n)_n$ is \textui{Cauchy} if:
            \begin{equation*}
            \begin{split}
                (\forall \epsilon > 0)(\exists N \in \bfN) \ni m,n \geq N \implies d(x_n,x_n) < \epsilon.
            \end{split}
            \end{equation*}
    \end{definition}

    \begin{example}
        Prove $\left(\frac{1}{n}\right)_n$ is Cauchy.
    \end{example}
        \begin{solution}
            For $n > m$:
                \begin{equation*}
                \begin{split}
                    |x_m - x_n| 
                    = \frac{1}{m} -\frac{1}{n} 
                    = \frac{n-m}{mn} 
                    < \frac{n}{nm} 
                    = \frac{1}{m}.
                \end{split}
                \end{equation*}
            We can start the proof. Given $\epsilon > 0$, by Archimedean property $2$ there exists $N$ large satisfying $\frac{1}{N} < \epsilon$. For $n > m \geq N$, we have $|x_m - x_n| < \frac{1}{m} \leq \frac{1}{N} < \epsilon$.
        \end{solution}

    \begin{proposition}
        Cauchy sequences are bounded.
    \end{proposition}
        \begin{proof}
            Pick $\epsilon = 1$. There exists $N \in  \bfN$ such that $m,n \geq N$ implies $|x_n - x_m| < 1$. Let $c = \max \{|x_1|,...,|x_N|\}$. For $n \geq N$, we have $|x_n| = |x_n - x_N + x_N| \leq |x_n - x_N| + |x_N| \leq 1+|x_N|$. So $|x_n| \leq c'$, where $c' = \max\{c, 1 + |x_N|\}$.
        \end{proof}

    \begin{exercise}\label{exercise:homework3}
        If $(x_n)_n$ is Cauchy and there exists a subsequence $(x_{n_k})_k$ with $(x_{n_k})_k \rightarrow x$, then $(x_n)_n \rightarrow x$.
    \end{exercise}

    \begin{theorem}
        Let $(x_n)_n$ be a sequence. $(x_n)_n$ is Cauchy if and only if $(x_n)_n$ converges.
    \end{theorem}
        \begin{proof}
            $(\Rightarrow)$ Suppose $(x_n)_n \rightarrow x$. Let $\epsilon > 0$. There exists $N \in \bfN$ such that $n \geq N$ implies $|x_n - x| < \frac{\epsilon}{2}$. For $m,n \geq N$, we have $|x_n - x_m| = |x_n - x + x_m - x| \leq |x_n - x| + |x_m - x| < \epsilon$.

            $(\Leftarrow)$ If $(x_n)_n$ is Cauchy then $(x_n)_n$ is bounded. The Bolzano-Weierstass theorem says gives there exists some convergent subsequence $(x_{n_k})_k$. By Exercise~\ref{exercise:homework3}, $(x_n)_n \rightarrow x$.
        \end{proof}

    \begin{example}
        We will show that $x_n = \sum_{k = 0}^n \frac{(-1)^k}{k!}$ is a convergent sequence. This sequence is not monotone, so we are unable to use the monotone convergence theorem. Instead, we will show that this sequence is Cauchy, which implies that it is convergent. For $m > n$, observe that:
            \begin{equation*}
            \begin{split}
                |x_m - x_n|
                & = \left|\sum_{k = n + 1}^m \frac{(-1)^k}{k!}\right| \\
                & \leq \sum_{k = n + 1}^m \left|\frac{(-1)^k}{k!}\right| \\
                & = \sum_{k = n+1}^m \frac{1}{k!} \\
                & \leq \sum_{k = n+1}^m \frac{1}{2^{k-1}} \\
                & = \frac{1}{2^n} + \frac{1}{2^{n+1}} + ... + \frac{1}{2^{m-1}} \\
                & = \frac{1}{2^n} \left(1 + \frac{1}{2} + ... + \frac{1}{2^{m-n-1}}\right) \\
                & \leq \frac{1}{2^n}\cdot 2 \\
                & = \frac{1}{2^{n-1}}.
            \end{split}
            \end{equation*}
        If $\epsilon > 0$ is given, choose $N$ large so that $\frac{1}{2^{N - 1}} < \epsilon$. If $m > n > N$, then $|x_m - x_n| < \epsilon$. Thus $x_n$ is Cauchy, implying that it is convergent.
    \end{example}

    \begin{definition}
        A sequence $(x_n)_n$ is \textui{contractive} if there exists $0 < \rho < 1$ with $|x_{n+1} - x_n| \leq \rho |x_n - x_{n-1}|$ for all $n \geq 2$. We say $\rho$ is the \textui{constant of contraction}.
    \end{definition}

    \begin{proposition}
        Contractive sequences are Cauchy.
    \end{proposition}
        \begin{proof}
            Observe that:
                \begin{equation*}
                \begin{split}
                    |x_3 - x_2| &\leq \rho |x_2 - x_1| \\
                    |x_4 - x_3| &\leq \rho |x_2 - x_1|  \leq \rho^2 |x_2 - x_1|\\
                \end{split}
                \end{equation*}
            Inductively, $|x_{n+1} - x_n| \leq \rho^{n-1}|x_2 - x_1|$. For $m > n$, we have:
                \begin{equation*}
                \begin{split}
                    |x_m - x_n|
                    & = 
                \end{split}
                \end{equation*}
                {\color{red} finish later}
        \end{proof}

    \begin{example}
        {\color{red} fibonacci finish later}
    \end{example}

\section{Sequences of Functions}
    \begin{definition}
        Let $\Omega$ be a nonempty set. The set of all functions $\Omega\xrightarrow{f} X$ is denoted $\cF(\Omega,X) = \{f \mid f:\Omega \rightarrow X\}$.
    \end{definition}

    \begin{exercise}
        Show that $\cF(\Omega,X)$ is an algebra under pointwise operations.
    \end{exercise}

    \begin{definition}
        A sequence in $(f_n)_n : \bfN \rightarrow \cF(\Omega,\bfR)$n converges \textui{pointwise} to $f \in \cF(\Omega,\bfR)$ if for all $x \in \Omega$, $(f_n(x))_n \rightarrow f(x)$. In particular:
            \begin{equation*}
            \begin{split}
                (\forall x \in \Omega)(\forall \epsilon > 0)(\exists N_{x,\epsilon} \in \bfN) \ni n \geq N \implies |f_n(x) - f(x)| < \epsilon.
            \end{split}
            \end{equation*}
    \end{definition}

    \begin{example}
        \phantom{a}
        \begin{enumerate}[label = (\arabic*)]
            \item Let $f_n \in \cF([0,1],\bfR)$ be defined by $f_n(x) = x^n$. Given $x \in [0,1)$, we can see that $(f_n(x))_n \rightarrow 0$. If $x = 1$, then $(f_n(x))_n \rightarrow 1$. So let $f: [0,1] \rightarrow \bfR$ be defined as $f = \delta_1$. Then $(f_n)_n \rightarrow f$ converges pointwise.
            \item Let $f_n \in \cF(\bfR,\bfR)$ be defined by $f_n(x) = \frac{nx}{1+n^2x^2}$. Note that:
                \begin{equation*}
                \begin{split}
                    |f_n(x)|
                    & = \left|\frac{nx}{1+n^2x^2}\right| \\
                    & = \frac{n|x|}{1 + n^2 x^2} \\
                    & \leq \frac{|x|}{nx^2} \\
                    & = \frac{1}{n|x|}. 
                \end{split}
                \end{equation*}
            Claim: $(f_n)_n \rightarrow 0_{\cF(\bfR, \bfR)}$. If $x = 0$, then $(f_n(0))_n = (0)_n \rightarrow 0$. If $x \neq 0$, then $(f_n(x))_n \rightarrow 0$ because $|f_n(x) - 0| \leq \frac{1}{|x|} \cdot \frac{1}{n}$. Since $x$ is fixed and $\left(\frac{1}{n}\right)_n \rightarrow 0$, by "Lemma" $(f_n(x))_n \rightarrow 0_{\cF(\bfR,\bfR)}$.
            \item Let $h_n \in \cF([0,\infty),\bfR)$ be defined by $h_n(x) = x^\frac{1}{n}$. If $x > 0$, then $(h_n(x))_n \rightarrow 1$. If $x = 0$, then $(h_n(x))_n \rightarrow 0$. Then $(h_n(x))_n \rightarrow \mathbf{1}_{(0,\infty)}$.
            \item Let $g_n \in \cF([0,\infty),\bfR)$ be defined by $g_n(x) = e^{-nx}$. If $x = 0$, then $(g_n(x))_n \rightarrow 1$. If $x > 0$, then $(g_n(x))_n \rightarrow 0$. So $(g_n(x))_n \rightarrow \delta_0$.
        \end{enumerate}
    \end{example}

    \begin{example}
        Our goal in this example is to construct a function $f_n \in \cF([0,\infty),\bfR$ satisfying $(f_n(x))_n \rightarrow 0_{\cF([0,\infty),\bfR)}$ and $\int_{0,\infty}f_n(x)dx = 1$. Let:
            \begin{equation*}
            \begin{split}
                f_n = \begin{cases} 4n^2x , & x < \frac{1}{2n}\\ 4n-4n^2x, & \frac{1}{2} \leq x < \frac{1}{n} \\ 0, & x \geq \frac{1}{n}. \end{cases}
            \end{split}
            \end{equation*}
        Given $x = 0$, $(f_n(0)) = (0)_n \rightarrow 0$. Given $x > 0$, then there exists $N \in \bfN$ such that $\frac{1}{N} < x$ (AP1). If $n \geq N$, then $f_n(x) = 0$, hence $(f_n(x))_n \rightarrow 0$. Thus $(f_n)_n \rightarrow 0_{\cF([0,\infty),\bfR)}$. Furthermore:
            \begin{equation*}
            \begin{split}
                \int_0 ^\infty f_n(x)dx &= \int_0^\frac{1}{2n}f_n(x)dx + \int_\frac{1}{2n}^\frac{1}{n}f_n(x)dx \\
                & = 1/2 + 1/2 \\
                & = 1.
            \end{split}
            \end{equation*}
    \end{example}

    \begin{definition}
        A sequence $(f_n)_n \in \cF(\Omega,\bfR)$ converges \textui{uniformly} to $f \in \cF(\Omega,\bfR)$ if:
            \begin{equation*}
            \begin{split}
                (\forall \epsilon > 0)(\exists N_\epsilon \in \bfN) \ni n \geq N &\implies |f_n(x) - f(x)| < \epsilon \hspace{6pt}\forall x \in \Omega \\
                & \implies \sup_{x \in \Omega}|f_n(x) - f(x)| \leq \epsilon.
            \end{split}
            \end{equation*}
    \end{definition}
