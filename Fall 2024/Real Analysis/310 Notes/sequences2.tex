\chapter{Sequences*}\label{chapter:sequences}
\vspace{12pt}

\section{Basic Definitions and Examples}
    \begin{definition}
        A \textui{sequence} in a metric space $X$ is a map $x:\bfN \rightarrow X$. We often write $x = (x_n)_n = (x_1,x_2,...)$, where $x_n = x(n)$. If $X = \bfR$, we call $x$ a \textui{real sequence}.
    \end{definition}

    \begin{example}
        \phantom{a}
        \begin{enumerate}[label = (\arabic*)]
            \item Sequences defined explicitly:
                \begin{enumerate}[label = (\roman*)]
                    \item Constant sequences: $x_n = t$, $(x_n)_n  = (t,t,t,...)$
                    \item Sequences defined by a function: $d_n = \left(1 + \frac{1}{n}\right)^n$.
                    \item Geometric sequences: fix $b \in \bfR$, then $(b^n)_n = (1,b,b^2,...)$.    
                \end{enumerate}
            \item Sequences defined recursively:
                \begin{enumerate}[label = (\roman*)]
                    \item Let $a_1 = 1$, $a_{n+1} = 2a_n + 1$. Then $(a_n)_n = (1,3,7,15,...)$.
                    \item Let $f_1 = 1$, $f_2 = 1$, $f_{n+1} = f_n + f_{n-1}$. Then $(f_n)_n = (1,1,2,3,5,8,...)$. This is the \textit{Fibonacci sequence}.
                    \item Let $X$ be a metric space and $f:X \rightarrow X$ be an endomorphism. Fix $x_0 \in X$. Then define:
                        \begin{equation*}
                        \begin{split}
                            x_1 &= f(x_0) \\
                            x_2 &= f(x_1) \\
                            &\vdots \\
                            x_n &= f(x_{n-1}).
                        \end{split}
                        \end{equation*}  
                \end{enumerate}
            \item New sequences from old:
                \begin{enumerate}[label = (\roman*)]
                    \item Let $(a_n)_n$ and $(b_n)_n$ be sequences. Define:
                        \begin{equation*}
                        \begin{split}
                            (a_n)_n + (b_n)_n &= (a_n + b_n)_n \\
                            t(a_n)_n &= (ta_n)_n \\
                            (a_n)_n \cdot (a_n)_n &= (a_n b_n)_n \\
                            \frac{(a_n)_n}{(b_n)_n} &= \left(\frac{a_n}{b_n}\right)_n, \hspace{4pt} (b_n)_n \neq 0 \mtext{for all $n$}.
                        \end{split}
                        \end{equation*}
                    \item Given $(x_n)_n$ and $k \in \bfN$, consider $(x_{n+k})_n = (x_k,x_{k+1},...)$. This is called a \textit{shift} or the \textit{$k^\text{th}$ tail} of $(x_n)_n$.
                    \item If $(a_n)_n$ is a sequence, $a_n \neq 0$ for all $n$, consider:
                        \begin{equation*}
                        \begin{split}
                            r_n = \frac{a_{n+1}}{a_n}.
                        \end{split}
                        \end{equation*}
                    So $(r_n)_n = \left(\frac{a_2}{a_1},\frac{a_3}{a_2},\frac{a_4}{a_3},...\right)$. These are called \textit{sequences of ratios}.
                    \item Given a real sequence $(x_k)_k$, consider the sequence $(s_n)_n$ where:
                        \begin{equation*}
                        \begin{split}
                            s_n = \sum_{k=1}^n x_k = s_{n-1} + x_k.
                        \end{split}
                        \end{equation*}
                    We call these \textit{$n^\text{th}$ partial sums}. An example of these are geometric sequences and telescoping sequences.
                \end{enumerate}
        \end{enumerate}
    \end{example}

    \begin{example}
        Let $F$ be a field. The set $F^\bfN = \{x\mid x:\bfN \rightarrow F\}$ is the set of all $F$-sequences. This forms an $F$-vector space under componentwise addition and scalar multiplication.
    \end{example}

    \begin{definition}
        Let $(x_n)_n$ be a sequence.
            \begin{enumerate}[label = (\arabic*)]
                \item $x_n$ is \textui{increasing} if $x_1 \leq x_2 \leq x_3 \leq ...$
                \item $x_n$ is \textui{decreasing} if $x_1 \geq x_2 \geq x_3 \geq ...$
                \item $x_n$ is \textui{strictly increasing} if $x_1 < x_2 < x_3 < ...$
                \item $x_n$ is \textui{strictly decreasing} if $x_1 > x_2 > x_3 > ...$
            \end{enumerate}
    \end{definition}

    \begin{definition}
        A sequence is said to \textui{eventually} have a certain property if it does not have the said property across all its ordered instances, but will after some instances have passed.
    \end{definition}

    \begin{definition}
        A sequence $(x_n)_n$ is \textui{monotone} if it is either increasing, decreasing, strictly increasing, or strictly decreasing.
    \end{definition}

\section{Convergence}
    \begin{definition}
        Let $(x_n)_n$ be a sequence in a metric space $X$.
        \begin{enumerate}[label = (\arabic*)]
            \item $(x_n)_n$ \textui{converges} to $x \in X$ if:
                \begin{equation*}
                \begin{split}
                    (\forall \epsilon > 0)(\exists N_\epsilon \in \bfN)\ni(\forall n \geq N_\epsilon)(d(x_n,x) < \epsilon)).
                \end{split}
                \end{equation*}
            We denote this as $(x_n)_n \rightarrow x$ or $\lim_{n \rightarrow \infty} x_n = x$.

            \item $(x_n)_n$ \textui{does not exist} if:
                \begin{equation*}
                \begin{split}
                    (\exists \epsilon_0 > 0)(\forall N \in \bfN)\ni (\exists n \geq N)(d(x_n,n)\geq \epsilon_0).
                \end{split}
                \end{equation*}
            We abbreviate this as D.N.E.

            \item $(x_n)_n$ \textui{diverges properly} to $+\infty$ if:
                \begin{equation*}
                \begin{split}
                    (\forall M > 0)(\exists N_M \in \bfN) \ni (\forall n \geq N_M)(x_n \geq M).
                \end{split}
                \end{equation*}
            We write $(x_n)_n \rightarrow +\infty$.

            \item $(x_n)_n$ \textui{diverges properly} to $-\infty$ if:
                \begin{equation*}
                \begin{split}
                    (\forall M < 0)(\exists N_M \in \bfN) \ni (\forall n \geq N_M)(x_n \leq M).
                \end{split}
                \end{equation*}
        \end{enumerate}      
    \end{definition} 

    \begin{example}
        Let $(x_n)_n$ be a real sequence. Then $d(x_n,x) < \epsilon \iff |x_n - x| < \epsilon \iff x_n \in V_\epsilon(x)$. We can visually represent a sequence as follows:
            \begin{center}
                \begin{tikzpicture}
                    \begin{axis}[
                        axis lines = middle,
                        xmin=0, xmax=11, % x-axis starts at 0 and extends to 10
                        ymin=-1, ymax=10, % y-axis range
                        xtick={1,2,3,4,5,6,7,8,9,10}, % add ticks 1 to 8 on the x-axis
                        xticklabels={}, % labels for the x-axis ticks
                        xlabel={$\bfN$},
                        ytick={4, 5, 6}, % positions for the y-axis ticks
                        yticklabels={$x-\epsilon$, $x$, $x+\epsilon$}, % labels for the y-axis ticks
                        ylabel={$\bfR$},
                    ]

                    
                    % Add dashed horizontal lines at x+\epsilon and x-\epsilon
                    \addplot[dashed, color=red] coordinates {(0, 4) (11, 4)}; % dashed line at x+\epsilon
                    \addplot[dashed, color=red] coordinates {(0, 6) (11, 6)}; % dashed line at x-\epsilon

                    \addplot[only marks,mark=*,mark options={fill=blue},color=blue]coordinates {(1, 7)};
                    \addplot[only marks,mark=*,mark options={fill=blue},color=blue]coordinates {(2, 2.5)};
                    \addplot[only marks,mark=*,mark options={fill=blue},color=blue]coordinates {(3, 6.5)};
                    \addplot[only marks,mark=*,mark options={fill=blue},color=blue]coordinates {(4, 3)};
                    \addplot[only marks,mark=*,mark options={fill=blue},color=blue]coordinates {(5, 6.2)};
                    \addplot[only marks,mark=*,mark options={fill=blue},color=blue]coordinates {(6, 4)};
                    \addplot[only marks,mark=*,mark options={fill=blue},color=blue]coordinates {(7, 5.8)};
                    \addplot[only marks,mark=*,mark options={fill=blue},color=blue]coordinates {(8, 4.4)};
                    \addplot[only marks,mark=*,mark options={fill=blue},color=blue]coordinates {(9, 5.3)};
                    \addplot[only marks,mark=*,mark options={fill=blue},color=blue]coordinates {(10, 5)};
                    \end{axis}
                \end{tikzpicture}
            \end{center}
        If a sequence is convergent it will eventually be contained between the two dashed lines.
    \end{example}

    \begin{example}
        \phantom{a}
        \begin{enumerate}[label = (\arabic*)]
            \item Prove $\left(\frac{1}{n}\right)_n \rightarrow 0$.
                \begin{solution}
                    Let $\epsilon > 0$. Find $N_\epsilon \in \bfN$ so that $\frac{1}{N_\epsilon} < \epsilon$. If $n \geq N_\epsilon$, then $\frac{1}{n} \leq \frac{1}{N_\epsilon} < \epsilon$. Hence $\frac{1}{n} = \left|\frac{1}{n} - 0\right| < \epsilon$.
                \end{solution}
            
            \item Prove $\left(\frac{5n-1}{3-n}\right)_{n=4}^\infty \rightarrow -5$.
                \begin{solution}
                    Note that:
                        \begin{equation*}
                        \begin{split}
                            |x_n - x| = \left|\frac{5n-1}{3-n}  + 5\right| = \frac{14}{|3-n|} = \frac{14}{n-3}.
                        \end{split}
                        \end{equation*}
                    Let $\epsilon > 0$. Find $N_\epsilon \in \bfN$ such that $N_\epsilon > \frac{14}{\epsilon} = 3$. If $n \geq N_\epsilon$, then $n > \frac{14}{\epsilon} + 3$ gives:
                        \begin{equation*}
                        \begin{split}
                            n-3 > \frac{14}{\epsilon} \implies \frac{14}{n-3} < \epsilon \implies |x_n - x| < \epsilon.
                        \end{split}
                        \end{equation*}
                \end{solution}
        \end{enumerate}
    \end{example}

    \begin{lemma}
        Let $(X,d)$ be a metric space. Then $(x_n)_n \rightarrow x$ if and only if $(d(x_n,x))_n \rightarrow 0$.
    \end{lemma}

    
    