%%%%%%%PACKAGES%%%%%%%
\documentclass[10pt,twoside,openany]{memoir}
\usepackage[T1]{fontenc}
\usepackage[utf8]{inputenc}
\usepackage{titlesec}
\usepackage{anyfontsize}
\usepackage{fancybox}
\usepackage[dvipsnames,svgnames,x11names,hyperref]{xcolor}
\usepackage{enumerate}
\usepackage{amsfonts}
\usepackage{amsthm}
\usepackage{amsmath}
\usepackage{amssymb}
\usepackage{hyperref}
\usepackage{fullpage}
\usepackage{bm}
\usepackage{cprotect}
\usepackage{calligra}
\usepackage{emptypage}
\usepackage{titleps}
\usepackage{microtype}
\usepackage{float}
\usepackage{ocgx}
\usepackage{appendix}
\usepackage{graphicx}
\usepackage{pdfcomment}
\usepackage{enumitem}
\usepackage{mathtools}
\usepackage{tikz-cd}
\usepackage{relsize}
\usepackage[font=footnotesize,labelfont=bf]{caption}
\usepackage{changepage}
\usepackage{xcolor}
\usepackage{ulem}
\usepackage{pgfplots}
\usepackage{marginnote}
    \newcommand*{\mnote}[1]{ % <----------
    \checkoddpage
    \ifoddpage
        \marginparmargin{left}
    \else
        \marginparmargin{right}
    \fi
        \marginnote{\tiny \textcolor{oorange}{#1}}
    }
\usepackage{tgbonum}
\usepackage{datetime}
    \newdateformat{specialdate}{\THEYEAR\ \monthname\ \THEDAY}
\usepackage[margin=0.9in]{geometry}
    \setlength{\voffset}{-0.4in}
    \setlength{\headsep}{30pt}
%\usepackage{fancyhdr}
%    \fancyhf{}
%    \pagestyle{fancy}
%    \cfoot{\footnotesize \thepage}
%    \fancyhead[R]{\footnotesize \rightmark}
%    \fancyhead[L]{\footnotesize \leftmark}
\usepackage[T1]{fontenc}% http://ctan.org/pkg/fontenc
\usepackage[outline]{contour}% http://ctan.org/pkg/contour
    \renewcommand{\arraystretch}{1.5}
    \contourlength{0.4pt}
    \contournumber{10}%
\usepackage{letterspace}
\usepackage{verbatim}


%%%%%%%%%%%%%%%%%%%%%%
%%%%%%%%MACROS%%%%%%%%
%%%%%%%%%%%%%%%%%%%%%%
%to make the correct symbol for Sha
%\newcommand\cyr{%
%\renewcommand\rmdefault{wncyr}%
%\renewcommand\sfdefault{wncyss}%
%\renewcommand\encodingdefault{OT2}%
%\normalfont \selectfont} \DeclareTextFontCommand{\textcyr}{\cyr}


\DeclareMathOperator{\ab}{ab}
\newcommand{\absgal}{\G_{\bbQ}}
\DeclareMathOperator{\ad}{ad}
\DeclareMathOperator{\adj}{adj}
\DeclareMathOperator{\alg}{alg}
\DeclareMathOperator{\Alt}{Alt}
\DeclareMathOperator{\Ann}{Ann}
\DeclareMathOperator{\arith}{arith}
\DeclareMathOperator{\Aut}{Aut}
\DeclareMathOperator{\Be}{B}
\DeclareMathOperator{\card}{card}
\DeclareMathOperator{\Char}{char}
\DeclareMathOperator{\csp}{csp}
\DeclareMathOperator{\codim}{codim}
\DeclareMathOperator{\coker}{coker}
\DeclareMathOperator{\coh}{H}
\DeclareMathOperator{\compl}{compl}
\DeclareMathOperator{\conj}{conj}
\DeclareMathOperator{\cont}{cont}
\DeclareMathOperator{\crys}{crys}
\DeclareMathOperator{\Crys}{Crys}
\DeclareMathOperator{\cusp}{cusp}
\DeclareMathOperator{\diag}{diag}
\DeclareMathOperator{\disc}{disc}
\DeclareMathOperator{\dR}{dR}
\DeclareMathOperator{\Eis}{Eis}
\DeclareMathOperator{\End}{End}
\DeclareMathOperator{\ev}{ev}
\DeclareMathOperator{\eval}{eval}
\DeclareMathOperator{\Eq}{Eq}
\DeclareMathOperator{\Ext}{Ext}
\DeclareMathOperator{\Fil}{Fil}
\DeclareMathOperator{\Fitt}{Fitt}
\DeclareMathOperator{\Frob}{Frob}
\DeclareMathOperator{\G}{G}
\DeclareMathOperator{\Gal}{Gal}
\DeclareMathOperator{\GL}{GL}
\DeclareMathOperator{\Gr}{Gr}
\DeclareMathOperator{\Graph}{Graph}
\DeclareMathOperator{\GSp}{GSp}
\DeclareMathOperator{\GUn}{GU}
\DeclareMathOperator{\Hom}{Hom}
\DeclareMathOperator{\id}{id}
\DeclareMathOperator{\Id}{Id}
\DeclareMathOperator{\Ik}{Ik}
\DeclareMathOperator{\IM}{Im}
\DeclareMathOperator{\Image}{im}
\DeclareMathOperator{\Ind}{Ind}
\DeclareMathOperator{\Inf}{inf}
\DeclareMathOperator{\Isom}{Isom}
\DeclareMathOperator{\J}{J}
\DeclareMathOperator{\Jac}{Jac}
\DeclareMathOperator{\lcm}{lcm}
\DeclareMathOperator{\length}{length}
\DeclareMathOperator{\Log}{Log}
\DeclareMathOperator{\M}{M}
\DeclareMathOperator{\Mat}{Mat}
\DeclareMathOperator{\N}{N}
\DeclareMathOperator{\Nm}{Nm}
\DeclareMathOperator{\NIk}{N-Ik}
\DeclareMathOperator{\NSK}{N-SK}
\DeclareMathOperator{\new}{new}
\DeclareMathOperator{\obj}{obj}
\DeclareMathOperator{\old}{old}
\DeclareMathOperator{\ord}{ord}
\DeclareMathOperator{\Or}{O}
\DeclareMathOperator{\PGL}{PGL}
\DeclareMathOperator{\PGSp}{PGSp}
\DeclareMathOperator{\rank}{rank}
\DeclareMathOperator{\Rel}{Rel}
\DeclareMathOperator{\Real}{Re}
\DeclareMathOperator{\RES}{res}
\DeclareMathOperator{\Res}{Res}
%\DeclareMathOperator{\Sha}{\textcyr{Sh}}
\DeclareMathOperator{\Sel}{Sel}
\DeclareMathOperator{\semi}{ss}
\DeclareMathOperator{\sgn}{sign}
\DeclareMathOperator{\SK}{SK}
\DeclareMathOperator{\SL}{SL}
\DeclareMathOperator{\SO}{SO}
\DeclareMathOperator{\Sp}{Sp}
\DeclareMathOperator{\Span}{span}
\DeclareMathOperator{\Spec}{Spec}
\DeclareMathOperator{\spin}{spin}
\DeclareMathOperator{\st}{st}
\DeclareMathOperator{\St}{St}
\DeclareMathOperator{\SUn}{SU}
\DeclareMathOperator{\supp}{supp}
\DeclareMathOperator{\Sup}{sup}
\DeclareMathOperator{\Sym}{Sym}
\DeclareMathOperator{\Tam}{Tam}
\DeclareMathOperator{\tors}{tors}
\DeclareMathOperator{\tr}{tr}
\DeclareMathOperator{\un}{un}
\DeclareMathOperator{\Un}{U}
\DeclareMathOperator{\val}{val}
\DeclareMathOperator{\vol}{vol}

\DeclareMathOperator{\Sets}{S \mkern1.04mu e \mkern1.04mu t \mkern1.04mu s}
    \newcommand{\cSets}{\scalebox{1.02}{\contour{black}{$\Sets$}}}
    
\DeclareMathOperator{\Groups}{G \mkern1.04mu r \mkern1.04mu o \mkern1.04mu u \mkern1.04mu p \mkern1.04mu s}
    \newcommand{\cGroups}{\scalebox{1.02}{\contour{black}{$\Groups$}}}

\DeclareMathOperator{\TTop}{T \mkern1.04mu o \mkern1.04mu p}
    \newcommand{\cTop}{\scalebox{1.02}{\contour{black}{$\TTop$}}}

\DeclareMathOperator{\Htp}{H \mkern1.04mu t \mkern1.04mu p}
    \newcommand{\cHtp}{\scalebox{1.02}{\contour{black}{$\Htp$}}}

\DeclareMathOperator{\Mod}{M \mkern1.04mu o \mkern1.04mu d}
    \newcommand{\cMod}{\scalebox{1.02}{\contour{black}{$\Mod$}}}

\DeclareMathOperator{\Ab}{A \mkern1.04mu b}
    \newcommand{\cAb}{\scalebox{1.02}{\contour{black}{$\Ab$}}}

\DeclareMathOperator{\Rings}{R \mkern1.04mu i \mkern1.04mu n \mkern1.04mu g \mkern1.04mu s}
    \newcommand{\cRings}{\scalebox{1.02}{\contour{black}{$\Rings$}}}

\DeclareMathOperator{\ComRings}{C \mkern1.04mu o \mkern1.04mu m \mkern1.04mu R \mkern1.04mu i \mkern1.04mu n \mkern1.04mu g \mkern1.04mu s}
    \newcommand{\cComRings}{\scalebox{1.05}{\contour{black}{$\ComRings$}}}

\DeclareMathOperator{\hHom}{H \mkern1.04mu o \mkern1.04mu m}
    \newcommand{\cHom}{\scalebox{1.02}{\contour{black}{$\hHom$}}}

         %  \item $\cGroups$
          %  \item $\cTop$
          %  \item $\cHtp$
          %  \item $\cMod$




\renewcommand{\k}{\kappa}
\newcommand{\Ff}{F_{f}}
\newcommand{\ts}{\,^{t}\!}


%Mathcal

\newcommand{\cA}{\mathcal{A}}
\newcommand{\cB}{\mathcal{B}}
\newcommand{\cC}{\mathcal{C}}
\newcommand{\cD}{\mathcal{D}}
\newcommand{\cE}{\mathcal{E}}
\newcommand{\cF}{\mathcal{F}}
\newcommand{\cG}{\mathcal{G}}
\newcommand{\cH}{\mathcal{H}}
\newcommand{\cI}{\mathcal{I}}
\newcommand{\cJ}{\mathcal{J}}
\newcommand{\cK}{\mathcal{K}}
\newcommand{\cL}{\mathcal{L}}
\newcommand{\cM}{\mathcal{M}}
\newcommand{\cN}{\mathcal{N}}
\newcommand{\cO}{\mathcal{O}}
\newcommand{\cP}{\mathcal{P}}
\newcommand{\cQ}{\mathcal{Q}}
\newcommand{\cR}{\mathcal{R}}
\newcommand{\cS}{\mathcal{S}}
\newcommand{\cT}{\mathcal{T}}
\newcommand{\cU}{\mathcal{U}}
\newcommand{\cV}{\mathcal{V}}
\newcommand{\cW}{\mathcal{W}}
\newcommand{\cX}{\mathcal{X}}
\newcommand{\cY}{\mathcal{Y}}
\newcommand{\cZ}{\mathcal{Z}}


%mathfrak (missing \fi)

\newcommand{\fa}{\mathfrak{a}}
\newcommand{\fA}{\mathfrak{A}}
\newcommand{\fb}{\mathfrak{b}}
\newcommand{\fB}{\mathfrak{B}}
\newcommand{\fc}{\mathfrak{c}}
\newcommand{\fC}{\mathfrak{C}}
\newcommand{\fd}{\mathfrak{d}}
\newcommand{\fD}{\mathfrak{D}}
\newcommand{\fe}{\mathfrak{e}}
\newcommand{\fE}{\mathfrak{E}}
\newcommand{\ff}{\mathfrak{f}}
\newcommand{\fF}{\mathfrak{F}}
\newcommand{\fg}{\mathfrak{g}}
\newcommand{\fG}{\mathfrak{G}}
\newcommand{\fh}{\mathfrak{h}}
\newcommand{\fH}{\mathfrak{H}}
\newcommand{\fI}{\mathfrak{I}}
\newcommand{\fj}{\mathfrak{j}}
\newcommand{\fJ}{\mathfrak{J}}
\newcommand{\fk}{\mathfrak{k}}
\newcommand{\fK}{\mathfrak{K}}
\newcommand{\fl}{\mathfrak{l}}
\newcommand{\fL}{\mathfrak{L}}
\newcommand{\fm}{\mathfrak{m}}
\newcommand{\fM}{\mathfrak{M}}
\newcommand{\fn}{\mathfrak{n}}
\newcommand{\fN}{\mathfrak{N}}
\newcommand{\fo}{\mathfrak{o}}
\newcommand{\fO}{\mathfrak{O}}
\newcommand{\fp}{\mathfrak{p}}
\newcommand{\fP}{\mathfrak{P}}
\newcommand{\fq}{\mathfrak{q}}
\newcommand{\fQ}{\mathfrak{Q}}
\newcommand{\fr}{\mathfrak{r}}
\newcommand{\fR}{\mathfrak{R}}
\newcommand{\fs}{\mathfrak{s}}
\newcommand{\fS}{\mathfrak{S}}
\newcommand{\ft}{\mathfrak{t}}
\newcommand{\fT}{\mathfrak{T}}
\newcommand{\fu}{\mathfrak{u}}
\newcommand{\fU}{\mathfrak{U}}
\newcommand{\fv}{\mathfrak{v}}
\newcommand{\fV}{\mathfrak{V}}
\newcommand{\fw}{\mathfrak{w}}
\newcommand{\fW}{\mathfrak{W}}
\newcommand{\fx}{\mathfrak{x}}
\newcommand{\fX}{\mathfrak{X}}
\newcommand{\fy}{\mathfrak{y}}
\newcommand{\fY}{\mathfrak{Y}}
\newcommand{\fz}{\mathfrak{z}}
\newcommand{\fZ}{\mathfrak{Z}}


%mathbf

\newcommand{\bfA}{\mathbf{A}}
\newcommand{\bfB}{\mathbf{B}}
\newcommand{\bfC}{\mathbf{C}}
\newcommand{\bfD}{\mathbf{D}}
\newcommand{\bfE}{\mathbf{E}}
\newcommand{\bfF}{\mathbf{F}}
\newcommand{\bfG}{\mathbf{G}}
\newcommand{\bfH}{\mathbf{H}}
\newcommand{\bfI}{\mathbf{I}}
\newcommand{\bfJ}{\mathbf{J}}
\newcommand{\bfK}{\mathbf{K}}
\newcommand{\bfL}{\mathbf{L}}
\newcommand{\bfM}{\mathbf{M}}
\newcommand{\bfN}{\mathbf{N}}
\newcommand{\bfO}{\mathbf{O}}
\newcommand{\bfP}{\mathbf{P}}
\newcommand{\bfQ}{\mathbf{Q}}
\newcommand{\bfR}{\mathbf{R}}
\newcommand{\bfS}{\mathbf{S}}
\newcommand{\bfT}{\mathbf{T}}
\newcommand{\bfU}{\mathbf{U}}
\newcommand{\bfV}{\mathbf{V}}
\newcommand{\bfW}{\mathbf{W}}
\newcommand{\bfX}{\mathbf{X}}
\newcommand{\bfY}{\mathbf{Y}}
\newcommand{\bfZ}{\mathbf{Z}}

\newcommand{\bfa}{\mathbf{a}}
\newcommand{\bfb}{\mathbf{b}}
\newcommand{\bfc}{\mathbf{c}}
\newcommand{\bfd}{\mathbf{d}}
\newcommand{\bfe}{\mathbf{e}}
\newcommand{\bff}{\mathbf{f}}
\newcommand{\bfg}{\mathbf{g}}
\newcommand{\bfh}{\mathbf{h}}
\newcommand{\bfi}{\mathbf{i}}
\newcommand{\bfj}{\mathbf{j}}
\newcommand{\bfk}{\mathbf{k}}
\newcommand{\bfl}{\mathbf{l}}
\newcommand{\bfm}{\mathbf{m}}
\newcommand{\bfn}{\mathbf{n}}
\newcommand{\bfo}{\mathbf{o}}
\newcommand{\bfp}{\mathbf{p}}
\newcommand{\bfq}{\mathbf{q}}
\newcommand{\bfr}{\mathbf{r}}
\newcommand{\bfs}{\mathbf{s}}
\newcommand{\bft}{\mathbf{t}}
\newcommand{\bfu}{\mathbf{u}}
\newcommand{\bfv}{\mathbf{v}}
\newcommand{\bfw}{\mathbf{w}}
\newcommand{\bfx}{\mathbf{x}}
\newcommand{\bfy}{\mathbf{y}}
\newcommand{\bfz}{\mathbf{z}}

%blackboard bold

\newcommand{\bbA}{\mathbb{A}}
\newcommand{\bbB}{\mathbb{B}}
\newcommand{\bbC}{\mathbb{C}}
\newcommand{\bbD}{\mathbb{D}}
\newcommand{\bbE}{\mathbb{E}}
\newcommand{\bbF}{\mathbb{F}}
\newcommand{\bbG}{\mathbb{G}}
\newcommand{\bbH}{\mathbb{H}}
\newcommand{\bbI}{\mathbb{I}}
\newcommand{\bbJ}{\mathbb{J}}
\newcommand{\bbK}{\mathbb{K}}
\newcommand{\bbL}{\mathbb{L}}
\newcommand{\bbM}{\mathbb{M}}
\newcommand{\bbN}{\mathbb{N}}
\newcommand{\bbO}{\mathbb{O}}
\newcommand{\bbP}{\mathbb{P}}
\newcommand{\bbQ}{\mathbb{Q}}
\newcommand{\bbR}{\mathbb{R}}
\newcommand{\bbS}{\mathbb{S}}
\newcommand{\bbT}{\mathbb{T}}
\newcommand{\bbU}{\mathbb{U}}
\newcommand{\bbV}{\mathbb{V}}
\newcommand{\bbW}{\mathbb{W}}
\newcommand{\bbX}{\mathbb{X}}
\newcommand{\bbY}{\mathbb{Y}}
\newcommand{\bbZ}{\mathbb{Z}}

\newcommand{\bmat}{\left( \begin{matrix}}
\newcommand{\emat}{\end{matrix} \right)}

\newcommand{\pmat}{\left( \begin{smallmatrix}}
\newcommand{\epmat}{\end{smallmatrix} \right)}

\newcommand{\lat}{\mathscr{L}}
\newcommand{\mat}[4]{\begin{pmatrix}{#1}&{#2}\\{#3}&{#4}\end{pmatrix}}
\newcommand{\ov}[1]{\overline{#1}}
\newcommand{\res}[1]{\underset{#1}{\RES}\,}
\newcommand{\up}{\upsilon}

\newcommand{\tac}{\textasteriskcentered}

%mahesh macros
\newcommand{\tm}{\textrm}

%Comments
\newcommand{\com}[1]{\vspace{5 mm}\par \noindent
\marginpar{\textsc{Comment}} \framebox{\begin{minipage}[c]{0.95
\textwidth} \tt #1 \end{minipage}}\vspace{5 mm}\par}

\newcommand{\Bmu}{\mbox{$\raisebox{-0.59ex}
  {$l$}\hspace{-0.18em}\mu\hspace{-0.88em}\raisebox{-0.98ex}{\scalebox{2}
  {$\color{white}.$}}\hspace{-0.416em}\raisebox{+0.88ex}
  {$\color{white}.$}\hspace{0.46em}$}{}}  %need graphicx and xcolor. this produces blackboard bold mu 

\newcommand{\hooktwoheadrightarrow}{%
  \hookrightarrow\mathrel{\mspace{-15mu}}\rightarrow
}

\makeatletter
\newcommand{\xhooktwoheadrightarrow}[2][]{%
  \lhook\joinrel
  \ext@arrow 0359\rightarrowfill@ {#1}{#2}%
  \mathrel{\mspace{-15mu}}\rightarrow
}
\makeatother

\renewcommand{\geq}{\geqslant}
    \renewcommand{\leq}{\leqslant}
    
    \newcommand{\bone}{\mathbf{1}}
    \newcommand{\sign}{\mathrm{sign}}
    \newcommand{\eps}{\varepsilon}
    \newcommand{\textui}[1]{\uline{\textit{#1}}}
    
    %\newcommand{\ov}{\overline}
    %\newcommand{\un}{\underline}
    \newcommand{\fin}{\mathrm{fin}}
    
    \newcommand{\chnum}{\titleformat
    {\chapter} % command
    [display] % shape
    {\centering} % format
    {\Huge \color{black} \shadowbox{\thechapter}} % label
    {-0.5em} % sep (space between the number and title)
    {\LARGE \color{black} \underline} % before-code
    }
    
    \newcommand{\chunnum}{\titleformat
    {\chapter} % command
    [display] % shape
    {} % format
    {} % label
    {0em} % sep
    { \begin{flushright} \begin{tabular}{r}  \Huge \color{black}
    } % before-code
    [
    \end{tabular} \end{flushright} \normalsize
    ] % after-code
    }

\newcommand{\littletaller}{\mathchoice{\vphantom{\big|}}{}{}{}}
\newcommand\restr[2]{{% we make the whole thing an ordinary symbol
  \left.\kern-\nulldelimiterspace % automatically resize the bar with \right
  #1 % the function
  \littletaller % pretend it's a little taller at normal size
  \right|_{#2} % this is the delimiter
  }}

\newcommand{\mtext}[1]{\hspace{6pt}\text{#1}\hspace{6pt}}

%This adds a "front cover" page.
%{\thispagestyle{empty}
%\vspace*{\fill}
%\begin{tabular}{l}
%\begin{tabular}{l}
%\includegraphics[scale=0.24]{oxy-logo.png}
%\end{tabular} \\
%\begin{tabular}{l}
%\Large \color{black} Module Theory, Linear Algebra, and Homological Algebra \\ \Large \color{black} Gianluca Crescenzo
%\end{tabular}
%\end{tabular}
%\newpage

\newcommand{\TBC}{\textbf{TO BE CONTINUED}}
\theoremstyle{plain}
\newtheorem{theorem}{Theorem}[section]
\newtheorem{proposition}[theorem]{Proposition}
\newtheorem{corollary}[theorem]{Corollary}
\newtheorem{lemma}[theorem]{Lemma}

\theoremstyle{definition}
\newtheorem{definition}{Definition}[section]
\newtheorem{example}{Example}[section]
\newtheorem{exercise}{Exercise}
\newtheorem{note}{Note}[section]

\theoremstyle{remark}
\newtheorem{remark}[theorem]{Remark}
\newtheorem*{noproof}{Proof omitted}
\numberwithin{equation}{section}

\newenvironment{solution}[1]{\noindent \textbf{#1}:}{}

\newcommand{\NN}{\mathbf{N}}
\newcommand{\ZZ}{\mathbf{Z}}
\newcommand{\QQ}{\mathbf{Q}}
\newcommand{\RR}{\mathbf{R}}
\newcommand{\CC}{\mathbf{C}}
\newcommand{\HH}{\mathbf{H}}
\newcommand{\KK}{\mathbf{K}}
\newcommand{\FF}{\mathbf{F}}

\newcommand{\bRR}{\overline{\RR}}
\newcommand{\bRRp}{\overline{\RR}_{\geq 0}}

\renewcommand{\geq}{\geqslant}
\renewcommand{\leq}{\leqslant}



\titlespacing*{\chapter}
{0pt} % Left margin
{*1} % Space before the chapter number (increase this value for more space)
{0pt} % Space after the chapter number and before the title

%%%%%%%%%%%%%%%%%%%%%%%%%%%%%%%%%%%%%%%%%%%%%%%%%%%%%%%%%%%%%%%%%
%%%%%%%%%%%%%%%%%%%%%%%%%%%%%%%%%%%%%%%%%%%%%%%%%%%%%%%%%%%%%%%%%%%%%%%%%%%%%%%%%%%%%%%%%%%%%%%%%%%%%%%%%%%%%%%%%%%%%%%%%%%%%%%%%%%%%%%%%%%%%%%%%%%%%%%%%%%%%%%%%%%%%%%%%%%%%%%%%%%%%%%%%%%%%%%%%%%%%%%%%%%%%%%%%%%%%%%%%%%%%%%%%%%%%%%%%%%%%%%%%%%%%%%%%%%%%%%%%%%%%%%%%%%%%%%%%%%%%%%%%%%%%%%%%%%%%%%%%%%%%%%%%%%%%%%%%%%%%%%%%%%%%%%%%%%%%%%%%%%%%%%%%%%%%%%%%%%%%%%%%%%%%%%%%%%%%%%%%%%%%%%%%%%%%%%%%%%%%%%%%%%%%%%%%%%%%%%%%%%%%%%%%%%%%%%%%%%%%%%%%%%%%%%%%%%%%%%%%%%%%%%%%%%%%%%%%%%%%%%%%%%%%%%%%%%%%%%%%%%%%%%%%%%%%%%%%%%%%%%%%%%%%%%%%%%%%%%%%%%%%%%%%%%%%%%%%%%%%%%%%%%%%%%%%%%%%%%%%%%%%%%%%%%%%%%%%%%%%%%%%%%%%%%%%%%%%%%%%%%%%%%%%%%%%%%%%%%%%%%%%%%%%%%%%%%%%%%%%%%%%%%%%%%%%%%%%%%%%%%%%%%%%%%%%%%%%%%%%%%%%%%%%%%%%%%%%%%%%%%%%%%%%%%%%%%%%%%%%%%%%%%%%%%%%%%%%%%%%%%%%%%%%%%%%%%%%%%%%%%%%%%%%%%%%%%%%%%%%%%%%%%%%%%%%%%%%%%%%%%%%%%%%%%%%%%%%%%%%%%%%%%%%%%%%%%%%%%%%%%%%%%%%%%%%%%%%%%%%%%%%%%%%%%%%%%%%%%%%%%%%%%%%%%%%%%%%%%%%%%%%%%%%%%%%%%%%%%%%%%%%%%%%%%%%%%%%%%%%%%%%%%%%%%%%%%%%%%%%%%%%%%%%%%%%%%%%%%%%%%%%%%%%%%%%%%%%%%%%%%%%%%%%%%%%%%%%%%%%%%%%%%%%%%%%%%%%%%%%%%%%%%%%%%%%%%%%%%%%%%%%%%%%%%%%%%%%%%%%%%%%%%%%%%%%%%%%%%%%%%%%%%%%%%%%%%%%%%%%%%%%%%%%%%%%%%%%%%%%%%%%%%%%%%%%%%%%%%%%%%%%%%%%%%%%%%%%%%%%%%%%%%%%%%%%%%%%%%%%%%%%%%%%%%%%%%%%%%
\begin{document}
\begin{center}
    { \Large Math 310 \\[0.1in]Homework 1 \\[0.1in]
    Due: 9/9/2024}\\[.25in]
    { Name:} {\underline{Gianluca Crescenzo\hspace*{2in}}}\\[0.15in]
    \end{center}
    \vspace{4pt}
%%%%%%%%%%%%%%%%%%%%%%%%%%%%%%%%%%%%%%%%%%%%%%%%%%%%%%%%%%%%%%%%%%%%%%%%%%%%%%%%%%%%%%%%%%%%%%%%%%%%%%%%%%%%%%%    
    \begin{exercise}
        If $F$ is a finite set and $k:F \rightarrow F$ is a self map, prove that $k$ is injective if and only if $k$ is surjective.
    \end{exercise}
        \begin{proof}
             Let $k$ be injective. Suppose towards contradiction that $k$ is not surjective. Then $k(F) \subset F$. But then there exists $f_i,f_j \in F$ such that $k(f_i) = k(f_j)$ with $f_i \neq f_j$, contradicting the fact that $k$ is injective. Hence $k$ must also be surjective. 

            Now suppose $k$ is not injective. Then there exists \textit{at least} two elements $f_i,f_j$ with $k(f_i) = k(f_j)$ and $f_i \neq f_j$. So $k(F) \subset F$, hence $k$ is not surjective.
        \end{proof}
%%%%%%%%%%%%%%%%%%%%%%%%%%%%%%%%%%%%%%%%%%%%%%%%%%%%%%%%%%%%%%%%%%%%%%%%%%%%%%%%%%%%%%%%%%%%%%%%%%%%%%%%%%%%%%%%
    \begin{exercise}\label{example:2}
        Prove that a set $A$ is infinite if and only if there is a non-surjective injection $f:A \rightarrow A$.
    \end{exercise}
        \begin{proof}
            Suppose $A$ is infinite. Then there exists an injection $\pi:\bfN \hookrightarrow A$ defined by $\pi(n) = a_n$. Define $f:A \rightarrow A$ by $f(\pi(n)) = \pi(n+1)$. Suppose $f(\pi(i)) = f(\pi(j))$, then $\pi(i+1) = \pi(j+i)$. Simplifying further yields $i+1 = j+1$, or equivalently $i = j$. Hence $\pi(i) = \pi(j)$, establishing $f$ as an injection. Now suppose there exists a surjection $A \twoheadrightarrow A$. This immediately leads to a contradiction, as there does not exist an element $x \in A$ such that $f(x) = \pi(1)$. Hence $f$ is a non-surjective injection.

            Conversely, suppose that $A$ is finite. Then by Exercise 1 there exists a bijection $A \hooktwoheadrightarrow A$.
        \end{proof}
%%%%%%%%%%%%%%%%%%%%%%%%%%%%%%%%%%%%%%%%%%%%%%%%%%%%%%%%%%%%%%%%%%%%%%%%%%%%%%%%%%%%%%%%%%%%%%%%%%%%%%%%%%%%%%%%
    \begin{exercise}
        Let $A,B$, and $C$ be sets and suppose $\card(A) < \card(B) \leq \card(C)$. Prove that $\card(A) < \card(C)$.
    \end{exercise}
        \begin{proof}
            Let $A \xrightarrow{f} B \xrightarrow{g} C$ be functions with $g$ injective and $f$ injective but not surjective. Then $g\circ f$ is injective, but note that $f(A) \subset B$ implies $(g\circ f)(A) \subset C$ (otherwise $g$ would not be a function). Hence there does not exist a surjection $g \circ f : A \rightarrow C$, establishing $\card(A) < \card(C)$.
        \end{proof}
%%%%%%%%%%%%%%%%%%%%%%%%%%%%%%%%%%%%%%%%%%%%%%%%%%%%%%%%%%%%%%%%%%%%%%%%%%%%%%%%%%%%%%%%%%%%%%%%%%%%%%%%%%%%%%%%
    \begin{exercise}
        If $A \subseteq B$ is an inclusion of sets with $A$ countable and $B$ uncountable, show that $B\setminus A$ is uncountable.
    \end{exercise}
        \begin{proof}
            Suppose towards contradiction that $B\setminus A$ is countable. Since countable unions of countable sets is countable, $(B\setminus A) \cup A = B$ is countable, which is a contradiction. Thus $B\setminus A$ is uncountable.
        \end{proof}
%%%%%%%%%%%%%%%%%%%%%%%%%%%%%%%%%%%%%%%%%%%%%%%%%%%%%%%%%%%%%%%%%%%%%%%%%%%%%%%%%%%%%%%%%%%%%%%%%%%%%%%%%%%%%%%%
    \begin{exercise}
        Is the set $\{x \in \bfR \mid x > 0 \hspace{4pt}\text{and}\hspace{4pt} x^2 \in \bfQ\}$ countable?
    \end{exercise}
        \begin{proof}
            Let $S$ be the above set. If $x \in S$, then $x>0$ and $x^2 \in \bfQ$. This implies that $x^2 = q$ for some $q \in \bfQ^+$, hence $x = \sqrt{q}$. Define $f:S \rightarrow \bfQ$ by $\sqrt{q} \mapsto q$. Let $f(\sqrt{m}) = f(\sqrt{n})$ for some $\sqrt{m},\sqrt{n} \in S$. Then $m = n$, and square-rooting both sides gives $\sqrt{m} = \sqrt{n}$, establishing an injection. Thus $S$ is countable.
        \end{proof}
%%%%%%%%%%%%%%%%%%%%%%%%%%%%%%%%%%%%%%%%%%%%%%%%%%%%%%%%%%%%%%%%%%%%%%%%%%%%%%%%%%%%%%%%%%%%%%%%%%%%%%%%%%%%%%%%
    \begin{exercise}
        Consider the set $\cF(\bfN)$ of all finite subsets of $\bfN$. Is $\cF(\bfN)$ countable?
    \end{exercise}
        \begin{proof}
            Let $A_n = \{A \subseteq \bfN \mid \card(A) = n\hspace{4pt}\text{for some $n \in \bfN$}\}$ Note that $A_n$ by our definition is finite, and $\bigcup_{n \in \bfN}A_n = \cF(\bfN)$. Thus $\cF(\bfN)$ is countable.
        \end{proof}
%%%%%%%%%%%%%%%%%%%%%%%%%%%%%%%%%%%%%%%%%%%%%%%%%%%%%%%%%%%%%%%%%%%%%%%%%%%%%%%%%%%%%%%%%%%%%%%%%%%%%%%%%%%%%%%%
    \begin{exercise}
        Let $k \in \bfN$.
            \begin{enumerate}[label = (\roman*)]
                \item Prove that $\bfN ^k : = \underbrace{\bfN \times \bfN \times ... \times \bfN}_{\text{$k$ times}}$ is countable.
                    \begin{proof}
                        Let $p_1,p_2,...,p_k$ denote the first $k$ primes. Define $f:\bfN^k \rightarrow \bfN$ by $(e_1,e_2,...,e_k) \mapsto p_1^{e_1} \cdot p_2^{e_2}\cdot ...\cdot p_k^{e_k}$. Then $f((e_1,e_2,...,e_k)) = f((r_1,r_2,...,r_k))$ is equivalent to $p_1^{e_1}\cdot p_2^{e_2}\cdot...\cdot p_k^{e_k} = p_1^{r_1}\cdot p_2^{r_2}\cdot...\cdot p_k^{r_k}$. By the fundamental theorem of arithmetic, every natural number is prime itself or the product of a \textit{unique} combination of prime numbers. Therefore $p_1^{e_1}\cdot p_2^{e_2}\cdot...\cdot p_k^{e_k} = p_1^{r_1}\cdot p_2^{r_2}\cdot...\cdot p_k^{r_k}$ implies $e_i = r_i$ for all $1 \leq i \leq k$. Hence $(e_1,e_2,...,e_k) = (r_1,r_2,...,r_k)$, establishing $f$ as an injection into the natural numbers. Thus $\bfN^k$ is countable.
                    \end{proof}
                \item Show that the set
                    \begin{equation*}
                    \begin{split}
                        \bfN^{\infty} := \{(n_k)_{k \geq 1} \mid n_k \in \bfN \}
                    \end{split}
                    \end{equation*}
                consisting of all sequences of natural numbers is uncountable.
                    \begin{proof}
                        Note that $2^\bfN = \{f \mid f:\bfN \rightarrow \{0,1\}\} \subseteq \bfN^\infty$. Since $2^\bfN$ is uncountable, $\bfN^\infty$ must be uncountable.
                    \end{proof}
                \item Prove that the set of \textbf{finitely-supported} natural sequences
                    \begin{equation*}
                    \begin{split}
                        c_c(\bfN) := \{(n_k)_{k\geq 1} \mid n_k \in \bfN, n_k = 0 \hspace{4pt}\text{for all but finitely many $k$}\}
                    \end{split}
                    \end{equation*}
                is countable.
                    \begin{proof}
                        Let $c_i(\bfN) = \{(n_k)_{k\geq 1} \mid n_k \in \bfN , n_k = 0 \hspace{4pt}\text{for all }\hspace{4pt}k > i \in \bfN\}$. Define $f:c_i(\bfN) \rightarrow \bfN^i$ by $(n_k)_{k\geq 1} \mapsto (n_1,n_2,...,n_i)$. If $f((n_k)_{k\geq 1}) = f((p_k)_{p \geq 1})$, then $(n_1,n_2,...,n_i) = (p_1,p_2,...,p_i)$; i.e., $n_j = p_j$ for all $1 \leq j \leq i$. Hence $(n_k)_{k\geq1} = (p_k)_{k\geq1}$. Since $f$ is injective, $c_i(\bfN)$ is countable, therefore $c_c(\bfN) = \bigcup_{i \in \bfN} c_i(\bfN)$ is countable.
                    \end{proof}
                \item Is the set of decreasing natural sequences
                    \begin{equation*}
                    \begin{split}
                        D := \{(n_k)_{k \geq 1} \mid n_k \in \bfN, n_{k+1} \leq n_k \hspace{4pt}\text{for all $k \geq 1$}    \}
                    \end{split}
                    \end{equation*}
                countable or uncountable?
                \begin{proof}
                    Let $c_{i,j}(\bfN) = \{(n_k)_{k\geq1} \mid n_k \in \bfN, n_{k+1} \leq n_k \hspace{4pt} \text{for all}\hspace{4pt}k \geq 1 \hspace{4pt}\text{terminating in $j$ for all $k >i \in \bfN$}\}$. By (iii) this set is countable, hence $D = \bigcup_{j \in \bfN} \left( \bigcup_{i \in \bfN} c_{i,j}(\bfN)\right)$ is countable.
                \end{proof}
            \end{enumerate} 
    \end{exercise}
%%%%%%%%%%%%%%%%%%%%%%%%%%%%%%%%%%%%%%%%%%%%%%%%%%%%%%%%%%%%%%%%%%%%%%%%%%%%%%%%%%%%%%%%%%%%%%%%%%%%%%%%%%%%%%%%
    \begin{exercise}
        Let $f: \bfR \rightarrow \bfR$ be a function that sends rational numbers to irrational  numbers and irrational numbers to rational numbers. Prove that $\Image{(f)}$ can't contain any interval.
    \end{exercise}
        \begin{proof}
            Note that $\bfR = (\bfR \setminus \bfQ)\cup \bfQ$. Then $f(\bfR) = f((\bfR \setminus \bfQ)\cup \bfQ) = f(\bfR \setminus \bfQ) \cup f(\bfQ)$. Since $f(\bfR \setminus \bfQ) \subseteq \bfQ$, it is countable. Likewise, since $\bfQ$ has a countable number of elements, $f(\bfQ)$ must get mapped to a countable subset of $\bfR\setminus \bfQ$ (otherwise $f$ would not be a function). Therefore $f(\bfQ)$ is countable, establishing $f(\bfR \setminus \bfQ) \cup f(\bfQ)$ as countable. Then $\card(\Image{(f)}) \leq \card({\bfN}) < \card(0,1)$, and Exercise 3 gives $\card(\Image{(f)}) < \card(0,1)$. Hence $\Image{(f)}$ cannot contain any interval.
        \end{proof}
%%%%%%%%%%%%%%%%%%%%%%%%%%%%%%%%%%%%%%%%%%%%%%%%%%%%%%%%%%%%%%%%%%%%%%%%%%%%%%%%%%%%%%%%%%%%%%%%%%%%%%%%%%%%%%%%
    \begin{exercise}
        Prove that the set
            \begin{equation*}
            \begin{split}
                \bfQ[x] = \left\{\sum_{k=0}^n a_k x^k \mid n\in \bfN_0 , a_k \in \bfQ \right\},
            \end{split}
            \end{equation*}
        consisting of all polynomials with rational coefficients, is countable.
    \end{exercise}
        \begin{proof}
            Let $P_n(\bfQ) = \{a_0 + a_1 x + ... + a_n x^n \mid a_i \in \bfQ$\} be the set of all polynomials of degree $n$. Define $f:P_n(\bfQ) \rightarrow \bfQ^{n+1}$ by $a_0 + a_1x + ...+ a_nx^n \mapsto (a_0, a_1, ..., a_n)$. Let $f(a_0 + a_1x + ... a_nx^n) = f(b_0 + b_1x + ... b_nx^n)$. We have that $(a_0,a_1,...,a_n) = (b_0,b_1,...,b_n)$, hence $a_i = b_i$ for all $0 \leq i \leq n$. This gives $a_0 + a_1x + ... + a_n x^n = b_0 + b_1x + ... b_nx^n$, establishing that $f$ is injective. Therefore $P_n(\bfQ)$ is countable, and since $\bigcup_{k \in \bfN} P_k(\bfQ) = \bfQ[x]$, we can conclude $\bfQ[x]$ is countable.
        \end{proof}
%%%%%%%%%%%%%%%%%%%%%%%%%%%%%%%%%%%%%%%%%%%%%%%%%%%%%%%%%%%%%%%%%%%%%%%%%%%%%%%%%%%%%%%%%%%%%%%%%%%%%%%%%%%%%%%%    
    \begin{exercise}
        A real number $t$ is called \textbf{algebraic} if there is a nonzero polynomial $p$ with rational coefficients such that $p(t) = 0$. If $t \in \bfR$ is not algebraic, it is called \textbf{transcendental}. For example, $\sqrt{2}$ is algebraic, but $\pi$ is transcendental. Show that the set of algebraic numbers is countable, and conclude that there are uncountably many transcendental numbers.
    \end{exercise}
        \begin{proof}
            The set containing all such algebraic numbers is denoted $\overline{\bfQ}$, referred to as the \textit{algebraic closure of $\bfQ$ in $\bfR$}. Let $A_n = \{t \mid p(t) = 0\hspace{4pt}\text{for some}\hspace{4pt} p(x) \in P_n(\bfQ)\}$. Since a polynomial of finite degree has a finite number of roots, $A_n$ is countable. Then $\bigcup_{k \in \bfN} A_k = \overline{\bfQ}$, establishing that the algebraic closure of $\bfQ$ over $\bfR$ is countable. Since $\overline{\bfQ} \subseteq \bfR$, Exercise 4 gives that the transcendental numbers $\bfR \setminus \overline{\bfQ}$ are uncountable.
        \end{proof}
\end{document}
