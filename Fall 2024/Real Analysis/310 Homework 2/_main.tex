%%%%%%%PACKAGES%%%%%%%
\documentclass[10pt,twoside,openany]{memoir}
%\usepackage[T1]{fontenc}
%\usepackage[utf8]{inputenc}
\usepackage{titlesec}
    \titlespacing*{\chapter}
    {0pt} % Left margin
    {*1} % Space before the chapter number (increase this value for more space)
    {0pt} % Space after the chapter number and before the title
\usepackage{anyfontsize}
\usepackage{fancybox}
\usepackage[dvipsnames,svgnames,x11names,hyperref]{xcolor}
\usepackage{enumerate}
\usepackage{comment}
\usepackage{amsfonts}
\usepackage{amsthm}
\usepackage{amsmath}
\usepackage{amssymb}
\usepackage{mathrsfs}
\usepackage{hyperref}
\usepackage{fullpage}
\usepackage{bm}
\usepackage{cprotect}
\usepackage{calligra}
\usepackage{emptypage}
\usepackage{titleps}
\usepackage{microtype}
\usepackage{float}
\usepackage{ocgx}
\usepackage{appendix}
\usepackage{graphicx}
\usepackage{pdfcomment}
\usepackage{enumitem}
\usepackage{mathtools}
\usepackage{tikz-cd}
\usepackage{relsize}
\usepackage[font=footnotesize,labelfont=bf]{caption}
\usepackage{changepage}
\usepackage{xcolor}
\usepackage{ulem}
\usepackage{pgfplots}
\usepackage{marginnote}
    \newcommand*{\mnote}[1]{ % <----------
    \checkoddpage
    \ifoddpage
        \marginparmargin{left}
    \else
        \marginparmargin{right}
    \fi
        \marginnote{\tiny \textcolor{oorange}{#1}}
    }
\usepackage{tgbonum}
\usepackage{datetime}
    \newdateformat{specialdate}{\THEYEAR\ \monthname\ \THEDAY}
\usepackage[margin=0.9in]{geometry}
    \setlength{\voffset}{-0.4in}
    \setlength{\headsep}{30pt}
\usepackage{fancyhdr}
    \fancyhf{}
    \cfoot{\footnotesize \thepage}
    \fancyhead[R]{\footnotesize \rightmark}
    \fancyhead[L]{\footnotesize \leftmark}
\usepackage[outline]{contour}% http://ctan.org/pkg/contour
    \renewcommand{\arraystretch}{1.5}
    \contourlength{0.4pt}
    \contournumber{10}%
\usepackage{letterspace}
    \linespread{1.25}
\usepackage{thmtools}
    \declaretheoremstyle[
        spaceabove=6pt, spacebelow=6pt,
        headfont=\normalfont\bfseries,
        notefont=\mdseries, notebraces={(}{)},
        bodyfont=\normalfont,
        postheadspace=1em
        %qed=\qedsymbol
        ]{mystyle}

    \declaretheorem[
        numberwithin=section,
        %shaded={
        %    rulecolor={RGB}{255,250,177},
        %    rulewidth=5pt, 
        %    bgcolor={RGB}{255,250,177}}
    ]{theorem}

    \declaretheorem[
        sibling=theorem,
        numberwithin=section,
        %shaded={
        %    rulecolor=Lavender,
        %    rulewidth=5pt, 
        %    bgcolor=Lavender}
    ]{lemma}

    \declaretheorem[
        sibling=theorem,
        numberwithin=section,
        %shaded={
        %    rulecolor=Lavender,
        %    rulewidth=5pt, 
        %    bgcolor=Lavender}
    ]{proposition}

    \declaretheorem[
        sibling=theorem,
        numberwithin=section,
        %shaded={
        %    rulecolor=Lavender,
        %    rulewidth=5pt, 
        %    bgcolor=Lavender}
    ]{corollary}

    \declaretheorem[
        numberwithin=section,
        style=mystyle,
        %shaded={
        %    rulecolor={RGB}{233,233,233},
        %    rulewidth=5pt, 
        %    bgcolor={RGB}{233,233,233}}
    ]{example}

    \declaretheorem[
        numberwithin=section,
        style=mystyle,
        %shaded={
        %    rulecolor={RGB}{255,250,177},
        %    rulewidth=5pt, 
        %    bgcolor={RGB}{255,250,177}}
    ]{definition}

    \declaretheorem[
        %shaded={
        %    rulecolor=Lavender,
        %    rulewidth=5pt, 
        %    bgcolor=Lavender}
    ]{exercise}
\declaretheorem[numbered=unless unique,style=mystyle]{Note}
    




\usepackage[math]{iwona}
\usepackage[T1]{fontenc}
%%%%%%%%%%%%%%%%%%%%%%
%%%%%%%%MACROS%%%%%%%%
%%%%%%%%%%%%%%%%%%%%%%
%to make the correct symbol for Sha
%\newcommand\cyr{%
%\renewcommand\rmdefault{wncyr}%
%\renewcommand\sfdefault{wncyss}%
%\renewcommand\encodingdefault{OT2}%
%\normalfont \selectfont} \DeclareTextFontCommand{\textcyr}{\cyr}


\DeclareMathOperator{\ab}{ab}
\newcommand{\absgal}{\G_{\bbQ}}
\DeclareMathOperator{\ad}{ad}
\DeclareMathOperator{\adj}{adj}
\DeclareMathOperator{\alg}{alg}
\DeclareMathOperator{\Alt}{Alt}
\DeclareMathOperator{\Ann}{Ann}
\DeclareMathOperator{\arith}{arith}
\DeclareMathOperator{\Aut}{Aut}
\DeclareMathOperator{\Be}{B}
\DeclareMathOperator{\card}{card}
\DeclareMathOperator{\Char}{char}
\DeclareMathOperator{\csp}{csp}
\DeclareMathOperator{\codim}{codim}
\DeclareMathOperator{\coker}{coker}
\DeclareMathOperator{\coh}{H}
\DeclareMathOperator{\compl}{compl}
\DeclareMathOperator{\conj}{conj}
\DeclareMathOperator{\cont}{cont}
\DeclareMathOperator{\crys}{crys}
\DeclareMathOperator{\Crys}{Crys}
\DeclareMathOperator{\cusp}{cusp}
\DeclareMathOperator{\diag}{diag}
\DeclareMathOperator{\disc}{disc}
\DeclareMathOperator{\dR}{dR}
\DeclareMathOperator{\Eis}{Eis}
\DeclareMathOperator{\End}{End}
\DeclareMathOperator{\ev}{ev}
\DeclareMathOperator{\eval}{eval}
\DeclareMathOperator{\Eq}{Eq}
\DeclareMathOperator{\Ext}{Ext}
\DeclareMathOperator{\Fil}{Fil}
\DeclareMathOperator{\Fitt}{Fitt}
\DeclareMathOperator{\Frob}{Frob}
\DeclareMathOperator{\G}{G}
\DeclareMathOperator{\Gal}{Gal}
\DeclareMathOperator{\GL}{GL}
\DeclareMathOperator{\Gr}{Gr}
\DeclareMathOperator{\Graph}{Graph}
\DeclareMathOperator{\GSp}{GSp}
\DeclareMathOperator{\GUn}{GU}
\DeclareMathOperator{\Hom}{Hom}
\DeclareMathOperator{\id}{id}
\DeclareMathOperator{\Id}{Id}
\DeclareMathOperator{\Ik}{Ik}
\DeclareMathOperator{\IM}{Im}
\DeclareMathOperator{\Image}{im}
\DeclareMathOperator{\Ind}{Ind}
\DeclareMathOperator{\Inf}{inf}
\DeclareMathOperator{\Isom}{Isom}
\DeclareMathOperator{\J}{J}
\DeclareMathOperator{\Jac}{Jac}
\DeclareMathOperator{\lcm}{lcm}
\DeclareMathOperator{\length}{length}
\DeclareMathOperator{\Log}{Log}
\DeclareMathOperator{\M}{M}
\DeclareMathOperator{\Mat}{Mat}
\DeclareMathOperator{\N}{N}
\DeclareMathOperator{\Nm}{Nm}
\DeclareMathOperator{\NIk}{N-Ik}
\DeclareMathOperator{\NSK}{N-SK}
\DeclareMathOperator{\new}{new}
\DeclareMathOperator{\obj}{obj}
\DeclareMathOperator{\old}{old}
\DeclareMathOperator{\ord}{ord}
\DeclareMathOperator{\Or}{O}
\DeclareMathOperator{\PGL}{PGL}
\DeclareMathOperator{\PGSp}{PGSp}
\DeclareMathOperator{\rank}{rank}
\DeclareMathOperator{\Rel}{Rel}
\DeclareMathOperator{\Real}{Re}
\DeclareMathOperator{\RES}{res}
\DeclareMathOperator{\Res}{Res}
%\DeclareMathOperator{\Sha}{\textcyr{Sh}}
\DeclareMathOperator{\Sel}{Sel}
\DeclareMathOperator{\semi}{ss}
\DeclareMathOperator{\sgn}{sign}
\DeclareMathOperator{\SK}{SK}
\DeclareMathOperator{\SL}{SL}
\DeclareMathOperator{\SO}{SO}
\DeclareMathOperator{\Sp}{Sp}
\DeclareMathOperator{\Span}{span}
\DeclareMathOperator{\Spec}{Spec}
\DeclareMathOperator{\spin}{spin}
\DeclareMathOperator{\st}{st}
\DeclareMathOperator{\St}{St}
\DeclareMathOperator{\SUn}{SU}
\DeclareMathOperator{\supp}{supp}
\DeclareMathOperator{\Sup}{sup}
\DeclareMathOperator{\Sym}{Sym}
\DeclareMathOperator{\Tam}{Tam}
\DeclareMathOperator{\tors}{tors}
\DeclareMathOperator{\tr}{tr}
\DeclareMathOperator{\un}{un}
\DeclareMathOperator{\Un}{U}
\DeclareMathOperator{\val}{val}
\DeclareMathOperator{\vol}{vol}

\DeclareMathOperator{\Sets}{S \mkern1.04mu e \mkern1.04mu t \mkern1.04mu s}
    \newcommand{\cSets}{\scalebox{1.02}{\contour{black}{$\Sets$}}}
    
\DeclareMathOperator{\Groups}{G \mkern1.04mu r \mkern1.04mu o \mkern1.04mu u \mkern1.04mu p \mkern1.04mu s}
    \newcommand{\cGroups}{\scalebox{1.02}{\contour{black}{$\Groups$}}}

\DeclareMathOperator{\TTop}{T \mkern1.04mu o \mkern1.04mu p}
    \newcommand{\cTop}{\scalebox{1.02}{\contour{black}{$\TTop$}}}

\DeclareMathOperator{\Htp}{H \mkern1.04mu t \mkern1.04mu p}
    \newcommand{\cHtp}{\scalebox{1.02}{\contour{black}{$\Htp$}}}

\DeclareMathOperator{\Mod}{M \mkern1.04mu o \mkern1.04mu d}
    \newcommand{\cMod}{\scalebox{1.02}{\contour{black}{$\Mod$}}}

\DeclareMathOperator{\Ab}{A \mkern1.04mu b}
    \newcommand{\cAb}{\scalebox{1.02}{\contour{black}{$\Ab$}}}

\DeclareMathOperator{\Rings}{R \mkern1.04mu i \mkern1.04mu n \mkern1.04mu g \mkern1.04mu s}
    \newcommand{\cRings}{\scalebox{1.02}{\contour{black}{$\Rings$}}}

\DeclareMathOperator{\ComRings}{C \mkern1.04mu o \mkern1.04mu m \mkern1.04mu R \mkern1.04mu i \mkern1.04mu n \mkern1.04mu g \mkern1.04mu s}
    \newcommand{\cComRings}{\scalebox{1.05}{\contour{black}{$\ComRings$}}}

\DeclareMathOperator{\hHom}{H \mkern1.04mu o \mkern1.04mu m}
    \newcommand{\cHom}{\scalebox{1.02}{\contour{black}{$\hHom$}}}

         %  \item $\cGroups$
          %  \item $\cTop$
          %  \item $\cHtp$
          %  \item $\cMod$




\renewcommand{\k}{\kappa}
\newcommand{\Ff}{F_{f}}
\newcommand{\ts}{\,^{t}\!}


%Mathcal

\newcommand{\cA}{\mathcal{A}}
\newcommand{\cB}{\mathcal{B}}
\newcommand{\cC}{\mathcal{C}}
\newcommand{\cD}{\mathcal{D}}
\newcommand{\cE}{\mathcal{E}}
\newcommand{\cF}{\mathcal{F}}
\newcommand{\cG}{\mathcal{G}}
\newcommand{\cH}{\mathcal{H}}
\newcommand{\cI}{\mathcal{I}}
\newcommand{\cJ}{\mathcal{J}}
\newcommand{\cK}{\mathcal{K}}
\newcommand{\cL}{\mathcal{L}}
\newcommand{\cM}{\mathcal{M}}
\newcommand{\cN}{\mathcal{N}}
\newcommand{\cO}{\mathcal{O}}
\newcommand{\cP}{\mathcal{P}}
\newcommand{\cQ}{\mathcal{Q}}
\newcommand{\cR}{\mathcal{R}}
\newcommand{\cS}{\mathcal{S}}
\newcommand{\cT}{\mathcal{T}}
\newcommand{\cU}{\mathcal{U}}
\newcommand{\cV}{\mathcal{V}}
\newcommand{\cW}{\mathcal{W}}
\newcommand{\cX}{\mathcal{X}}
\newcommand{\cY}{\mathcal{Y}}
\newcommand{\cZ}{\mathcal{Z}}


%mathfrak (missing \fi)

\newcommand{\fa}{\mathfrak{a}}
\newcommand{\fA}{\mathfrak{A}}
\newcommand{\fb}{\mathfrak{b}}
\newcommand{\fB}{\mathfrak{B}}
\newcommand{\fc}{\mathfrak{c}}
\newcommand{\fC}{\mathfrak{C}}
\newcommand{\fd}{\mathfrak{d}}
\newcommand{\fD}{\mathfrak{D}}
\newcommand{\fe}{\mathfrak{e}}
\newcommand{\fE}{\mathfrak{E}}
\newcommand{\ff}{\mathfrak{f}}
\newcommand{\fF}{\mathfrak{F}}
\newcommand{\fg}{\mathfrak{g}}
\newcommand{\fG}{\mathfrak{G}}
\newcommand{\fh}{\mathfrak{h}}
\newcommand{\fH}{\mathfrak{H}}
\newcommand{\fI}{\mathfrak{I}}
\newcommand{\fj}{\mathfrak{j}}
\newcommand{\fJ}{\mathfrak{J}}
\newcommand{\fk}{\mathfrak{k}}
\newcommand{\fK}{\mathfrak{K}}
\newcommand{\fl}{\mathfrak{l}}
\newcommand{\fL}{\mathfrak{L}}
\newcommand{\fm}{\mathfrak{m}}
\newcommand{\fM}{\mathfrak{M}}
\newcommand{\fn}{\mathfrak{n}}
\newcommand{\fN}{\mathfrak{N}}
\newcommand{\fo}{\mathfrak{o}}
\newcommand{\fO}{\mathfrak{O}}
\newcommand{\fp}{\mathfrak{p}}
\newcommand{\fP}{\mathfrak{P}}
\newcommand{\fq}{\mathfrak{q}}
\newcommand{\fQ}{\mathfrak{Q}}
\newcommand{\fr}{\mathfrak{r}}
\newcommand{\fR}{\mathfrak{R}}
\newcommand{\fs}{\mathfrak{s}}
\newcommand{\fS}{\mathfrak{S}}
\newcommand{\ft}{\mathfrak{t}}
\newcommand{\fT}{\mathfrak{T}}
\newcommand{\fu}{\mathfrak{u}}
\newcommand{\fU}{\mathfrak{U}}
\newcommand{\fv}{\mathfrak{v}}
\newcommand{\fV}{\mathfrak{V}}
\newcommand{\fw}{\mathfrak{w}}
\newcommand{\fW}{\mathfrak{W}}
\newcommand{\fx}{\mathfrak{x}}
\newcommand{\fX}{\mathfrak{X}}
\newcommand{\fy}{\mathfrak{y}}
\newcommand{\fY}{\mathfrak{Y}}
\newcommand{\fz}{\mathfrak{z}}
\newcommand{\fZ}{\mathfrak{Z}}


%mathbf

\newcommand{\bfA}{\mathbf{A}}
\newcommand{\bfB}{\mathbf{B}}
\newcommand{\bfC}{\mathbf{C}}
\newcommand{\bfD}{\mathbf{D}}
\newcommand{\bfE}{\mathbf{E}}
\newcommand{\bfF}{\mathbf{F}}
\newcommand{\bfG}{\mathbf{G}}
\newcommand{\bfH}{\mathbf{H}}
\newcommand{\bfI}{\mathbf{I}}
\newcommand{\bfJ}{\mathbf{J}}
\newcommand{\bfK}{\mathbf{K}}
\newcommand{\bfL}{\mathbf{L}}
\newcommand{\bfM}{\mathbf{M}}
\newcommand{\bfN}{\mathbf{N}}
\newcommand{\bfO}{\mathbf{O}}
\newcommand{\bfP}{\mathbf{P}}
\newcommand{\bfQ}{\mathbf{Q}}
\newcommand{\bfR}{\mathbf{R}}
\newcommand{\bfS}{\mathbf{S}}
\newcommand{\bfT}{\mathbf{T}}
\newcommand{\bfU}{\mathbf{U}}
\newcommand{\bfV}{\mathbf{V}}
\newcommand{\bfW}{\mathbf{W}}
\newcommand{\bfX}{\mathbf{X}}
\newcommand{\bfY}{\mathbf{Y}}
\newcommand{\bfZ}{\mathbf{Z}}

\newcommand{\bfa}{\mathbf{a}}
\newcommand{\bfb}{\mathbf{b}}
\newcommand{\bfc}{\mathbf{c}}
\newcommand{\bfd}{\mathbf{d}}
\newcommand{\bfe}{\mathbf{e}}
\newcommand{\bff}{\mathbf{f}}
\newcommand{\bfg}{\mathbf{g}}
\newcommand{\bfh}{\mathbf{h}}
\newcommand{\bfi}{\mathbf{i}}
\newcommand{\bfj}{\mathbf{j}}
\newcommand{\bfk}{\mathbf{k}}
\newcommand{\bfl}{\mathbf{l}}
\newcommand{\bfm}{\mathbf{m}}
\newcommand{\bfn}{\mathbf{n}}
\newcommand{\bfo}{\mathbf{o}}
\newcommand{\bfp}{\mathbf{p}}
\newcommand{\bfq}{\mathbf{q}}
\newcommand{\bfr}{\mathbf{r}}
\newcommand{\bfs}{\mathbf{s}}
\newcommand{\bft}{\mathbf{t}}
\newcommand{\bfu}{\mathbf{u}}
\newcommand{\bfv}{\mathbf{v}}
\newcommand{\bfw}{\mathbf{w}}
\newcommand{\bfx}{\mathbf{x}}
\newcommand{\bfy}{\mathbf{y}}
\newcommand{\bfz}{\mathbf{z}}

%blackboard bold

\newcommand{\bbA}{\mathbb{A}}
\newcommand{\bbB}{\mathbb{B}}
\newcommand{\bbC}{\mathbb{C}}
\newcommand{\bbD}{\mathbb{D}}
\newcommand{\bbE}{\mathbb{E}}
\newcommand{\bbF}{\mathbb{F}}
\newcommand{\bbG}{\mathbb{G}}
\newcommand{\bbH}{\mathbb{H}}
\newcommand{\bbI}{\mathbb{I}}
\newcommand{\bbJ}{\mathbb{J}}
\newcommand{\bbK}{\mathbb{K}}
\newcommand{\bbL}{\mathbb{L}}
\newcommand{\bbM}{\mathbb{M}}
\newcommand{\bbN}{\mathbb{N}}
\newcommand{\bbO}{\mathbb{O}}
\newcommand{\bbP}{\mathbb{P}}
\newcommand{\bbQ}{\mathbb{Q}}
\newcommand{\bbR}{\mathbb{R}}
\newcommand{\bbS}{\mathbb{S}}
\newcommand{\bbT}{\mathbb{T}}
\newcommand{\bbU}{\mathbb{U}}
\newcommand{\bbV}{\mathbb{V}}
\newcommand{\bbW}{\mathbb{W}}
\newcommand{\bbX}{\mathbb{X}}
\newcommand{\bbY}{\mathbb{Y}}
\newcommand{\bbZ}{\mathbb{Z}}

\newcommand{\bmat}{\left( \begin{matrix}}
\newcommand{\emat}{\end{matrix} \right)}

\newcommand{\pmat}{\left( \begin{smallmatrix}}
\newcommand{\epmat}{\end{smallmatrix} \right)}

\newcommand{\lat}{\mathscr{L}}
\newcommand{\mat}[4]{\begin{pmatrix}{#1}&{#2}\\{#3}&{#4}\end{pmatrix}}
\newcommand{\ov}[1]{\overline{#1}}
\newcommand{\res}[1]{\underset{#1}{\RES}\,}
\newcommand{\up}{\upsilon}

\newcommand{\tac}{\textasteriskcentered}

%mahesh macros
\newcommand{\tm}{\textrm}

%Comments
\newcommand{\com}[1]{\vspace{5 mm}\par \noindent
\marginpar{\textsc{Comment}} \framebox{\begin{minipage}[c]{0.95
\textwidth} \tt #1 \end{minipage}}\vspace{5 mm}\par}

\newcommand{\Bmu}{\mbox{$\raisebox{-0.59ex}
  {$l$}\hspace{-0.18em}\mu\hspace{-0.88em}\raisebox{-0.98ex}{\scalebox{2}
  {$\color{white}.$}}\hspace{-0.416em}\raisebox{+0.88ex}
  {$\color{white}.$}\hspace{0.46em}$}{}}  %need graphicx and xcolor. this produces blackboard bold mu 

\newcommand{\hooktwoheadrightarrow}{%
  \hookrightarrow\mathrel{\mspace{-15mu}}\rightarrow
}

\makeatletter
\newcommand{\xhooktwoheadrightarrow}[2][]{%
  \lhook\joinrel
  \ext@arrow 0359\rightarrowfill@ {#1}{#2}%
  \mathrel{\mspace{-15mu}}\rightarrow
}
\makeatother

\renewcommand{\geq}{\geqslant}
    \renewcommand{\leq}{\leqslant}
    
    \newcommand{\bone}{\mathbf{1}}
    \newcommand{\sign}{\mathrm{sign}}
    \newcommand{\eps}{\varepsilon}
    \newcommand{\textui}[1]{\uline{\textit{#1}}}
    
    %\newcommand{\ov}{\overline}
    %\newcommand{\un}{\underline}
    \newcommand{\fin}{\mathrm{fin}}
    
    \newcommand{\chnum}{\titleformat
    {\chapter} % command
    [display] % shape
    {\centering} % format
    {\Huge \color{black} \shadowbox{\thechapter}} % label
    {-0.5em} % sep (space between the number and title)
    {\LARGE \color{black} \underline} % before-code
    }
    
    \newcommand{\chunnum}{\titleformat
    {\chapter} % command
    [display] % shape
    {} % format
    {} % label
    {0em} % sep
    { \begin{flushright} \begin{tabular}{r}  \Huge \color{black}
    } % before-code
    [
    \end{tabular} \end{flushright} \normalsize
    ] % after-code
    }

\newcommand{\littletaller}{\mathchoice{\vphantom{\big|}}{}{}{}}
\newcommand\restr[2]{{% we make the whole thing an ordinary symbol
  \left.\kern-\nulldelimiterspace % automatically resize the bar with \right
  #1 % the function
  \littletaller % pretend it's a little taller at normal size
  \right|_{#2} % this is the delimiter
  }}

\newcommand{\mtext}[1]{\hspace{6pt}\text{#1}\hspace{6pt}}

%This adds a "front cover" page.
%{\thispagestyle{empty}
%\vspace*{\fill}
%\begin{tabular}{l}
%\begin{tabular}{l}
%\includegraphics[scale=0.24]{oxy-logo.png}
%\end{tabular} \\
%\begin{tabular}{l}
%\Large \color{black} Module Theory, Linear Algebra, and Homological Algebra \\ \Large \color{black} Gianluca Crescenzo
%\end{tabular}
%\end{tabular}
%\newpage





%%%%%%%%%%%%%%%%%%%%%%
%%%%DOCUMENT SETUP%%%%
%%%%%%%%%%%%%%%%%%%%%%
\setsecnumdepth{subsection}
\definecolor{darkgreen}{rgb}{0, 0.5976, 0}
\hypersetup{pdfauthor=Gianluca Crescenzo, pdftitle=Real Analysis I Notes, pdfstartview=FitH, colorlinks=true, linkcolor=darkgreen, citecolor=darkgreen}


%This adds a "front cover" page.
%{\thispagestyle{empty}
%\vspace*{\fill}
%\begin{tabular}{l}
%\begin{tabular}{l}
%\includegraphics[scale=0.24]{oxy-logo.png}
%\end{tabular} \\
%\begin{tabular}{l}
%\Large \color{black} Module Theory, Linear Algebra, and Homological Algebra \\ \Large \color{black} Gianluca Crescenzo
%\end{tabular}
%\end{tabular}
\begin{document}
\begin{center}
    { \Large Math 310 \\[0.1in]Homework 2 \\[0.1in]
    Due: 9/20/2024}\\[.25in]
    { Name:} {\underline{Gianluca Crescenzo\hspace*{2in}}}\\[0.15in]
    \end{center}
    \vspace{4pt}
%%%%%%%%%%%%%%%%%%%%%%%%%%%%%%%%%%%%%%%%%%%%%%%%%%%%%%%%%%%%%%%%%%%%%%%%%%%%%%%%%%%%%%%%%%%%%%%%%%%%%%%%%%%%%%%   
    \begin{exercise}
        Let $F$ be a field. Show that the following hold:
            \begin{enumerate}[label = (\roman*)]
                \item $-1(a) = -a$.
                \item $-(-a) = a$.
                \item $-(a+b) = (-a) + (-b)$.
                \item $(-a)^{-1} = -(a^{-1})$.
                \item $(ab)^{-1} = a^{-1}b^{-1}$.
            \end{enumerate}
    \end{exercise}
        \begin{proof}
            (i) Note that $(-1 +1)a = 0$. Distributing $a$ gives $(-1)a + a = 0$. Hence $(-1)a = -a$.

            (ii) $-(-a) = -1(-a)$ by part (i). Adding $1(-a)$ to both sides gives $-(-a) + (-a) = 0$. So $-(-a)$ is the additive inverse of $-a$; we denote this $a$. Hence $-(-a) = a$.

            (iii) $-(a+b) = (-1)(a+b) = (-1)a + (-1)b = (-a) + (-b)$.

            (iv) Note that $(-a)\cdot 1 = a$ implies $(-a)(a \cdot a^{-1}) = 1$. Further simplification yields $(-a)(-(a^{-1})) = 1$. So $-(a^{-1})$ is the multiplicative inverse of $-a$, which is denoted $(-a)^{-1}$. Hence $(-a)^{-1} = -(a^{-1})$.

            (v) From $ab = ab$, we have that $1 = ab(ab)^{-1}$. Then $a^{-1} = b(ab)^{-1}$, hence $a^{-1}b^{-1} = (ab)^{-1}$.

        
        \end{proof}
%%%%%%%%%%%%%%%%%%%%%%%%%%%%%%%%%%%%%%
    \begin{exercise}
        Consider the set $K := \{a+b\sqrt{2} \mid a,b \in \bfQ\}$. Show that:
            \begin{enumerate}[label = (\roman*)]
                \item If $x,y \in K$, then $x+y \in K$ and $xy \in K$.
                \item If $x \neq 0$, then $x^{-1} \in K$.
            \end{enumerate}
    \end{exercise}
        \begin{proof}
            Let $x,y \in K$. Then $x = a+b\sqrt{2}$ and $y = c+d\sqrt{2}$. So $x + y = a+b\sqrt{2} + c + d\sqrt{2} = (a+c) + (b+d)\sqrt{2} \in K$ (since $\bfQ$ is closed under addition). Similarly, $xy = (a+b\sqrt{2})(c+d\sqrt{2}) = (ac+2bd) + (ad+bc)\sqrt{2} \in K$.

            Let $x = (a+b\sqrt{2})$. Suppose there exists a $y \neq 0$ such that $y = c+d\sqrt{2}$. Let $(a+b\sqrt{2})(c+d\sqrt{2}) = 1$. Then $ac + 2bd + (bc  + ad)\sqrt{2} = 1$. We have the following system of equations:
                \begin{equation*}
                \begin{split}
                    bc + ad &= 0 \\
                    ac+2bd &= 1.
                \end{split}
                \end{equation*}
            Solving for $c$ and $d$ yields:
                \begin{equation*}
                \begin{split}
                    c &= \frac{-a}{-(a^2)+2b^2} \in \bfQ\\
                    d &= \frac{b}{-(a^2)+2b^2} \in \bfQ.
                \end{split}
                \end{equation*}
            Hence $\displaystyle\left(\frac{-a}{-(a^2)+2b^2}\right) + \left(\frac{b}{-(a^2)+2b^2}\right)\sqrt{2} = x^{-1} \in K$.
        \end{proof}
%%%%%%%%%%%%%%%%%%%%%%%%%%%%%%%%%%%%%%
    \begin{exercise}
        Suppose $F$ is a field admitting a subset $P \subseteq F$ with properties:
            \begin{enumerate}[label = (\arabic*)]
                \item If $x,y \in P$, then $x+y \in P$ and $xy \in P$.
                \item For all $x \in F$, $x \in P$ or $-x \in P$.
                \item If $x, -x \in P$, then $x=0$. 
            \end{enumerate}
        Show that there is an ordering on $F$ making it into an ordered field.
    \end{exercise}
        \begin{proof}
            Define $a \leq_P b$ if and only if $b - a \in P$. This ordering is reflexive because $a \leq_P a$ if and only if $a - a = 0 \in P$. The ordering is also transitive as follows: if $x \leq_p y \leq_p z$, then $y-x \in P$ and $z - y \in P$. Since $P$ is closed under addition, $y-x + (z-y) = z - x \in P$ implies $x \leq_p z$. The ordering is antisymmetric as follows: Let $x \leq_p y$ and $y \leq_p x$. Then $x-y \in P$ and $y-x = -(x-y) \in P$. Hence $x-y = 0$, implying $x = y$. Note that this ordering is total as well: if $x,y\in F$ then $x-y \in F$ by closure of fields under addition. Then either $x-y \in P$ or $-(x-y)=y-x \in P$. Hence $x \leq_P y$ or $y \leq_P x$. Thus $\leq_P$ is a total ordering on $F$. 
            
            
            
            
            Suppose $x \leq_P y$ and $s \leq_P t$. Then $y-x \in P$ and $t-s \in P$. Since $P$ is closed under addition, $(y-x) + (t-s) = (y+t) - (x + s) \in P$. Then by our definition $x + s \leq_P y+t$. Now consider $z \in P$, $z \neq 0$. Since $P$ is closed under multiplication, $(y-x)z \in P$. Simplifying yields $yz - xz \in P$, or equivalently $xz \leq_P yz$. This establishes $F$ as an ordered field.
        \end{proof}
%%%%%%%%%%%%%%%%%%%%%%%%%%%%%%%%%%%%%%
    \begin{exercise}
        Let $a,b \in \bfR$.
            \begin{enumerate}[label = (\roman*)]
                \item If $0 \leq a \leq \epsilon$ for all $\epsilon > 0$, then $a = 0$.
                \item If $a \leq b + \epsilon$ for all $\epsilon > 0 $, then $a \leq b$.
            \end{enumerate}
    \end{exercise}
        \begin{proof}
            (i) If $a = 0$ we are done. If $a > 0$, then $0 \leq \frac{1}{2}a \leq a$. Pick $\epsilon = \frac{1}{2}a$, then $a \leq \frac{1}{2}a$, a contradiction. Thus $a = 0$. (ii) If $a \leq b$ then we are done. If $a > b$, then $a-b \in \bfR^+$, hence $\frac{1}{2}(a-b) \in \bfR^+$. Take $\epsilon = \frac{1}{2}(a-b)$. Then $a \leq b + \frac{1}{2}(a-b)$ is equivalent to $a \leq \frac{1}{2}(a+b) \leq b$, which is a contradiction. Thus $a \leq b$.
        \end{proof}
%%%%%%%%%%%%%%%%%%%%%%%%%%%%%%%%%%%%%%
    \begin{exercise}
        If $a,b \in \bfR$, show that:
            \begin{equation*}
            \begin{split}
            \left(\frac{1}{2}(a+b)\right)^2 \leq \frac{1}{2}(a^2 + b^2)
            \end{split}
            \end{equation*}
        with equality if and only if $a = b$.
    \end{exercise}
        \begin{proof}
            Observe that:
                \begin{equation*}
                \begin{split}
                    0
                    &\leq \frac{1}{4}(a-b)^2 \\
                    &= \frac{1}{4}(a^2 - 2ab + b^2) \\
                    &= \frac{1}{4}a^2 - \frac{1}{2}ab + \frac{1}{4}b^2. \\
                \end{split}
                \end{equation*}
            Adding $\frac{1}{4}a^2 + \frac{1}{2}ab + \frac{1}{4}b^2$ to both sides and yields:
                \begin{equation*}
                \begin{split}
                    \frac{1}{4}(a^2 + 2ab + b^2) \leq \frac{1}{2}(a^2 + b^2).
                \end{split}
                \end{equation*}
            And upon further simplification we get the desired result:
                \begin{equation*}
                \begin{split}
                    \left(\frac{1}{2}(a+b)\right)^2 \leq \frac{1}{2}(a^2 + b^2).
                \end{split}
                \end{equation*}
            Note that we have equality if and only if $a=b$ by the first equation.
        \end{proof}
%%%%%%%%%%%%%%%%%%%%%%%%%%%%%%%%%%%%%%
    \begin{exercise}
        For $x \in \bfR$, show that $\sqrt{x^2} = |x|$.
    \end{exercise}
        \begin{proof}
            Observe the following maps:
                \begin{equation*}
                \begin{split}
                    \cdot^2:\bfR \rightarrow \bfR+ \mtext{defined by} x \mapsto x^2 \\
                    \sqrt{\cdot}:\bfR^+ \rightarrow \bfR^+ \mtext{defined by} x^2 \mapsto x.
                \end{split}
                \end{equation*}
            Let $x \in \bfR^+$. Then $\sqrt{x^2} = x$. Now let $-x \in \bfR^+$. Then $\sqrt{(-x)^2} = -x$. Hence $\sqrt{x^2} = |x|$.
        \end{proof}
%%%%%%%%%%%%%%%%%%%%%%%%%%%%%%%%%%%%%%
    \begin{exercise}
          Let $x,y,a,b \in \bfR$ and $\epsilon > 0$.  
    \end{exercise}
        \begin{enumerate}[label = (\roman*)]
            \item Show that $|x-a| < \epsilon$ if and only if $a - \epsilon < x < a + \epsilon$.
            \item If $a < x < b$, then $|x| < \max{\{|a|,|b|\}}$.
            \item If $a<x<b$ and $a < y < b$, show that $|x - y| < b- a$.
        \end{enumerate}
            \begin{proof}
                (i) By definition this is equal to $ - \epsilon < x - a < \epsilon$, which is equivalent to $a -\epsilon < x < a + \epsilon$.

                (ii) Let $a < x < b$. Suppose $x < 0$. Then $a < 0$. So $|x| = -x < -a = |a| \leq \max{\left\{ |a|,|b| \right\}}$. Thus $|x| < |a|$. Now suppose $x > 0$. Then $b > 0$. So $|x| = x < b = |b| \leq \max{\left\{ |a|,|b| \right\}}$. Hence $|x| < \max{\left\{ |a|,|b| \right\}}$.

                (iii) Note that $a < x < b$ is equivalent to $-b < -x < -a$. Hence $a-b < y-x < b-a$. Similarly, $a < y < b$ is equivalent to $-b < -y < -a$. Hence $a-b < x -y < b-a$. Thus $|x-y| < b-a$.
            \end{proof}
%%%%%%%%%%%%%%%%%%%%%%%%%%%%%%%%%%%%%%
    \begin{exercise}
        Find all $x \in \bfR$ that satisfy
            \begin{equation*}
            \begin{split}
                4 < \left| x+2 \right| + \left| x-1 \right| < 5.
            \end{split}
            \end{equation*}
    \end{exercise}
        \begin{proof}
            We proceed with cases. Case 1: $x+2 \in \bfR^+$ and $x-1 \in \bfR^+$. Then:
                \begin{equation*}
                    4 < x+2 + x-1 < 5,
                \end{equation*}
                \begin{equation*}
                    4 < 2x-1 < 5,
                \end{equation*}
                \begin{equation*}
                    \frac{5}{2} < x < 3.
                \end{equation*}
            Case 2: $-(x+2) \in \bfR^+$ and $x-1 \in \bfR^+$. Then:
                \begin{equation*}
                    4 < -x-2 + x + 1 <5,
                \end{equation*}
                \begin{equation*}
                    4 < -3 < 5,
                \end{equation*}
            which is a contradiction. Case 3: $x+2 \in \bfR^+$ and $-(x-1) \in \bfR^+$. Then:
                \begin{equation*}
                    4 < x+2 - x + 1 < 5,
                \end{equation*}
                \begin{equation*}
                    4 < 3 < 5,
                \end{equation*}
            which is a contradiction. Case 4: $-(x+2) \in \bfR^+$ and $-(x-1) \in \bfR^+$. Then:
                \begin{equation*}
                    4 < -x-2-x+1 < 5,
                \end{equation*}
                \begin{equation*}
                    5< -2x < 6,
                \end{equation*}
                \begin{equation*}
                    -5 > 2x > -6,
                \end{equation*}
                \begin{equation*}
                    -\frac{5}{2} > x > -3.
                \end{equation*}.
            Thus $x \in (-3,-\frac{5}{2}) \cup (\frac{5}{2},3)$.
        \end{proof}
%%%%%%%%%%%%%%%%%%%%%%%%%%%%%%%%%%%%%%
    \begin{exercise}
        Let $a,b \in \bfR$. Show that:
            \begin{equation*}
            \begin{split}
                \max{\{a,b\}} = \frac{1}{2}(a+b+\left|a-b\right|)\mtext{and}\min{\{a,b\}} = \frac{1}{2}(a+b-\left|a-b\right|).
            \end{split}
            \end{equation*}
    \end{exercise}
        \begin{proof}
          Without loss of generality, suppose $a \leq b$. Then $b - a \in \bfR^+$. Observe that:
            \begin{equation*}
            \begin{split}
                \max{\{a,b\}} = b &= \frac{1}{2}a + \frac{1}{2}b - \frac{1}{2}a +  \frac{1}{2}b \\
                & =\frac{1}{2}(a + b - a + b) \\
                & = \frac{1}{2}(a + b  + |a-b|). \\
            \end{split}
            \end{equation*}  

            \begin{equation*}
            \begin{split}
                \min{\{a,b\}} = a &= \frac{1}{2}a + \frac{1}{2}b + \frac{1}{2}a -  \frac{1}{2}b \\
                & = \frac{1}{2}(a+b+a-b) \\
                & = \frac{1}{2}(a+b-(b-a))\\
                & = \frac{1}{2}(a+b-|a-b|).\\
            \end{split}
            \end{equation*}
        \end{proof}
%%%%%%%%%%%%%%%%%%%%%%%%%%%%%%%%%%%%%%
    \begin{exercise}
        If $x\neq y \in \bfR$, show that there is a $\delta > 0$ such that $V_\delta(x) \cap V_\delta(y) = \emptyset$.
    \end{exercise}
        \begin{proof}
            Without loss of generality suppose $x < y$. Pick $\delta = \frac{|x-y|}{3}$. Suppose towards contradiction $t \in V_\delta(x) \cap V_\delta(y)$. Then $t \in V_\delta(x)$ and $t \in V_\delta(y)$. Hence $t \in (x- \frac{|x-y|}{3}, x + \frac{|x-y|}{3})$ and $t \in (y - \frac{|x-y|}{3}, y + \frac{|x-y|}{3})$. But this gives:
                \begin{equation*}
                \begin{split}
                    t < x+ \frac{|x-y|}{3} < y - \frac{|x-y|}{3} < t,
                \end{split}
                \end{equation*}
            which is a contradiction. Hence $V_\delta(x) \cap V_\delta(y) = \emptyset$.
        \end{proof}
\end{document}
