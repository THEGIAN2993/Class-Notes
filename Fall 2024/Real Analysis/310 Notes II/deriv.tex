\chapter{Differentiation}
Throughout, $I$ is an open interval.
\section{Differentiation}

    \begin{definition}
        \phantom{a}
        \begin{enumerate}[label = (\arabic*)]
            \item Let $f:I \rightarrow \bfR$ and $c \in I$ a cluster point. Then $f$ is \textit{differentiable at $c$} if:
            \begin{equation*}
            \begin{split}
                \limit_{x \rightarrow c} \frac{f(x)-f(c)}{x-c} := f'(c)
            \end{split}
            \end{equation*}
        exists and is finite. We say $f'(c)$ is the \textit{derivative of $f$ at $x = c$}.

            \item If $f$ is differentiable at every $c \in I$, we say $f$ is differentiable at $I$.
        \end{enumerate}
    \end{definition}

    \begin{example}
        Let $f(x) = ax + b$. Then:
            \begin{equation*}
            \begin{split}
                f'(c) = \limit_{x \rightarrow c} \frac{f(x) - f(c)}{x-c} = \limit_{x \rightarrow c}\frac{a(x-c)}{(x-c)} = a.
            \end{split}
            \end{equation*}
    \end{example}

    \begin{example}
        Let $f(x) = |x|$ and $c = 0$. Then:
            \begin{equation*}
            \begin{split}
                f'(0) = \limit_{x \rightarrow 0}\frac{|x| - |0|}{x-0} = \limit_{x \rightarrow 0}\frac{|x|}{x}.
            \end{split}
            \end{equation*}
        Since this limit does not exist, $f$ is not differentiable at $c = 0$.
    \end{example}

    \begin{example}
        Let $f(x) = \begin{cases} x^2\sin \left(\frac{1}{x}\right), & x \neq 0 \\ 0, & x = 0\end{cases}$. Is $f$ differentiable at $c = 0$?
    \end{example}
        \begin{solution}
            Observe that:
                \begin{equation*}
                \begin{split}
                    f'(0) = \limit_{x \rightarrow 0}\frac{f(x) - f(0)}{x-0} = \limit_{x \rightarrow 0} x \sin \left(\frac{1}{x}\right) = 0.
                \end{split}
                \end{equation*}
            Hence $f$ is differentiable at $c=0$.
        \end{solution}

    \begin{proposition}
        If $f$ is differentiable at $x=c$, then $f$ is continuous at $x= c$.
    \end{proposition}
        \begin{proof}
            Let $(x_n)_n$ be a sequence with $(x_n)_n \rightarrow c$, $x_n \neq c$. Note that:
                \begin{equation*}
                \begin{split}
                    |f(x_n) - f(c)| = \left|\frac{f(x_n) - f(c)}{x_n - c} ( x_n - c)\right| = f'(c)|x_n - c|.
                \end{split}
                \end{equation*}
            Since $f'(c)$ is a constant and $(x_n - c)_n \rightarrow 0$, by "Lemma" $(f(x_n))_n \rightarrow f(c)$.
        \end{proof}

    \begin{theorem}
        Let $f,g:I \rightarrow \bfR$ be differentiable at $x = c$.
            \begin{enumerate}[label = (\arabic*)]
                \item $(\alpha f + g)'(c) = \alpha f'(c) + g'(c)$;
                \item $(f\cdot g)'(c) = f'(c)g(c) = f(c)g'(c)$;
                \item $\ds \left(\frac{f}{g}\right)'(c) = \frac{f'(c)g(c) - f(c)g'(c)}{g(c)^2}$ provided $g(c) \neq 0$.
            \end{enumerate}
    \end{theorem}
        \begin{proof}
            (2) Let $(x_n)_n \in I^\bfN$ with $(x_n)_n \rightarrow c$, $x_n \neq c$. We have:
                \begin{equation*}
                \begin{split}
                    \frac{fg(x_n) - fg(c)}{x_n - c} 
                    &= \frac{f(x_n)g(x_n) - f(c)g(c)}{x_n - c} \\
                    & = \frac{f(x_n)g(x_n)- f(x_n)g(c) + f(x_n)g(c) - f(c)g(c)}{x_n - c} \\
                    & = f(x_n) \left(\frac{g(x_n) - g(c)}{x_n -c}\right) + g(c) \left( \frac{f(x_n) - f(c)}{x_n - c}\right) \\
                    & \xrightarrow{n \rightarrow \infty} f(c)g'(c) + g(c)f'(c). \qedhere
                \end{split}
                \end{equation*}
        \end{proof}

    \begin{proposition}[Power Rule]
        \phantom{a}
        \begin{enumerate}[label = (\arabic*)]
            \item If $f(x) = x^n$ for $n \in \bfN_0$, then $f'(x) = nx^{n-1}$.
            \item If $f(x) = x^n$ for $n \in \bfZ$, then $f'(x) = nx^{n-1}$.
            \item If $f(x) = x^r$ for $r \in \bfQ$ then $f'(x) = rx^{r-1}$.
        \end{enumerate}
    \end{proposition}
        \begin{proof}
            (1) Induction and product rule. (2) Induction and quotient rule. (3) Inverse function theorem.
        \end{proof}

    \begin{proposition}[Chain Rule]
        Let $I \xrightarrow{f} J \xrightarrow{g} \bfR$ and $\Ran(f) \subseteq J$. Then $(g \circ f)'(c) = g'(f(c)) \cdot f'(c)$ whenever $f$ is differentiable at $c$ and $g$ is differentiable at $f(c)$.
    \end{proposition}
        \begin{proof}
            Apply Careterodry's Theorem.
        \end{proof}

\section{The Pillars of Differentiation}
    \begin{definition}
        Let $I$ be an open interval and $f:I \rightarrow \bfR$.
        \begin{enumerate}[label = (\arabic*)]
            \item $f$ has a \textit{local minimum} if $(\exists \delta > 0)\ni (\forall x\in V_\delta(c))(f(x) \geq f(c))$
            \item $f$ has a \textit{local maximum} if $(\exists \delta > 0)\ni (\forall x\in V_\delta(c))(f(x) \leq f(c))$
        \end{enumerate}
    \end{definition}

    \begin{theorem}[Fermat's Theorem]
        If $f(x) = y$ has a local minimum or maximum at $x = c$, then $f'(c) = 0$ or $f'(c)$ does not exist.
    \end{theorem}
        \begin{proof}
            If $f'(c)$ does not exist, we are done. Assume $f'(c)$ exists and is finite. Assume $f'(c)$ is a local maximum. For $n$ large enough, $x_n \in V_\delta(c)$. Then:
                \begin{equation*}
                \begin{split}
                    \exists \delta>0 \ni f(x) \leq f(c) \forall x \in V_\delta(c).
                \end{split}
                \end{equation*}
            Let $(x_n)_n$ be a decreasing sequence with $(x_n)_n \rightarrow c$, $x_n \neq c$. Then:
                \begin{equation*}
                \begin{split}
                    \frac{f(x_n) - f(c)}{x_n - c} \leq 0,
                \end{split}
                \end{equation*}
            which implies $f'(c) \leq 0$. Now let $(x_n)_n$ be an increasing sequence with $(x_n)_n \rightarrow c$, $x_n \neq c$. For $n$ large enough, $x_n \in V_\delta(c)$. Then:
                \begin{equation*}
                \begin{split}
                    \frac{f(x_n) - f(c)}{x_n - c} \geq 0,
                \end{split}
                \end{equation*}
            which implies $f'(c) \geq 0$. By antisymmetry, $f'(c) = 0$.
        \end{proof}

    \begin{theorem}[Rolle's Theorem]
        Let $f:[a,b] \rightarrow \bfR$ be continuous (on $[a,b]$) and differentiable on $(a,b)$. Suppose that $f(a) = f(b)$. Then there exists $c \in (a,b)$ with $f'(c) = 0$.
    \end{theorem}
        \begin{proof}
            By the Extreme Value Theorem, there exists $x_M \in [a,b]$ with $\sup_{x \in [a,b]}f(x) = f(x_M)$. Again by the Extreme Value Theorem, there exists $x_m \in [a,b]$ with $\inf_{x \in [a,b]}f(x) = f(x_m)$.\nl
            
            If $x_M \neq a,b$, then by Fermat's Theorem, $f'(x_M) = 0$. \nl
            
            If $x_m \neq a,b$, then by Fermat's Theorem, $f'(x_m) = 0$. \nl

            If both of the above cases fail, then by our condition $f(x_m) = f(x_M)$. So $f(x) = K$ for some $k \in \bfR$. Then clearly $f'(c) = 0$ for all $c \in (a,b)$.
        \end{proof}

    \begin{theorem}[Mean Value Theorem]
        Let $f:[a,b]\rightarrow \bfR$ be continuous and differentiable on $(a,b)$. There exists $c \in (a,b)$ such that $f'(c) = \frac{f(b)-f(a)}{b-a}$.
    \end{theorem}
        \begin{proof}
            Consider $g:[a,b] \rightarrow \bfR$ defined by:
                \begin{equation*}
                \begin{split}
                    g(x) = f(x) - \left(\frac{f(b)-f(a)}{b-a}\right)(x-a)
                \end{split}
                \end{equation*}
            Then $g$ is certainly continuous on $[a,b]$ and differentiable on $(a,b)$ because $f$ is. \nl
            
            Note that $g(a) = f(a)$ and $g(b) = f(b)$. By Rolle's Theorem, there exists $c \in (a,b)$ such that:
                \begin{equation*}
                \begin{split}
                    0 &= g'(c) \\
                    & = f'(c) - \left(\frac{f(b)-f(a)}{b-a}\right)
                \end{split}
                \end{equation*}
            Whence $f'(c) = \frac{f(b)-f(a)}{b-a}$.
        \end{proof}

    \begin{corollary}
        ***Let $f:[a,b] \rightarrow \bfR$ be continuous and differentiable on $(a,b)$ with $f'(x) = 0$ for all $x \in (a,b)$. Then $f$ is constant.
    \end{corollary}
        \begin{proof}
            Let $x_1,x_2 \in [a,b]$ with $x_1 \neq x_2$. Without loss of generality, suppose $x_1 < x_2$. \nl
            
            Apply the Mean Value Theorem to $f$ on $[x_1,x_2]$. Then there exists $c \in (x_1,x_2)$ with $0 =f'(c) = \frac{f(x_2)-f(x_1)}{x_2 - x_1}$. \nl

            Simplifying the above equation gives $f(x_1) = f(x_2)$. Whence $f$ is constant.
        \end{proof}

    \begin{theorem}
        Let $I$ be an open interval and $f:I \rightarrow \bfR$ differentiable.
            \begin{enumerate}[label = (\arabic*)]
                \item $f$ is increasing on $I$ if and only if $f' \geq 0$;
                \item $f$ is decreasing on $I$ if and only if $f' \leq 0$.
            \end{enumerate}
    \end{theorem}
        \begin{proof}
            $\stackrel{\text{\tiny (1)}}{(\Rightarrow)}$ Let $c \in I$. Since $f$ is differentiable, the limit $f'(c)$ is defined, allowing us to write:
                \begin{equation*}
                \begin{split}
                    f'(c) = \limit_{x \rightarrow c^+}\frac{f(x) - f(c)}{x-c}.
                \end{split}
                \end{equation*}
            Since we are approaching $c$ from the right, and since $f$ is increasing, it must be that $f'(c) = \limit_{x \rightarrow c^+}\frac{f(x) - f(c)}{x-c} \geq 0$. \nl

            $\stackrel{\text{\tiny (1)}}{(\Leftarrow)}$ Let $x_1,x_2 \in I$ with $x_1 < x_2$. Apply the Mean Value Theorem to $f$ on $[x_1,x_2] \subseteq I$. Then there exists $c \in (x_1,x_2)$ with $f'(c) = \frac{f(x_2) - f(x_1)}{x_2 - x_1} \geq 0$. Since $x_2 - x_1 \geq 0$, it must be that $f(x_2) - f(x_1) \geq 0$. Thus $f(x_2) \geq f(x_1)$, establishing that $f$ is increasing. \nl
            
            $\stackrel{\text{\tiny (2)}}{(\Rightarrow)}$ This direction follows similarly. \nl
            
            $\stackrel{\text{\tiny (2)}}{(\Leftarrow)}$ This direction follows similarly.
        \end{proof}

    \begin{example}
        Show that $\sin : \bfR \rightarrow \bfR$ is Lipschitz.
    \end{example}
        \begin{solution}
            Pick $x,y \in \bfR$ and suppose without loss of generality that $x < y$. \nl
            
            Apply the Mean Value Theorem to $\sin$ on $[x,y]$. Then there exists $c \in (x,y)$ with:
                \begin{equation*}
                \begin{split}
                    \frac{\sin(y) - \sin(x)}{y-x} = \sin'(c) = \cos(c).
                \end{split}
                \end{equation*}
            Applying the absolute value to both sides gives:
                \begin{equation*}
                \begin{split}
                    \left|\frac{\sin(y) - \sin(x)}{y-x}\right| = |cos(c)| \leq 1.
                \end{split}
                \end{equation*}
            Multiplying $|y-x|$ on both sides gives the desired result.
        \end{solution}

    \begin{exercise}
        If $f:I \rightarrow \bfR$ is differentiable with $f'$ bounded on $I$, then $f$ is Lipschitz.
    \end{exercise}

    \iffalse
    \begin{example}
        Discuss $\sum_{k = 1}^\infty \sin \left(\frac{1}{k^2}\right)$.
    \end{example}
        \begin{solution}
            Observe that:
                \begin{equation*}
                \begin{split}
                    \sum_k \left|\sin \left(\frac{1}{k^2}\right)\right| \leq \sum_k \frac{1}{k^2}.
                \end{split}
                \end{equation*}
            By comparison, $\sum_{k} \sin \left(\frac{1}{k^2}\right)$ converges absolutely.
        \end{solution}
    \fi
    
    \begin{lemma}
        Let $I\subseteq \bfR$ be an interval, $f:I \rightarrow \bfR$ continuous, and $c \in I$ with $f$ differentiable at $x=c$.
            \begin{enumerate}[label = (\arabic*)]
                \item If $f'(c) > 0$, then there exists $\delta > 0$ such that $f(x) > f(c)$ for all $x \in (c,c+\delta)$.
                \item If $f'(c) < 0$, then there exists $\delta > 0$ such that $f(x) > f(c)$ for all $x \in (c-\delta,c)$.
            \end{enumerate}
    \end{lemma}
        \begin{proof}
            (1) $0 < f'(c) = \limit_{x \rightarrow c}\frac{f(x) - f(c)}{x-c} = \limit_{x \rightarrow c^+}\frac{f(x) - f(c)}{x-c}$ (if the limit exists then the one-sided limit exists). Setting $\epsilon = \frac{f'(c)}{2}$, there exists $\delta > 0$ such that for $x \in (c,c+\delta)$, $\frac{f(x) - f(c)}{x-c} \in V_\epsilon(f'(c))$. Equivalently, $\frac{f(x)-f(c)}{x-c} > \frac{f'(c)}{2} > 0$. Since $x-c > 0$ for $x \in (c,c+\delta)$, we get that $f(x) - f(c) > 0$ for $x \in (c,c+\delta)$. \nl
            
            (2) This follow similarly.
        \end{proof}

    \begin{theorem}[Darboux's Theorem]
        Let $f:[a,b] \rightarrow \bfR$ be differentiable. Let $k$ be a number strictly between $f'(a)$ and $f'(b)$. Then there exists $c \in (a,b)$ with $f'(c) = k$.
    \end{theorem}
        \begin{proof}
            Let $h(x) = kx - f(x)$ on $[a,b]$. Then $h$ is continuous on $[a,b]$. \nl
            
            By the Extreme Value Theorem, $h$ attains its supremum; i.e., there exists $c \in [a,b]$ with $h(c) \geq h(x)$ for all $x \in [a,b]$.\nl
            
            $h'(a) = k-f'(a)$ and $h'(b) = k-f'(b)$. Without loss of generality, suppose:
                \begin{equation*}
                \begin{split}
                    h'(a) &= k-f'(a) > 0 \\
                    h'(b) &= k-f'(b) < 0.
                \end{split}
                \end{equation*}
            By the previous lemma, there exists $\delta > 0$ such that $h(x) > h(a)$ on $(a,a + \delta)$. \nl
            
            By the previous lemma, there exists $\eta  > 0$ such that $h(x) > h(b)$ on $(b - \eta,b)$. \nl
            
            So $c \neq a$ and $c \neq b$. So $c \in (a,b)$. Thus $h'(c) = 0$ by Fermat's theorem. \nl
            
            Thus $f'(c) = k$.
        \end{proof}

    \begin{question}
        Does there exists $f:\bfR \rightarrow \bfR$ differentiable with $f'(x) = \sgn(x)$?
    \end{question}
        \begin{answer}
            No! $\sgn(x)$ does not satisfy the Intermediate Value Theorem on $[-1,1]$.
        \end{answer}

    \begin{corollary}
        Let $f:I \rightarrow \bfR$ be differentiable and $f' \neq 0$ on $I$. Then $f$ is monotone.
    \end{corollary}
        \begin{proof}
            Homework
        \end{proof}

    