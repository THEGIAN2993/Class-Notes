%%%%%%%PACKAGES%%%%%%%
\documentclass[10pt,twoside,openany]{memoir}
\usepackage[T1]{fontenc}
\usepackage[utf8]{inputenc}
\usepackage{titlesec}
\usepackage{anyfontsize}
\usepackage{fancybox}
\usepackage[dvipsnames,svgnames,x11names,hyperref]{xcolor}
\usepackage{enumerate}
\usepackage{amsfonts}
\usepackage{amsthm}
\usepackage{amsmath}
\usepackage{amssymb}
\usepackage{hyperref}
\usepackage{fullpage}
\usepackage{bm}
\usepackage{cprotect}
\usepackage{calligra}
\usepackage{emptypage}
\usepackage{titleps}
\usepackage{microtype}
\usepackage{float}
\usepackage{ocgx}
\usepackage{appendix}
\usepackage{graphicx}
\usepackage{pdfcomment}
\usepackage{enumitem}
\usepackage{mathtools}
\usepackage{tikz-cd}
\usepackage{relsize}
\usepackage[font=footnotesize,labelfont=bf]{caption}
\usepackage{changepage}
\usepackage{xcolor}
\usepackage{ulem}
\usepackage{pgfplots}
\usepackage{marginnote}
    \newcommand*{\mnote}[1]{ % <----------
    \checkoddpage
    \ifoddpage
        \marginparmargin{left}
    \else
        \marginparmargin{right}
    \fi
        \marginnote{\tiny \textcolor{oorange}{#1}}
    }
\usepackage{tgbonum}
\usepackage{datetime}
    \newdateformat{specialdate}{\THEYEAR\ \monthname\ \THEDAY}
\usepackage[margin=0.9in]{geometry}
    \setlength{\voffset}{-0.4in}
    \setlength{\headsep}{30pt}
%\usepackage{fancyhdr}
%    \fancyhf{}
%    \pagestyle{fancy}
%    \cfoot{\footnotesize \thepage}
%    \fancyhead[R]{\footnotesize \rightmark}
%    \fancyhead[L]{\footnotesize \leftmark}
\usepackage[T1]{fontenc}% http://ctan.org/pkg/fontenc
\usepackage[outline]{contour}% http://ctan.org/pkg/contour
    \renewcommand{\arraystretch}{1.5}
    \contourlength{0.4pt}
    \contournumber{10}%
\usepackage{letterspace}
\usepackage{verbatim}


%%%%%%%%%%%%%%%%%%%%%%
%%%%%%%%MACROS%%%%%%%%
%%%%%%%%%%%%%%%%%%%%%%
%to make the correct symbol for Sha
%\newcommand\cyr{%
%\renewcommand\rmdefault{wncyr}%
%\renewcommand\sfdefault{wncyss}%
%\renewcommand\encodingdefault{OT2}%
%\normalfont \selectfont} \DeclareTextFontCommand{\textcyr}{\cyr}


\DeclareMathOperator{\ab}{ab}
\newcommand{\absgal}{\G_{\bbQ}}
\DeclareMathOperator{\ad}{ad}
\DeclareMathOperator{\adj}{adj}
\DeclareMathOperator{\alg}{alg}
\DeclareMathOperator{\Alt}{Alt}
\DeclareMathOperator{\Ann}{Ann}
\DeclareMathOperator{\arith}{arith}
\DeclareMathOperator{\Aut}{Aut}
\DeclareMathOperator{\Be}{B}
\DeclareMathOperator{\card}{card}
\DeclareMathOperator{\Char}{char}
\DeclareMathOperator{\csp}{csp}
\DeclareMathOperator{\codim}{codim}
\DeclareMathOperator{\coker}{coker}
\DeclareMathOperator{\coh}{H}
\DeclareMathOperator{\compl}{compl}
\DeclareMathOperator{\conj}{conj}
\DeclareMathOperator{\cont}{cont}
\DeclareMathOperator{\crys}{crys}
\DeclareMathOperator{\Crys}{Crys}
\DeclareMathOperator{\cusp}{cusp}
\DeclareMathOperator{\diag}{diag}
\DeclareMathOperator{\disc}{disc}
\DeclareMathOperator{\dR}{dR}
\DeclareMathOperator{\Eis}{Eis}
\DeclareMathOperator{\End}{End}
\DeclareMathOperator{\ev}{ev}
\DeclareMathOperator{\eval}{eval}
\DeclareMathOperator{\Eq}{Eq}
\DeclareMathOperator{\Ext}{Ext}
\DeclareMathOperator{\Fil}{Fil}
\DeclareMathOperator{\Fitt}{Fitt}
\DeclareMathOperator{\Frob}{Frob}
\DeclareMathOperator{\G}{G}
\DeclareMathOperator{\Gal}{Gal}
\DeclareMathOperator{\GL}{GL}
\DeclareMathOperator{\Gr}{Gr}
\DeclareMathOperator{\Graph}{Graph}
\DeclareMathOperator{\GSp}{GSp}
\DeclareMathOperator{\GUn}{GU}
\DeclareMathOperator{\Hom}{Hom}
\DeclareMathOperator{\id}{id}
\DeclareMathOperator{\Id}{Id}
\DeclareMathOperator{\Ik}{Ik}
\DeclareMathOperator{\IM}{Im}
\DeclareMathOperator{\Image}{im}
\DeclareMathOperator{\Ind}{Ind}
\DeclareMathOperator{\Inf}{inf}
\DeclareMathOperator{\Isom}{Isom}
\DeclareMathOperator{\J}{J}
\DeclareMathOperator{\Jac}{Jac}
\DeclareMathOperator{\lcm}{lcm}
\DeclareMathOperator{\length}{length}
\DeclareMathOperator{\Log}{Log}
\DeclareMathOperator{\M}{M}
\DeclareMathOperator{\Mat}{Mat}
\DeclareMathOperator{\N}{N}
\DeclareMathOperator{\Nm}{Nm}
\DeclareMathOperator{\NIk}{N-Ik}
\DeclareMathOperator{\NSK}{N-SK}
\DeclareMathOperator{\new}{new}
\DeclareMathOperator{\obj}{obj}
\DeclareMathOperator{\old}{old}
\DeclareMathOperator{\ord}{ord}
\DeclareMathOperator{\Or}{O}
\DeclareMathOperator{\PGL}{PGL}
\DeclareMathOperator{\PGSp}{PGSp}
\DeclareMathOperator{\rank}{rank}
\DeclareMathOperator{\Rel}{Rel}
\DeclareMathOperator{\Real}{Re}
\DeclareMathOperator{\RES}{res}
\DeclareMathOperator{\Res}{Res}
%\DeclareMathOperator{\Sha}{\textcyr{Sh}}
\DeclareMathOperator{\Sel}{Sel}
\DeclareMathOperator{\semi}{ss}
\DeclareMathOperator{\sgn}{sign}
\DeclareMathOperator{\SK}{SK}
\DeclareMathOperator{\SL}{SL}
\DeclareMathOperator{\SO}{SO}
\DeclareMathOperator{\Sp}{Sp}
\DeclareMathOperator{\Span}{span}
\DeclareMathOperator{\Spec}{Spec}
\DeclareMathOperator{\spin}{spin}
\DeclareMathOperator{\st}{st}
\DeclareMathOperator{\St}{St}
\DeclareMathOperator{\SUn}{SU}
\DeclareMathOperator{\supp}{supp}
\DeclareMathOperator{\Sup}{sup}
\DeclareMathOperator{\Sym}{Sym}
\DeclareMathOperator{\Tam}{Tam}
\DeclareMathOperator{\tors}{tors}
\DeclareMathOperator{\tr}{tr}
\DeclareMathOperator{\un}{un}
\DeclareMathOperator{\Un}{U}
\DeclareMathOperator{\val}{val}
\DeclareMathOperator{\vol}{vol}

\DeclareMathOperator{\Sets}{S \mkern1.04mu e \mkern1.04mu t \mkern1.04mu s}
    \newcommand{\cSets}{\scalebox{1.02}{\contour{black}{$\Sets$}}}
    
\DeclareMathOperator{\Groups}{G \mkern1.04mu r \mkern1.04mu o \mkern1.04mu u \mkern1.04mu p \mkern1.04mu s}
    \newcommand{\cGroups}{\scalebox{1.02}{\contour{black}{$\Groups$}}}

\DeclareMathOperator{\TTop}{T \mkern1.04mu o \mkern1.04mu p}
    \newcommand{\cTop}{\scalebox{1.02}{\contour{black}{$\TTop$}}}

\DeclareMathOperator{\Htp}{H \mkern1.04mu t \mkern1.04mu p}
    \newcommand{\cHtp}{\scalebox{1.02}{\contour{black}{$\Htp$}}}

\DeclareMathOperator{\Mod}{M \mkern1.04mu o \mkern1.04mu d}
    \newcommand{\cMod}{\scalebox{1.02}{\contour{black}{$\Mod$}}}

\DeclareMathOperator{\Ab}{A \mkern1.04mu b}
    \newcommand{\cAb}{\scalebox{1.02}{\contour{black}{$\Ab$}}}

\DeclareMathOperator{\Rings}{R \mkern1.04mu i \mkern1.04mu n \mkern1.04mu g \mkern1.04mu s}
    \newcommand{\cRings}{\scalebox{1.02}{\contour{black}{$\Rings$}}}

\DeclareMathOperator{\ComRings}{C \mkern1.04mu o \mkern1.04mu m \mkern1.04mu R \mkern1.04mu i \mkern1.04mu n \mkern1.04mu g \mkern1.04mu s}
    \newcommand{\cComRings}{\scalebox{1.05}{\contour{black}{$\ComRings$}}}

\DeclareMathOperator{\hHom}{H \mkern1.04mu o \mkern1.04mu m}
    \newcommand{\cHom}{\scalebox{1.02}{\contour{black}{$\hHom$}}}

         %  \item $\cGroups$
          %  \item $\cTop$
          %  \item $\cHtp$
          %  \item $\cMod$




\renewcommand{\k}{\kappa}
\newcommand{\Ff}{F_{f}}
\newcommand{\ts}{\,^{t}\!}


%Mathcal

\newcommand{\cA}{\mathcal{A}}
\newcommand{\cB}{\mathcal{B}}
\newcommand{\cC}{\mathcal{C}}
\newcommand{\cD}{\mathcal{D}}
\newcommand{\cE}{\mathcal{E}}
\newcommand{\cF}{\mathcal{F}}
\newcommand{\cG}{\mathcal{G}}
\newcommand{\cH}{\mathcal{H}}
\newcommand{\cI}{\mathcal{I}}
\newcommand{\cJ}{\mathcal{J}}
\newcommand{\cK}{\mathcal{K}}
\newcommand{\cL}{\mathcal{L}}
\newcommand{\cM}{\mathcal{M}}
\newcommand{\cN}{\mathcal{N}}
\newcommand{\cO}{\mathcal{O}}
\newcommand{\cP}{\mathcal{P}}
\newcommand{\cQ}{\mathcal{Q}}
\newcommand{\cR}{\mathcal{R}}
\newcommand{\cS}{\mathcal{S}}
\newcommand{\cT}{\mathcal{T}}
\newcommand{\cU}{\mathcal{U}}
\newcommand{\cV}{\mathcal{V}}
\newcommand{\cW}{\mathcal{W}}
\newcommand{\cX}{\mathcal{X}}
\newcommand{\cY}{\mathcal{Y}}
\newcommand{\cZ}{\mathcal{Z}}


%mathfrak (missing \fi)

\newcommand{\fa}{\mathfrak{a}}
\newcommand{\fA}{\mathfrak{A}}
\newcommand{\fb}{\mathfrak{b}}
\newcommand{\fB}{\mathfrak{B}}
\newcommand{\fc}{\mathfrak{c}}
\newcommand{\fC}{\mathfrak{C}}
\newcommand{\fd}{\mathfrak{d}}
\newcommand{\fD}{\mathfrak{D}}
\newcommand{\fe}{\mathfrak{e}}
\newcommand{\fE}{\mathfrak{E}}
\newcommand{\ff}{\mathfrak{f}}
\newcommand{\fF}{\mathfrak{F}}
\newcommand{\fg}{\mathfrak{g}}
\newcommand{\fG}{\mathfrak{G}}
\newcommand{\fh}{\mathfrak{h}}
\newcommand{\fH}{\mathfrak{H}}
\newcommand{\fI}{\mathfrak{I}}
\newcommand{\fj}{\mathfrak{j}}
\newcommand{\fJ}{\mathfrak{J}}
\newcommand{\fk}{\mathfrak{k}}
\newcommand{\fK}{\mathfrak{K}}
\newcommand{\fl}{\mathfrak{l}}
\newcommand{\fL}{\mathfrak{L}}
\newcommand{\fm}{\mathfrak{m}}
\newcommand{\fM}{\mathfrak{M}}
\newcommand{\fn}{\mathfrak{n}}
\newcommand{\fN}{\mathfrak{N}}
\newcommand{\fo}{\mathfrak{o}}
\newcommand{\fO}{\mathfrak{O}}
\newcommand{\fp}{\mathfrak{p}}
\newcommand{\fP}{\mathfrak{P}}
\newcommand{\fq}{\mathfrak{q}}
\newcommand{\fQ}{\mathfrak{Q}}
\newcommand{\fr}{\mathfrak{r}}
\newcommand{\fR}{\mathfrak{R}}
\newcommand{\fs}{\mathfrak{s}}
\newcommand{\fS}{\mathfrak{S}}
\newcommand{\ft}{\mathfrak{t}}
\newcommand{\fT}{\mathfrak{T}}
\newcommand{\fu}{\mathfrak{u}}
\newcommand{\fU}{\mathfrak{U}}
\newcommand{\fv}{\mathfrak{v}}
\newcommand{\fV}{\mathfrak{V}}
\newcommand{\fw}{\mathfrak{w}}
\newcommand{\fW}{\mathfrak{W}}
\newcommand{\fx}{\mathfrak{x}}
\newcommand{\fX}{\mathfrak{X}}
\newcommand{\fy}{\mathfrak{y}}
\newcommand{\fY}{\mathfrak{Y}}
\newcommand{\fz}{\mathfrak{z}}
\newcommand{\fZ}{\mathfrak{Z}}


%mathbf

\newcommand{\bfA}{\mathbf{A}}
\newcommand{\bfB}{\mathbf{B}}
\newcommand{\bfC}{\mathbf{C}}
\newcommand{\bfD}{\mathbf{D}}
\newcommand{\bfE}{\mathbf{E}}
\newcommand{\bfF}{\mathbf{F}}
\newcommand{\bfG}{\mathbf{G}}
\newcommand{\bfH}{\mathbf{H}}
\newcommand{\bfI}{\mathbf{I}}
\newcommand{\bfJ}{\mathbf{J}}
\newcommand{\bfK}{\mathbf{K}}
\newcommand{\bfL}{\mathbf{L}}
\newcommand{\bfM}{\mathbf{M}}
\newcommand{\bfN}{\mathbf{N}}
\newcommand{\bfO}{\mathbf{O}}
\newcommand{\bfP}{\mathbf{P}}
\newcommand{\bfQ}{\mathbf{Q}}
\newcommand{\bfR}{\mathbf{R}}
\newcommand{\bfS}{\mathbf{S}}
\newcommand{\bfT}{\mathbf{T}}
\newcommand{\bfU}{\mathbf{U}}
\newcommand{\bfV}{\mathbf{V}}
\newcommand{\bfW}{\mathbf{W}}
\newcommand{\bfX}{\mathbf{X}}
\newcommand{\bfY}{\mathbf{Y}}
\newcommand{\bfZ}{\mathbf{Z}}

\newcommand{\bfa}{\mathbf{a}}
\newcommand{\bfb}{\mathbf{b}}
\newcommand{\bfc}{\mathbf{c}}
\newcommand{\bfd}{\mathbf{d}}
\newcommand{\bfe}{\mathbf{e}}
\newcommand{\bff}{\mathbf{f}}
\newcommand{\bfg}{\mathbf{g}}
\newcommand{\bfh}{\mathbf{h}}
\newcommand{\bfi}{\mathbf{i}}
\newcommand{\bfj}{\mathbf{j}}
\newcommand{\bfk}{\mathbf{k}}
\newcommand{\bfl}{\mathbf{l}}
\newcommand{\bfm}{\mathbf{m}}
\newcommand{\bfn}{\mathbf{n}}
\newcommand{\bfo}{\mathbf{o}}
\newcommand{\bfp}{\mathbf{p}}
\newcommand{\bfq}{\mathbf{q}}
\newcommand{\bfr}{\mathbf{r}}
\newcommand{\bfs}{\mathbf{s}}
\newcommand{\bft}{\mathbf{t}}
\newcommand{\bfu}{\mathbf{u}}
\newcommand{\bfv}{\mathbf{v}}
\newcommand{\bfw}{\mathbf{w}}
\newcommand{\bfx}{\mathbf{x}}
\newcommand{\bfy}{\mathbf{y}}
\newcommand{\bfz}{\mathbf{z}}

%blackboard bold

\newcommand{\bbA}{\mathbb{A}}
\newcommand{\bbB}{\mathbb{B}}
\newcommand{\bbC}{\mathbb{C}}
\newcommand{\bbD}{\mathbb{D}}
\newcommand{\bbE}{\mathbb{E}}
\newcommand{\bbF}{\mathbb{F}}
\newcommand{\bbG}{\mathbb{G}}
\newcommand{\bbH}{\mathbb{H}}
\newcommand{\bbI}{\mathbb{I}}
\newcommand{\bbJ}{\mathbb{J}}
\newcommand{\bbK}{\mathbb{K}}
\newcommand{\bbL}{\mathbb{L}}
\newcommand{\bbM}{\mathbb{M}}
\newcommand{\bbN}{\mathbb{N}}
\newcommand{\bbO}{\mathbb{O}}
\newcommand{\bbP}{\mathbb{P}}
\newcommand{\bbQ}{\mathbb{Q}}
\newcommand{\bbR}{\mathbb{R}}
\newcommand{\bbS}{\mathbb{S}}
\newcommand{\bbT}{\mathbb{T}}
\newcommand{\bbU}{\mathbb{U}}
\newcommand{\bbV}{\mathbb{V}}
\newcommand{\bbW}{\mathbb{W}}
\newcommand{\bbX}{\mathbb{X}}
\newcommand{\bbY}{\mathbb{Y}}
\newcommand{\bbZ}{\mathbb{Z}}

\newcommand{\bmat}{\left( \begin{matrix}}
\newcommand{\emat}{\end{matrix} \right)}

\newcommand{\pmat}{\left( \begin{smallmatrix}}
\newcommand{\epmat}{\end{smallmatrix} \right)}

\newcommand{\lat}{\mathscr{L}}
\newcommand{\mat}[4]{\begin{pmatrix}{#1}&{#2}\\{#3}&{#4}\end{pmatrix}}
\newcommand{\ov}[1]{\overline{#1}}
\newcommand{\res}[1]{\underset{#1}{\RES}\,}
\newcommand{\up}{\upsilon}

\newcommand{\tac}{\textasteriskcentered}

%mahesh macros
\newcommand{\tm}{\textrm}

%Comments
\newcommand{\com}[1]{\vspace{5 mm}\par \noindent
\marginpar{\textsc{Comment}} \framebox{\begin{minipage}[c]{0.95
\textwidth} \tt #1 \end{minipage}}\vspace{5 mm}\par}

\newcommand{\Bmu}{\mbox{$\raisebox{-0.59ex}
  {$l$}\hspace{-0.18em}\mu\hspace{-0.88em}\raisebox{-0.98ex}{\scalebox{2}
  {$\color{white}.$}}\hspace{-0.416em}\raisebox{+0.88ex}
  {$\color{white}.$}\hspace{0.46em}$}{}}  %need graphicx and xcolor. this produces blackboard bold mu 

\newcommand{\hooktwoheadrightarrow}{%
  \hookrightarrow\mathrel{\mspace{-15mu}}\rightarrow
}

\makeatletter
\newcommand{\xhooktwoheadrightarrow}[2][]{%
  \lhook\joinrel
  \ext@arrow 0359\rightarrowfill@ {#1}{#2}%
  \mathrel{\mspace{-15mu}}\rightarrow
}
\makeatother

\renewcommand{\geq}{\geqslant}
    \renewcommand{\leq}{\leqslant}
    
    \newcommand{\bone}{\mathbf{1}}
    \newcommand{\sign}{\mathrm{sign}}
    \newcommand{\eps}{\varepsilon}
    \newcommand{\textui}[1]{\uline{\textit{#1}}}
    
    %\newcommand{\ov}{\overline}
    %\newcommand{\un}{\underline}
    \newcommand{\fin}{\mathrm{fin}}
    
    \newcommand{\chnum}{\titleformat
    {\chapter} % command
    [display] % shape
    {\centering} % format
    {\Huge \color{black} \shadowbox{\thechapter}} % label
    {-0.5em} % sep (space between the number and title)
    {\LARGE \color{black} \underline} % before-code
    }
    
    \newcommand{\chunnum}{\titleformat
    {\chapter} % command
    [display] % shape
    {} % format
    {} % label
    {0em} % sep
    { \begin{flushright} \begin{tabular}{r}  \Huge \color{black}
    } % before-code
    [
    \end{tabular} \end{flushright} \normalsize
    ] % after-code
    }

\newcommand{\littletaller}{\mathchoice{\vphantom{\big|}}{}{}{}}
\newcommand\restr[2]{{% we make the whole thing an ordinary symbol
  \left.\kern-\nulldelimiterspace % automatically resize the bar with \right
  #1 % the function
  \littletaller % pretend it's a little taller at normal size
  \right|_{#2} % this is the delimiter
  }}

\newcommand{\mtext}[1]{\hspace{6pt}\text{#1}\hspace{6pt}}

%This adds a "front cover" page.
%{\thispagestyle{empty}
%\vspace*{\fill}
%\begin{tabular}{l}
%\begin{tabular}{l}
%\includegraphics[scale=0.24]{oxy-logo.png}
%\end{tabular} \\
%\begin{tabular}{l}
%\Large \color{black} Module Theory, Linear Algebra, and Homological Algebra \\ \Large \color{black} Gianluca Crescenzo
%\end{tabular}
%\end{tabular}
%\newpage

\newcommand{\TBC}{\textbf{TO BE CONTINUED}}
\theoremstyle{plain}
\newtheorem{theorem}{Theorem}[section]
\newtheorem{proposition}[theorem]{Proposition}
\newtheorem{corollary}[theorem]{Corollary}
\newtheorem{lemma}[theorem]{Lemma}

\theoremstyle{definition}
\newtheorem{definition}{Definition}[section]
\newtheorem{example}{Example}[section]
\newtheorem{exercise}{Exercise}
\newtheorem{note}{Note}[section]

\theoremstyle{remark}
\newtheorem{remark}[theorem]{Remark}
\newtheorem*{noproof}{Proof omitted}
\numberwithin{equation}{section}

\newenvironment{solution}[1]{\noindent \textbf{#1}:}{}

\newcommand{\NN}{\mathbf{N}}
\newcommand{\ZZ}{\mathbf{Z}}
\newcommand{\QQ}{\mathbf{Q}}
\newcommand{\RR}{\mathbf{R}}
\newcommand{\CC}{\mathbf{C}}
\newcommand{\HH}{\mathbf{H}}
\newcommand{\KK}{\mathbf{K}}
\newcommand{\FF}{\mathbf{F}}

\newcommand{\bRR}{\overline{\RR}}
\newcommand{\bRRp}{\overline{\RR}_{\geq 0}}

\renewcommand{\geq}{\geqslant}
\renewcommand{\leq}{\leqslant}

\newcommand{\bone}{\mathbf{1}}
\newcommand{\sign}{\mathrm{sign}}
\newcommand{\eps}{\varepsilon}
\newcommand{\textui}[1]{\uline{\textit{#1}}}

%\newcommand{\ov}{\overline}
%\newcommand{\un}{\underline}
\newcommand{\fin}{\mathrm{fin}}

\newcommand{\chnum}{\titleformat
{\chapter} % command
[display] % shape
{\centering} % format
{\Huge \color{black} \shadowbox{\thechapter}} % label
{-0.5em} % sep (space between the number and title)
{\LARGE \color{black} \underline} % before-code
}

\newcommand{\chunnum}{\titleformat
{\chapter} % command
[display] % shape
{} % format
{} % label
{0em} % sep
{ \begin{flushright} \begin{tabular}{r}  \Huge \color{black}
} % before-code
[
\end{tabular} \end{flushright} \normalsize
] % after-code
}

\titlespacing*{\chapter}
{0pt} % Left margin
{*1} % Space before the chapter number (increase this value for more space)
{0pt} % Space after the chapter number and before the title

%%%%%%%%%%%%%%%%%%%%%%%%%%%%%%%%%%%%%%%%%%%%%%%%%%%%%%%%%%%%%%%%%
%%%%%%%%%%%%%%%%%%%%%%%%%%%%%%%%%%%%%%%%%%%%%%%%%%%%%%%%%%%%%%%%%%%%%%%%%%%%%%%%%%%%%%%%%%%%%%%%%%%%%%%%%%%%%%%%%%%%%%%%%%%%%%%%%%%%%%%%%%%%%%%%%%%%%%%%%%%%%%%%%%%%%%%%%%%%%%%%%%%%%%%%%%%%%%%%%%%%%%%%%%%%%%%%%%%%%%%%%%%%%%%%%%%%%%%%%%%%%%%%%%%%%%%%%%%%%%%%%%%%%%%%%%%%%%%%%%%%%%%%%%%%%%%%%%%%%%%%%%%%%%%%%%%%%%%%%%%%%%%%%%%%%%%%%%%%%%%%%%%%%%%%%%%%%%%%%%%%%%%%%%%%%%%%%%%%%%%%%%%%%%%%%%%%%%%%%%%%%%%%%%%%%%%%%%%%%%%%%%%%%%%%%%%%%%%%%%%%%%%%%%%%%%%%%%%%%%%%%%%%%%%%%%%%%%%%%%%%%%%%%%%%%%%%%%%%%%%%%%%%%%%%%%%%%%%%%%%%%%%%%%%%%%%%%%%%%%%%%%%%%%%%%%%%%%%%%%%%%%%%%%%%%%%%%%%%%%%%%%%%%%%%%%%%%%%%%%%%%%%%%%%%%%%%%%%%%%%%%%%%%%%%%%%%%%%%%%%%%%%%%%%%%%%%%%%%%%%%%%%%%%%%%%%%%%%%%%%%%%%%%%%%%%%%%%%%%%%%%%%%%%%%%%%%%%%%%%%%%%%%%%%%%%%%%%%%%%%%%%%%%%%%%%%%%%%%%%%%%%%%%%%%%%%%%%%%%%%%%%%%%%%%%%%%%%%%%%%%%%%%%%%%%%%%%%%%%%%%%%%%%%%%%%%%%%%%%%%%%%%%%%%%%%%%%%%%%%%%%%%%%%%%%%%%%%%%%%%%%%%%%%%%%%%%%%%%%%%%%%%%%%%%%%%%%%%%%%%%%%%%%%%%%%%%%%%%%%%%%%%%%%%%%%%%%%%%%%%%%%%%%%%%%%%%%%%%%%%%%%%%%%%%%%%%%%%%%%%%%%%%%%%%%%%%%%%%%%%%%%%%%%%%%%%%%%%%%%%%%%%%%%%%%%%%%%%%%%%%%%%%%%%%%%%%%%%%%%%%%%%%%%%%%%%%%%%%%%%%%%%%%%%%%%%%%%%%%%%%%%%%%%%%%%%%%%%%%%%%%%%%%%%%%%%%%%%%%%%%%%%%%%%%%%%%%%%%%%%%%%%%%%%%%%%%%%%%%%%%%%%%%%%%%%%%%%%%%%%%%%%%%%%%%%%%%%%%%%%
\begin{document}
\begin{center}
    { \Large Math 395 \\[0.1in]Homework 1 \\[0.1in]
    Due: 9/6/2024}\\[.25in]
    { Name:} {\underline{Gianluca Crescenzo\hspace*{2in}}}\\[0.15in]
    { Collaborators:} {\underline{ Avinash Iyer, Noah Smith, Carly Venenciano, Ben Langer \hspace*{1in}}} \\
    \end{center}
    For these problems $F$ is assumed to be a field.
    \vspace{4pt}
    
    \begin{comment}
    \begin{exercise}
        Define
            \begin{equation*}
            \begin{split}
                \mathfrak{sl}_n(\bfQ) = \{X = (x_{i,j}) \in \Mat_n{(\bfQ)} \mid \Tr{X} = \sum_{i=1}^n x_{i,i} = 0 \}.
            \end{split}
            \end{equation*}
        Show that $\mathfrak{sl}_n(\bfQ)$ is a $\bfQ$-vector space.
    \end{exercise}
    \end{comment}
%%%%%%%%%%%%%%%%%%%%%%%%%%%%%%%%%%%%%%%%%%%%%%%%%%%%%%%%%%%%%
    \begin{comment}
    \begin{exercise}
            Consider the vector space $F^3$. Determine and justify your answer, whether each of the following are subspaces of $F^3$:
    \end{exercise}
    \end{comment}
%%%%%%%%%%%%%%%%%%%%%%%%%%%%%%%%%%%%%%%%%%%%%%%%%%%%%%%%%%%%%
    \addtocounter{exercise}{2}
    \begin{exercise}
        Let $V$ be an $F$-vector space.
        \begin{enumerate}[label = (\alph*)]
            \item Prove that an arbitrary intersection of subspaces of $V$ is again a subspace of $V$
            \item Prove that the union of two subspaces of $V$ is a subspace of $V$ if and only if one of the subspaces is contained in the other.
        \end{enumerate}
    \end{exercise}
        \begin{proof}
            Let $\{W_i\}_{i \in I}$ be an arbitrary collection of subspaces of $V$. Let $x,y \in \bigcap_{i \in I}W_i$ and $\alpha \in F$. Then $x,y \in W_i$ for all $i$. Hence $x+\alpha y \in W_i$ for all $i$ which gives $x+\alpha y \in \bigcap_{i \in I}W_i$, establishing (a).

            Let $U,W$ be subspaces of $V$. Let $u,w \in U \cup W$ and $\alpha \in F$. Without loss of generality suppose $U \subseteq W$ with $u \in U$ and $w\in W$. Then it is also the case that $u \in W$, hence $u + \alpha w \in W \subseteq U \cup W$. Conversely, suppose $U \cup W$ is an $F$-subspace of $V$. Assume towards contradiction that $U \not\subseteq W$ and $W \not\subseteq U$. Let $u \in U$,$u \not\in W$ and $w \in W$, $w \not\in U$. Since $U \cup W$ is an $F$-vector space, $u + w \in  U \cup W$. Without loss of generality, let $u + w \in U$, then $(-u) +u + w = w \in U$, a contradiction. Hence $U \subseteq W$ or $W \subseteq U$, establishing (b).
        \end{proof}
%%%%%%%%%%%%%%%%%%%%%%%%%%%%%%%%%%%%%%%%%%%%%%%%%%%%%%%%%%%%%
    \vspace{10pt}
    \begin{exercise}
        Let $T \in \Hom_F{(F,F)}$. Prove there exists $\alpha \in F$ so that $T(v) = \alpha v $ for every $v \in F$.
    \end{exercise}
        \begin{proof}
            Let $\beta \in F$, $\beta \neq 0$. Then $\{\beta\}$ forms a basis for $F$ as an $F$-vector space. Let $v \in F$, then $v = \beta v_0$ for some $v_0 \in F$. Observe that:
                \begin{equation*}
                \begin{split}
                    T(v)
                    & = T(\beta v_0) \\
                    & = v_0 T(\beta) \\
                    & = v \beta^{-1}T(\beta) \\
                    & = v T(\beta^{-1} \beta) \\
                    & = v T(1).
                \end{split}
                \end{equation*}
            From this, "$\alpha$" is uniquely determined by where $T(1)$ gets mapped to.
        \end{proof}
%%%%%%%%%%%%%%%%%%%%%%%%%%%%%%%%%%%%%%%%%%%%%%%%%%%%%%%%%%%%%
    \begin{comment}
    \begin{exercise}
        Let $U$, $V$, and $W$ be $F$-vector spaces. Let $S \in \Hom_F{(U,V)}$ and $T \in \Hom_F{(V,W)}$. Prove that $T \circ S \in \Hom_F{(U,W)}$.
    \end{exercise}
        {\color{red} \begin{proof}
            dude fix this shit


            Let $S$ and $T$ be as given and let $\alpha \in F$ and $x,y \in U$. Then:
            \begin{equation*}
            \begin{split}
                (T \circ S)(rx+y)
                & = \psi(\varphi(rx+y)) \\
                & = \psi(r\varphi(x) + \varphi(y)) \\
                & = r\psi(\varphi(x)) + \psi(\varphi(y)) \\
                & = r(\psi \circ \varphi)(x) + (\psi \circ \varphi)(y).
            \end{split}
            \end{equation*}
        Thus $\psi \circ \varphi$ is an $R$-module homomorphism.
        \end{proof}}
    \end{comment}
%%%%%%%%%%%%%%%%%%%%%%%%%%%%%%%%%%%%%%%%%%%%%%%%%%%%%%%%%%%%%
    \begin{comment}
    \begin{exercise}
        Let $V$ be an $F$-vector space. Prove that if $\{v_1,...,v_n\}$ is linearly independent in $V$, then so if the set $\{v_1 - v_2, v_2 - v_3,..., v_{n-1} - v_n , v_n \}$.
    \end{exercise}
    \end{comment}
%%%%%%%%%%%%%%%%%%%%%%%%%%%%%%%%%%%%%%%%%%%%%%%%%%%%%%%%%%%%%
    \begin{comment}
    \begin{exercise}
        Let $V$ be the subspace of $\bfR^5$ defined by 
            \begin{equation*}
            \begin{split}
                V = \{ (x_1,x_2,...,x_5) \in \bfR^5 \mid x_1 = 4x_4, x_2 = 5x_5 \}.
            \end{split}
            \end{equation*}
        Find a basis for $V$.
    \end{exercise}
    \end{comment}
%%%%%%%%%%%%%%%%%%%%%%%%%%%%%%%%%%%%%%%%%%%%%%%%%%%%%%%%%%%%%
    \begin{comment}
    \begin{exercise}
        Prove that there does not exist a $T \in \Hom_F{(F^5, F^2)}$ so that
            \begin{equation*}
            \begin{split}
                \ker{(T)} = \{(x_1,x_2,...,x_5) \in F^5 \mid x_1 = x_2 \hspace{4pt} \text{and} \hspace{4pt} x_3 = x_4 = x_5 \}.
            \end{split}
            \end{equation*}
    \end{exercise}
    \end{comment}
%%%%%%%%%%%%%%%%%%%%%%%%%%%%%%%%%%%%%%%%%%%%%%%%%%%%%%%%%%%%%
    \newpage
    \addtocounter{exercise}{4}
    \begin{exercise}
        Let $V$ be a finite dimensional vector space and $T \in \Hom_F{(V,V)}$ with $T^2 = T$.
            \begin{enumerate}[label = (\alph*)]
                \item Prove that $\Image{(T)} \cap \ker{(T)} = \{0\}$.
                \item Prove that $V = \Image{(T)} \oplus \ker{(T)}$.
            \end{enumerate}
    \end{exercise}
        \begin{proof}
            Let $v \in \Image{(T)} \cap \ker{(T)}$. Then $v \in \Image{(T)}$ and $v \in \ker{(T)}$. So there exists an element $w \in V$ such that $T(w) = v$, and $T(v) = 0$. Observe that $v = T(w) = T(T(w)) = T(v) = 0$.

            Let $x+y \in \Image{(T)} + \ker{(T)}$. Since $x \in \Image{(T)} \subseteq V$ and $y \in \ker{(T)} \subseteq V$, $x+y \in V$. Now let $v \in V$. Then $T(v) = w$ for some $w \in \Image{(T)}$. Let $k = v - T(w)$. Then $T(k) = T(v - T(w)) = T(v) - T(T(w)) = T(v) - T(v) = 0$, so $k \in \ker{(T)}$. Hence $v = T(w) + k \in \Image{(T)} + \ker{(T)}$, which gives $V = \Image{(T)} + \ker{(T)}$.
            
            We must now show that $\Image{(T)}$ and $\ker{(T)}$ are independent. If $T(w) + k = 0$, then $k = T(-w)$ implies $k \in \Image{(T)}$. So $k \in \Image{(T)} \cap \ker{(T)}$, and by (a) it must be that $k = 0$. Similarly, $T(T(w) + k) = 0$ is equivalent to $T(T(w)) + T(k) = 0$, which simplifies to $T(T(w)) = 0$; i.e., $T(w) \in \ker{(T)}$. So $T(w) \in \Image{(T)} \cap \ker{(T)}$, which gives that $T(w) = 0$. Thus $\Image{(T)}$ and $\ker{(T)}$ are independent, giving $V = \Image{(T)} \oplus \ker{(T)}$.
        \end{proof}
%%%%%%%%%%%%%%%%%%%%%%%%%%%%%%%%%%%%%%%%%%%%%%%%%%%%%%%%%%%%%
    \begin{comment}
    \begin{exercise}
        Let $T \in \Hom_F{(V,F)}$. Prove that if $v \in V$ is not in $\ker{(T)}$, then
            \begin{equation*}
            \begin{split}
                V = \ker{(T)} \oplus \{cv \mid c \in F\}.
            \end{split}
            \end{equation*}
    \end{exercise}
    \end{comment}
%%%%%%%%%%%%%%%%%%%%%%%%%%%%%%%%%%%%%%%%%%%%%%%%%%%%%%%%%%%%%
    \begin{comment}
    \begin{exercise}
        Let $V_1$,$V_2$ be subspaces of the finite dimensional vector space $V$. Prove
            \begin{equation*}
            \begin{split}
                \dim_F{(V_1 + V_2)} = \dim_F{(V_1)} + \dim_F{(V_2)} - \dim_F{(V_1 \cap V_2)}.
            \end{split}
            \end{equation*}
    \end{exercise} 
    \end{comment}
%%%%%%%%%%%%%%%%%%%%%%%%%%%%%%%%%%%%%%%%%%%%%%%%%%%%%%%%%%%%%
    \begin{comment}
    \begin{exercise}
        Suppose that $V$ and $W$ are both $5$-dimensional $\bfR$-subspaces of $\bfR^9$. Prove that $V \cap W \neq \{0\}$. 
    \end{exercise}
    \end{comment}
%%%%%%%%%%%%%%%%%%%%%%%%%%%%%%%%%%%%%%%%%%%%%%%%%%%%%%%%%%%%%
    \begin{comment}
    \begin{exercise}
        Let $p$ be a prime and $V$ a dimension $n$ vector space over $\bfF_p$. Show there are
            \begin{equation*}
            \begin{split}
                (p^n - 1)(p^n - p)(p^n - p^2)...(p^n - p^{n-1})
            \end{split}
            \end{equation*}
        distinct bases of $V$.
    \end{exercise}
    \end{comment}
%%%%%%%%%%%%%%%%%%%%%%%%%%%%%%%%%%%%%%%%%%%%%%%%%%%%%%%%%%%%%
    \addtocounter{exercise}{4}
    \vspace{10pt}
    \begin{exercise}
        Let $V$ be an $F$-vector space of dimension $n$. Let $T \in \Hom_F{(V,V)}$ so that $T^2 = 0$. Prove that the image of $T$ is contained in the kernel of $T$ and hence the dimension of the image of $T$ is at most $n/2$.
    \end{exercise}
        \begin{proof}
            Let $v \in \Image{(T)}$. Then there exists a $w \in V$ such that $T(w) = v$. But observe that $T(v) = T(T(w)) = T^2(w) = 0$, hence $v \in \ker{(T)}$ which establishes the containment $\Image{(T)} \subseteq \ker{(T)}$. By the rank-nullity theorem $n = \dim_F{(\Image{(T)})} + \dim_F{(\ker{(T)})}$. If $\Image{(T)} = \ker{(T)}$ then it must be the case that $\dim_F{(\Image{(T)})} = \dim_F{(\ker{(T)})} = n/2$, otherwise (when $\Image{(T)} \subset \ker{(T)}$) $\dim_F{(\Image{(T)})} < n/2$.
        \end{proof}
%%%%%%%%%%%%%%%%%%%%%%%%%%%%%%%%%%%%%%%%%%%%%%%%%%%%%%%%%%%%%
    \begin{comment}
    \begin{exercise}
        Let $T \in \Hom_F{(V,V)}$.
            \begin{enumerate}[label = (\alph*)]
                \item Give an example to show that one does not always have $V \cong \ker{(T)} \oplus \Image{(T)}$.
                \item Show that $\ker{(T^j)} \subset \ker{(T^{j+1})}$ for all $j \geq 1$. Prove that this sequence stabilizes; i.e., there exists $m \geq 1$ so that $\ker{(T^{m+j})} = \ker{(T^m)}$ for all $j \geq 1$. The subspace $\ker{(T^m)}$ is called the \textui{eventual kernel} and denoted $\ker{(T^\infty)}$.
                \item Show that $\Image{(T^{j})} \supset \Image{(T^{j+1})}$ for all $j \geq 1$. Prove that this sequence stabilizes; i.e., there exists $m \geq 1$ so that $\Image{(T^{m+j})} = \Image{(T^{m})}$ for all $j \geq 1$. The subspace $\Image{(T^{m})}$ is called the \textui{eventual image} and denoted $\Image{(T^{\infty})}$.
                \item Prove that $V \cong \ker{(T^\infty)} \oplus \Image{(T^\infty)}$.
            \end{enumerate}
    \end{exercise}
    \end{comment}
%%%%%%%%%%%%%%%%%%%%%%%%%%%%%%%%%%%%%%%%%%%%%%%%%%%%%%%%%%%%
    \begin{exercise}
        Let $W$ be a subspace of a finite dimensional vector space $V$. Let $T \in \Hom_F{(V,V)}$ so that $T(W) \subset W$. Show that $T$ induces a linear transformation $\overline{T} \in \Hom_F{(V/W,V/W)}$. Prove that $T$ is nonsingular (i.e., injective) on $V$ if and only if $T$ is restricted to $W$ and $\overline{T}$ on $V/W$ are both nonsingular.
    \end{exercise}
        \begin{proof}
            Define $\overline{T}: V/W \rightarrow V/W$ by $v +W \mapsto T(v) + W$. We must first show that $\overline{T}$ is well-defined. Suppose $v_1 + W = v_2 + W$, then $v_1 = v_2 + w$ for some $w \in W$. Observe that:
                \begin{equation*}
                \begin{split}
                    \overline{T}(v_1 + W)
                    & = T(v_1) + W \\
                    & = T(v_2 + w) + W \\
                    & = T(v_2) + T(w) + W \hspace{15pt} \text{\tiny Since $T \in \Hom_F{(V,V)}$}\\
                    & = T(v_2) + W \hspace{53pt} \text{\tiny Since $T(W) \subset W$} \\
                    & = \overline{T}(v_2 + W).
                \end{split}
                \end{equation*}
            Let $v_1 + W , v_2 + W \in V/W$, and $\alpha \in F$. Then:
                \begin{equation*}
                \begin{split}
                    \overline{T}((v_1 + W) + \alpha(v_2 + W))
                    & = \overline{T}((v_1 + W) + (\alpha v_2 + W)) \\
                    & = \overline{T}((v_1 + \alpha v_2) + W) \\
                    & = T(v_1 + \alpha v_2) + W \\
                    & = T(v_1) + \alpha T(v_2) + W \\
                    & = (T(v_1) + W) + \alpha(T(v_2) + W) \\
                    & = \overline{T}(v_1 + W) + \alpha \overline{T}(v_2 + W),
                \end{split}
                \end{equation*}
            hence $\overline{T} \in \Hom_F{(V/W, V/W)}$. Now consider the maps $V \xrightarrow{T} V \xrightarrow{\pi} V/W$, where $\pi:V \rightarrow V/W$ is the projection map. It can be proved that $\pi \circ T = \overline{T} \circ \pi$ as follows: let $v \in V$ and observe that $\pi(T(v)) = T(v) + W$, which is equivalent to $\overline{T}(\pi(v)) = \overline{T}(v + W) = T(v) + W$. We've established that the following diagram commutes:
            \begin{center}
                \begin{tikzcd}
                    V \arrow[d, "\pi"'] \arrow[r, "T"] & V \arrow[d, "\pi"] \\
                    V/W \arrow[r, "\overline{T}"']     & V/W \hspace{4pt}.              
                    \end{tikzcd}
            \end{center}

            
            Now assume $T$ is injective \textemdash by inspection one can see that $T\mid_W : W \rightarrow V$ is injective. Since $\overline{T}$ is defined by $v + W \mapsto T(v)+W$, it must be injective as well. Conversely, let $T\mid_W$ and $\overline{T}$ be injective. Let $v \in V$ with $v \in \ker{(T)}$. Then $T(v) = 0$ is equivalent to $\pi(T(v)) = 0 + W$. Using the fact that the above diagram commutes, we can write $\overline{T}(\pi(v)) = 0 + W$. Since $\overline{T}$ is injective, its kernel is trivial, hence it must be the case that $\pi(v) = 0 +W$. Thus $v \in W$, giving $T\mid_W (v) = 0$. Again, since $T\mid_W$ is injective, its kernel is trivial, establshing that $v= 0$. Thus $T$ is injective.
        \end{proof}
\end{document}
