%%%%%%%PACKAGES%%%%%%%
\documentclass[10pt,twoside,openany]{memoir}


\usepackage[p,osf]{scholax}
% T1 and textcomp are loaded by package. Change that here, if you want
% load sans and typewriter packages here, if needed
\usepackage{amsmath,amsthm}% must be loaded before newtxmath
% amssymb should not be loaded
\usepackage[scaled=1.075,ncf,vvarbb]{newtxmath}% need to scale up math package
% vvarbb selects the STIX version of blackboard bold.


\usepackage{titlesec}
    \titlespacing*{\chapter}
    {0pt} % Left margin
    {*1} % Space before the chapter number (increase this value for more space)
    {0pt} % Space after the chapter number and before the title
\usepackage{anyfontsize}
\usepackage{fancybox}
\usepackage[dvipsnames,svgnames,x11names,hyperref]{xcolor}
\usepackage{enumerate}
\usepackage{comment}
\usepackage{amsfonts}
\usepackage{amsthm}
\usepackage{mathrsfs}
\usepackage{hyperref}
\usepackage{fullpage}
\usepackage{bm}
\usepackage{cprotect}
\usepackage{calligra}
\usepackage{emptypage}
\usepackage{titleps}
\usepackage{microtype}
\usepackage{float}
\usepackage{ocgx}
\usepackage{appendix}
\usepackage{graphicx}
\usepackage{pdfcomment}
\usepackage{enumitem}
\usepackage{mathtools}
\usepackage{tikz-cd}
\usepackage{relsize}
\usepackage[font=footnotesize,labelfont=bf]{caption}
\usepackage{changepage}
\usepackage{xcolor}
\usepackage{ulem}
\usepackage{pgfplots}
\usepackage{marginnote}
    \newcommand*{\mnote}[1]{ % <----------
    \checkoddpage
    \ifoddpage
        \marginparmargin{left}
    \else
        \marginparmargin{right}
    \fi
        \marginnote{\tiny \textcolor{oorange}{#1}}
    }
\usepackage{datetime}
    \newdateformat{specialdate}{\THEYEAR\ \monthname\ \THEDAY}
\usepackage[margin=0.9in]{geometry}
    \setlength{\voffset}{-0.4in}
    \setlength{\headsep}{30pt}
\usepackage{fancyhdr}
    \fancyhf{}
    \cfoot{\footnotesize \thepage}
    \fancyhead[R]{\footnotesize \rightmark}
    \fancyhead[L]{\footnotesize \leftmark}
\usepackage[outline]{contour}% http://ctan.org/pkg/contour
    \renewcommand{\arraystretch}{1.5}
    \contourlength{0.4pt}
    \contournumber{10}%
\usepackage{letterspace}
    \linespread{1.25}
\usepackage{thmtools}
    \declaretheoremstyle[
        spaceabove=6pt, spacebelow=6pt,
        headfont=\normalfont\bfseries,
        notefont=\mdseries, notebraces={(}{)},
        bodyfont=\normalfont,
        postheadspace=1em
        %qed=\qedsymbol
        ]{mystyle}

    \declaretheorem[
        numberwithin=section,
        %shaded={
        %    rulecolor={RGB}{255,250,177},
        %    rulewidth=5pt, 
        %    bgcolor={RGB}{255,250,177}}
    ]{theorem}

    \declaretheorem[
        sibling=theorem,
        numberwithin=section,
        %shaded={
        %    rulecolor=Lavender,
        %    rulewidth=5pt, 
        %    bgcolor=Lavender}
    ]{lemma}

    \declaretheorem[
        sibling=theorem,
        numberwithin=section,
        %shaded={
        %    rulecolor=Lavender,
        %    rulewidth=5pt, 
        %    bgcolor=Lavender}
    ]{proposition}

    \declaretheorem[
        sibling=theorem,
        numberwithin=section,
        %shaded={
        %    rulecolor=Lavender,
        %    rulewidth=5pt, 
        %    bgcolor=Lavender}
    ]{corollary}

    \declaretheorem[
        numberwithin=section,
        style=mystyle,
        %shaded={
        %    rulecolor={RGB}{233,233,233},
        %    rulewidth=5pt, 
        %    bgcolor={RGB}{233,233,233}}
    ]{example}

    \declaretheorem[
        numberwithin=section,
        style=mystyle,
        %shaded={
        %    rulecolor={RGB}{255,250,177},
        %    rulewidth=5pt, 
        %    bgcolor={RGB}{255,250,177}}
    ]{definition}

    \declaretheorem[
        %shaded={
        %    rulecolor=Lavender,
        %    rulewidth=5pt, 
        %    bgcolor=Lavender}
    ]{exercise}
\declaretheorem[numbered=unless unique,style=mystyle]{Note}

%%%%%%%%MACROS%%%%%%%%
%to make the correct symbol for Sha
%\newcommand\cyr{%
%\renewcommand\rmdefault{wncyr}%
%\renewcommand\sfdefault{wncyss}%
%\renewcommand\encodingdefault{OT2}%
%\normalfont \selectfont} \DeclareTextFontCommand{\textcyr}{\cyr}


\DeclareMathOperator{\ab}{ab}
\newcommand{\absgal}{\G_{\bbQ}}
\DeclareMathOperator{\ad}{ad}
\DeclareMathOperator{\adj}{adj}
\DeclareMathOperator{\alg}{alg}
\DeclareMathOperator{\Alt}{Alt}
\DeclareMathOperator{\Ann}{Ann}
\DeclareMathOperator{\arith}{arith}
\DeclareMathOperator{\Aut}{Aut}
\DeclareMathOperator{\Be}{B}
\DeclareMathOperator{\card}{card}
\DeclareMathOperator{\Char}{char}
\DeclareMathOperator{\csp}{csp}
\DeclareMathOperator{\codim}{codim}
\DeclareMathOperator{\coker}{coker}
\DeclareMathOperator{\coh}{H}
\DeclareMathOperator{\compl}{compl}
\DeclareMathOperator{\conj}{conj}
\DeclareMathOperator{\cont}{cont}
\DeclareMathOperator{\crys}{crys}
\DeclareMathOperator{\Crys}{Crys}
\DeclareMathOperator{\cusp}{cusp}
\DeclareMathOperator{\diag}{diag}
\DeclareMathOperator{\disc}{disc}
\DeclareMathOperator{\dR}{dR}
\DeclareMathOperator{\Eis}{Eis}
\DeclareMathOperator{\End}{End}
\DeclareMathOperator{\ev}{ev}
\DeclareMathOperator{\eval}{eval}
\DeclareMathOperator{\Eq}{Eq}
\DeclareMathOperator{\Ext}{Ext}
\DeclareMathOperator{\Fil}{Fil}
\DeclareMathOperator{\Fitt}{Fitt}
\DeclareMathOperator{\Frob}{Frob}
\DeclareMathOperator{\G}{G}
\DeclareMathOperator{\Gal}{Gal}
\DeclareMathOperator{\GL}{GL}
\DeclareMathOperator{\Gr}{Gr}
\DeclareMathOperator{\Graph}{Graph}
\DeclareMathOperator{\GSp}{GSp}
\DeclareMathOperator{\GUn}{GU}
\DeclareMathOperator{\Hom}{Hom}
\DeclareMathOperator{\id}{id}
\DeclareMathOperator{\Id}{Id}
\DeclareMathOperator{\Ik}{Ik}
\DeclareMathOperator{\IM}{Im}
\DeclareMathOperator{\Image}{im}
\DeclareMathOperator{\Ind}{Ind}
\DeclareMathOperator{\Inf}{inf}
\DeclareMathOperator{\Isom}{Isom}
\DeclareMathOperator{\J}{J}
\DeclareMathOperator{\Jac}{Jac}
\DeclareMathOperator{\lcm}{lcm}
\DeclareMathOperator{\length}{length}
\DeclareMathOperator{\Log}{Log}
\DeclareMathOperator{\M}{M}
\DeclareMathOperator{\Mat}{Mat}
\DeclareMathOperator{\N}{N}
\DeclareMathOperator{\Nm}{Nm}
\DeclareMathOperator{\NIk}{N-Ik}
\DeclareMathOperator{\NSK}{N-SK}
\DeclareMathOperator{\new}{new}
\DeclareMathOperator{\obj}{obj}
\DeclareMathOperator{\old}{old}
\DeclareMathOperator{\ord}{ord}
\DeclareMathOperator{\Or}{O}
\DeclareMathOperator{\PGL}{PGL}
\DeclareMathOperator{\PGSp}{PGSp}
\DeclareMathOperator{\rank}{rank}
\DeclareMathOperator{\Rel}{Rel}
\DeclareMathOperator{\Real}{Re}
\DeclareMathOperator{\RES}{res}
\DeclareMathOperator{\Res}{Res}
%\DeclareMathOperator{\Sha}{\textcyr{Sh}}
\DeclareMathOperator{\Sel}{Sel}
\DeclareMathOperator{\semi}{ss}
\DeclareMathOperator{\sgn}{sign}
\DeclareMathOperator{\SK}{SK}
\DeclareMathOperator{\SL}{SL}
\DeclareMathOperator{\SO}{SO}
\DeclareMathOperator{\Sp}{Sp}
\DeclareMathOperator{\Span}{span}
\DeclareMathOperator{\Spec}{Spec}
\DeclareMathOperator{\spin}{spin}
\DeclareMathOperator{\st}{st}
\DeclareMathOperator{\St}{St}
\DeclareMathOperator{\SUn}{SU}
\DeclareMathOperator{\supp}{supp}
\DeclareMathOperator{\Sup}{sup}
\DeclareMathOperator{\Sym}{Sym}
\DeclareMathOperator{\Tam}{Tam}
\DeclareMathOperator{\tors}{tors}
\DeclareMathOperator{\tr}{tr}
\DeclareMathOperator{\un}{un}
\DeclareMathOperator{\Un}{U}
\DeclareMathOperator{\val}{val}
\DeclareMathOperator{\vol}{vol}

\DeclareMathOperator{\Sets}{S \mkern1.04mu e \mkern1.04mu t \mkern1.04mu s}
    \newcommand{\cSets}{\scalebox{1.02}{\contour{black}{$\Sets$}}}
    
\DeclareMathOperator{\Groups}{G \mkern1.04mu r \mkern1.04mu o \mkern1.04mu u \mkern1.04mu p \mkern1.04mu s}
    \newcommand{\cGroups}{\scalebox{1.02}{\contour{black}{$\Groups$}}}

\DeclareMathOperator{\TTop}{T \mkern1.04mu o \mkern1.04mu p}
    \newcommand{\cTop}{\scalebox{1.02}{\contour{black}{$\TTop$}}}

\DeclareMathOperator{\Htp}{H \mkern1.04mu t \mkern1.04mu p}
    \newcommand{\cHtp}{\scalebox{1.02}{\contour{black}{$\Htp$}}}

\DeclareMathOperator{\Mod}{M \mkern1.04mu o \mkern1.04mu d}
    \newcommand{\cMod}{\scalebox{1.02}{\contour{black}{$\Mod$}}}

\DeclareMathOperator{\Ab}{A \mkern1.04mu b}
    \newcommand{\cAb}{\scalebox{1.02}{\contour{black}{$\Ab$}}}

\DeclareMathOperator{\Rings}{R \mkern1.04mu i \mkern1.04mu n \mkern1.04mu g \mkern1.04mu s}
    \newcommand{\cRings}{\scalebox{1.02}{\contour{black}{$\Rings$}}}

\DeclareMathOperator{\ComRings}{C \mkern1.04mu o \mkern1.04mu m \mkern1.04mu R \mkern1.04mu i \mkern1.04mu n \mkern1.04mu g \mkern1.04mu s}
    \newcommand{\cComRings}{\scalebox{1.05}{\contour{black}{$\ComRings$}}}

\DeclareMathOperator{\hHom}{H \mkern1.04mu o \mkern1.04mu m}
    \newcommand{\cHom}{\scalebox{1.02}{\contour{black}{$\hHom$}}}

         %  \item $\cGroups$
          %  \item $\cTop$
          %  \item $\cHtp$
          %  \item $\cMod$




\renewcommand{\k}{\kappa}
\newcommand{\Ff}{F_{f}}
\newcommand{\ts}{\,^{t}\!}


%Mathcal

\newcommand{\cA}{\mathcal{A}}
\newcommand{\cB}{\mathcal{B}}
\newcommand{\cC}{\mathcal{C}}
\newcommand{\cD}{\mathcal{D}}
\newcommand{\cE}{\mathcal{E}}
\newcommand{\cF}{\mathcal{F}}
\newcommand{\cG}{\mathcal{G}}
\newcommand{\cH}{\mathcal{H}}
\newcommand{\cI}{\mathcal{I}}
\newcommand{\cJ}{\mathcal{J}}
\newcommand{\cK}{\mathcal{K}}
\newcommand{\cL}{\mathcal{L}}
\newcommand{\cM}{\mathcal{M}}
\newcommand{\cN}{\mathcal{N}}
\newcommand{\cO}{\mathcal{O}}
\newcommand{\cP}{\mathcal{P}}
\newcommand{\cQ}{\mathcal{Q}}
\newcommand{\cR}{\mathcal{R}}
\newcommand{\cS}{\mathcal{S}}
\newcommand{\cT}{\mathcal{T}}
\newcommand{\cU}{\mathcal{U}}
\newcommand{\cV}{\mathcal{V}}
\newcommand{\cW}{\mathcal{W}}
\newcommand{\cX}{\mathcal{X}}
\newcommand{\cY}{\mathcal{Y}}
\newcommand{\cZ}{\mathcal{Z}}


%mathfrak (missing \fi)

\newcommand{\fa}{\mathfrak{a}}
\newcommand{\fA}{\mathfrak{A}}
\newcommand{\fb}{\mathfrak{b}}
\newcommand{\fB}{\mathfrak{B}}
\newcommand{\fc}{\mathfrak{c}}
\newcommand{\fC}{\mathfrak{C}}
\newcommand{\fd}{\mathfrak{d}}
\newcommand{\fD}{\mathfrak{D}}
\newcommand{\fe}{\mathfrak{e}}
\newcommand{\fE}{\mathfrak{E}}
\newcommand{\ff}{\mathfrak{f}}
\newcommand{\fF}{\mathfrak{F}}
\newcommand{\fg}{\mathfrak{g}}
\newcommand{\fG}{\mathfrak{G}}
\newcommand{\fh}{\mathfrak{h}}
\newcommand{\fH}{\mathfrak{H}}
\newcommand{\fI}{\mathfrak{I}}
\newcommand{\fj}{\mathfrak{j}}
\newcommand{\fJ}{\mathfrak{J}}
\newcommand{\fk}{\mathfrak{k}}
\newcommand{\fK}{\mathfrak{K}}
\newcommand{\fl}{\mathfrak{l}}
\newcommand{\fL}{\mathfrak{L}}
\newcommand{\fm}{\mathfrak{m}}
\newcommand{\fM}{\mathfrak{M}}
\newcommand{\fn}{\mathfrak{n}}
\newcommand{\fN}{\mathfrak{N}}
\newcommand{\fo}{\mathfrak{o}}
\newcommand{\fO}{\mathfrak{O}}
\newcommand{\fp}{\mathfrak{p}}
\newcommand{\fP}{\mathfrak{P}}
\newcommand{\fq}{\mathfrak{q}}
\newcommand{\fQ}{\mathfrak{Q}}
\newcommand{\fr}{\mathfrak{r}}
\newcommand{\fR}{\mathfrak{R}}
\newcommand{\fs}{\mathfrak{s}}
\newcommand{\fS}{\mathfrak{S}}
\newcommand{\ft}{\mathfrak{t}}
\newcommand{\fT}{\mathfrak{T}}
\newcommand{\fu}{\mathfrak{u}}
\newcommand{\fU}{\mathfrak{U}}
\newcommand{\fv}{\mathfrak{v}}
\newcommand{\fV}{\mathfrak{V}}
\newcommand{\fw}{\mathfrak{w}}
\newcommand{\fW}{\mathfrak{W}}
\newcommand{\fx}{\mathfrak{x}}
\newcommand{\fX}{\mathfrak{X}}
\newcommand{\fy}{\mathfrak{y}}
\newcommand{\fY}{\mathfrak{Y}}
\newcommand{\fz}{\mathfrak{z}}
\newcommand{\fZ}{\mathfrak{Z}}


%mathbf

\newcommand{\bfA}{\mathbf{A}}
\newcommand{\bfB}{\mathbf{B}}
\newcommand{\bfC}{\mathbf{C}}
\newcommand{\bfD}{\mathbf{D}}
\newcommand{\bfE}{\mathbf{E}}
\newcommand{\bfF}{\mathbf{F}}
\newcommand{\bfG}{\mathbf{G}}
\newcommand{\bfH}{\mathbf{H}}
\newcommand{\bfI}{\mathbf{I}}
\newcommand{\bfJ}{\mathbf{J}}
\newcommand{\bfK}{\mathbf{K}}
\newcommand{\bfL}{\mathbf{L}}
\newcommand{\bfM}{\mathbf{M}}
\newcommand{\bfN}{\mathbf{N}}
\newcommand{\bfO}{\mathbf{O}}
\newcommand{\bfP}{\mathbf{P}}
\newcommand{\bfQ}{\mathbf{Q}}
\newcommand{\bfR}{\mathbf{R}}
\newcommand{\bfS}{\mathbf{S}}
\newcommand{\bfT}{\mathbf{T}}
\newcommand{\bfU}{\mathbf{U}}
\newcommand{\bfV}{\mathbf{V}}
\newcommand{\bfW}{\mathbf{W}}
\newcommand{\bfX}{\mathbf{X}}
\newcommand{\bfY}{\mathbf{Y}}
\newcommand{\bfZ}{\mathbf{Z}}

\newcommand{\bfa}{\mathbf{a}}
\newcommand{\bfb}{\mathbf{b}}
\newcommand{\bfc}{\mathbf{c}}
\newcommand{\bfd}{\mathbf{d}}
\newcommand{\bfe}{\mathbf{e}}
\newcommand{\bff}{\mathbf{f}}
\newcommand{\bfg}{\mathbf{g}}
\newcommand{\bfh}{\mathbf{h}}
\newcommand{\bfi}{\mathbf{i}}
\newcommand{\bfj}{\mathbf{j}}
\newcommand{\bfk}{\mathbf{k}}
\newcommand{\bfl}{\mathbf{l}}
\newcommand{\bfm}{\mathbf{m}}
\newcommand{\bfn}{\mathbf{n}}
\newcommand{\bfo}{\mathbf{o}}
\newcommand{\bfp}{\mathbf{p}}
\newcommand{\bfq}{\mathbf{q}}
\newcommand{\bfr}{\mathbf{r}}
\newcommand{\bfs}{\mathbf{s}}
\newcommand{\bft}{\mathbf{t}}
\newcommand{\bfu}{\mathbf{u}}
\newcommand{\bfv}{\mathbf{v}}
\newcommand{\bfw}{\mathbf{w}}
\newcommand{\bfx}{\mathbf{x}}
\newcommand{\bfy}{\mathbf{y}}
\newcommand{\bfz}{\mathbf{z}}

%blackboard bold

\newcommand{\bbA}{\mathbb{A}}
\newcommand{\bbB}{\mathbb{B}}
\newcommand{\bbC}{\mathbb{C}}
\newcommand{\bbD}{\mathbb{D}}
\newcommand{\bbE}{\mathbb{E}}
\newcommand{\bbF}{\mathbb{F}}
\newcommand{\bbG}{\mathbb{G}}
\newcommand{\bbH}{\mathbb{H}}
\newcommand{\bbI}{\mathbb{I}}
\newcommand{\bbJ}{\mathbb{J}}
\newcommand{\bbK}{\mathbb{K}}
\newcommand{\bbL}{\mathbb{L}}
\newcommand{\bbM}{\mathbb{M}}
\newcommand{\bbN}{\mathbb{N}}
\newcommand{\bbO}{\mathbb{O}}
\newcommand{\bbP}{\mathbb{P}}
\newcommand{\bbQ}{\mathbb{Q}}
\newcommand{\bbR}{\mathbb{R}}
\newcommand{\bbS}{\mathbb{S}}
\newcommand{\bbT}{\mathbb{T}}
\newcommand{\bbU}{\mathbb{U}}
\newcommand{\bbV}{\mathbb{V}}
\newcommand{\bbW}{\mathbb{W}}
\newcommand{\bbX}{\mathbb{X}}
\newcommand{\bbY}{\mathbb{Y}}
\newcommand{\bbZ}{\mathbb{Z}}

\newcommand{\bmat}{\left( \begin{matrix}}
\newcommand{\emat}{\end{matrix} \right)}

\newcommand{\pmat}{\left( \begin{smallmatrix}}
\newcommand{\epmat}{\end{smallmatrix} \right)}

\newcommand{\lat}{\mathscr{L}}
\newcommand{\mat}[4]{\begin{pmatrix}{#1}&{#2}\\{#3}&{#4}\end{pmatrix}}
\newcommand{\ov}[1]{\overline{#1}}
\newcommand{\res}[1]{\underset{#1}{\RES}\,}
\newcommand{\up}{\upsilon}

\newcommand{\tac}{\textasteriskcentered}

%mahesh macros
\newcommand{\tm}{\textrm}

%Comments
\newcommand{\com}[1]{\vspace{5 mm}\par \noindent
\marginpar{\textsc{Comment}} \framebox{\begin{minipage}[c]{0.95
\textwidth} \tt #1 \end{minipage}}\vspace{5 mm}\par}

\newcommand{\Bmu}{\mbox{$\raisebox{-0.59ex}
  {$l$}\hspace{-0.18em}\mu\hspace{-0.88em}\raisebox{-0.98ex}{\scalebox{2}
  {$\color{white}.$}}\hspace{-0.416em}\raisebox{+0.88ex}
  {$\color{white}.$}\hspace{0.46em}$}{}}  %need graphicx and xcolor. this produces blackboard bold mu 

\newcommand{\hooktwoheadrightarrow}{%
  \hookrightarrow\mathrel{\mspace{-15mu}}\rightarrow
}

\makeatletter
\newcommand{\xhooktwoheadrightarrow}[2][]{%
  \lhook\joinrel
  \ext@arrow 0359\rightarrowfill@ {#1}{#2}%
  \mathrel{\mspace{-15mu}}\rightarrow
}
\makeatother

\renewcommand{\geq}{\geqslant}
    \renewcommand{\leq}{\leqslant}
    
    \newcommand{\bone}{\mathbf{1}}
    \newcommand{\sign}{\mathrm{sign}}
    \newcommand{\eps}{\varepsilon}
    \newcommand{\textui}[1]{\uline{\textit{#1}}}
    
    %\newcommand{\ov}{\overline}
    %\newcommand{\un}{\underline}
    \newcommand{\fin}{\mathrm{fin}}
    
    \newcommand{\chnum}{\titleformat
    {\chapter} % command
    [display] % shape
    {\centering} % format
    {\Huge \color{black} \shadowbox{\thechapter}} % label
    {-0.5em} % sep (space between the number and title)
    {\LARGE \color{black} \underline} % before-code
    }
    
    \newcommand{\chunnum}{\titleformat
    {\chapter} % command
    [display] % shape
    {} % format
    {} % label
    {0em} % sep
    { \begin{flushright} \begin{tabular}{r}  \Huge \color{black}
    } % before-code
    [
    \end{tabular} \end{flushright} \normalsize
    ] % after-code
    }

\newcommand{\littletaller}{\mathchoice{\vphantom{\big|}}{}{}{}}
\newcommand\restr[2]{{% we make the whole thing an ordinary symbol
  \left.\kern-\nulldelimiterspace % automatically resize the bar with \right
  #1 % the function
  \littletaller % pretend it's a little taller at normal size
  \right|_{#2} % this is the delimiter
  }}

\newcommand{\mtext}[1]{\hspace{6pt}\text{#1}\hspace{6pt}}

%This adds a "front cover" page.
%{\thispagestyle{empty}
%\vspace*{\fill}
%\begin{tabular}{l}
%\begin{tabular}{l}
%\includegraphics[scale=0.24]{oxy-logo.png}
%\end{tabular} \\
%\begin{tabular}{l}
%\Large \color{black} Module Theory, Linear Algebra, and Homological Algebra \\ \Large \color{black} Gianluca Crescenzo
%\end{tabular}
%\end{tabular}
%\newpage


%%%%DOCUMENT SETUP%%%%
\setsecnumdepth{subsection}
\definecolor{darkgreen}{rgb}{0, 0.5976, 0}
\hypersetup{pdfauthor=Gianluca Crescenzo, pdftitle=Real Analysis I Notes, pdfstartview=FitH, colorlinks=true, linkcolor=darkgreen, citecolor=darkgreen}

\begin{document}
\begin{center}
    { \Large Math 395 \\[0.1in]Homework 6 \\[0.1in]
    Due: 11/05/2024}\\[.25in]
    { Name:} {\underline{Gianluca Crescenzo\hspace*{2in}}}\\[0.15in]
    { Collaborators:} {\underline{Noah Smith, Carly Venenciano, Avinash Iyer, Tim Rainone\hspace*{1in}}} \\
    \end{center}
    \vspace{4pt}
%%%%%%%%%%%%%%%%%%%%%%%%%%%%%%%%%%%%%%%%%%%%%%%%%%
    \begin{exercise}
        Let $V$ be an $\bfR$-vector space. Prove that $\bfC \otimes_\bfR V \cong V_\bfC$
    \end{exercise}
        \begin{proof}
            Define $t:V \rightarrow V \oplus V$ by $v \mapsto (v,0_V)$. Clearly $t \in \Hom_\bfR(V, V \oplus V)$. This extends to a map $T \in \Hom_\bfC(\bfC \otimes_\bfR V, V_\bfC)$ satisfying $t = T \circ \iota$, where $\iota:V \rightarrow \bfC \otimes_\bfR V$ is defined by $v \mapsto 1 \otimes v$. Claim: defining $T$ by $1 \otimes v_1 + i \otimes v_2 \mapsto (v_1,v_2)$ satisfies the universal property. Observe that:
                \begin{equation*}
                \begin{split}
                    T(\iota(v)) = T(1 \otimes v) = (v,0_V) = t(v).
                \end{split}
                \end{equation*}
            Furthermore, given $a \in \bfC$ we have:
                \begin{equation*}
                \begin{split}
                    T(a \otimes v)
                    & = T(a (1 \otimes v)) \\
                    & = aT(1 \otimes v) \\
                    & = a(v,0_V).
                \end{split}
                \end{equation*}
            Define $S: V_\bfC \rightarrow \bfC \otimes_\bfR V$ by $(v_1,v_2) \mapsto 1 \otimes v_1 + i \otimes v_2$. Given $v_1,v_2,v_1',v_2' \in V$ and $a+bi \in \bfC$, observe that:
                \begin{equation*}
                \begin{split}
                    S((v_1,v_2)+(a+bi)(v_1',v_2'))
                    & = S((v_1,v_2) + (av_1' - bv_2',bv_1' + av_2')) \\
                    & = S((v_1 + av_1' - bv_2',v_2 + bv_1' + av_2')) \\
                    & = 1 \otimes (v_1 + av_1' - bv_2') + i \otimes (v_2 + bv_1' + av_2') \\
                    & = 1 \otimes v_1 + 1 \otimes (av_1' - bv_2') + i \otimes v_2 + i \otimes (bv_1' + av_2')\\
                    & = (1\otimes v_1 + i \otimes v_2) + (a \otimes v_1' + bi \otimes v_1' + ai\otimes v_2' + (-b)\otimes v_2') \\
                    & = (1\otimes v_1 + i \otimes v_2) + ((a+bi) \otimes v_1' + (ai-b) \otimes v_2') \\
                    & = (1\otimes v_1 + i \otimes v_2) + ((a+bi)(1 \otimes v_1') + (a+bi)(i \otimes v_2')) \\
                    & = (1\otimes v_1 + i \otimes v_2) + (a+bi)(1 \otimes v_1' + i \otimes v_2') \\
                    & = S((v_1,v_2)) + (a+bi)S((v_1',v_2')).
                \end{split}
                \end{equation*}
            Hence $S \in \Hom_\bfC(V_\bfC, \bfC \otimes_\bfR V)$. Now consider:
                \begin{equation*}
                \begin{split}
                    T(S((v_1,v_2)))
                    & = T(1 \otimes v_1 + i \otimes v_2) \\
                    & = T(1 \otimes v_1) + T(i \otimes v_2) \\
                    & = (v_1,0_V) + (0_V,v_2) \\
                    & = (v_1,v_2).
                \end{split}
                \end{equation*}
            Moreover, since $\{1 \otimes v_k , i \otimes v_j\}_{k,j}$ is a basis of $\bfC \otimes_\bfR V$, it suffices to show:
                \begin{equation*}
                \begin{split}
                    S(T(1 \otimes v_k)) &= S((v_k,0_V)) = 1 \otimes v_k \\
                    S(T(i \otimes v_j)) &= S((0_V,v_j)) = i \otimes v_j.
                \end{split}
                \end{equation*}
            Thus $T \circ S = \id_{V_\bfC}$ and $S \circ T = \id_{\bfC \otimes_\bfR V}$, establishing $\bfC \otimes_\bfR V \cong V_\bfC$.
        \end{proof}
%%%%%%%%%%%%%%%%%%%%%%%%%%%%%%%%%%%%%%%%%%%%%%%%%%
    \begin{exercise}
        Let $t: \bfR^3 \times \bfR^3 \rightarrow \bfR^3$ be defined by $t(v,w) = v \times w$. Let $\cE_3$ be the standard basis of $\bfR^3$ and $\cB = \{e_i \otimes e_j\}_{1 \leq i,j \leq 3}$. Let $T \in \Hom_F(\bfR^3 \otimes_\bfR \bfR^3 , \bfR^3)$ be the linear map associated to $t$. Calculate $\left[T\right]_\cB^{\cE_3}$.
    \end{exercise}
        \begin{proof}
            We have:
                \begin{equation*}
                \begin{split}
                    T(e_1 \otimes e_1) &=e_1 \times e_1 = 0 \\
                    T(e_1 \otimes e_2) &=e_1 \times e_2 = e_3\\
                    T(e_1 \otimes e_3) &=e_1 \times e_3 = -e_2\\
                    T(e_2 \otimes e_1) &=e_2 \times e_1 = -e_3\\
                    T(e_2 \otimes e_2) &=e_2 \times e_2 = 0 \\
                    T(e_2 \otimes e_3) &=e_2 \times e_3  =e_1\\
                    T(e_3 \otimes e_1) &=e_3 \times e_1 = e_2\\
                    T(e_3 \otimes e_2) &=e_3 \times e_2  = -e_1\\
                    T(e_3 \otimes e_3) &=e_3 \times e_3 = 0.
                \end{split}
                \end{equation*}
            Hence:
                \begin{equation*}
                \begin{split}
                    \left[T\right]_\cB^{\cE_3}=
                    \bmat
                    0 &  0  &  0   &  0   &  0     &   1   &  0   &  -1 &  0\\
                    0 &  0  &  -1  &  0   &   0    &   0   &  1   &  0  &  0 \\
                    0 &  1  &  0   &  -1  &   0    &   0   &  0   &  0  &  0
                    \emat
                \end{split}
                \end{equation*}
        \end{proof}
%%%%%%%%%%%%%%%%%%%%%%%%%%%%%%%%%%%%%%%%%%%%%%%%%%
    \begin{exercise}
        Let $V$ and $W$ be $F$-vector spaces. Prove that $V \otimes_F W \cong W \otimes_F V$.
    \end{exercise}
        \begin{proof}
            Define $t_1:V \times W \rightarrow W \otimes_F V$ by $(v,w) \mapsto w \otimes v$. We have:
                \begin{equation*}
                \begin{split}
                    t_1(v_1 + cv_2,w)
                    & = w \otimes (v_1 + cv_2) \\
                    & = w \otimes v_1 + c(w \otimes v_2) \\
                    & = t_1(v_1,w) + ct(v_2,w).
                \end{split}
                \end{equation*}
                \begin{equation*}
                \begin{split}
                    t_1(v,w_1 + cw_2)
                    & = (w_1 + cw_2) \otimes v \\
                    & = w_1 \otimes v + c(w_2 \otimes v) \\
                    &= t_1(v,w_1) + ct_1(v,w_2).
                \end{split}
                \end{equation*}
            Thus $t_1 \in \Hom_F(V,W;W \otimes_F V)$. This extends to a map $T \in \Hom_F(V \otimes_F W, W \otimes_F V)$ defined by $v \otimes w \mapsto w \otimes v$. Now define $t_2:W \times V \mapsto V \otimes_F W$ by $(w,v) \mapsto v \otimes w$. We have:
                \begin{equation*}
                \begin{split}
                    t_2(w_1 + cw_2,v)
                    & = v \otimes (w_1 + cw_2) \\
                    & = v \otimes w_1 + c(v \otimes w_2) \\
                    & = t_2(w_1,v) + ct(w_1,v).
                \end{split}
                \end{equation*}
                \begin{equation*}
                \begin{split}
                    t_2(w,v_1 + cv_2)
                    & = (v_1 + cv_2) \otimes w \\
                    & = v_1 \otimes w + c(v_2 \otimes w) \\
                    & = t_2(w,v_1) + ct(w, v_2).
                \end{split}
                \end{equation*}
            Thus $t_2 \in \Hom_F(W,V;V \otimes_F W)$. This extends to a map $S \in \Hom_F(W \otimes_F V, V \otimes_F W)$. Now consider:
                \begin{equation*}
                \begin{split}
                    S(T(v \otimes w)) &= S(w \otimes v) = v \otimes w \\
                    T(S(w \otimes v)) &= T(v \otimes w) = w \otimes v.
                \end{split}
                \end{equation*}
            Thus $S \circ T = \id_{V \otimes_F W}$ and $T \circ S = \id_{W \otimes_F V}$, establishing $V \otimes_F W \cong W \otimes_F V$.
        \end{proof}
%%%%%%%%%%%%%%%%%%%%%%%%%%%%%%%%%%%%%%%%%%%%%%%%%%
    \begin{exercise}
        \phantom{a}
        \begin{enumerate}[label = (\alph*)]
            \item Let $\varphi \in V^\vee$ and $\psi \in W^\vee$. Define a map
                \begin{equation*}
                \begin{split}
                    B_{\varphi,\psi}: V \times W \rightarrow F \mtext{by} (v,w) \mapsto \varphi(v)\psi(w).
                \end{split}
                \end{equation*}
            Show that $B_{\varphi,\psi}$ is a bilinear form.
            \item Prove that there is a natural isomorphism between $(V \otimes_F W)^\vee$ and $V^\vee \otimes_F W^\vee$ (note that a natural isomorphism means it does not depend on a choice of basis).
        \end{enumerate}
    \end{exercise}
        \begin{proof}
            Given $v,v_1,v_2 \in V$, $w,w_1,w_2 \in W$, and $c \in F$, we have:
                \begin{equation*}
                \begin{split}
                    B_{\varphi,\psi}(v_1 + cv_2,w)
                    & = \varphi(v_1 + cv_2)\psi(w)\\
                    & = \varphi(v_1)\psi(w) + c\varphi(v_2)\psi(w) \\
                    & = B_{\varphi,\psi}(v_1,w) + c B_{\varphi,\psi}(v_2,w).
                \end{split}
                \end{equation*}
                \begin{equation*}
                \begin{split}
                    B_{\varphi,\psi}(v,w_1 + cw_2)
                    & = \varphi(v)\psi(w_1 +c w_2) \\
                    & = \varphi(v)\psi(w_1) + c\varphi(v)\psi(w_2) \\
                    & = B_{\varphi,\psi}(v,w_1) +c B_{\varphi,\psi}(v,w_2).
                \end{split}
                \end{equation*}
            Thus $B_{\varphi,\psi} \in \Hom_F(V,W;F)$. This induces a unique $T_{\varphi,\psi} \in \Hom_F(V \otimes_F W,F)$ defined by $T_{\varphi,\psi}(v \otimes w) = \varphi(v)\psi(w)$. Define $s:V^\vee \times W^\vee \rightarrow (V \otimes_F W)^\vee$ by $t(\varphi,\psi) \mapsto T_{\varphi,\psi}$. Given $\varphi,\varphi_1,\varphi_2 \in V^\vee$, $\psi,\psi_1,\psi_2 \in W^\vee$, and $c \in F$, we have:
                \begin{equation*}
                \begin{split}
                    s(\varphi,\psi_1 + c\psi_2)
                    & = T_{\varphi,\psi_1 + c \psi_2}(v \otimes w) \\
                    & = \varphi(v)(\psi_1 + c\psi_2)(w) \\
                    & = \varphi(v)\psi_1(w) + c \varphi(v)\psi_2(w) \\
                    & = T_{\varphi,\psi_1}(v \otimes w) + c T_{\varphi,\psi_2}(v \otimes w) \\
                    & = s(\varphi,\psi_1) + c s(\varphi,\psi_2).
                \end{split}
                \end{equation*}
                \begin{equation*}
                \begin{split}
                    s(\varphi_1 + c \varphi_2,\psi)
                    & = T_{\varphi_1 + c\varphi_2,\psi}(v \otimes w) \\
                    & = (\varphi_1 + c\varphi_2)(v)\psi(w) \\
                    & = \varphi_1(v)\psi(w) + c\varphi_2(v)\psi(w) \\
                    & = T_{\varphi_1,\psi}(v \otimes w) + cT_{\varphi_2,\psi}(v \otimes w) \\
                    & = s(\varphi_1,\psi) + c s(\varphi_2,\psi).
                \end{split}
                \end{equation*}
            Thus $s \in \Hom_F(V^\vee,W^\vee;(V \otimes_F W)^\vee )$. This induces a unique $S \in \Hom_F(V^\vee \otimes_F W^\vee, (V \otimes_F W)^\vee)$ defined by $\varphi \otimes \psi \mapsto T_{\varphi,\psi}$.

            Let $\{v_i\}_{i \in I}$ be a basis for $V$ and $\{w_j\}_{j \in I}$ a basis for $W$. We have that $\{v_i^\vee \otimes w_j^\vee\}_{i,j}$ is a basis for $V^\vee \otimes_F W^\vee$ and $\{(v_i \otimes w_j)^\vee \}_{i,j}$ is a basis for $(V \otimes_F W)^\vee$. Given $v \in V$ and $w \in W$, we have:
                \begin{equation*}
                \begin{split}
                    S(v_i^\vee \otimes w_j^\vee)(v \otimes w)
                    & = T_{{v_i^\vee},{w_j^\vee}}(v \otimes w) \\
                    & = v_i^\vee (v)w_j^\vee(w) \\
                    & = \begin{cases} 1, & v_i = v \mtext{and} w_j = w \\ 0, & \text{otherwise}\end{cases} \\
                    & = (v_i \otimes w_j)^\vee
                    (v \otimes w)
                \end{split}
                \end{equation*}
            we have that $\Image(S) \supseteq \{(v_i \otimes w_j)\}_{i,j}$, which implies that $S$ is surjective. Now suppose:
                \begin{equation*}
                \begin{split}
                    S\left(\sum_{\text{\tiny finite}}a_{i,j} (v_i^\vee \otimes w_j ^\vee) \right) = 0.
                \end{split}
                \end{equation*}
            Then:
                \begin{equation*}
                \begin{split}
                    \sum_{\text{\tiny finite}}a_{i,j}(v_i \otimes w_j)^\vee = 0,
                \end{split}
                \end{equation*}
            which implies that $a_{i,j} = 0$ for all $i,j$ since $\{(v_i \otimes w_j)^\vee \}_{i,j}$ is a basis. Hence $\sum_{\text{\tiny finite}}a_{i,j} (v_i^\vee \otimes w_j ^\vee) = 0$. Thus $V^\vee \otimes_F W^\vee \cong (V \otimes_F W)^\vee$.
        \end{proof}
%%%%%%%%%%%%%%%%%%%%%%%%%%%%%%%%%%%%%%%%%%%%%%%%%%
\end{document}



