%%%%%%%PACKAGES%%%%%%%
\documentclass[10pt,twoside,openany]{memoir}
%\usepackage[T1]{fontenc} font1
%\usepackage[utf8]{inputenc}font1
%\usepackage[math]{tgbonum} font 2
\usepackage[math]{iwona}
\usepackage[T1]{fontenc}
\usepackage{titlesec}
    \titlespacing*{\chapter}
    {0pt} % Left margin
    {*1} % Space before the chapter number (increase this value for more space)
    {0pt} % Space after the chapter number and before the title
\usepackage{anyfontsize}
\usepackage{fancybox}
\usepackage[dvipsnames,svgnames,x11names,hyperref]{xcolor}
\usepackage{enumerate}
\usepackage{comment}
\usepackage{amsfonts}
\usepackage{amsthm}
\usepackage{amsmath}
\usepackage{amssymb}
\usepackage{mathrsfs}
\usepackage{hyperref}
\usepackage{fullpage}
\usepackage{bm}
\usepackage{cprotect}
\usepackage{calligra}
\usepackage{emptypage}
\usepackage{titleps}
\usepackage{microtype}
\usepackage{float}
\usepackage{ocgx}
\usepackage{appendix}
\usepackage{graphicx}
\usepackage{pdfcomment}
\usepackage{enumitem}
\usepackage{mathtools}
\usepackage{tikz-cd}
\usepackage{relsize}
\usepackage[font=footnotesize,labelfont=bf]{caption}
\usepackage{changepage}
\usepackage{xcolor}
\usepackage{ulem}
\usepackage{pgfplots}
\usepackage{marginnote}
    \newcommand*{\mnote}[1]{ % <----------
    \checkoddpage
    \ifoddpage
        \marginparmargin{left}
    \else
        \marginparmargin{right}
    \fi
        \marginnote{\tiny \textcolor{oorange}{#1}}
    }

\usepackage{datetime}
    \newdateformat{specialdate}{\THEYEAR\ \monthname\ \THEDAY}
\usepackage[margin=0.9in]{geometry}
    \setlength{\voffset}{-0.4in}
    \setlength{\headsep}{30pt}
\usepackage{fancyhdr}
    \fancyhf{}
    \cfoot{\footnotesize \thepage}
    \fancyhead[R]{\footnotesize \rightmark}
    \fancyhead[L]{\footnotesize \leftmark}
\usepackage[T1]{fontenc}% http://ctan.org/pkg/fontenc
\usepackage[outline]{contour}% http://ctan.org/pkg/contour
    \renewcommand{\arraystretch}{1.5}
    \contourlength{0.4pt}
    \contournumber{10}%
\usepackage{letterspace}
    \linespread{1.25}
\usepackage{thmtools}
    \declaretheoremstyle[
        spaceabove=6pt, spacebelow=6pt,
        headfont=\normalfont\bfseries,
        notefont=\mdseries, notebraces={(}{)},
        bodyfont=\normalfont,
        postheadspace=1em
        %qed=\qedsymbol
        ]{mystyle}

    \declaretheorem[
        numberwithin=section,
        %shaded={
        %    rulecolor={RGB}{255,250,177},
        %    rulewidth=5pt, 
        %    bgcolor={RGB}{255,250,177}}
    ]{theorem}

    \declaretheorem[
        sibling=theorem,
        numberwithin=section,
        %shaded={
        %    rulecolor=Lavender,
        %    rulewidth=5pt, 
        %    bgcolor=Lavender}
    ]{lemma}

    \declaretheorem[
        sibling=theorem,
        numberwithin=section,
        %shaded={
        %    rulecolor=Lavender,
        %    rulewidth=5pt, 
        %    bgcolor=Lavender}
    ]{proposition}

    \declaretheorem[
        sibling=theorem,
        numberwithin=section,
        %shaded={
        %    rulecolor=Lavender,
        %    rulewidth=5pt, 
        %    bgcolor=Lavender}
    ]{corollary}

    \declaretheorem[
        numberwithin=section,
        style=mystyle,
        %shaded={
        %    rulecolor={RGB}{233,233,233},
        %    rulewidth=5pt, 
        %    bgcolor={RGB}{233,233,233}}
    ]{example}

    \declaretheorem[
        numberwithin=section,
        style=mystyle,
        %shaded={
        %    rulecolor={RGB}{255,250,177},
        %    rulewidth=5pt, 
        %    bgcolor={RGB}{255,250,177}}
    ]{definition}

    \declaretheorem[ numbered=unless unique]{exercise}

    \declaretheorem[numbered=no, name=Theorem]{theorem*}
    \declaretheorem[numbered=no, name=Lemma]{lemma*}
    \declaretheorem[numbered=no, name=Proposition]{proposition*}
    \declaretheorem[numbered=no, name=Corollary]{corollary*}
    \declaretheorem[numbered=no, name=Exercise]{exercise*}

\declaretheorem[numbered=unless unique,style=mystyle]{note}
%%%%%%%%%%%%%%%%%%%%%%%%%%%%%%%%%%%%%%%%%%%%%%%%%%%%%%%%%%%
%%%%%%%%%%%%%%%%%%%%%%%%%%%%%%%%%%%%%%%%%%%%%%%%%%%%%%%%%%%
%%%%%%%%%%%%%%%%%%%%%%%%%%%%%%%%%%%%%%%%%%%%%%%%%%%%%%%%%%%
%%%%%%%%%%%%%%%%%%%%%%%%%%%%%%%%%%%%%%%%%%%%%%%%%%%%%%%%%%%
%%%%%%%%%%%%%%%%%%%%%%%%%%%%%%%%%%%%%%%%%%%%%%%%%%%%%%%%%%%
%%%%%%%%%%%%%%%%%%%%%%%%%%%%%%%%%%%%%%%%%%%%%%%%%%%%%%%%%%%
%%%%%%%%%%%%%%%%%%%%%%%%%%%%%%%%%%%%%%%%%%%%%%%%%%%%%%%%%%%
%%%%%%%%%%%%%%%%%%%%%%%%%%%%%%%%%%%%%%%%%%%%%%%%%%%%%%%%%%%
%%%%%%%%%%%%%%%%%%%%%%%%%%%%%%%%%%%%%%%%%%%%%%%%%%%%%%%%%%%



%to make the correct symbol for Sha
%\newcommand\cyr{%
%\renewcommand\rmdefault{wncyr}%
%\renewcommand\sfdefault{wncyss}%
%\renewcommand\encodingdefault{OT2}%
%\normalfont \selectfont} \DeclareTextFontCommand{\textcyr}{\cyr}


\DeclareMathOperator{\ab}{ab}
\newcommand{\absgal}{\G_{\bbQ}}
\DeclareMathOperator{\ad}{ad}
\DeclareMathOperator{\adj}{adj}
\DeclareMathOperator{\alg}{alg}
\DeclareMathOperator{\Alt}{Alt}
\DeclareMathOperator{\Ann}{Ann}
\DeclareMathOperator{\arith}{arith}
\DeclareMathOperator{\Aut}{Aut}
\DeclareMathOperator{\Be}{B}
\DeclareMathOperator{\card}{card}
\DeclareMathOperator{\Char}{char}
\DeclareMathOperator{\csp}{csp}
\DeclareMathOperator{\codim}{codim}
\DeclareMathOperator{\coker}{coker}
\DeclareMathOperator{\coh}{H}
\DeclareMathOperator{\compl}{compl}
\DeclareMathOperator{\conj}{conj}
\DeclareMathOperator{\cont}{cont}
\DeclareMathOperator{\crys}{crys}
\DeclareMathOperator{\Crys}{Crys}
\DeclareMathOperator{\cusp}{cusp}
\DeclareMathOperator{\diag}{diag}
\DeclareMathOperator{\disc}{disc}
\DeclareMathOperator{\dR}{dR}
\DeclareMathOperator{\Eis}{Eis}
\DeclareMathOperator{\End}{End}
\DeclareMathOperator{\ev}{ev}
\DeclareMathOperator{\eval}{eval}
\DeclareMathOperator{\Eq}{Eq}
\DeclareMathOperator{\Ext}{Ext}
\DeclareMathOperator{\Fil}{Fil}
\DeclareMathOperator{\Fitt}{Fitt}
\DeclareMathOperator{\Frob}{Frob}
\DeclareMathOperator{\G}{G}
\DeclareMathOperator{\Gal}{Gal}
\DeclareMathOperator{\GL}{GL}
\DeclareMathOperator{\Gr}{Gr}
\DeclareMathOperator{\Graph}{Graph}
\DeclareMathOperator{\GSp}{GSp}
\DeclareMathOperator{\GUn}{GU}
\DeclareMathOperator{\Hom}{Hom}
\DeclareMathOperator{\id}{id}
\DeclareMathOperator{\Id}{Id}
\DeclareMathOperator{\Ik}{Ik}
\DeclareMathOperator{\IM}{Im}
\DeclareMathOperator{\Image}{im}
\DeclareMathOperator{\Ind}{Ind}
\DeclareMathOperator{\Inf}{inf}
\DeclareMathOperator{\Isom}{Isom}
\DeclareMathOperator{\J}{J}
\DeclareMathOperator{\Jac}{Jac}
\DeclareMathOperator{\lcm}{lcm}
\DeclareMathOperator{\length}{length}
\DeclareMathOperator{\Log}{Log}
\DeclareMathOperator{\M}{M}
\DeclareMathOperator{\Mat}{Mat}
\DeclareMathOperator{\N}{N}
\DeclareMathOperator{\Nm}{Nm}
\DeclareMathOperator{\NIk}{N-Ik}
\DeclareMathOperator{\NSK}{N-SK}
\DeclareMathOperator{\new}{new}
\DeclareMathOperator{\obj}{obj}
\DeclareMathOperator{\old}{old}
\DeclareMathOperator{\ord}{ord}
\DeclareMathOperator{\Or}{O}
\DeclareMathOperator{\PGL}{PGL}
\DeclareMathOperator{\PGSp}{PGSp}
\DeclareMathOperator{\rank}{rank}
\DeclareMathOperator{\Rel}{Rel}
\DeclareMathOperator{\Real}{Re}
\DeclareMathOperator{\RES}{res}
\DeclareMathOperator{\Res}{Res}
%\DeclareMathOperator{\Sha}{\textcyr{Sh}}
\DeclareMathOperator{\Sel}{Sel}
\DeclareMathOperator{\semi}{ss}
\DeclareMathOperator{\sgn}{sign}
\DeclareMathOperator{\SK}{SK}
\DeclareMathOperator{\SL}{SL}
\DeclareMathOperator{\SO}{SO}
\DeclareMathOperator{\Sp}{Sp}
\DeclareMathOperator{\Span}{span}
\DeclareMathOperator{\Spec}{Spec}
\DeclareMathOperator{\spin}{spin}
\DeclareMathOperator{\st}{st}
\DeclareMathOperator{\St}{St}
\DeclareMathOperator{\SUn}{SU}
\DeclareMathOperator{\supp}{supp}
\DeclareMathOperator{\Sup}{sup}
\DeclareMathOperator{\Sym}{Sym}
\DeclareMathOperator{\Tam}{Tam}
\DeclareMathOperator{\tors}{tors}
\DeclareMathOperator{\tr}{tr}
\DeclareMathOperator{\un}{un}
\DeclareMathOperator{\Un}{U}
\DeclareMathOperator{\val}{val}
\DeclareMathOperator{\vol}{vol}

\DeclareMathOperator{\Sets}{S \mkern1.04mu e \mkern1.04mu t \mkern1.04mu s}
    \newcommand{\cSets}{\scalebox{1.02}{\contour{black}{$\Sets$}}}
    
\DeclareMathOperator{\Groups}{G \mkern1.04mu r \mkern1.04mu o \mkern1.04mu u \mkern1.04mu p \mkern1.04mu s}
    \newcommand{\cGroups}{\scalebox{1.02}{\contour{black}{$\Groups$}}}

\DeclareMathOperator{\TTop}{T \mkern1.04mu o \mkern1.04mu p}
    \newcommand{\cTop}{\scalebox{1.02}{\contour{black}{$\TTop$}}}

\DeclareMathOperator{\Htp}{H \mkern1.04mu t \mkern1.04mu p}
    \newcommand{\cHtp}{\scalebox{1.02}{\contour{black}{$\Htp$}}}

\DeclareMathOperator{\Mod}{M \mkern1.04mu o \mkern1.04mu d}
    \newcommand{\cMod}{\scalebox{1.02}{\contour{black}{$\Mod$}}}

\DeclareMathOperator{\Ab}{A \mkern1.04mu b}
    \newcommand{\cAb}{\scalebox{1.02}{\contour{black}{$\Ab$}}}

\DeclareMathOperator{\Rings}{R \mkern1.04mu i \mkern1.04mu n \mkern1.04mu g \mkern1.04mu s}
    \newcommand{\cRings}{\scalebox{1.02}{\contour{black}{$\Rings$}}}

\DeclareMathOperator{\ComRings}{C \mkern1.04mu o \mkern1.04mu m \mkern1.04mu R \mkern1.04mu i \mkern1.04mu n \mkern1.04mu g \mkern1.04mu s}
    \newcommand{\cComRings}{\scalebox{1.05}{\contour{black}{$\ComRings$}}}

\DeclareMathOperator{\hHom}{H \mkern1.04mu o \mkern1.04mu m}
    \newcommand{\cHom}{\scalebox{1.02}{\contour{black}{$\hHom$}}}

         %  \item $\cGroups$
          %  \item $\cTop$
          %  \item $\cHtp$
          %  \item $\cMod$




\renewcommand{\k}{\kappa}
\newcommand{\Ff}{F_{f}}
\newcommand{\ts}{\,^{t}\!}


%Mathcal

\newcommand{\cA}{\mathcal{A}}
\newcommand{\cB}{\mathcal{B}}
\newcommand{\cC}{\mathcal{C}}
\newcommand{\cD}{\mathcal{D}}
\newcommand{\cE}{\mathcal{E}}
\newcommand{\cF}{\mathcal{F}}
\newcommand{\cG}{\mathcal{G}}
\newcommand{\cH}{\mathcal{H}}
\newcommand{\cI}{\mathcal{I}}
\newcommand{\cJ}{\mathcal{J}}
\newcommand{\cK}{\mathcal{K}}
\newcommand{\cL}{\mathcal{L}}
\newcommand{\cM}{\mathcal{M}}
\newcommand{\cN}{\mathcal{N}}
\newcommand{\cO}{\mathcal{O}}
\newcommand{\cP}{\mathcal{P}}
\newcommand{\cQ}{\mathcal{Q}}
\newcommand{\cR}{\mathcal{R}}
\newcommand{\cS}{\mathcal{S}}
\newcommand{\cT}{\mathcal{T}}
\newcommand{\cU}{\mathcal{U}}
\newcommand{\cV}{\mathcal{V}}
\newcommand{\cW}{\mathcal{W}}
\newcommand{\cX}{\mathcal{X}}
\newcommand{\cY}{\mathcal{Y}}
\newcommand{\cZ}{\mathcal{Z}}


%mathfrak (missing \fi)

\newcommand{\fa}{\mathfrak{a}}
\newcommand{\fA}{\mathfrak{A}}
\newcommand{\fb}{\mathfrak{b}}
\newcommand{\fB}{\mathfrak{B}}
\newcommand{\fc}{\mathfrak{c}}
\newcommand{\fC}{\mathfrak{C}}
\newcommand{\fd}{\mathfrak{d}}
\newcommand{\fD}{\mathfrak{D}}
\newcommand{\fe}{\mathfrak{e}}
\newcommand{\fE}{\mathfrak{E}}
\newcommand{\ff}{\mathfrak{f}}
\newcommand{\fF}{\mathfrak{F}}
\newcommand{\fg}{\mathfrak{g}}
\newcommand{\fG}{\mathfrak{G}}
\newcommand{\fh}{\mathfrak{h}}
\newcommand{\fH}{\mathfrak{H}}
\newcommand{\fI}{\mathfrak{I}}
\newcommand{\fj}{\mathfrak{j}}
\newcommand{\fJ}{\mathfrak{J}}
\newcommand{\fk}{\mathfrak{k}}
\newcommand{\fK}{\mathfrak{K}}
\newcommand{\fl}{\mathfrak{l}}
\newcommand{\fL}{\mathfrak{L}}
\newcommand{\fm}{\mathfrak{m}}
\newcommand{\fM}{\mathfrak{M}}
\newcommand{\fn}{\mathfrak{n}}
\newcommand{\fN}{\mathfrak{N}}
\newcommand{\fo}{\mathfrak{o}}
\newcommand{\fO}{\mathfrak{O}}
\newcommand{\fp}{\mathfrak{p}}
\newcommand{\fP}{\mathfrak{P}}
\newcommand{\fq}{\mathfrak{q}}
\newcommand{\fQ}{\mathfrak{Q}}
\newcommand{\fr}{\mathfrak{r}}
\newcommand{\fR}{\mathfrak{R}}
\newcommand{\fs}{\mathfrak{s}}
\newcommand{\fS}{\mathfrak{S}}
\newcommand{\ft}{\mathfrak{t}}
\newcommand{\fT}{\mathfrak{T}}
\newcommand{\fu}{\mathfrak{u}}
\newcommand{\fU}{\mathfrak{U}}
\newcommand{\fv}{\mathfrak{v}}
\newcommand{\fV}{\mathfrak{V}}
\newcommand{\fw}{\mathfrak{w}}
\newcommand{\fW}{\mathfrak{W}}
\newcommand{\fx}{\mathfrak{x}}
\newcommand{\fX}{\mathfrak{X}}
\newcommand{\fy}{\mathfrak{y}}
\newcommand{\fY}{\mathfrak{Y}}
\newcommand{\fz}{\mathfrak{z}}
\newcommand{\fZ}{\mathfrak{Z}}


%mathbf

\newcommand{\bfA}{\mathbf{A}}
\newcommand{\bfB}{\mathbf{B}}
\newcommand{\bfC}{\mathbf{C}}
\newcommand{\bfD}{\mathbf{D}}
\newcommand{\bfE}{\mathbf{E}}
\newcommand{\bfF}{\mathbf{F}}
\newcommand{\bfG}{\mathbf{G}}
\newcommand{\bfH}{\mathbf{H}}
\newcommand{\bfI}{\mathbf{I}}
\newcommand{\bfJ}{\mathbf{J}}
\newcommand{\bfK}{\mathbf{K}}
\newcommand{\bfL}{\mathbf{L}}
\newcommand{\bfM}{\mathbf{M}}
\newcommand{\bfN}{\mathbf{N}}
\newcommand{\bfO}{\mathbf{O}}
\newcommand{\bfP}{\mathbf{P}}
\newcommand{\bfQ}{\mathbf{Q}}
\newcommand{\bfR}{\mathbf{R}}
\newcommand{\bfS}{\mathbf{S}}
\newcommand{\bfT}{\mathbf{T}}
\newcommand{\bfU}{\mathbf{U}}
\newcommand{\bfV}{\mathbf{V}}
\newcommand{\bfW}{\mathbf{W}}
\newcommand{\bfX}{\mathbf{X}}
\newcommand{\bfY}{\mathbf{Y}}
\newcommand{\bfZ}{\mathbf{Z}}

\newcommand{\bfa}{\mathbf{a}}
\newcommand{\bfb}{\mathbf{b}}
\newcommand{\bfc}{\mathbf{c}}
\newcommand{\bfd}{\mathbf{d}}
\newcommand{\bfe}{\mathbf{e}}
\newcommand{\bff}{\mathbf{f}}
\newcommand{\bfg}{\mathbf{g}}
\newcommand{\bfh}{\mathbf{h}}
\newcommand{\bfi}{\mathbf{i}}
\newcommand{\bfj}{\mathbf{j}}
\newcommand{\bfk}{\mathbf{k}}
\newcommand{\bfl}{\mathbf{l}}
\newcommand{\bfm}{\mathbf{m}}
\newcommand{\bfn}{\mathbf{n}}
\newcommand{\bfo}{\mathbf{o}}
\newcommand{\bfp}{\mathbf{p}}
\newcommand{\bfq}{\mathbf{q}}
\newcommand{\bfr}{\mathbf{r}}
\newcommand{\bfs}{\mathbf{s}}
\newcommand{\bft}{\mathbf{t}}
\newcommand{\bfu}{\mathbf{u}}
\newcommand{\bfv}{\mathbf{v}}
\newcommand{\bfw}{\mathbf{w}}
\newcommand{\bfx}{\mathbf{x}}
\newcommand{\bfy}{\mathbf{y}}
\newcommand{\bfz}{\mathbf{z}}

%blackboard bold

\newcommand{\bbA}{\mathbb{A}}
\newcommand{\bbB}{\mathbb{B}}
\newcommand{\bbC}{\mathbb{C}}
\newcommand{\bbD}{\mathbb{D}}
\newcommand{\bbE}{\mathbb{E}}
\newcommand{\bbF}{\mathbb{F}}
\newcommand{\bbG}{\mathbb{G}}
\newcommand{\bbH}{\mathbb{H}}
\newcommand{\bbI}{\mathbb{I}}
\newcommand{\bbJ}{\mathbb{J}}
\newcommand{\bbK}{\mathbb{K}}
\newcommand{\bbL}{\mathbb{L}}
\newcommand{\bbM}{\mathbb{M}}
\newcommand{\bbN}{\mathbb{N}}
\newcommand{\bbO}{\mathbb{O}}
\newcommand{\bbP}{\mathbb{P}}
\newcommand{\bbQ}{\mathbb{Q}}
\newcommand{\bbR}{\mathbb{R}}
\newcommand{\bbS}{\mathbb{S}}
\newcommand{\bbT}{\mathbb{T}}
\newcommand{\bbU}{\mathbb{U}}
\newcommand{\bbV}{\mathbb{V}}
\newcommand{\bbW}{\mathbb{W}}
\newcommand{\bbX}{\mathbb{X}}
\newcommand{\bbY}{\mathbb{Y}}
\newcommand{\bbZ}{\mathbb{Z}}

\newcommand{\bmat}{\left( \begin{matrix}}
\newcommand{\emat}{\end{matrix} \right)}

\newcommand{\pmat}{\left( \begin{smallmatrix}}
\newcommand{\epmat}{\end{smallmatrix} \right)}

\newcommand{\lat}{\mathscr{L}}
\newcommand{\mat}[4]{\begin{pmatrix}{#1}&{#2}\\{#3}&{#4}\end{pmatrix}}
\newcommand{\ov}[1]{\overline{#1}}
\newcommand{\res}[1]{\underset{#1}{\RES}\,}
\newcommand{\up}{\upsilon}

\newcommand{\tac}{\textasteriskcentered}

%mahesh macros
\newcommand{\tm}{\textrm}

%Comments
\newcommand{\com}[1]{\vspace{5 mm}\par \noindent
\marginpar{\textsc{Comment}} \framebox{\begin{minipage}[c]{0.95
\textwidth} \tt #1 \end{minipage}}\vspace{5 mm}\par}

\newcommand{\Bmu}{\mbox{$\raisebox{-0.59ex}
  {$l$}\hspace{-0.18em}\mu\hspace{-0.88em}\raisebox{-0.98ex}{\scalebox{2}
  {$\color{white}.$}}\hspace{-0.416em}\raisebox{+0.88ex}
  {$\color{white}.$}\hspace{0.46em}$}{}}  %need graphicx and xcolor. this produces blackboard bold mu 

\newcommand{\hooktwoheadrightarrow}{%
  \hookrightarrow\mathrel{\mspace{-15mu}}\rightarrow
}

\makeatletter
\newcommand{\xhooktwoheadrightarrow}[2][]{%
  \lhook\joinrel
  \ext@arrow 0359\rightarrowfill@ {#1}{#2}%
  \mathrel{\mspace{-15mu}}\rightarrow
}
\makeatother

\renewcommand{\geq}{\geqslant}
    \renewcommand{\leq}{\leqslant}
    
    \newcommand{\bone}{\mathbf{1}}
    \newcommand{\sign}{\mathrm{sign}}
    \newcommand{\eps}{\varepsilon}
    \newcommand{\textui}[1]{\uline{\textit{#1}}}
    
    %\newcommand{\ov}{\overline}
    %\newcommand{\un}{\underline}
    \newcommand{\fin}{\mathrm{fin}}
    
    \newcommand{\chnum}{\titleformat
    {\chapter} % command
    [display] % shape
    {\centering} % format
    {\Huge \color{black} \shadowbox{\thechapter}} % label
    {-0.5em} % sep (space between the number and title)
    {\LARGE \color{black} \underline} % before-code
    }
    
    \newcommand{\chunnum}{\titleformat
    {\chapter} % command
    [display] % shape
    {} % format
    {} % label
    {0em} % sep
    { \begin{flushright} \begin{tabular}{r}  \Huge \color{black}
    } % before-code
    [
    \end{tabular} \end{flushright} \normalsize
    ] % after-code
    }

\newcommand{\littletaller}{\mathchoice{\vphantom{\big|}}{}{}{}}
\newcommand\restr[2]{{% we make the whole thing an ordinary symbol
  \left.\kern-\nulldelimiterspace % automatically resize the bar with \right
  #1 % the function
  \littletaller % pretend it's a little taller at normal size
  \right|_{#2} % this is the delimiter
  }}

\newcommand{\mtext}[1]{\hspace{6pt}\text{#1}\hspace{6pt}}

%This adds a "front cover" page.
%{\thispagestyle{empty}
%\vspace*{\fill}
%\begin{tabular}{l}
%\begin{tabular}{l}
%\includegraphics[scale=0.24]{oxy-logo.png}
%\end{tabular} \\
%\begin{tabular}{l}
%\Large \color{black} Module Theory, Linear Algebra, and Homological Algebra \\ \Large \color{black} Gianluca Crescenzo
%\end{tabular}
%\end{tabular}
%\newpage


%%%%%%%%%%%%%%%%%%%%%%%%%%%%%%%%%%%%%%%%%%%%%%%%%%%%%%%%%%%%%%%%%
%%%%%%%%%%%%%%%%%%%%%%%%%%%%%%%%%%%%%%%%%%%%%%%%%%%%%%%%%%%%%%%%%%%%%%%%%%%%%%%%%%%%%%%%%%%%%%%%%%%%%%%%%%%%%%%%%%%%%%%%%%%%%%%%%%%%%%%%%%%%%%%%%%%%%%%%%%%%%%%%%%%%%%%%%%%%%%%%%%%%%%%%%%%%%%%%%%%%%%%%%%%%%%%%%%%%%%%%%%%%%%%%%%%%%%%%%%%%%%%%%%%%%%%%%%%%%%%%%%%%%%%%%%%%%%%%%%%%%%%%%%%%%%%%%%%%%%%%%%%%%%%%%%%%%%%%%%%%%%%%%%%%%%%%%%%%%%%%%%%%%%%%%%%%%%%%%%%%%%%%%%%%%%%%%%%%%%%%%%%%%%%%%%%%%%%%%%%%%%%%%%%%%%%%%%%%%%%%%%%%%%%%%%%%%%%%%%%%%%%%%%%%%%%%%%%%%%%%%%%%%%%%%%%%%%%%%%%%%%%%%%%%%%%%%%%%%%%%%%%%%%%%%%%%%%%%%%%%%%%%%%%%%%%%%%%%%%%%%%%%%%%%%%%%%%%%%%%%%%%%%%%%%%%%%%%%%%%%%%%%%%%%%%%%%%%%%%%%%%%%%%%%%%%%%%%%%%%%%%%%%%%%%%%%%%%%%%%%%%%%%%%%%%%%%%%%%%%%%%%%%%%%%%%%%%%%%%%%%%%%%%%%%%%%%%%%%%%%%%%%%%%%%%%%%%%%%%%%%%%%%%%%%%%%%%%%%%%%%%%%%%%%%%%%%%%%%%%%%%%%%%%%%%%%%%%%%%%%%%%%%%%%%%%%%%%%%%%%%%%%%%%%%%%%%%%%%%%%%%%%%%%%%%%%%%%%%%%%%%%%%%%%%%%%%%%%%%%%%%%%%%%%%%%%%%%%%%%%%%%%%%%%%%%%%%%%%%%%%%%%%%%%%%%%%%%%%%%%%%%%%%%%%%%%%%%%%%%%%%%%%%%%%%%%%%%%%%%%%%%%%%%%%%%%%%%%%%%%%%%%%%%%%%%%%%%%%%%%%%%%%%%%%%%%%%%%%%%%%%%%%%%%%%%%%%%%%%%%%%%%%%%%%%%%%%%%%%%%%%%%%%%%%%%%%%%%%%%%%%%%%%%%%%%%%%%%%%%%%%%%%%%%%%%%%%%%%%%%%%%%%%%%%%%%%%%%%%%%%%%%%%%%%%%%%%%%%%%%%%%%%%%%%%%%%%%%%%%%%%%%%%%%%%%%%%%%%%%%%%%%%%%%%%%%%%%%%%%%%%%%%%%%%%%%%%%%%%%
\begin{document}
\begin{center}
    { \Large Math 395 \\[0.1in]Homework 3 \\[0.1in]
    Due: 9/24/2024}\\[.25in]
    { Name:} {\underline{Gianluca Crescenzo\hspace*{2in}}}\\[0.15in]
    { Collaborators:} {\underline{ Avinash Iyer, Noah Smith, Carly Venenciano \hspace*{2in}}} \\
    \end{center}
    \vspace{3pt}
    For these problems $V$ is a finite-dimensional $F$-vector space.
    \vspace{3pt}
%%%%%%%%%%%%%%%%%%%%%%%%%%%%%%%%%%%%%%%%%%%%%%%%%%%%%%%%%%%%
\begin{exercise*}
    \phantom{a}
    \begin{enumerate}[label = (\arabic*)]
        \item Let $\cE$ be a basis of $U$, $\cF$ a basis of $V$ and $\cG$ a basis of $W$. Let $T_B \in \Hom_F{(U,V)}$ and $T_A \in \Hom_F{(V,W)}$. Show
            \begin{equation*}
            \begin{split}
                \left[ T_A \circ T_B \right]_\cE ^ \cG = \left[ T_A \right]_\cF ^\cG \left[ T_B \right]_\cE ^ \cF.
            \end{split}
            \end{equation*}
        \item Let $\left[ T_A \right]_\cF ^\cG = A \in \Mat_{p,m}(F)$ and $\left[ T_B \right]_\cE ^ \cF = B \in \Mat_{m,n}(F)$. Show that you can recover the definition of matrix multiplication by using part (1).
    \end{enumerate}
\end{exercise*}
    \begin{proof}
        Since the following diagram commutes:
            \begin{center}
                \begin{tikzcd}
                    U \arrow[r, "T_B"] \arrow[d, "T_\cE"']            & V \arrow[r, "T_A"] \arrow[d, "T_\cF"']       & W \arrow[d, "T_\cG"'] \\
                    F^n \arrow[r, "{\left[ T_B\right]_{\cE}^{\cF}}"'] & F^m \arrow[r, "{\left[T_A\right]_\cF^\cG}"'] & F^p ,                 
                    \end{tikzcd}
            \end{center}
        we have that $\left[ T_A \circ T_B \right]_\cE ^ \cG = \left[ T_A \right]_\cF ^\cG \left[ T_B \right]_\cE ^ \cF$. Let $\cE = \{e_1,...,e_n\}$, $\cF = \{f_1,...,f_m\}$, and $\cG = \{g_1,...,g_p\}$. The equations
            \begin{equation*}
            \begin{split}
                T_B(e_j) &= \sum_{k = 1}^m b_{kj}f_k \\
                T_B(f_k) &= \sum_{i = 1}^p a_{ik}g_i \\
            \end{split}
            \end{equation*}
        give rise to the following:
            \begin{equation*}
            \begin{split}
                T_A(T_B(e_j)) &= \sum_{i=1}^p\left(\sum_{k=1}^m a_{ik}b_{kj}\right)g_i \\
                & := \sum_{i=1}^pc_{ij}g_i.
            \end{split}
            \end{equation*}
        Hence $\left[ T_A \circ T_B \right]_\cE ^ \cG = AB = (c_{ij}) \in \Mat_{p,n}(F)$.
    \end{proof}
%%%%%%%%%%%%%%%%%%%%%%%%%%%%%%%%%%%%%%%%%%%%%%%%%%%%%%%%%%%%
    \newpage
    \begin{exercise}
        Let $V = P_n(F)$. Let $\cB = \{1,x,x^2,...,x^n\}$ be a basis of $V$. Let $\lambda \in F$ and set \newline$\cC = \{1 , (x-\lambda), (x-\lambda)^2,...,(x-\lambda)^n\}$. Define a linear transformation $T \in \Hom_F{(V,V)}$ by defining $T(x^j) = (x-\lambda)^j$. Determine the matrix of this linear transformation. Use this to conclude that $\cC$ is also a basis of $V$.
    \end{exercise}
        \begin{proof}
            Note that $T(x^j) = (x-\lambda)^j = \sum_{k=0}^j {j \choose k}(-\lambda)^{j-k} x^k$ for all $0 \leq j \leq n$. Hence:
            \begin{equation*}
                \begin{split}
                    \left[T\right]_\cB^\cB =
                    \bmat 
                    1 & -\lambda & \lambda^2 & ... & (-\lambda)^n  \\
                    0 & 1 &  -2\lambda & ... & {n \choose 1}(-\lambda)^{n-1}  \\
                    0 &  0 & 1 & ... & {n \choose 2}(-\lambda)^{n-2} \\
                    \vdots & & & \ddots & \vdots  \\
                    0 & 0 & 0 & ... & 1 \\
                    \emat.
                \end{split}
                \end{equation*}
            Note that this matrix is non-singular, hence it is an isomorphism. Thus there exists a $T^{-1}$ defined by $T^{-1}((x-\lambda)^j) = x^j$, establishing that $\cC$ forms a basis of V.
        \end{proof}
%%%%%%%%%%%%%%%%%%%%%%%%%%%%%%%%%%%%%%%%%%%%%%%%%%%%%%%%%%%%
    \addtocounter{exercise}{2}
    \begin{exercise}
        Let $V = P_5(\bfQ)$ and let $\cB = \{1,x,...,x^5\}$. Prove that the following are elements of $V^\vee$ and express them as linear combinations of the dual basis:
            \begin{enumerate}[label = (\arabic*)]
                \item $\phi_1:V \rightarrow \bfQ$ defined by $\phi(p(x)) = \int_0^1 t^2p(t)dt$.
                \item $\phi_2:V \rightarrow \bfQ$ defined by $\phi(p(x)) = p'(5)$ where $p'(x)$ denotes the derivative of $p(x)$.
            \end{enumerate}
    \end{exercise}
        \begin{proof}
            Let $p_1,p_2 \in P_5(\bfQ)$. Then:
                \begin{equation*}
                \begin{split}
                    \phi_1((p_1 + cp_2)(x)) 
                    & = \int_0^1t^2(p_1 + cp_2)(x) dt \\
                    & = \int_0^1 t^2(p_1(x) + cp_2(x))dt \\
                    & = \int_0^1 t^2p_1(x)dt + c \int_0^1 t^2p_2(x)dt \\
                    & = \phi_1(p_1(x)) + c\phi_1(p_2(x)).
                \end{split}
                \end{equation*}
                
                \begin{equation*}
                \begin{split}
                    \phi_2((p_1 + c p_2)(x))
                    & = (p_1 + cp_2)'(5)\\
                    & = p_1'(5) + cp_2'(5) \\
                    & = \phi_2(p_1(x)) + c \phi_2(p_2(x)).
                \end{split}
                \end{equation*}
            Thus $\phi_1,\phi_2 \in V^\vee$. Note that elements of the dual basis $\cB^\vee = \{1^\vee, x^\vee,...,{x^5}^\vee\}$ are defined as follows:
                \begin{equation*}
                    {x^i}^\vee(x^j) = 
                \begin{cases}
                    1, & i = j \\
                    0, & \text{otherwise},
                \end{cases}
                \end{equation*}
            and furthermore ${x^i}^\vee (p) = a_i$ for some $p \in P_5(\bfQ)$ and $i \in \{0,1,...,5\}$. Thus we can express any $\phi \in V^\vee$ in terms of its dual basis as follows:
                \begin{equation*}
                \begin{split}
                    \phi(p) &= \phi \left(\sum_{i=0}^5 a_i x^i\right)\\
                    &=  \sum_{i=0}^5 a_i \phi(x^i) \\
                    & = \sum_{i=0}^5{x^i}^\vee(p)\phi(x^i).
                \end{split}
                \end{equation*}
            Hence:
                \begin{equation*}
                \begin{split}
                    \phi_1(p) &= \sum_{i=0}^5 \left[{x^i}^\vee(p) \left( \int_0^1 t^{2+i}dt\right)\right] = \sum_{i=0}^5 {x^i}^\vee(p)\cdot \frac{1}{3+i} \\
                    \phi_2(p) &= \sum_{i=0}^5 {x^i}^\vee(p) \cdot ix^{i-1}
                \end{split}
                \end{equation*}

            \iffalse
            We can express $\phi_1$ in terms of its dual basis elements as follows:
                \begin{equation*}
                \begin{split}
                    \phi_1(p)
                    & = \phi_1(a_0 + a_1x + ... + a_5 x^5) \\
                    & = a_0\phi_1(1) + a_1 \phi_1(x) + ... + a_5 \phi_1(x^5) \\
                    & = 1^\vee(p)\int_0^1t^2\hspace{2pt}dt \hspace{4pt}+\hspace{4pt} x^\vee(p)\int_0^1t^2x\hspace{2pt}dt \hspace{4pt}+\hspace{4pt}...\hspace{4pt}+\hspace{4pt}{x^5}^\vee(p)\int_0^1t^2x^5\hspace{2pt}dt. 
                \end{split}
                \end{equation*}
            Similarly:
                \begin{equation*}
                \begin{split}
                    \phi_2((p_1 + c p_2)(x))
                    & = (p_1 + cp_2)'(5)\\
                    & = p_1'(5) + cp_2'(5) \\
                    & = \phi_2(p_1(x)) + c \phi_2(p_2(x))
                \end{split}
                \end{equation*}
            Thus $\phi_2 \in V^\vee$. We can express $\phi_2$ in terms of its dual basis as follows:
                \begin{equation*}
                \begin{split}
                    \phi_2(p)
                    & = \phi_2(a_0 + a_1x + a_2x^2 + ... + a_5x^5) \\
                    & = a_0\phi_2(1) + a_1\phi_2(x) + a_2 \phi_2(x^2) + ... + a_5\phi_2(x^5) \\
                    & = 1^\vee(p)\cdot 0 + x^\vee(p)\cdot 1 + {x^2}^\vee(p) \cdot 2(5) + ... + {x^5}^\vee(p) \cdot 5(5)^4 \\
                \end{split}
                \end{equation*}

                \begin{equation*}
                \begin{split}
                    \phi_2(p)
                    & = \sum_{i=0}^5 a_i \phi_2(x^i) \\
                    & = \sum_{i=0}^5 {x^j}^\vee (p) \cdot \phi_2(x^i)
                \end{split}
                \end{equation*}
            \fi

        \end{proof}
%%%%%%%%%%%%%%%%%%%%%%%%%%%%%%%%%%%%%%%%%%%%%%%%%%%%%%%%%%%%
    \begin{exercise}
        Let $V$ be a vector space over $F$ and let $T \in \Hom_F{(V,V)}$. A nonzero $v \in V$ satisfying $T(v) = \lambda v$ for some $\lambda \in F$ is called an eigenvector of T with eigenvalue $\lambda$.
            \begin{enumerate}[label = (\alph*)]
                \item Prove that for any fixed $\lambda \in F$ the collection of eigenvectors of $T$ with eigenvalue $\lambda$ together with $0_V$ forms a subspace of $V$.
                \item Prove that if $V$ has a basis $\cB$ consisting of eigenvectors for $T$ then $\left[T\right]_\cB ^\cB$ is a diagonal matrix with the eigenvalues of $T$ as diagonal entries.
            \end{enumerate}
    \end{exercise}
        \begin{proof}
            Let $E = \{v_i \mid T(v_i) = \lambda v_i,\hspace{3pt} v_i \in V\} \cup \{0_V\}$. Clearly $E$ is nonempty. Let $v_1 ,v_2 \in E$. Then $T(v_1 + cv_2) = T(v_1) + cT(v_2) = \lambda v_1 + c \lambda v_2 = \lambda(v_1 + c v_2)$. Thus $E$ is a subspace of $V$.

            Now let $\cB = \{v_1,...,v_n\}$ where each $v_i$ is an eigenvector associated with a unique eigenvalue $\lambda_i$. Observe that:
                \begin{equation*}
                \begin{split}
                    T(v_1) & = \lambda_1 v_1 = \lambda_1 \cdot v_1 + 0 \cdot v_2 + 0 \cdot v_3 + ... + 0 \cdot v_n \\
                    T(v_2) & = \lambda_2 v_2 = 0 \cdot v_1 + \lambda_2 \cdot v_2 + 0 \cdot v_3 + ... + 0 \cdot v_n \\
                    T(v_3) & = \lambda_3 v_3 = 0 \cdot v_1 + 0 \cdot v_2 + \lambda_3 \cdot v_3  + ... + 0 \cdot v_n \\
                    &\vdots \\
                    T(v_n) & = \lambda_n v_n = 0 \cdot v_1 + 0 \cdot v_2 + 0 \cdot v_3 + ... + \lambda_n v_n.
                \end{split}
                \end{equation*}
            Hence:
                \begin{equation*}
                \begin{split}
                    \left[T\right]_\cB^\cB =
                    \bmat 
                    \lambda_1 & 0 & ... & 0   \\
                    0 & \lambda_2 &  ... &  0  \\
                    \vdots &  \vdots & \ddots & \vdots \\
                    0 & 0& ... & \lambda_n   \\
                    \emat.
                \end{split}
                \end{equation*}
        \end{proof}
\end{document}
