\chapter{Generalized Eigenvectors and Jordan Canonical Form}
\vspace{12pt}

\section{Diagonalization}
    \begin{recall}
        We say $A \sim B$ if and only if $A = PBP^{-1}$ for some $P \in \GL_n(F)$. In particular, this means $A = [T]_\cA$ and $B = [T]_\cB$ for some bases $\cA$ and $\cB$ (Example~\ref{example:similar-matrices}).
    \end{recall}

    \begin{definition}
        We say $A$ is \textui{diagonalizable} if $A \sim D$ for some diagonal matrix $D$. In terms of linear transformations, $A = [T]_\cA$ is diagonalizable if there is a basis $\cB$ such that $[T]_\cB = D$.
    \end{definition}

    \begin{example}
        If $A \sim B$ then $A$ is diagonalizable if and only if $B$ is diagonalizable. If $A$ and $B$ are diagonalizable, they must be similar to the same diagonal matrix up to reordering the diagonals. 
    \end{example}

    \begin{example}
        Let $V = F^2$ and $T \in \Hom_F{(V,V)}$. Let $T(e_1) = 3e_1$ and $T(e_2) = -2e_2$. We have that:
            \begin{equation*}
            \begin{split}
                \left[T\right]_{\cE_2} = \bmat 3 & 0 \\ 0 & -2 \emat.
            \end{split}
            \end{equation*}
        It follows that $V = V_1 \oplus V_2$, where $V_1 = \Span_F{(e_1)}$ and $V_2 = \Span_F{(e_2)}$. In this case, we have that $T(V_1) \subseteq V_1$ and $T(V_2) \subseteq V_2$, allowing us to write $T$ as a diagonal matrix.
    \end{example}

    \begin{example}
        Let $V = F^2$ and $T \in \Hom_F{(V,V)}$. Consider $T(e_1) = 3e_1$ and $T(e_2) = e_1 + 3e_2$. Then:
            \begin{equation*}
            \begin{split}
                \left[T\right]_{\cE_2} = \bmat 3 & 1 \\ 0 & 3 \emat.
            \end{split}
            \end{equation*}
        Then $V = V_1 \oplus V_2$ with $V_1 = \Span_F(e_1)$ and $V_2 = \Span_F{(e_2)}$. But while we have $T(V_1) \subseteq V_1$, we do not have $T(V_2) \subseteq V_2$.

        Suppose towards contradiction we have $W_1,W_2 \neq \{0\}$ with $T(W_1) \subseteq W_1$ and $T(W_2) \subseteq W_2$. Write $W_i = \Span_F{(w_i)}$. In particular, this means we can write $T(w_1) = \alpha w_1$ and $T(w_2) = \beta w_2$. For $\cB = \{w_1,w_2\}$, we have:
            \begin{equation*}
            \begin{split}
                \left[T\right]_\cB = \bmat \alpha & 0 \\ 0 & \beta \emat.
            \end{split}
            \end{equation*}
        Write $w_1 = ae_1 + be_2$ and $w_2 = ce_1 + de_2$. Then:
            \begin{equation*}
            \begin{split}
                \alpha w_1
                & = T(w_1) \\
                & = aT(e_1) + bT(e_2) \\
                & = a(3e_1) + b (e_1 + 3e_2) \\
                & = (3a+b)e_1 + (3b)e_2.
            \end{split}
            \end{equation*}
        Thus, $\alpha(ae_1 + be_2) = (3a+b)e_1 + (3b) e_2$, meaning $\alpha a = 3b+b$ and $\alpha b = 3b$. Either $b = 0$ or $\alpha = 3$. It must be the case that $\alpha = 3$, hence $T(w_1) = 3w_1$. A similar argument for $w_1$ gives:
            \begin{equation*}
            \begin{split}
                \beta w_2
                & = T(w_2) \\
                & = ... \\
                & = (3c+d)e_1 + (3d)e_2.
            \end{split}
            \end{equation*}
        This implies $\beta c = ec + d$ and $\beta d = 3d$. If $\beta = 3$, then this contradicts the first equation. If $w_2 = ce_1$, this contradicts $w_1,w_2$ being a basis.
    \end{example}

    \begin{example}\label{example:field-extension-invertible}
        Let $A = \pmat 1 & 2 \\ 3 & 4 \epmat$. Let $F = \bfQ$. Let $P \in GL_2(\bfQ)$, where $P = \pmat a & b \\ c & d \epmat$. We have:
            \begin{equation*}
            \begin{split}
                P^{-1}AP = \frac{1}{ad-bc}\bmat ad - 2ab + 2cd - 4bc & -3bd - 3b^2 + 2d^2 \\ 3ac + 3a^2 - 2c^2 & -bc + 3ab - 2cd + 4ad \emat.
            \end{split}
            \end{equation*}
        We must have that $3a^2 + 4ac - 2c^2 = 0$. If $c= 0$, then $a = 0$, which contradicts $P$ being invertible. So $c\neq 0$, meaning we can divide by $c^2$ and set $x = \frac{a}{c}$. Then the roots of $3x^2 + 3x - 2 = 0$ are:
            \begin{equation*}
            \begin{split}
                x = \frac{-3 \pm \sqrt{33}}{6},
            \end{split}
            \end{equation*}
        which gives:
            \begin{equation*}
            \begin{split}
                a = \frac{-3 \pm \sqrt{33}}{6} c.
            \end{split}
            \end{equation*}
        Since $c \neq 0$, $a \not\in \bfQ$. Thus we cannot diagonalize $A$ over $\bfQ$. But if we were to take $F = \bfQ(\sqrt{33})$, then we have that:
            \begin{equation*}
            \begin{split}
                \cB = \{v_1 = \pmat 1 \\ \frac{3 + \sqrt{33}}{4} \epmat, v_2 = \pmat 1 \\ \frac{3 - \sqrt{33}}{4} \epmat \},
            \end{split}
            \end{equation*}
            \begin{equation*}
            \begin{split}
                \left[T\right]_\cB = \bmat \frac{5 + \sqrt{33}}{2} & 0 \\ 0 & \frac{5 - \sqrt{33}}{2} \emat .
            \end{split}
            \end{equation*}
    \end{example}

    \begin{definition}
        Let $V$ be an $F$-vector space and $T \in \Hom_F{(V,V)}$. A subspace $W \subseteq V$ is said to be \textui{$T$-invariant} or \textui{$T$-stable} if $T(W) \subseteq W$.
    \end{definition}

    \begin{theorem}\label{thm:stable-block-diagonal}
        Let $\dim_F{(V)} = n$ and $W \subseteq V$ a $k$-dimensional subspace. Let $\cB_W = \{v_1,...,v_k\}$ be a basis of $W$ and extend to a basis $\cB = \{v_1,...,v_n\}$ of $V$. Let $T \in \Hom_F{(V,V)}$. We have $W$ is $T$-stable if and only if $\left[T\right]_\cB$ is block upper-triangular of the form
            \begin{equation*}
            \begin{split}
                \bmat A & B \\ 0& D \emat
            \end{split}
            \end{equation*}
        where $A = \left[\restr{T}{W}\right]_{\cB_W}$.
    \end{theorem}

    \begin{example}
        Let $V = \bfQ^4$ with basis $\cE_4 = \{e_1,e_2,...,e_4\}$ and define $T$ by:
            \begin{equation*}
            \begin{split}
                T(e_1) &= 2e_1 + 3e_3 \\
                T(e_2) &= e_1 + e_4 \\
                T(e_3) &= e_1 - e_3 \\
                T(e_4) &= 2e_1 - 2e_2 + 5e_3 - 4e_4.
            \end{split}
            \end{equation*}
        Set $W = \Span_\bfQ(e_1,e_3)$, then $W$ is $T$-stable. Since $\cB_W = \{e_1,e_3\}$ and $\cB = \{e_1,e_2,e_3,e_4\}$, we have:
            \begin{equation*}
            \begin{split}
                \left[T\right]_\cB
                &= \bmat 2 & 1 & 1 & 2 \\ 3 & -1 & 0 & 5 \\ 0 & 0 & 0 & -2 \\ 0 & 0 & 1 & -4 \emat
                \begin{tikzpicture}[overlay, remember picture]
                    \draw[gray, thick] (-1.85,-1.32) rectangle (-2.9,-0.1); % Adjust the coordinates for placement
                \end{tikzpicture}
            \end{split}
            \end{equation*}
    \end{example}

    \begin{example}
        A special case is when $\dim_F{W} = 1$. If $W = \Span_F{(w_1)}$ and $W$ is $T$-stable, then $T(w_1) \in W_1$; i.e., $T(w_1) = \lambda w_1$ for some $\lambda \in F$ Equivalently, this can be written as $(T-\lambda \id_V)(w_1) = 0_V$, meaning $w_1 \in \ker{(T-\lambda \id_V)}$.
    \end{example}

\section{Eigenvalues and Eigenvectors}

    \begin{definition}
        Let $T \in \Hom_F{(V,V)}$ and $\lambda \in F$. If $\ker{(T-\lambda \id_V)} \neq \{0_V\}$, we say $\lambda$ is an \textui{eigenvalue} of $T$. Any nonzero vector in $\ker{(T-\lambda \id_V)}$ is called a \textui{$\lambda$-eigenvector}. The set $E_\lambda^1 = \ker{(T - \lambda \id_V)}$ is called the \textui{eigenspace} associated with $\lambda$.
    \end{definition}

    \begin{exercise}
        Show that $E_\lambda^1$ is a subspace.
    \end{exercise}
    
    \begin{exercise}
        Let $T \in \Hom_F{(V,V)}$. If $\lambda_1,\lambda_2 \in F$ with $\lambda_1 \neq \lambda_2$, then $E_{\lambda_1}^1 \cap E_{\lambda_2}^1 = \{0_V\}$.
    \end{exercise}

    \begin{example}
        Let $A = \pmat 12 & 35 \\ -6 & 17 \epmat \in \Mat_2{(\bfQ)}$ and $T_A \in \Hom_\bfQ{(\bfQ^2,\bfQ^2)}$. We have:
            \begin{equation*}
            \begin{split}
                \phantom{a}\\
                \bmat -12 & 35 \\ -6 & 17 \emat \bmat 1 \\ \frac{2}{5} \emat &= 2 \bmat 1 \\ \frac{2}{5} \emat \\
                \phantom{a}\\
                \bmat -12 & 35 \\ -6 & 17 \emat \bmat 1 \\ \frac{3}{7} \emat &= 3 \bmat 1 \\ \frac{3}{7} \emat \\
            \end{split}
            \end{equation*}
        So $T_A$ has eigenvalues of $2$ and $3$. Then 
            \begin{equation*}
            \begin{split}
                E_2^1 &= \Span_\bfQ{ \left(v_1 =\pmat 1\\ 2/5 \epmat\right)}\\
                E_3^1 &= \Span_\bfQ{ \left(v_2 =\pmat 1\\ 3/7 \epmat\right)}\\
            \end{split}
            \end{equation*}
        gives:
            \begin{equation*}
            \begin{split}
                \left[T_A\right]_{\{v_1,v_2\}} = \bmat 2 & 0 \\ 0 & 3 \emat.
            \end{split}
            \end{equation*}
    \end{example}

    \begin{example}[$F \mathrlap{\lfloor}\lceil x \mathrlap{\rfloor}\rceil$-Modules]
        Let $T \in \Hom_F{(V,V)}$. Note that $V$ is by definition an $F$-module, but we are able to view $V$ as an $F[x]$-module given some linear transformation $T$. The action $ F[x] \times V \rightarrow V$ is defined by $(f(x),v) \mapsto f(T)(v)$.

        Write $T^m = \underbrace{T \circ T \circ ... \circ T}_{m-\text{times}}$. Write $f(x) \in F[x]$ as $f(x) = a_mx^m + ... + a_1x + a_0$. Then
            \begin{equation*}
            \begin{split}
                f(T) = a_m T^m + ... + a_1 T + a_0 \id_V \in \Hom_F{(V,V)}.
            \end{split}
            \end{equation*}
        For example, let $g(x) = 2x^2 + 3 \in \bfR[x]$. Then $g(T) = 2T^2 + 3\id_V$ and $g(T)(v) = 2T(T(v)) + 3v$. If $f(x) = g(x)h(x)$ for some $g(x),h(x) \in F[x]$, then $f(T) = g(T) \circ h(T)$. Instead of writing $f(T)(v) = g(T)(h(T)(v))$, we will abuse notation and write $g(T)h(T)(v)$. Normally function composition does not commute, but these do {\color{red} for some reason}.
    \end{example}

    \begin{theorem}
        Let $\dim_F{(V)} = n$ and $T \in \Hom_F{(V,V)}$. There is a unique monic polynomial $m_T(x) \in F[x]$ of lowest degree so that $m_T(T)(v) = 0_V$ for all $v \in V$. Moreover, $\deg_{m_T}{(T)}\leq n^2$.
    \end{theorem}
        \begin{proof}
            Recall that $\Hom_F{(V,V)}$ is an $F$-vector space. We have $\Hom_F{(V,V)} \cong \Mat_n{(F)}$, hence $\dim_F{(\Hom_F{(V,V)})} = n^2$.

            Given $T \in \Hom_F{(V,V)}$, consider the set $\{\id_V,T,T^2,...,T^{n^2}\} \subseteq \Hom_F{(V,V)}$. This has $n^2 + 1$ elements, so it must be linearly dependent (meaning a linear combination of some subset can equal $0$). Let $m$ be the smallest integer so that
                \begin{equation*}
                \begin{split}
                    a_mT^m + ... + a_1T + a_0 \id_V\footnotemark = 0_{\Hom_F{(V,V)}}.
                \end{split}
                \end{equation*}
            \footnotetext{This seems kind of out of nowhere, so think of it like this: Let $I_T =\{p \in F[x] \mid p(T)(v) = 0_V \mtext{for all}v \in V\}$. $F[x]$ is a P.I.D., so every ideal is generated by a single element. The minimal polynomial $m_T(x)$ is the generator of this ideal.}We obtain a set $\{\id_V,T,T^2,...,T^m\}$. Since $m$ is minimal, $a_m \neq 0$. Define:
                \begin{equation*}
                \begin{split}
                    m_T(x) = x^m + b_{m-1}x^{m-1} + ... + b_1x + b_0 \in F[x], \mtext{where} b_i = \frac{a_i}{a_m}.
                \end{split}
                \end{equation*}
            This gives $m_T(T) = 0_{\Hom_F{(V,V)}}$; i.e., $m_T(T)(v) = 0_V$ for all $v \in V$. 
                It remains to that $m_T(x)$ is unique. Suppose there exists an $f(x) \in F[x]$ which satisfies $f(T)(v) = 0_V$ for all $v \in V$. Write:
                    \begin{equation*}
                    \begin{split}
                        f(x) = m_T(x)q(x) + r(x)
                    \end{split}
                    \end{equation*}
                for some $q(x), r(x) \in F[x]$ with $r(x) = 0$ or $\deg{(r(x))} < \deg{(m_T(x))}$. We have for all $v \in V$:
                    \begin{equation*}
                    \begin{split}
                        0_V
                        & = f(T)(v) \\
                        & = q(T)m_T(T)(v) + r(T)(v) \\
                        & = q(T)(0_V) + r(T)(v) \\
                        & = r(T)(v)
                    \end{split}
                    \end{equation*}
                It must be the case that $r(x) = 0$, otherwise we have a polynomial of lower degree than $m_T(x)$ which kills all vectors. So $f(x) = m_T(x)q(x)$; i.e., $m_T(x) \mid f(x)$. But if $m_T(x)$ and $f(x)$ are both monic and of minimal degree, it must be the case that they are the same degree. This gives $m_T(x) = f(x)$.
        \end{proof}

    \begin{definition}
        The unique monic polynomial $m_T(x)$ is called the \textui{minimal polynomial} of $T$.
    \end{definition}

    \begin{corollary}
        If $f(x) \in F[x]$ satisfies $f(T)(v) = 0_V$ for all $v \in V$, then $m_T(x) \mid f(x)$.
    \end{corollary}
        \begin{proof}
            {\color{red} wat}
        \end{proof}
    
    \begin{example}
        Let $F = \bfQ$ and $A = \pmat 1 & 2  \\ 3 & 4 \epmat$. We can see that:
            \begin{equation*}
            \begin{split}
                A - a_0 1_2 \neq 0_2 \mtext{for any} a_0 \in F.
            \end{split}
            \end{equation*}
        But $A^2 = \pmat 7 & 10 \\ 15 & 22 \epmat$ gives $A^2 -5A -2\cdot1_2 = 0_2$. Hence $m_A(x) = x^2 - 5x - 2$. Note the relationship between this example and Example~\ref{example:field-extension-invertible}.
    \end{example}

    \begin{example}
        Let $V = \bfQ^3$, $\cE_3 = \{e_1,e_2,e_3\}$, and 
            \begin{equation*}
            \begin{split}
                \left[T_A\right]_{\cE_3} = A = \bmat 1 & 2 & 3 \\  0 & 1 & 4 \\ 0 & 0 & -1 \emat.
            \end{split}
            \end{equation*}
        Let $W = \Span_\bfQ(e_1)$ Then $T(W) = T(\alpha e_1) = \alpha e_1 \in W$. Hence $T(W) \subseteq W$, meaning $W$ is $T$-stable. This gives $1$ as an eigenvalue. On a completely unrelated note, $m_{T_A}(x) = (x-1)^2(x+1)$.
    \end{example}

    \begin{theorem}
        Let $V$ be an $F$-vector space and $T \in \Hom_F{(V,V)}$. We have $\lambda$ is an eigenvalue if and only if $\lambda$ is the root of $m_T(x)$. In particular, if $(x-\lambda) \mid m_T(x)$, then $E_\lambda^1 \neq \{0_V\}$ (i.e., there is a nonzero $v \in V$ such that $T(v) = \lambda v$).
    \end{theorem}
        \begin{proof}
            Let $\lambda$ be an eigenvalue with eigenvector $v$ and write $m_T(x) = x^m + ... + a_1 x + a_0$. We have:
                \begin{equation*}
                \begin{split}
                    0_V
                    & = m_T(T)(v) \\
                    & = (T^m + a_{m-1}T^{m-1} + ... + a_1 T + a_0 \id_V)(v) \\
                    & = T^m(v) + a_{m-1}T^{m-1}(v) + ... + a_1T(v) + a_0v \\
                    & = \lambda^m v + a_{m-1}\lambda^{m-1}v + ... + a_1 \lambda v + a_0 v \\
                    & = (\lambda^m  + a_{m-1}\lambda^{m-1} + ... + a_1 \lambda  + a_0)v \\
                    & = m_T(\lambda)\cdot v.
                \end{split}
                \end{equation*}
            Since $v \neq 0$ and $m_T(\lambda) \in F$, it must be the case that $m_T(\lambda) = 0$. Hence $\lambda$ is a root.

            Now suppose $m_T(\lambda) = 0$. This gives $m_T(x) = (x-\lambda)f(x)$ for some $f(x) \in F[x]$. Since $\deg{f(x)} < \deg{m_T(x)}$, this gives a nonzero vector $v \in V$ so that $f(T)(v) \neq 0$ (since $m_T(x)$ is the smallest polynomial that satisfies $m_T(T)(v) = 0_V$, it must be the case that there is a nonzero $v \in V$ that satisfies $f(T)(v) \neq 0$). Set $w = f(T)(v)$, then:
                \begin{equation*}
                \begin{split}
                    0_V
                    & = (T-\lambda \id_V)f(T) \\
                    & = (T - \lambda \id_V)w,
                \end{split}
                \end{equation*}
            which simplifies to $T(w) = \lambda w$. Thus $\lambda$ is an eigenvalue.
        \end{proof}

    \begin{corollary}
        Let $\lambda_1,...,\lambda_n \in F$ be distinct eigenvalues of $T$. For each $i$, let $v_i$ be an eigenvector with eigenvalue $\lambda_i$. The set $\{v_1,...,v_m\}$ is linearly independent.
    \end{corollary}
        \begin{proof}
            We have $m_T(x) = (x-\lambda_1)(x-\lambda_2)...(x-\lambda_m)f(x)$ for some $f(x) \in F[x]$. Suppose $a_1v_1 + ... + a_mv_m = 0_V$ for $a_i \in F$. Define $g_1(x) = (x-\lambda_2)...(x-\lambda_m)f(x)$. Note that $g_1(T)(v_i) = 0_V$ for $2 \leq i \leq m$. Then:
                \begin{equation*}
                \begin{split}
                    0_V 
                    & = g_1(T)(0_V) \\
                    & = \sum_{j=1}^m a_j g_1(T)(v_j) \\
                    & = a_1 g_1(T)(v_1) \\
                    & = a_1 g_1 (\lambda_1) v_1
                \end{split}
                \end{equation*}
            But $g_1(\lambda_1) \neq 0$ and $v\neq 0$,  so it must be that case that $a_1 =0$. Inductively, it follows for $2,...,m$.
        \end{proof}

    \begin{corollary}
        If $\deg{(m_T(x))} = \dim_F(V)$ and $m_T(x)$ has distinct roots, all of which are in $F$, then we can find a basis $\cB$ so that $[T]_\cB$ is diagonal.
    \end{corollary}

    \begin{example}
        Let $A = \pmat 1 & 0 & 0 \\ 0 & 2 & 0 \\ 0 & 0 & 2 \epmat$ and $B = \pmat 1 & 0 & 0 \\ 0 & 1 & 0 \\ 0 & 0 & 2 \epmat$. These matrices are not similar, however $m_A(x) = m_B(x) = (x-1)(x-2)$. The minimal polynomial is not enough information on the similarity of matrices.
    \end{example}

    \begin{example}\label{example:gen-eigen-vec}
        Let:
            \begin{equation*}
            \begin{split}
                A = \bmat 1 & 2 & 3 \\ 0 & 1 & 4 \\ 0 & 0 & -1 \emat.
            \end{split}
            \end{equation*}
        We have that $m_A(x) = (x-1)^2(x+1)$. Note that $Ae_1 = e_1$, so $E_1^1 \supseteq \Span_F{(e_1)}$ (or, more simply, $e_1 \in E_1^1$). Note that $Ae_2 = \pmat 2 \\ 1 \\ 0 \epmat$. So $e_2 \not\in E_1^1$ (another way of saying this is $(A-1_3)e_2 \neq \pmat 0 \\ 0 \\ 0 \epmat$). But now consider:
            \begin{equation*}
            \begin{split}
                 (A - 1_3)^2 = \bmat 0 & 0 & 2 \\ 0 & 0 & -8 \\ 0 & 0 & 4 \emat.
            \end{split}
            \end{equation*}
        We have $(A-1_3)^2 e_2 = \pmat 0 \\ 0\\ 0 \epmat$. Thus $e_1,e_2 \in \ker{(A-\id_{F^3})^2}$.
    \end{example}

    \begin{definition}
        Let $T \in \Hom_F{(V,V)}$. For $k \geq 1$, the \textui{$k^\text{th}$ generalized eigenspace} of $T$ associated to $\lambda$ is $E_\lambda^k = \ker(T - \lambda \id_V)^k = \{v \in V \mid (T-\lambda \id_V)^k v = 0_V \}$. Elements of $E_\lambda^k$ are called \textui{generalized eigenvectors}. Set $E_\lambda^\infty = \bigcup_{k \geq 1} E_\lambda^k$.
    \end{definition}

    \begin{example}
        Continuing Example~\ref{example:gen-eigen-vec}, let $\alpha e_1 + \beta e_2 \in \Span_F{(e_1,e_2)}$. Then:
            \begin{equation*}
            \begin{split}
                (A - 1_3)^2 (\alpha e_1 + \beta e_2)
                & = \alpha (A - 1_3)^2  e_1 + \beta(A - 1_3)^2  e_2
                 = \pmat 0 \\ 0 \\ 0 \epmat.
            \end{split}
            \end{equation*}
        So $\Span_F{(e_1,e_2)} \subseteq E_1^2$. We also have $-1$ as an eigenvalue with eigenvector $v_3 = \pmat \frac{1}{2} \\ -2 \\ 1 \epmat$. Check that $v_3 \not\in E_1^2$. So $\dim_F{(E_1^2)} \leq 2$; i.e., $E_1^2 = \Span_F{(e_1,e_2)}$. {\color{red} why does $v_3 \not\in E_1^2$ imply the dimension which implies containment in the other direction}.
    \end{example}
    
    \begin{lemma}\label{lemma:eventual-kernel}
        Let $V$ be a finite dimensional $F$-vector space, $\dim_F{(V)} = n$, and $T \in \Hom_F{(V,V)}$. There exists $m$ with $1 \leq m \leq n$ such that $\ker(T^,) = \ker(T^{m+1})$. Moreover, for such an $m$, $\ker(T^m) = \ker(T^{m+j})$ for all $j \geq 0$.
    \end{lemma}
        \begin{proof}
            We have $\ker(T^1) \subseteq \ker(T^2) \subseteq...$ If these containments are always strict, then the dimension increases indefinitely, which contradicts $\dim_F(V) = n$. Hence we have an $m$ with $1 \leq m \leq n$ and $\ker(T^m) = \ker(T^{m+1})$.

            Let $m$ be the smallest value where $\ker(T^m) = \ker(T^{m+1})$. We use induction on $j$. Base case of $j=1$ is what defines $m$. Assume $\ker(T^m) = \ker(T^{m+j})$ for all $1 \leq j \leq N$. Let $v \in \ker(T^{m+N+1})$. This gives:
                \begin{equation*}
                \begin{split}
                    0_V 
                    & = T^{m+N+1}(v) \\
                    & = T^{m+1}(T^N(v)).
                \end{split}
                \end{equation*}
            So $T^N(v) \in \ker(T^{m+1})$. However $\ker(T^{m+1}) = \ker(T^{m})$, so $T^N(v) \in \ker(T^{m})$. Hence:
                \begin{equation*}
                \begin{split}
                    0_V 
                    & = T^m(T^n(v)) \\
                    & = T^{m+N}(v), 
                \end{split}
                \end{equation*}
            so $v \in \ker(T^{m+N})$. Induction hypothesis gives $\ker(T^{m+N}) = \ker(T^m)$, giving $v \in \ker{(T^m)}$. Thus $\ker(T^{m+N+1}) \subseteq \ker(T^m)$. The other direction of containment is trivial.
        \end{proof}

    \begin{example}
        Let $\cB = \{v_1,...,v_n\}$ be a basis of $V$ and $T \in \Hom_F{(V,V)}$, $\lambda \in F$ such that:
            \begin{equation*}
            \begin{split}
                [T]_\cB = 
                \bmat
                \lambda & 1 & 0 & ... & 0 \\
                0 & \lambda  & 1 & ... & 0 \\
                0 & 0 & \lambda & ... & 0 \\
                \vdots & \vdots & \vdots & \ddots & 1 \\
                0 & 0 & 0 & 0 & \lambda
                \emat.
            \end{split}
            \end{equation*}
        In other words, $[T]_\cB$ contains $\lambda$ along the diagonal and $1$ along the super-diagonal. Let $A = [T]_\cB$. Consider:
            \begin{equation*}
            \begin{split}
                (A-\lambda 1_n) =
                \bmat
                0 & 1 & 0 & ... & 0 \\
                0 & 0  & 1 & ... & 0 \\
                0 & 0 & 0 & ... & 0 \\
                \vdots & \vdots & \vdots & \ddots & 1 \\
                0 & 0 & 0 & 0 & 0
                \emat.
            \end{split}
            \end{equation*}
        We get:
            \begin{equation*}
            \begin{split}
                (A - \lambda 1_n)v_1 &= 0_V \\
                (A - \lambda 1_n)v_2 &= v_1 \\
                &\vdots \\
                (A - \lambda 1_n)v_n &= v_{n-1}.
            \end{split}
            \end{equation*}
        This gives $E_\lambda^1 = \Span_F{(v_1)}$ (by the first equation). Now observe:
            \begin{equation*}
            \begin{split}
                (A - \lambda 1_n)^2 v_1 &= 0_V \\
                \phantom{a} \\
                (A - \lambda 1_n)^2 v_2 &= (A - \lambda 1_n)(A - \lambda 1_n)v_2 \\
                & = (A - \lambda 1_n)v_1 \\
                & = 0_V \\
                \phantom{a} \\
                (A - \lambda 1_n)^2 v_3 &= v_1 \\
                &\vdots \\
                (A - \lambda 1_n)^2 v_n &= v_{n-2}. 
            \end{split}
            \end{equation*}
        So $E_\lambda^2 = \Span_F{(v_1,v_2)}$. In general, we have that $E_\lambda^k = \Span_F{(v_1,...,v_k)}$. Moreover,  Lemma~\ref{lemma:eventual-kernel} gives $E_\lambda^1 \subseteq E_\lambda^2 \subseteq ... \subseteq E_\lambda^k$.
    \end{example}

    \begin{corollary}
        If $\dim_F(V) = n$ and $T \in \Hom_F{(V,V)}$, there exists an $m$ with $1 \leq m \leq n$ so that for any $\lambda \in F$, $E_\lambda^\infty = E_\lambda^m$.
    \end{corollary}

    \begin{theorem}
        Let $T \in \Hom_F{(V,V)}$, and $\lambda \in F$ with $(x-\lambda)^k \mid m_T(x)$. We have:
            \begin{equation*}
            \begin{split}
                \dim_F(E_\lambda^k) \geq k.
            \end{split}
            \end{equation*}
    \end{theorem}
        \begin{proof}
            Write $m_T(x) = (x-\lambda)^k f(x)$ where $f(x) \in F[x]$, $f(\lambda) \neq 0$. Define $g_k(x) = (x-\lambda)^k$. We have that $(x-\lambda)^{k-1}f(x) = g_{k-1}(x)f(x)$ is \textit{not} the minimal polynomial. So there is a $v \in V$ with $v \neq 0_V$ such that:
                \begin{equation*}
                \begin{split}
                    g_{k-1}(T)f(T)(v) \neq 0_V.
                \end{split}
                \end{equation*}
            Set $v_k = f(T)(v)$. Observe that:
                \begin{equation*}
                \begin{split}
                    \left(T - \lambda \id_V\right)^k (v_k)
                    & = (T-\lambda \id_V)^k f(T)(v)\\
                    & = m_T(T)(v) \\
                    & = 0_V.
                \end{split}
                \end{equation*}
            So $v_k \in E_\lambda^k$. Moreover, by our construction:
                \begin{equation*}
                \begin{split}
                    (T-\lambda \id_V)^{k-1}(v_k)
                    & = g_{k-1}(T)(v_k) \\
                    & = g_{k-1}(T)f(T)(v) \\
                    & \neq 0_V.
                \end{split}
                \end{equation*}
            Hence $v_k \in E_\lambda^{k} \setminus E_\lambda^{k-1}$. Now set $v_{k-1} = (T - \lambda \id_V)v_k = (T - \lambda \id_V)f(T)(v)$. Note:
                \begin{equation*}
                \begin{split}
                    (T - \lambda \id_V)^{k-1}(v_{k-1}) 
                    & = (T - \lambda \id_V)^{k-1} (T - \lambda \id_V)(v_k) \\
                    & = (T - \lambda \id_V)^k (v_k)\\
                    & = (T - \lambda \id_V)^k f(T)(v) \\
                    & = m_T(T)(v) \\
                    & = 0_V.
                \end{split}
                \end{equation*}
            So $v_{k-1} \in E_\lambda^{k-1}$. Again, by our construction:
                \begin{equation*}
                \begin{split}
                    (T - \lambda \id_V)^{k-2}(v_{k-1})
                    & = (T - \lambda \id_V)^{k-2}(T - \lambda \id_V)(v_k) \\
                    & =(T - \lambda \id_V)^{k-1}(v_k) \\
                    & \neq 0_V.
                \end{split}
                \end{equation*}
            So $v_{k-1} \in E_\lambda^{k-1} \setminus E_\lambda^{k-2}$. Setting $v_{k-2} = (T - \lambda \id_V)^2v_k$ gives a similar result. By this construction, we obtain a set $\{v_k,v_{k-1},...,v_2,v_1\}$. Claim: this set is linearly independent. Suppose towards contradiction it's not, that is, $a_1v_1 + ... + a_k v_k = 0_V$ does not imply $a_1 = ... = a_k = 0$. This gives $v_k = \frac{-1}{a_k}(a_1v_1 + ... + a_{k-1}v_{k-1}) \in E_\lambda^{k-1}$, which is a contradiction. It follows that $a_1 = ... = a_k = 0$, hence $\{v_k,v_{k-1},...,v_2,v_1\}$ is linearly independent (linear independent set $\subseteq$ a basis, so thats why the theorem is established).
        \end{proof}

    \begin{example}
        Let $T_A \in \Hom_F{(F^3,F^3)}$ be defined by:
            \begin{equation*}
            \begin{split}
                A = \bmat 2 & 1 & 3  \\ 0 & 2 & 4 \\ 0 & 0 & 2 \emat.
            \end{split}
            \end{equation*}
        We have that $m_T(x) = (x-2)^3$. Now observe:
            \begin{equation*}
            \begin{split}
                (A - 2\cdot 1_3)^2 = \bmat 0 & 0 & 4 \\ 0 & 0 & 0 \\ 0 & 0 & 0 \emat.
            \end{split}
            \end{equation*}
        Note $(A - 2\cdot 1_3)^2 e_3 = 4e_3 \neq 0_F^3$, but $(A - 2 \cdot 1_3)^3 e_3 = 0_{F^3}$. Set $v_3 = e_3$, we have $v_3 \in E_2^3$. Now observe:
            \begin{equation*}
            \begin{split}
                v_2 &= (A - 2 \cdot 1_3)(v_3) \\
                & = \bmat 0 & 1 & 3 \\ 0 & 0 & 4 \\ 0 & 0 & 0 \emat \bmat 0 \\ 0 \\ 1 \emat \\
                & = \bmat 3 \\ 4\\ 0 \emat.
            \end{split}
            \end{equation*}
        Similarly:
            \begin{equation*}
            \begin{split}
                v_1 &= (A - 2 \cdot 1_3)(v_2) \\
                & = ... \\
                & = \bmat 4 \\ 0\\ 0 \emat.
            \end{split}
            \end{equation*}
        Hence:
            \begin{equation*}
            \begin{split}
                E_2^3 &= \Span_F{(\pmat 0 \\ 0 \\ 1\epmat ,\pmat 3 \\ 4 \\ 0\epmat , \pmat 4 \\ 0 \\ 0\epmat)} \\
                E_2^2 & = \Span_F{(\pmat 3 \\ 4 \\ 0\epmat , \pmat 4 \\ 0 \\ 0\epmat)} \\
                E_2^1 & = \Span_F{(\pmat 4 \\ 0 \\ 0\epmat)}.
            \end{split}
            \end{equation*}
        Setting $\cB = \{v_1,v_2,v_3\}$, we have:
            \begin{equation*}
            \begin{split}
                \left[T_A\right]_\cB = \bmat 2& 1 & 0 \\ 0 & 2 & 1 \\ 0 & 0 & 2 \emat.
            \end{split}
            \end{equation*}
    \end{example}

    \iffalse
    \begin{definition}
        Let $A$ be a defective matrix, that is to say, it contains repeated eigenvalues. Then $A$ is similar to a matrix $J$, where $J$ is of the form:
            \begin{equation*}
            \begin{split}
                J = 
                \bmat
                \lambda_1 & 1 & \\
                 & \lambda_1 & 1 \\
                 & & \lambda_1 \\
                 & & & \lambda_2 & 1 \\
                 & & & & \lambda_2 \\
                 & & & & & \lambda_3 \\
                 & & & & & & \ddots \\
                 & & & & & & & \lambda_n & 1\\
                 & & & & & & & & \lambda_n  \\ 
                \emat.
                \begin{tikzpicture}[overlay, remember picture]
                    \draw[gray, thick] (-1.7,-3.35) rectangle (-0.4,-2); % Adjust the coordinates for placement
                \end{tikzpicture}
                \begin{tikzpicture}[overlay, remember picture]
                    \draw[gray, thick] (-3.18,-1.04) rectangle (-2.6,-0.5); % Adjust the coordinates for placement
                \end{tikzpicture}
                \begin{tikzpicture}[overlay, remember picture]
                    \draw[gray, thick] (-4.64,-0.3) rectangle (-3.4,1.1); % Adjust the coordinates for placement
                \end{tikzpicture}
                \begin{tikzpicture}[overlay, remember picture]
                    \draw[gray, thick] (-6.8,1.3) rectangle (-4.8,3.5); % Adjust the coordinates for placement
                \end{tikzpicture}
            \end{split}
            \end{equation*}
        We say $J$ is the \textui{Jordan normal form} of $A$. The outlined squares are known as \textui{Jordan blocks}.
    \end{definition}
    \fi

\section{Characteristic Polynomials}
    \begin{definition}
        Let $A \in \Mat_n(F)$. The \textui{characteristic polynomial} is $c_A(x) = \det(x 1_n - A)$.
    \end{definition}

    \begin{definition}
        Let $f(x) = x^n + a_{n-1}x^{n-1} + ... + a_1 x + a_0 \in F[x]$. The \textui{companion matrix} of $f(x)$ is given by:
            \begin{equation*}
            \begin{split}
                C(f(x)) = 
                \bmat
                -a_0 & 0 & 0 & ... & 0\\
                -a_1 & 1 & 0 & ... & 0\\
                -a_2 & 0 & 1 & ... & 0\\
                \vdots & \vdots & \vdots & \ddots & \vdots \\
                -a_{n-1} & 0 & 0 & ... & 1
                \emat
            \end{split}
            \end{equation*}
        The companion matrix shows that any polynomial $f(x) \in F[x]$ can be realized as the characteristic polynomial of a matrix.
    \end{definition}

    \begin{lemma}
        If $A = C(f(x))$, then $c_A(x) = f(x)$.
    \end{lemma}

    \begin{lemma}
        Let $A,B \in \Mat_n(F)$ be similar matrices. Then $c_A(x) = c_B(x)$.
    \end{lemma}
        \begin{proof}
            Let $A = PBP^{-1}$ for some $P \in GL_n(F)$. We have:
                \begin{equation*}
                \begin{split}
                    c_A(x)
                    & = \det(x1_n - A) \\
                    & = \det(x1_n - PBP^{-1}) \\
                    & = \det(P(x1_n)P^{-1} - PBP^{-1}) \\
                    & = \det(P(x1_n - B)P^{-1}) \\
                    & = \det(P)\det(x1_n - B)\det(P^{-1}) \\
                    & = \det(x1_n - B) \\
                    & = c_B(x).
                \end{split}
                \end{equation*}
        \end{proof}

    \begin{definition}
        For $T \in \Hom_F(V,V)$, let $\cB$ be a basis of $V$ and set $c_T(x) = c_{\left[T\right]_\cB}(x)$.
    \end{definition}

    \begin{theorem}
        Let $v \in V$, $v\neq 0_V$. Let $\dim_F(V) = n$. Then there is a unique monic polynomial $m_{T,v}(x) \in F[x]$ so that $m_{T,v}(T)(v) = 0_V$. Moreover, if $f(x) \in F[x]$ with $f(T)(v) = 0_V$, then $m_{T,v}(x) \mid f(x)$.
    \end{theorem}
        \begin{proof}
            Consider the set $\{v,T(v),T^2(v),...,T^n(v)\}$. Since this set contains $n+1$ elements and the dimension of $V$ is $n$, the set must be linearly dependent. Write:
                \begin{equation*}
                \begin{split}
                    a_mT^m(v) + ... + a_1T(v) + a_0 = 0_V
                \end{split}
                \end{equation*}
            for some $m \leq n$ of minimal order and $a_i \neq 0$ for all $i$. Set:
                \begin{equation*}
                \begin{split}
                    p(x) = x^m + \frac{a_{m-1}}{a_m}x^{m-1} + ... + \frac{a_{1}}{a_m}x + \frac{a_{0}}{a_m} \in F[x].
                \end{split}
                \end{equation*}
            By construction $p(T)(v) = 0_V$. Set $I_v = \{g(x) \in F[x] \mid g(T)(v) = 0_V \}$. We have that $p(x)$ is a monic nonzero polynomial in $I_v$ of minimal degree. Set $m_{T,v}(x) = p(x)$.

            Let $f(x) \in I_v$. We'd like to show that $m_{T,v}(x) \mid f(x)$. Write:
                \begin{equation*}
                \begin{split}
                    f(x) = q(x)m_{T,v}(x) + r(x),
                \end{split}
                \end{equation*}
            with $q(x),r(x) \in F[x]$ and $\deg(r(x)) = 0$ or $deg(r) < \deg(m_{T,v}(x))$. Observe that:
                \begin{equation*}
                \begin{split}
                    r(T)(v) &= f(T)(v) - q(T)m_{T,v}(T)(v) \\
                    & = 0_V - q(T)0_V \\
                    & = 0_V.
                \end{split}
                \end{equation*}
            So $r(x) \in I_v$. But $m_{T,v}(x)$ had minimal degree, so it must be the case that $r(x) = 0$. Thus $f(x) = q(x)m_{T,v}(x)$, implying $m_{T,v}(x) \mid f(x)$\footnote{The proof of $F[x]$ being a P.I.D. follows identically. Instead of considering $I_v$ we would consider an arbitrary polynomial in $F[x]$.}. Now suppose $h(x) \in I_v$ with $\deg(h(x)) = \deg(m_{T,v}(x))$. Since both polynomials are monic and of equal degree, if $m_{T,v}(x) \mid h(x)$ then $m_{T,v}(x) = h(x)$.
        \end{proof}

    \begin{definition}
        We refer to $m_{T,v}(x)$ as the \textui{$T$-annihilator} of $v$.
    \end{definition}

    \begin{example}
        Let $V = F^n$ and $\cE_n = \{e_1,...,e_n\}$. Define $T \in \Hom_F{(V,V)}$ by:
            \begin{equation*}
            \begin{split}
                T(e_1) &= 0_v \\ 
                T(e_j) &= e_{j-1} \mtext{for} 2 \leq j \leq n.
            \end{split}
            \end{equation*}
        Consider $f(x) = x$. Then $f(T)(e_1) = T(e_1) = 0_V$. Hence $m_{T,e_1}(x) \mid x$. So either $m_{T,e_1}(x) = 1$ or $m_{T,e_1}(x) = x$. But $\id_V(e_1) = e_1 \neq 0_V$, hence it must be the case that $m_{T,e_1}(x) = x$.

        Now consider $g(x) = x^2$. Then $g(T)(e_2) = T^2(e_2) = T(T(e_2)) = T(e_1) = 0_V$. Hence $m_{T,e_2}(x) \mid x^2$. So $m_{T,e_2}(x)= 1$ or $x$ or $x^2$. If $m_{T,e_2}(x) = 1$, then $\id_V(e_2) = e_2 \neq 0_V$. If $m_{T,e_2}(x) = x$, then $T(e_2) = e_1 \neq 0$. So $m_{T,e_2}(x) = x^2$. It follows for $i \leq j \leq n$, $m_{T,e_j}(x)=x^j$.
    \end{example}

    \begin{example}
        Let $V = \bfQ^2$. Define $T \in \Hom_\bfQ{(\bfQ^2,\bfQ^2)}$ by:
            \begin{equation*}
            \begin{split}
                T(e_1) &= e_1 + 3e_2 \\
                T(e_2) &= 2e_1 + 4e_2.
            \end{split}
            \end{equation*}
        We are trying to find $m_{T,e_1}(x)$. Since $V$ is two-dimensional, $\deg(m_{T,e_1}(x)) = 1$ or 2. Write $m_{T,e_1}(x) = x +a$. Then:
            \begin{equation*}
            \begin{split}
                m_{T,e_1}(T)(e_1)
                & = T(e_1) + ae_1 \\
                & = e_1 + 3e_2 + ae_1 \\
                & \neq 0_V.
            \end{split}
            \end{equation*}
        So it must be that $\deg(m_{T,e_1}(x)) = 2$. Note that:
            \begin{equation*}
            \begin{split}
                T^2(e_1)
                & = T(e_1 + 3e_2) \\
                & = T(e_1) + 3T(e_2) \\
                & = 7e_1 + 15e_2.
            \end{split}
            \end{equation*}
        Now let:
            \begin{equation*}
            \begin{split}
                T^2(e_1) + bT(e_1) + ce_1 = 0_V,
            \end{split}
            \end{equation*}
        for some $b,c \in \bfQ$. This will yield a system of equations, and solving for it gives:
            \begin{equation*}
            \begin{split}
                b & = -5 \\
                c & = -2.
            \end{split}
            \end{equation*}
        Hence $m_{T,e_1}(x) = x^2 - 5x - 2$.
    \end{example}

    \begin{exercise}
        \phantom{a}
        \begin{enumerate}
            \item Show $m_{T,e_2}(x) = x^2 - 5x -2$.
            \item Calculate $m_{T,e_1}(x)$ and $m_{T,e_2}(x)$ of $F = \bfF_3$.
        \end{enumerate}
    \end{exercise}

    \begin{theorem}
        Let $\dim_F{(V)} = n$ and $\cB = \{v_1,...,v_n\}$ be a basis of $V$. Let $T \in \Hom_F{(V,V)}$. We have:
            \begin{equation*}
            \begin{split}
                m_T(x) = \underset{1 \leq i \leq n}\lcm m_{T,v_i}(x).
            \end{split}
            \end{equation*}
    \end{theorem}
        \begin{proof}
            Let $f(x) = \underset{1 \leq i \leq n}\lcm m_{T,v_i}(x)$. Note that $m_T(T)(v_i) = 0_V$, so $m_{T,v_i}(x) \mid m_T(x)$ for each $i$. Hence $f(x) \mid m_T(x)$.

            Now let $v \in V$. Write $v = \sum_{i = 1}^n a_i v_i$. We have:
                \begin{equation*}
                \begin{split}
                    f(T)(v)
                    & = f(T)(\sum_{i = 1}^n a_i v_i) \\
                    & = \sum_{i = 1}^n a_i f(T)(v_i) \\
                    & = 0_V,
                \end{split}
                \end{equation*}
            because $m_{T,v_i}(x) \mid f(x)$ for all $i$. Hence $m_T(x)\mid f(x)$. {\color{red} i dont quite get this number theory stuff}
        \end{proof}
    
    \begin{lemma}
        Let $T \in \Hom_F{(V,V)}$. Let $v_1,...,v_k \in V$, and set $p_i(x) = m_{T,v_i}(x)$. Suppose $p_i(x)$ are pairwise relatively prime. Set $v = v_1+...+v_k$. Then:
            \begin{equation*}
            \begin{split}
                m_{T,v}(x) = p_1(x)...p_k(x).
            \end{split}
            \end{equation*}
    \end{lemma}
        \begin{proof}
            We prove this for $k \geq 2$; i.e., $m_{T,v_1 + v_2}(x) = m_{T,v_1}(x)m_{T,v_2}(x)$. Since $p_1(x)$ and $p_2(x)$ are relatively prime, there exists $q_1(x),q_2(x) \in F[x]$ so that $1 = p_1(x)q_1(x) + p_2(x)q_2(x)$. In particular, $\id_V = p_1(T)q_1(T) + p_2(T)q_2(T)$. Set $v = v_1 + v_2$. We have:
                \begin{equation*}
                \begin{split}
                    v
                    & = \id_V(v) \\
                    & = (p_1(T)q_1(T) + p_2(T)q_2(T))(v) \\
                    & = p_1(T)q_1(T)(v) + p_2(T)q_2(T)(v) \\
                    & = p_1(T)q_1(T)(v_1 + v_2) + p_2(T)q_2(T)(v_1 + v_2) \\
                    & = p_1(T)q_1(T)(v_2) + p_2(T)q_2(T)(v_2).
                \end{split}
                \end{equation*}
            Write $w_1 = p_1(T)q_1(T)(v_2)$ and $w_2 = p_2(T)q_2(T)(v_1)$. This means $v = w_1 + w_2$. Note:
                \begin{equation*}
                \begin{split}
                    p_1(T)(w_1)
                    & = p_1(T)p_2(T)q_2(T)(v_1) \\
                    & = q_2(T)p_2(T)\underbrace{p_1(T)(v_1)} \\
                    & = 0_V.
                \end{split}
                \end{equation*}
            Hence $w_1 \in \ker(p_1(T))$. It follows similarly that $w_1 \in \ker(p_2(T))$. Let $r(x) \in F[x]$ with $r(T)(v) = 0_V$. We have $v = w_1 + w_2$ and $w_2 \in \ker(p_2(T))$, so:
                \begin{equation*}
                \begin{split}
                    p_2(T)(v)
                    &  = p_2(T)(w_1 + w_2) \\
                    & = p_2(T)(w_1).
                \end{split}
                \end{equation*}
            Thus:
                \begin{equation*}
                \begin{split}
                    0_V 
                    & = p_2(T)q_2(T)(0_V) \\
                    & = p_2(T)q_2(T)r(T)(v) \\
                    & = r(T)p_2(T)q_2(T)(v) \\
                    & = r(T)p_2(T)q_2(T)(w_1).
                \end{split}
                \end{equation*}
            We also know $r(T)q_1(T)p_1(T)(w_1) = 0_V$ because $w_1 \in \ker(p_1(T))$. Hence:
                \begin{equation*}
                \begin{split}
                    0_V 
                    & = r(T)p_2(T)q_2(T)(w_1) + r(T)p_1(T)q_1(T)(w_1) \\
                    & = r(T)\underbrace{\left(p_2(T)q_2(T) + p_1(T)q_1(T)\right)}_{\id_V} (w_1) \\
                    & = r(T)(w_1).
                \end{split}
                \end{equation*}
            This gives:
                \begin{equation*}
                \begin{split}
                    0_V 
                    & = r(T)(w_1) \\
                    & = r(T)p_2(T)q_2(T)(v_1).
                \end{split}
                \end{equation*}
            So $r(T)p_2(T)q_2(T)(v_1) = 0_V$. Thus $p_1(x)\mid r(x)p_2(x)q_2(x)$. Now note that:
                \begin{equation*}
                \begin{split}
                    \gcd{(p_1(x),p_2(x)q_2(x))} = 1,
                \end{split}
                \end{equation*}
            which means $p_1(x)\mid r(x)$. A similar argument shows $p_2(x) \mid r(x)$. And since $\gcd(p_1(x),p_2(x)) = 1$, this gives $\lcm(p_1(x),p_2(x)) = p_1(x)p_2(x)$. So $p_1(x)p_2(x) \mid r(x)$. Since $r(x)$ was arbitrary, take $r(x) = m_{T,v}(x)$. Then $p_1(x)p_2(x) \mid m_{T,v}(x)$. Finally, since $p_1(x)p_2(x)(v) = 0_V$, $m_{T,v}(x) \mid p_1(x)p_2(x)$, establishing the lemma.
        \end{proof}

    \begin{exercise}
        Show inductively that $m_{T,v} = p_1(x)p_2(x)...p_k(x)$ {\color{red} does k have to be even? what does pairwise coprime mean if k = 3}.
    \end{exercise}

    \begin{theorem}
        Let $T \in \Hom_F(V,V)$. There exists $v \in V$ such that $m_{T,v}(x) = m_T(x)$. In particular, $\deg(m_T(x)) \leq n$.
    \end{theorem}
        \begin{proof}
            Let $\cB = \{v_1,...,v_n\}$ be a basis. We know:
            \begin{equation*}
                \begin{split}
                    m_T(x) = \underset{1 \leq i \leq n}\lcm m_{T,v_i}(x).
                \end{split}
                \end{equation*}
            Factor $m_T(x) = p_1(x)^{e_1}...p_k(x)^{e_k}$, with each $p_i(x)$ relatively prime and $e_1 \geq 1$. For $1 \leq j \leq k$, there exists $i_j \in \{1,...,n\}$ and $q_{i_j}(x) \in F[x]$ with:
                \begin{equation*}
                \begin{split}
                    m_{T,v_{i_j}}(x) = p_j(x)^{e_j}q_{i_j}(x).
                \end{split}
                \end{equation*}
            Set $w_j = q_{i_j}(T)(v_{i_j})$. This gives:
                \begin{equation*}
                \begin{split}
                    m_{T,w_j}(x) = p_j(x)^{e_j}.
                \end{split}
                \end{equation*}
            Now set $w = w_1 + ... + w_k$. The previous result gives $m_{T,w}(x) = p_1(x)^{e_1}...p_k(x)^{e_k} = m_T(x)${\color{red} ????}.
        \end{proof}
    
    \begin{lemma}
        Let $W \subseteq V$ be a $T$-invariant subspace. Then there is an induced map $\overline{T} \in \Hom_F(V/W,V/W)$ defined by $\overline{T}(v+W) = T(v)+W$.
    \end{lemma}

    \begin{lemma}
        Let $v \in V$. Then $m_{\overline{T},[v]}(x) \mid m_{T,v}(x)$. Similarly, $m_{\overline{T}}(x) \mid m_T(x)$.
    \end{lemma}
        \begin{proof}
            We have:
                \begin{equation*}
                \begin{split}
                    m_{T,v}(\overline{T})([v])
                    & = m_{T,v}(\overline{T})(v+W) \\
                    & = m_{T,v}(T)(v) + W \\
                    & = 0_V + W \\
                    & = 0_{V/W}.
                \end{split}
                \end{equation*}
            Then by definition of (in this case, $m_{\overline{T},[v]}(x)$) annihilator polynomials, $m_{\overline{T},[v]}(x) \mid m_{T,v}(x)$.
        \end{proof}

    \begin{definition}
        Let $T \in \Hom_F(V,V)$ and $\cA = \{v_1,...,v_k\}$ a set of vectors in $V$. The \textui{$T$-span} of $\cA$ is the subspace:
            \begin{equation*}
            \begin{split}
                W = \left\{\sum_{i=1}^k p_i(T)(v_i) \mid p_i(x) \in F[x]\right\}.
            \end{split}
            \end{equation*}
        We say the subset $W$ is \textui{$T$-generated} by $\cA$.
    \end{definition}

    \begin{exercise}\label{exercise:t-span}
        Show $W$ is a $T$-invariant subspace of $V$. Moreover, show it is the smallest $T$-invariant subspace with respect to inclusion of $V$ that contains $\cA$.
    \end{exercise}

    \begin{example}
        Let $V = \bfQ^4$. Define $T \in \Hom_\bfQ{(\bfQ^4,\bfQ^4)}$ by:
            \begin{equation*}
            \begin{split}
                T(e_1) &= 2e_1 + 3e_3 \\
                T(e_2) &= e_1 + e_2 \\
                T(e_3) &= e_1 - e_3 \\
                T(e_4) &= 2e_1 - 2e_2 + 5e_3 - 4e_4.\\
            \end{split}
            \end{equation*}
        Let $\cA = \{e_1\}$. Our goal is to find $\Span_{T,\bfQ}(\cA)$. Set $p(x) = 1$, then $p(T)(e_1) = \id_V(e_1) =e_1$. Hence $e_1 \in \Span_{T,F}(\cA)$. Now set $q(x) = \frac{1}{3}(x-2)$. Then:
            \begin{equation*}
            \begin{split}
                q(T)(e_1)
                & = \frac{1}{3}(T-2\id_V)(e_1)\\
                & = \frac{1}{3}(T(e_1) - 2e_1) \\
                & = e_3.
            \end{split}
            \end{equation*}
        Hence $e_3 \in \Span_{T,F}(\cA)$. So $\Span_F{(e_1,e_3)} \subseteq \Span_{T,F}(\cA)$ {\color{red} Why}. Note that $\Span_F(e_1,e_3)$ is $T$-invariant. By Exercise~\ref{exercise:t-span}, since $\Span_{T,F}(\cA)$ is the smallest $T$-invariant subspace by inclusion, it must be the case that $\Span_{T,F}(\cA) \subseteq \Span_F{(e_1,e_3)}$. Hence $\Span_{T,F}(\cA) = \Span_F{(e_1,e_3)}$.
    \end{example}

    \begin{lemma}\label{lemma:4.3.9}
        Let $T \in \Hom_F{(V,V)}$, $w \in V$, and $W$ the subspace of $V$ that is $T$-generated by $\{w\}$. Then $\dim_F{(W)} = \deg(m_{T,w}(x))$.
    \end{lemma}
        \begin{proof}
            Let $\deg(m_{T,w}(x)) = k$. Consider the set $\{w,T(w),...,T^{k-1}(w)\}$. This is a basis of the $T$-span of $\{w\}$.
        \end{proof}

    \begin{theorem}
        Let $\dim_F(V) = n$.
        \begin{enumerate}[label = (\arabic*)]
            \item We have $m_T(x) \mid c_T(x)$.
            \item Every irreducible factor of $c_T(x)$ is a factor of $m_T(x)$.
        \end{enumerate}
    \end{theorem}
        \begin{proof}
            (1) Let $\deg(m_T(x)) = k \leq n$. Let $v \in V$ with $m_T(x) = m_{T,v}(x)$. Let $W_1$ be the $T$-span of $\{v\}$. By Lemma~\ref{lemma:4.3.9}, $\dim_F(W_1) = k$. Set $v_k = v$ and $v_{k-i} = T^i(v)$. We have $\cB_1 = \{v_1,...,v_k\}$ is a basis of $W_1$ (see proof of previous lemma), and furthermore $\left[\restr{T}{W_1}\right]_\cB = C(m_T(x))$ (See Theorem~\ref{thm:stable-block-diagonal}) {\color{red} I would like an explanation of this}. We proceed with cases.

            Case 1: $k = n$. Then $W_1 = V$, and $[T]_{\cB_1} = C(m_T(x))$, which has characteristic polynomial $m_T(x)$, meaning $m_T(x) = c_T(x)$.

            Case 2: $k < n$. Expand $\cB_1$ to a full basis of $V$ as follows: Let $\cB_2 = \{v_{k+1},...,v_n\}$ and write:
                \begin{equation*}
                \begin{split}
                    \cB = \cB_1 \cup \cB_2.
                \end{split}
                \end{equation*}
            We have:
                \begin{equation*}
                \begin{split}
                    \left[T\right]_\cB = \bmat A & B \\ 0 & D \emat, \hspace{4pt} A = C(m_T(x)).
                \end{split}
                \end{equation*}
            Note that, since $\cB_1$ is a basis for $W_1$ and $W_1$ is $T$-invariant, by Theorem~\ref{thm:stable-block-diagonal} $[T]_\cB$ is block diagonal. We have:
                \begin{equation*}
                \begin{split}
                    c_T(x)
                    & = \det(x1_n - \left[T\right]_\cB) \\
                    & = \det \bmat x1_k - A & -B \\ 0 & x1_{n-k}-D \emat \\
                    & = \det(x1_k - A)\det(x1_{n-k})-D \\
                    & = c_A(x)\det(x1_{n-k}-D).
                \end{split}
                \end{equation*}
            Thus $m_T(x) \mid c_T(x)$.

            For (2), we induct on $\dim_F(V) = n$. If $n = 1$, then both the characteristic polynomial and minimal polynomial are monic and of degree $1$, hence they are equal. If $\deg{(m_T(x))} = n$, then $m_T(x) \mid c_T(x)$. Since both are degree $n$ and monic, they must be equal. Now suppose $\deg{m_T(x)} = k < n$ {\color{red}  The rest of this proof is hard and I dont get it. I atleast understand why $m_T(x) \mid c_T(x)$. Maybe I'll do the rest later}.
        \end{proof}

    \begin{example}
        Consider:
            \begin{equation*}
            \begin{split}
                A = 
                \bmat
                1& 2 \\
                3 & 4 \\
                & & 3 & 7 \\
                & & -1 & 2 \\
                & & & & -5 & 6 \\
                & & & & 2 & -3 \\
                \emat \in \Mat_9(\bfQ).
            \end{split}
            \end{equation*}
        We can verify that $c_A(x) = (x^2-5x-2)(x^2-x-1)(x^2+8x+3)$. Since every irreducible factor of $c_T(x)$ is a factor of $m_T(x)$, we have that $m_T(x) = (x^2-5x-2)(x^2-x-1)(x^2+8x+3)$.
    \end{example}

    \begin{theorem}[Cayley-Hamilton]\label{thm:cayley-hamilton}
        \phantom{a}
        \begin{enumerate}[label = (\arabic*)]
            \item Let $T \in \Hom_F(V,V)$ and $\dim_F(V) < \infty$. Then $c_T(T) = 0_{\Hom_F{(V,V)}}$.
            \item Let $A \in \Mat_n(F)$. Then $c_A(A) = 0_n$.
        \end{enumerate}
    \end{theorem}
        \begin{proof}
            Write $c_T(x) = f(x)m_T(x)$. Then for any $v \in V$:
                \begin{equation*}
                \begin{split}
                    c_T(T)(v)
                    & = f(T)m_T(T)(v) \\
                    & = f(T)(0_V) \\
                    & = 0_V.
                \end{split}
                \end{equation*}
        \end{proof}

\section{Jordan Canonical Form}
For this section $V$ is always finite-dimensional.

    \begin{definition}
        Let $T \in \Hom_F(V,V)$. A \textui{Jordan basis} for $V$ with respect to $T$ is a basis $\cB$ so that:
            \begin{equation*}
            \begin{split}
                \left[T\right]_\cB = 
                        \bmat
                        \lambda_1 & 1 & \\
                        & \lambda_1 & 1 \\
                        & & \lambda_1 \\
                        & & & \lambda_2 & 1 \\
                        & & & & \lambda_2 \\
                        & & & & & \lambda_3 \\
                        & & & & & & \ddots \\
                        & & & & & & & \lambda_n & 1\\
                        & & & & & & & & \lambda_n  \\ 
                        \emat.
                        \begin{tikzpicture}[overlay, remember picture]
                            \draw[gray, thick] (-1.7,-3.35) rectangle (-0.4,-2); % Adjust the coordinates for placement
                        \end{tikzpicture}
                        \begin{tikzpicture}[overlay, remember picture]
                            \draw[gray, thick] (-3.27,-1.04) rectangle (-2.8,-0.5); % Adjust the coordinates for placement
                        \end{tikzpicture}
                        \begin{tikzpicture}[overlay, remember picture]
                            \draw[gray, thick] (-4.9,-0.3) rectangle (-3.5,1.1); % Adjust the coordinates for placement
                        \end{tikzpicture}
                        \begin{tikzpicture}[overlay, remember picture]
                            \draw[gray, thick] (-7.14,1.3) rectangle (-5.1,3.5); % Adjust the coordinates for placement
                        \end{tikzpicture}
            \end{split}
            \end{equation*}
        for some $\lambda_1,...,\lambda_n \in F$. More generally, if $V = V_1 \oplus ... \oplus V_k$ is a decomposition into $T$-invariant subspaces, and each $V_i$ has a Jordan basis $\cB_i$, we say $\cB = \bigcup_{i= 1 }^k\cB_i$ is a Jordan basis for $V$.
    \end{definition}

    \begin{definition}
        A matrix of the form:
            \begin{equation*}
            \begin{split}
                J_i = 
                \begin{bmatrix}
                    \lambda_i & 1            &      &   \\
                            & \lambda_i    & \ddots &   \\
                            &            & \ddots & 1   \\
                            &            &      & \lambda_i       
                \end{bmatrix}
            \end{split}
            \end{equation*}
        is called a \textui{Jordan block} associated to $\lambda_i$. We say a matrix $J$ is in \textui{Jordan canonical form} if it is a block diagonal matrix with each block representing a Jordan block.
            \begin{equation*}
            \begin{split}
                J = 
                \begin{bmatrix}
                    J_1 & \;     & \; \\
                    \;  & \ddots & \; \\ 
                    \;  & \;     & J_p
                \end{bmatrix}.
            \end{split}
            \end{equation*}
    \end{definition}



    

    


