\documentclass[11pt,twoside,openany]{memoir}
\usepackage{mlmodern}
%\usepackage{tgpagella} % text only
%\usepackage{mathpazo}  % math & text
\usepackage[T1]{fontenc}
\usepackage[hidelinks]{hyperref}
\usepackage{amsmath}
\usepackage{amsthm}
\usepackage{amssymb}
\usepackage{mathtools}
%\usepackage{newpxtext}
%\usepackage{eulerpx}
%\usepackage{eucal}
\usepackage{datetime}
    \newdateformat{specialdate}{\THEYEAR\ \monthname\ \THEDAY}
\usepackage[margin=1in]{geometry}
\usepackage{fancyhdr}
    \fancyhf{}
    \pagestyle{fancy}
    \cfoot{\scriptsize \thepage}
    \fancyhead[R]{\scriptsize \rightmark}
    \fancyhead[L]{\scriptsize \leftmark}
    \renewcommand{\headrulewidth}{0pt}
    \renewcommand{\footrulewidth}{0pt} % if you also want to remove the footer rule
\usepackage{thmtools}
    \declaretheoremstyle[
        spaceabove=10pt,
        spacebelow=10pt,
        headfont=\normalfont\bfseries,
        notefont=\mdseries, notebraces={(}{)},
        bodyfont=\normalfont,
        postheadspace=0.5em
        %qed=\qedsymbol
        ]{defs}

    \declaretheoremstyle[ 
        spaceabove=10pt, % space above the theorem
        spacebelow=10pt,
        headfont=\normalfont\bfseries,
        bodyfont=\normalfont\itshape,
        postheadspace=0.5em
        ]{thmstyle}
    
    \declaretheorem[
        style=thmstyle,
        numberwithin=section
    ]{theorem}

    \declaretheorem[
        style=thmstyle,
        sibling=theorem,
    ]{proposition}

    \declaretheorem[
        style=thmstyle,
        sibling=theorem,
    ]{lemma}

    \declaretheorem[
        style=thmstyle,
        sibling=theorem,
    ]{corollary}

    \declaretheorem[
        numberwithin=section,
        style=defs,
    ]{example}

    \declaretheorem[
        numberwithin=section,
        style=defs,
    ]{definition}

    \declaretheorem[
        style=defs,
        numbered=unless unique,
    ]{problem}

    \declaretheorem[
        numbered=unless unique,
        shaded={rulecolor=black,
    rulewidth=1pt, bgcolor={rgb}{1,1,1}}
    ]{axiom}

    \declaretheorem[numberwithin=section,style=defs]{note}
    \declaretheorem[numbered=unless unique,style=defs]{question}
    \declaretheorem[numbered=no,style=defs]{recall}
    \declaretheorem[numbered=no,style=remark]{answer}
    \declaretheorem[numbered=no,style=remark]{solution}

    \declaretheorem[numbered=no,style=defs]{remark}
\usepackage{enumitem}
\usepackage{titlesec}
    \titleformat{\chapter}[display]
    {\bfseries\LARGE\raggedright}
    {Chapter {\thechapter}}
    {1ex minus .1ex}
    {\Huge}
    \titlespacing{\chapter}
    {3pc}{*3}{40pt}[3pc]

    \titleformat{\section}[block]
    {\normalfont\bfseries\Large}
    {\S\ \thesection.}{.5em}{}[]
    \titlespacing{\section}
    {0pt}{3ex plus .1ex minus .2ex}{3ex plus .1ex minus .2ex}
\usepackage[utf8x]{inputenc}
\usepackage{tikz}
\usepackage{tikz-cd}
\usepackage{wasysym}
\usepackage{pgf}

\linespread{1}
%to make the correct symbol for Sha
%\newcommand\cyr{%
%\renewcommand\rmdefault{wncyr}%
%\renewcommand\sfdefault{wncyss}%
%\renewcommand\encodingdefault{OT2}%
%\normalfont \selectfont} \DeclareTextFontCommand{\textcyr}{\cyr}


\DeclareMathOperator{\ab}{ab}
\newcommand{\absgal}{\G_{\bbQ}}
\DeclareMathOperator{\ad}{ad}
\DeclareMathOperator{\adj}{adj}
\DeclareMathOperator{\alg}{alg}
\DeclareMathOperator{\Alt}{Alt}
\DeclareMathOperator{\Ann}{Ann}
\DeclareMathOperator{\arith}{arith}
\DeclareMathOperator{\Aut}{Aut}
\DeclareMathOperator{\Be}{B}
\DeclareMathOperator{\card}{card}
\DeclareMathOperator{\Char}{char}
\DeclareMathOperator{\csp}{csp}
\DeclareMathOperator{\codim}{codim}
\DeclareMathOperator{\coker}{coker}
\DeclareMathOperator{\coh}{H}
\DeclareMathOperator{\compl}{compl}
\DeclareMathOperator{\conj}{conj}
\DeclareMathOperator{\cont}{cont}
\DeclareMathOperator{\crys}{crys}
\DeclareMathOperator{\Crys}{Crys}
\DeclareMathOperator{\cusp}{cusp}
\DeclareMathOperator{\diag}{diag}
\DeclareMathOperator{\disc}{disc}
\DeclareMathOperator{\dR}{dR}
\DeclareMathOperator{\Eis}{Eis}
\DeclareMathOperator{\End}{End}
\DeclareMathOperator{\ev}{ev}
\DeclareMathOperator{\eval}{eval}
\DeclareMathOperator{\Eq}{Eq}
\DeclareMathOperator{\Ext}{Ext}
\DeclareMathOperator{\Fil}{Fil}
\DeclareMathOperator{\Fitt}{Fitt}
\DeclareMathOperator{\Frob}{Frob}
\DeclareMathOperator{\G}{G}
\DeclareMathOperator{\Gal}{Gal}
\DeclareMathOperator{\GL}{GL}
\DeclareMathOperator{\Gr}{Gr}
\DeclareMathOperator{\Graph}{Graph}
\DeclareMathOperator{\GSp}{GSp}
\DeclareMathOperator{\GUn}{GU}
\DeclareMathOperator{\Hom}{Hom}
\DeclareMathOperator{\id}{id}
\DeclareMathOperator{\Id}{Id}
\DeclareMathOperator{\Ik}{Ik}
\DeclareMathOperator{\IM}{Im}
\DeclareMathOperator{\Image}{im}
\DeclareMathOperator{\Ind}{Ind}
\DeclareMathOperator{\Inf}{inf}
\DeclareMathOperator{\Isom}{Isom}
\DeclareMathOperator{\J}{J}
\DeclareMathOperator{\Jac}{Jac}
\DeclareMathOperator{\lcm}{lcm}
\DeclareMathOperator{\length}{length}
\DeclareMathOperator{\Log}{Log}
\DeclareMathOperator{\M}{M}
\DeclareMathOperator{\Mat}{Mat}
\DeclareMathOperator{\N}{N}
\DeclareMathOperator{\Nm}{Nm}
\DeclareMathOperator{\NIk}{N-Ik}
\DeclareMathOperator{\NSK}{N-SK}
\DeclareMathOperator{\new}{new}
\DeclareMathOperator{\obj}{obj}
\DeclareMathOperator{\old}{old}
\DeclareMathOperator{\ord}{ord}
\DeclareMathOperator{\Or}{O}
\DeclareMathOperator{\PGL}{PGL}
\DeclareMathOperator{\PGSp}{PGSp}
\DeclareMathOperator{\rank}{rank}
\DeclareMathOperator{\Rel}{Rel}
\DeclareMathOperator{\Real}{Re}
\DeclareMathOperator{\RES}{res}
\DeclareMathOperator{\Res}{Res}
%\DeclareMathOperator{\Sha}{\textcyr{Sh}}
\DeclareMathOperator{\Sel}{Sel}
\DeclareMathOperator{\semi}{ss}
\DeclareMathOperator{\sgn}{sign}
\DeclareMathOperator{\SK}{SK}
\DeclareMathOperator{\SL}{SL}
\DeclareMathOperator{\SO}{SO}
\DeclareMathOperator{\Sp}{Sp}
\DeclareMathOperator{\Span}{span}
\DeclareMathOperator{\Spec}{Spec}
\DeclareMathOperator{\spin}{spin}
\DeclareMathOperator{\st}{st}
\DeclareMathOperator{\St}{St}
\DeclareMathOperator{\SUn}{SU}
\DeclareMathOperator{\supp}{supp}
\DeclareMathOperator{\Sup}{sup}
\DeclareMathOperator{\Sym}{Sym}
\DeclareMathOperator{\Tam}{Tam}
\DeclareMathOperator{\tors}{tors}
\DeclareMathOperator{\tr}{tr}
\DeclareMathOperator{\un}{un}
\DeclareMathOperator{\Un}{U}
\DeclareMathOperator{\val}{val}
\DeclareMathOperator{\vol}{vol}

\DeclareMathOperator{\Sets}{S \mkern1.04mu e \mkern1.04mu t \mkern1.04mu s}
    \newcommand{\cSets}{\scalebox{1.02}{\contour{black}{$\Sets$}}}
    
\DeclareMathOperator{\Groups}{G \mkern1.04mu r \mkern1.04mu o \mkern1.04mu u \mkern1.04mu p \mkern1.04mu s}
    \newcommand{\cGroups}{\scalebox{1.02}{\contour{black}{$\Groups$}}}

\DeclareMathOperator{\TTop}{T \mkern1.04mu o \mkern1.04mu p}
    \newcommand{\cTop}{\scalebox{1.02}{\contour{black}{$\TTop$}}}

\DeclareMathOperator{\Htp}{H \mkern1.04mu t \mkern1.04mu p}
    \newcommand{\cHtp}{\scalebox{1.02}{\contour{black}{$\Htp$}}}

\DeclareMathOperator{\Mod}{M \mkern1.04mu o \mkern1.04mu d}
    \newcommand{\cMod}{\scalebox{1.02}{\contour{black}{$\Mod$}}}

\DeclareMathOperator{\Ab}{A \mkern1.04mu b}
    \newcommand{\cAb}{\scalebox{1.02}{\contour{black}{$\Ab$}}}

\DeclareMathOperator{\Rings}{R \mkern1.04mu i \mkern1.04mu n \mkern1.04mu g \mkern1.04mu s}
    \newcommand{\cRings}{\scalebox{1.02}{\contour{black}{$\Rings$}}}

\DeclareMathOperator{\ComRings}{C \mkern1.04mu o \mkern1.04mu m \mkern1.04mu R \mkern1.04mu i \mkern1.04mu n \mkern1.04mu g \mkern1.04mu s}
    \newcommand{\cComRings}{\scalebox{1.05}{\contour{black}{$\ComRings$}}}

\DeclareMathOperator{\hHom}{H \mkern1.04mu o \mkern1.04mu m}
    \newcommand{\cHom}{\scalebox{1.02}{\contour{black}{$\hHom$}}}

         %  \item $\cGroups$
          %  \item $\cTop$
          %  \item $\cHtp$
          %  \item $\cMod$




\renewcommand{\k}{\kappa}
\newcommand{\Ff}{F_{f}}
\newcommand{\ts}{\,^{t}\!}


%Mathcal

\newcommand{\cA}{\mathcal{A}}
\newcommand{\cB}{\mathcal{B}}
\newcommand{\cC}{\mathcal{C}}
\newcommand{\cD}{\mathcal{D}}
\newcommand{\cE}{\mathcal{E}}
\newcommand{\cF}{\mathcal{F}}
\newcommand{\cG}{\mathcal{G}}
\newcommand{\cH}{\mathcal{H}}
\newcommand{\cI}{\mathcal{I}}
\newcommand{\cJ}{\mathcal{J}}
\newcommand{\cK}{\mathcal{K}}
\newcommand{\cL}{\mathcal{L}}
\newcommand{\cM}{\mathcal{M}}
\newcommand{\cN}{\mathcal{N}}
\newcommand{\cO}{\mathcal{O}}
\newcommand{\cP}{\mathcal{P}}
\newcommand{\cQ}{\mathcal{Q}}
\newcommand{\cR}{\mathcal{R}}
\newcommand{\cS}{\mathcal{S}}
\newcommand{\cT}{\mathcal{T}}
\newcommand{\cU}{\mathcal{U}}
\newcommand{\cV}{\mathcal{V}}
\newcommand{\cW}{\mathcal{W}}
\newcommand{\cX}{\mathcal{X}}
\newcommand{\cY}{\mathcal{Y}}
\newcommand{\cZ}{\mathcal{Z}}


%mathfrak (missing \fi)

\newcommand{\fa}{\mathfrak{a}}
\newcommand{\fA}{\mathfrak{A}}
\newcommand{\fb}{\mathfrak{b}}
\newcommand{\fB}{\mathfrak{B}}
\newcommand{\fc}{\mathfrak{c}}
\newcommand{\fC}{\mathfrak{C}}
\newcommand{\fd}{\mathfrak{d}}
\newcommand{\fD}{\mathfrak{D}}
\newcommand{\fe}{\mathfrak{e}}
\newcommand{\fE}{\mathfrak{E}}
\newcommand{\ff}{\mathfrak{f}}
\newcommand{\fF}{\mathfrak{F}}
\newcommand{\fg}{\mathfrak{g}}
\newcommand{\fG}{\mathfrak{G}}
\newcommand{\fh}{\mathfrak{h}}
\newcommand{\fH}{\mathfrak{H}}
\newcommand{\fI}{\mathfrak{I}}
\newcommand{\fj}{\mathfrak{j}}
\newcommand{\fJ}{\mathfrak{J}}
\newcommand{\fk}{\mathfrak{k}}
\newcommand{\fK}{\mathfrak{K}}
\newcommand{\fl}{\mathfrak{l}}
\newcommand{\fL}{\mathfrak{L}}
\newcommand{\fm}{\mathfrak{m}}
\newcommand{\fM}{\mathfrak{M}}
\newcommand{\fn}{\mathfrak{n}}
\newcommand{\fN}{\mathfrak{N}}
\newcommand{\fo}{\mathfrak{o}}
\newcommand{\fO}{\mathfrak{O}}
\newcommand{\fp}{\mathfrak{p}}
\newcommand{\fP}{\mathfrak{P}}
\newcommand{\fq}{\mathfrak{q}}
\newcommand{\fQ}{\mathfrak{Q}}
\newcommand{\fr}{\mathfrak{r}}
\newcommand{\fR}{\mathfrak{R}}
\newcommand{\fs}{\mathfrak{s}}
\newcommand{\fS}{\mathfrak{S}}
\newcommand{\ft}{\mathfrak{t}}
\newcommand{\fT}{\mathfrak{T}}
\newcommand{\fu}{\mathfrak{u}}
\newcommand{\fU}{\mathfrak{U}}
\newcommand{\fv}{\mathfrak{v}}
\newcommand{\fV}{\mathfrak{V}}
\newcommand{\fw}{\mathfrak{w}}
\newcommand{\fW}{\mathfrak{W}}
\newcommand{\fx}{\mathfrak{x}}
\newcommand{\fX}{\mathfrak{X}}
\newcommand{\fy}{\mathfrak{y}}
\newcommand{\fY}{\mathfrak{Y}}
\newcommand{\fz}{\mathfrak{z}}
\newcommand{\fZ}{\mathfrak{Z}}


%mathbf

\newcommand{\bfA}{\mathbf{A}}
\newcommand{\bfB}{\mathbf{B}}
\newcommand{\bfC}{\mathbf{C}}
\newcommand{\bfD}{\mathbf{D}}
\newcommand{\bfE}{\mathbf{E}}
\newcommand{\bfF}{\mathbf{F}}
\newcommand{\bfG}{\mathbf{G}}
\newcommand{\bfH}{\mathbf{H}}
\newcommand{\bfI}{\mathbf{I}}
\newcommand{\bfJ}{\mathbf{J}}
\newcommand{\bfK}{\mathbf{K}}
\newcommand{\bfL}{\mathbf{L}}
\newcommand{\bfM}{\mathbf{M}}
\newcommand{\bfN}{\mathbf{N}}
\newcommand{\bfO}{\mathbf{O}}
\newcommand{\bfP}{\mathbf{P}}
\newcommand{\bfQ}{\mathbf{Q}}
\newcommand{\bfR}{\mathbf{R}}
\newcommand{\bfS}{\mathbf{S}}
\newcommand{\bfT}{\mathbf{T}}
\newcommand{\bfU}{\mathbf{U}}
\newcommand{\bfV}{\mathbf{V}}
\newcommand{\bfW}{\mathbf{W}}
\newcommand{\bfX}{\mathbf{X}}
\newcommand{\bfY}{\mathbf{Y}}
\newcommand{\bfZ}{\mathbf{Z}}

\newcommand{\bfa}{\mathbf{a}}
\newcommand{\bfb}{\mathbf{b}}
\newcommand{\bfc}{\mathbf{c}}
\newcommand{\bfd}{\mathbf{d}}
\newcommand{\bfe}{\mathbf{e}}
\newcommand{\bff}{\mathbf{f}}
\newcommand{\bfg}{\mathbf{g}}
\newcommand{\bfh}{\mathbf{h}}
\newcommand{\bfi}{\mathbf{i}}
\newcommand{\bfj}{\mathbf{j}}
\newcommand{\bfk}{\mathbf{k}}
\newcommand{\bfl}{\mathbf{l}}
\newcommand{\bfm}{\mathbf{m}}
\newcommand{\bfn}{\mathbf{n}}
\newcommand{\bfo}{\mathbf{o}}
\newcommand{\bfp}{\mathbf{p}}
\newcommand{\bfq}{\mathbf{q}}
\newcommand{\bfr}{\mathbf{r}}
\newcommand{\bfs}{\mathbf{s}}
\newcommand{\bft}{\mathbf{t}}
\newcommand{\bfu}{\mathbf{u}}
\newcommand{\bfv}{\mathbf{v}}
\newcommand{\bfw}{\mathbf{w}}
\newcommand{\bfx}{\mathbf{x}}
\newcommand{\bfy}{\mathbf{y}}
\newcommand{\bfz}{\mathbf{z}}

%blackboard bold

\newcommand{\bbA}{\mathbb{A}}
\newcommand{\bbB}{\mathbb{B}}
\newcommand{\bbC}{\mathbb{C}}
\newcommand{\bbD}{\mathbb{D}}
\newcommand{\bbE}{\mathbb{E}}
\newcommand{\bbF}{\mathbb{F}}
\newcommand{\bbG}{\mathbb{G}}
\newcommand{\bbH}{\mathbb{H}}
\newcommand{\bbI}{\mathbb{I}}
\newcommand{\bbJ}{\mathbb{J}}
\newcommand{\bbK}{\mathbb{K}}
\newcommand{\bbL}{\mathbb{L}}
\newcommand{\bbM}{\mathbb{M}}
\newcommand{\bbN}{\mathbb{N}}
\newcommand{\bbO}{\mathbb{O}}
\newcommand{\bbP}{\mathbb{P}}
\newcommand{\bbQ}{\mathbb{Q}}
\newcommand{\bbR}{\mathbb{R}}
\newcommand{\bbS}{\mathbb{S}}
\newcommand{\bbT}{\mathbb{T}}
\newcommand{\bbU}{\mathbb{U}}
\newcommand{\bbV}{\mathbb{V}}
\newcommand{\bbW}{\mathbb{W}}
\newcommand{\bbX}{\mathbb{X}}
\newcommand{\bbY}{\mathbb{Y}}
\newcommand{\bbZ}{\mathbb{Z}}

\newcommand{\bmat}{\left( \begin{matrix}}
\newcommand{\emat}{\end{matrix} \right)}

\newcommand{\pmat}{\left( \begin{smallmatrix}}
\newcommand{\epmat}{\end{smallmatrix} \right)}

\newcommand{\lat}{\mathscr{L}}
\newcommand{\mat}[4]{\begin{pmatrix}{#1}&{#2}\\{#3}&{#4}\end{pmatrix}}
\newcommand{\ov}[1]{\overline{#1}}
\newcommand{\res}[1]{\underset{#1}{\RES}\,}
\newcommand{\up}{\upsilon}

\newcommand{\tac}{\textasteriskcentered}

%mahesh macros
\newcommand{\tm}{\textrm}

%Comments
\newcommand{\com}[1]{\vspace{5 mm}\par \noindent
\marginpar{\textsc{Comment}} \framebox{\begin{minipage}[c]{0.95
\textwidth} \tt #1 \end{minipage}}\vspace{5 mm}\par}

\newcommand{\Bmu}{\mbox{$\raisebox{-0.59ex}
  {$l$}\hspace{-0.18em}\mu\hspace{-0.88em}\raisebox{-0.98ex}{\scalebox{2}
  {$\color{white}.$}}\hspace{-0.416em}\raisebox{+0.88ex}
  {$\color{white}.$}\hspace{0.46em}$}{}}  %need graphicx and xcolor. this produces blackboard bold mu 

\newcommand{\hooktwoheadrightarrow}{%
  \hookrightarrow\mathrel{\mspace{-15mu}}\rightarrow
}

\makeatletter
\newcommand{\xhooktwoheadrightarrow}[2][]{%
  \lhook\joinrel
  \ext@arrow 0359\rightarrowfill@ {#1}{#2}%
  \mathrel{\mspace{-15mu}}\rightarrow
}
\makeatother

\renewcommand{\geq}{\geqslant}
    \renewcommand{\leq}{\leqslant}
    
    \newcommand{\bone}{\mathbf{1}}
    \newcommand{\sign}{\mathrm{sign}}
    \newcommand{\eps}{\varepsilon}
    \newcommand{\textui}[1]{\uline{\textit{#1}}}
    
    %\newcommand{\ov}{\overline}
    %\newcommand{\un}{\underline}
    \newcommand{\fin}{\mathrm{fin}}
    
    \newcommand{\chnum}{\titleformat
    {\chapter} % command
    [display] % shape
    {\centering} % format
    {\Huge \color{black} \shadowbox{\thechapter}} % label
    {-0.5em} % sep (space between the number and title)
    {\LARGE \color{black} \underline} % before-code
    }
    
    \newcommand{\chunnum}{\titleformat
    {\chapter} % command
    [display] % shape
    {} % format
    {} % label
    {0em} % sep
    { \begin{flushright} \begin{tabular}{r}  \Huge \color{black}
    } % before-code
    [
    \end{tabular} \end{flushright} \normalsize
    ] % after-code
    }

\newcommand{\littletaller}{\mathchoice{\vphantom{\big|}}{}{}{}}
\newcommand\restr[2]{{% we make the whole thing an ordinary symbol
  \left.\kern-\nulldelimiterspace % automatically resize the bar with \right
  #1 % the function
  \littletaller % pretend it's a little taller at normal size
  \right|_{#2} % this is the delimiter
  }}

\newcommand{\mtext}[1]{\hspace{6pt}\text{#1}\hspace{6pt}}

%This adds a "front cover" page.
%{\thispagestyle{empty}
%\vspace*{\fill}
%\begin{tabular}{l}
%\begin{tabular}{l}
%\includegraphics[scale=0.24]{oxy-logo.png}
%\end{tabular} \\
%\begin{tabular}{l}
%\Large \color{black} Module Theory, Linear Algebra, and Homological Algebra \\ \Large \color{black} Gianluca Crescenzo
%\end{tabular}
%\end{tabular}
%\newpage

\begin{document}
\begin{center}
{\large Econ 272 \\[0.1in]Homework 4 \\[0.1in]}
{Name:} {\underline{Gianluca Crescenzo\hspace*{2in}}}\\[0.15in]
\end{center}
\vspace{4pt}
%%%%%%%%%%%%%%%%%%%%%%%%%%%%%%%%%%%%%%%%%%%%%%%%%%%%%%%%%%%%%
    \begin{question}
        Consider the population regression testing the relationship whether college GPA is affected by hours of studying. $\text{GPA}_i = \beta_0 + \beta_1 \text{Study}_i + \delta \bfX_i + \epsilon_i$. $\bfX$ refers to other controls in the regression. Upon estimation in the data, the $95\%$ confidence interval corresponding to $\beta_1$ is $(0.012,0.024)$. Assume you have $200$ degrees of freedom ($(n-k-1)=200$).
            \begin{enumerate}[label = (\alph*),itemsep=1pt,topsep=3pt]
                \item Can you reject the null hypothesis of $\beta_1 = 0$ at the $5\%$ level of significance in a $2$-sided test? What about the null hypothesis $\beta_1 = 0.02$ in a $2$-sided test? Justify your reasoning.
                    {\color{blue} \begin{solution}
                        Since $\beta_1 = 0$ is not within the interval $(0.012,0.024)$, we reject the null-hypothesis. Since $\beta_2 = 0.02$ is within the interval $(0.012,0.024)$, we fail to reject the null-hypothesis.
                    \end{solution}}

                \item Can you reject the null hypothesis of $\beta_1 = 0$ at the $1\%$ level significance in a $2$-sided test? What about the null hypothesis of $\beta_1 = 0$ at the $10\%$ level of significance in a $1$-sided test? Justify your reasoning and show any steps.
                    {\color{blue} \begin{solution}
                    \end{solution}}
            \end{enumerate}
    \end{question}
%%%%%%%%%%%%%%%%%%%%%%%%%%%%%%%%%%%%%%%%%%%%%%%%%%%%%%%%%%%%%
    \begin{question}
        Consider the population regression testing the relationship whether college GPA is affected by hours of studying. $\text{GPA}_i = \beta_0 + \beta_1 \text{Study}_i + \delta \bfX_i + \epsilon_i$. $\bfX$ refers to other controls in the regression. Upon estimation in the data, $\widehat{\beta_1} = 0.025$ and the accompanying standard error is $\text{SE}(\widehat{\beta_1}) = 0.013$. Assume you have 200 degrees of freedom.
            \begin{enumerate}[label = (\alph*),itemsep=1pt,topsep=3pt]
                \item Can you reject the null hypothesis of $\beta_1 = 0$ in a two-sided test at the 10 percent level of significance?
                    {\color{blue} \begin{solution}
                        Our $t$-value is:
                            \begin{equation*}
                            \begin{split}
                                t_k = \frac{\widehat{\beta_1}}{\text{SE}(\widehat{\beta_1})} = \frac{0.025}{0.013} \approx 1.923
                            \end{split}
                            \end{equation*}
                        The critical value for a two-sided test with a $10\%$ level of significance is $t_c \approx 1.652$. Since $t_k > t_c$, we reject the null-hypothesis.
                    \end{solution}}

                \item Given the estimated coefficient and standard error, what is the minimum level of Type-1 error with which you can reject the null of $\beta_1 = 0$ in a two-sided test? What would be your answer if you were considering a one-sided test? (The alternate hypothesis being $\beta_1 > 0$) Your answer in both instances should be a probability.
                    {\color{blue} \begin{solution}
                        We can use $t_k$ from part (a). For a two-sided test, we have that $2P(T >|t_k|) \approx 0.056$, so the minimum level of Type-1 error is $5.6\%$. For a one-sided test, we have $P(T > t_k) \approx 0.028$, so the minimum level of Type-1 error is $2.8\%$.
                    \end{solution}}

                \item Can you reject the null hypothesis of $\beta_1 = 0.04$ at the 10 percent level in a two-sided test? What about a one-sided test? Show your steps.
                    {\color{blue} \begin{solution}
                        Our $t$-value is:
                            \begin{equation*}
                            \begin{split}
                                t_k = \frac{\widehat{\beta_1} - \beta_1}{\text{SE}(\widehat{\beta_1})} = \frac{0.025 - 0.04}{0.013} \approx -1.154
                            \end{split}
                            \end{equation*}
                        Using $t_c$ from part (a), since $|t_k| \not >  t_c$, we fail to reject the null-hypothesis. For a one-sided test, $t_c \approx 2.345$. Since $|t_k| > t_c$, we reject the null-hypothesis.
                    \end{solution}}
            \end{enumerate}
    \end{question}
%%%%%%%%%%%%%%%%%%%%%%%%%%%%%%%%%%%%%%%%%%%%%%%%%%%%%%%%%%%%%
    \begin{question}
        Use \textit{hw4data} for the following question. The data refers to manufacturing firms. A list of variable definitions are provided at the end of the problem set. Consider the population regression function:
            \begin{equation*}
            \begin{split}
                \ln(\text{Output}_i) = \beta_0 + \beta_1 &\ln(\text{Wages}_i) + \beta_2 \ln(\text{Capital}_i) + \beta_3 \ln(\text{Materials}_i) \\
                & + \beta_4 \text{Importer}_i + \beta_5 \text{Rural}_i + \beta_6 \text{Listed}_i + \beta_7 \text{Age}_i + \beta_8 \text{Age}^2_i + \epsilon_i
            \end{split}
            \end{equation*} 
        \begin{enumerate}[label = (\alph*),itemsep=1pt,topsep=3pt]
            \item How much additional output can a firm currently aged 5 years expect when it reaches the age of 10 years. How much additional output can a firm currently aged 50 years expect when it reaches the age of 55 years? (Note: you will first need to compute the natural log of output, wages paid, capital, and materials, and also generate the squared age variable).
                {\color{blue} \begin{solution}
                    From the regression, we found that $\widehat{\beta_7} = -0.0108897$ and $\widehat{\beta_8} = 0.0001206$. So we have:
                        \begin{equation*}
                        \begin{split}
                            \Delta \widehat{\ln(\text{Output})_i} &= \bigl(\widehat{\beta_7}\cdot10 + \widehat{\beta_8}\cdot 10^2 \bigr) - \bigl(\widehat{\beta_7}\cdot5 + \widehat{\beta_8}\cdot 5^2 \bigr) = -.04540269,\\
                            \Delta \widehat{\ln(\text{Output})_i} &= \bigl(\widehat{\beta_7}\cdot55 + \widehat{\beta_8}\cdot 55^2 \bigr) - \bigl(\widehat{\beta_7}\cdot50 + \widehat{\beta_8}\cdot 50^2 \bigr) = .00887165.
                        \end{split}
                        \end{equation*}
                \end{solution}}

            \item Based on your regression estimates, from what point can firm owners expect output to display an increasing relationship with age?
                {\color{blue} \begin{solution}
                    Consider:
                        \begin{equation*}
                        \begin{split}
                            \frac{\partial \ln(\widehat{\text{Output}_i})}{\partial \text{Age}_i} = \beta_7 + 2\beta_8 \text{Age}_i = 0
                        \end{split}
                        \end{equation*}
                        Solving for $\text{Age}_i$ gives:
                            \begin{equation*}
                            \begin{split}
                                \text{Age}_i = -\frac{\beta_7}{2\beta_8}
                            \end{split}
                            \end{equation*}
                        This will be a local-minimum, whence after age 45 firm owners can expect output to display an increasing relationship with age.
                \end{solution}}

            \item Based on your regression estimates, what will be the minimum Type-1 error you would need to tolerate to reject the null of $\beta_2 = 0.3$ using a two-sided test?
                {\color{blue} \begin{solution}
                    Our $t$-value is:
                        \begin{equation*}
                        \begin{split}
                            t_{\widehat{\beta_2}} = \frac{\widehat{\beta_2} - \beta_2}{\text{SE}(\widehat{\beta_2})} = \frac{0.207 - 0.3}{0.006} \approx -15.5
                        \end{split}
                        \end{equation*}
                    The probability that a Type-1 error occurs is basically 0.
                \end{solution}}
            
            \item Based on your regression estimates, can you reject the null of $\beta_6 = -0.1$ at the $5\%$ level using a two-sided test?
                {\color{blue} \begin{solution}
                    Our $t$-value is:
                        \begin{equation*}
                        \begin{split}
                            t_{\widehat{\beta_6}} = \frac{\widehat{\beta_6} - \beta_6}{\text{SE}(\widehat{\beta_6})} = \frac{-.0773319 + 0.1}{.0251632} \approx 0.9008
                        \end{split}
                        \end{equation*}
                    At the $5\%$ level, the critical value is $t_c = 1.96$. Thus we fail to reject the null hypothesis.
                \end{solution}}

            \item Based on your regression estimates, test the null hypothesis of $\beta_1 + \beta_2 + \beta_3 = 1$. Can you reject the null hypothesis at the $5\%$ level of significance using a two-sided test? What about the $1\%$ level of significance?
                {\color{blue} \begin{solution}
                    Using the \textit{test} command in Stata gave $\text{Prob} > F = 0.0144$. So at the $5\%$ level of significance, we reject the null hypothesis. For the $1\%$ level of significance, we fail to reject the null hypothesis.
                \end{solution}}
            
            \item Use an $F$-test to argue whether $\text{Age}_i$ and $\text{Age}_i^2$ should be a part of the population regression function.
                {\color{blue} \begin{solution}
                    The value Stata reported is very small. This means $\text{Age}_i$ and $\text{{Age}}_i^2$ are jointly statistically significant and should be included in the model.
                \end{solution}}
        \end{enumerate}
    \end{question}
%%%%%%%%%%%%%%%%%%%%%%%%%%%%%%%%%%%%%%%%%%%%%%%%%%%%%%%%%%%%%
    

\end{document}