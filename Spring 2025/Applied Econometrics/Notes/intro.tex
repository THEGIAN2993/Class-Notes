\chapter{Statistics for the Working Economist}

\section{Measurements}
    \begin{definition}
        \phantom{a}
        \begin{enumerate}[label = (\arabic*),itemsep=1pt,topsep=3pt]
            \item The large body of data that is the target of our interest is called the \textit{population}.
            \item A subset selected from a given population is called a \textit{sample}.
        \end{enumerate}
    \end{definition}

    \begin{definition}
        The \textit{mean} of a sample of $n$ measured responses $y_1,...,y_n$ is given by:
            \begin{equation*}
            \begin{split}
                \overline{y} = \frac{1}{n} \sum_{i = 1}^n y_i.
            \end{split}
            \end{equation*}
        The corresponding \textit{population mean} is denoted $\mu$.
    \end{definition}

    \begin{remark}
        We usually cannot measure the value of the population mean, $\mu$; rather, $\mu$ is an unknown constant that we may want to estimate using sample information.
    \end{remark}

    \begin{definition}
        The \textit{variance} of a sample of measurements $y_1,...,y_n$ is given by:
            \begin{equation*}
            \begin{split}
                s^2 = \frac{1}{n-1}\sum_{i = 1}^n (y_i - \overline{y})^2.
            \end{split}
            \end{equation*}
        The corresponding \textit{population variance} is denoted by $\sigma^2$. The larger the variance of a set of measurements, the greater will be the amount of variation within the set
    \end{definition}

    \begin{definition}
        The \textit{standard deviation} of a sample of measurements is given by:
            \begin{equation*}
            \begin{split}
                s = \sqrt{s^2}.
            \end{split}
            \end{equation*}
        The corresponding \textit{population standard deviation} is denoted by $\sigma = \sqrt{\sigma^2}$
    \end{definition}

    \begin{definition}
        An \textit{estimator} is a formula that tells how to calculate the value of an estimate based on the measurements contained in a sample.
    \end{definition}

    \begin{example}
        The sample mean $\overline{y}$ is one possible point estimator of a population mean $\mu$.
    \end{example}

\section{Linear Models}
    In this chapter, we undertake a study of inferential procedures that can be used
    when a random variable $Y$, called the \textit{dependent variable}, has a mean that is a function of one or more non-random variables $x_1,...,x_k$ called \textit{independent variables}.

    \begin{definition}
        A \textit{linear statistical model} relating a random response $Y$ to a set of independent variables $x_1,...,x_k$ is of the form:
            \begin{equation*}
            \begin{split}
                Y = \beta_0 + \beta_1 x_1 + ... + \beta_k x_k + \epsilon,
            \end{split}
            \end{equation*}
        where $\beta_0,...,\beta_k$ are unknown parameters, $epsilon$ is a random variable (typically an error of some sort), and the variables $x_1,...,x_k$ are known values. 
    \end{definition}

    \begin{remark}
        Although unreasonable, we will assume that $E(\epsilon) = 0$, hence that:
            \begin{equation*}
            \begin{split}
                E(Y) = \beta_0 + \beta_1 x_1 + ... + \beta_k x_k.
            \end{split}
            \end{equation*}
    \end{remark}

