\documentclass[11pt,twoside,openany]{memoir}
\usepackage{mlmodern}
%\usepackage{tgpagella} % text only
%\usepackage{mathpazo}  % math & text
\usepackage[T1]{fontenc}
\usepackage[hidelinks]{hyperref}
\usepackage{amsmath}
\usepackage{amsthm}
\usepackage{amssymb}
\usepackage{mathtools}
%\renewcommand*{\mathbf}[1]{\varmathbb{#1}}
%\usepackage{newpxtext}
%\usepackage{eulerpx}
%\usepackage{eucal}
\usepackage{datetime}
    \newdateformat{specialdate}{\THEYEAR\ \monthname\ \THEDAY}
\usepackage[margin=1in]{geometry}
\usepackage{fancyhdr}
    \fancyhf{}
    \pagestyle{fancy}
    \cfoot{\scriptsize \thepage}
    \fancyhead[R]{\scriptsize \rightmark}
    \fancyhead[L]{\scriptsize \leftmark}
    \renewcommand{\headrulewidth}{0pt}
    \renewcommand{\footrulewidth}{0pt} % if you also want to remove the footer rule
\usepackage{thmtools}
    \declaretheoremstyle[
        spaceabove=10pt,
        spacebelow=10pt,
        headfont=\normalfont\bfseries,
        notefont=\mdseries, notebraces={(}{)},
        bodyfont=\normalfont,
        postheadspace=0.5em
        %qed=\qedsymbol
        ]{defs}

    \declaretheoremstyle[ 
        spaceabove=10pt, % space above the theorem
        spacebelow=10pt,
        headfont=\normalfont\bfseries,
        bodyfont=\normalfont\itshape,
        postheadspace=0.5em
        ]{thmstyle}
    
    \declaretheorem[
        style=thmstyle,
        numberwithin=section
    ]{theorem}

    \declaretheorem[
        style=thmstyle,
        sibling=theorem,
    ]{proposition}

    \declaretheorem[
        style=thmstyle,
        sibling=theorem,
    ]{lemma}

    \declaretheorem[
        style=thmstyle,
        sibling=theorem,
    ]{corollary}

    \declaretheorem[
        numberwithin=section,
        style=defs,
    ]{example}

    \declaretheorem[
        numberwithin=section,
        style=defs,
    ]{definition}

    \declaretheorem[
        style=defs,
        numbered=unless unique,
    ]{exercise}

    \declaretheorem[
        numbered=unless unique,
        shaded={rulecolor=black,
    rulewidth=1pt, bgcolor={rgb}{1,1,1}}
    ]{axiom}

    \declaretheorem[numberwithin=section,style=defs]{note}
    \declaretheorem[numbered=no,style=defs]{question}
    \declaretheorem[numbered=no,style=defs]{recall}
    \declaretheorem[numbered=no,style=remark]{answer}
    \declaretheorem[numbered=no,style=remark]{solution}

    \declaretheorem[numbered=no,style=defs]{remark}
\usepackage{enumitem}
\usepackage{titlesec}
    \titleformat{\chapter}[display]
    {\bfseries\LARGE\raggedright}
    {Chapter {\thechapter}}
    {1ex minus .1ex}
    {\Huge}
    \titlespacing{\chapter}
    {3pc}{*3}{40pt}[3pc]

    \titleformat{\section}[block]
    {\normalfont\bfseries\Large}
    {\S\ \thesection.}{.5em}{}[]
    \titlespacing{\section}
    {0pt}{3ex plus .1ex minus .2ex}{3ex plus .1ex minus .2ex}
\usepackage[utf8x]{inputenc}
\usepackage{tikz}
\usepackage{tikz-cd}
\usepackage{wasysym}
\renewcommand{\int}{\varint}

\linespread{1}
%to make the correct symbol for Sha
%\newcommand\cyr{%
%\renewcommand\rmdefault{wncyr}%
%\renewcommand\sfdefault{wncyss}%
%\renewcommand\encodingdefault{OT2}%
%\normalfont \selectfont} \DeclareTextFontCommand{\textcyr}{\cyr}


\DeclareMathOperator{\ab}{ab}
\newcommand{\absgal}{\G_{\bbQ}}
\DeclareMathOperator{\ad}{ad}
\DeclareMathOperator{\adj}{adj}
\DeclareMathOperator{\alg}{alg}
\DeclareMathOperator{\Alt}{Alt}
\DeclareMathOperator{\Ann}{Ann}
\DeclareMathOperator{\arith}{arith}
\DeclareMathOperator{\Aut}{Aut}
\DeclareMathOperator{\Be}{B}
\DeclareMathOperator{\card}{card}
\DeclareMathOperator{\Char}{char}
\DeclareMathOperator{\csp}{csp}
\DeclareMathOperator{\codim}{codim}
\DeclareMathOperator{\coker}{coker}
\DeclareMathOperator{\coh}{H}
\DeclareMathOperator{\compl}{compl}
\DeclareMathOperator{\conj}{conj}
\DeclareMathOperator{\cont}{cont}
\DeclareMathOperator{\crys}{crys}
\DeclareMathOperator{\Crys}{Crys}
\DeclareMathOperator{\cusp}{cusp}
\DeclareMathOperator{\diag}{diag}
\DeclareMathOperator{\disc}{disc}
\DeclareMathOperator{\dR}{dR}
\DeclareMathOperator{\Eis}{Eis}
\DeclareMathOperator{\End}{End}
\DeclareMathOperator{\ev}{ev}
\DeclareMathOperator{\eval}{eval}
\DeclareMathOperator{\Eq}{Eq}
\DeclareMathOperator{\Ext}{Ext}
\DeclareMathOperator{\Fil}{Fil}
\DeclareMathOperator{\Fitt}{Fitt}
\DeclareMathOperator{\Frob}{Frob}
\DeclareMathOperator{\G}{G}
\DeclareMathOperator{\Gal}{Gal}
\DeclareMathOperator{\GL}{GL}
\DeclareMathOperator{\Gr}{Gr}
\DeclareMathOperator{\Graph}{Graph}
\DeclareMathOperator{\GSp}{GSp}
\DeclareMathOperator{\GUn}{GU}
\DeclareMathOperator{\Hom}{Hom}
\DeclareMathOperator{\id}{id}
\DeclareMathOperator{\Id}{Id}
\DeclareMathOperator{\Ik}{Ik}
\DeclareMathOperator{\IM}{Im}
\DeclareMathOperator{\Image}{im}
\DeclareMathOperator{\Ind}{Ind}
\DeclareMathOperator{\Inf}{inf}
\DeclareMathOperator{\Isom}{Isom}
\DeclareMathOperator{\J}{J}
\DeclareMathOperator{\Jac}{Jac}
\DeclareMathOperator{\lcm}{lcm}
\DeclareMathOperator{\length}{length}
\DeclareMathOperator{\Log}{Log}
\DeclareMathOperator{\M}{M}
\DeclareMathOperator{\Mat}{Mat}
\DeclareMathOperator{\N}{N}
\DeclareMathOperator{\Nm}{Nm}
\DeclareMathOperator{\NIk}{N-Ik}
\DeclareMathOperator{\NSK}{N-SK}
\DeclareMathOperator{\new}{new}
\DeclareMathOperator{\obj}{obj}
\DeclareMathOperator{\old}{old}
\DeclareMathOperator{\ord}{ord}
\DeclareMathOperator{\Or}{O}
\DeclareMathOperator{\PGL}{PGL}
\DeclareMathOperator{\PGSp}{PGSp}
\DeclareMathOperator{\rank}{rank}
\DeclareMathOperator{\Rel}{Rel}
\DeclareMathOperator{\Real}{Re}
\DeclareMathOperator{\RES}{res}
\DeclareMathOperator{\Res}{Res}
%\DeclareMathOperator{\Sha}{\textcyr{Sh}}
\DeclareMathOperator{\Sel}{Sel}
\DeclareMathOperator{\semi}{ss}
\DeclareMathOperator{\sgn}{sign}
\DeclareMathOperator{\SK}{SK}
\DeclareMathOperator{\SL}{SL}
\DeclareMathOperator{\SO}{SO}
\DeclareMathOperator{\Sp}{Sp}
\DeclareMathOperator{\Span}{span}
\DeclareMathOperator{\Spec}{Spec}
\DeclareMathOperator{\spin}{spin}
\DeclareMathOperator{\st}{st}
\DeclareMathOperator{\St}{St}
\DeclareMathOperator{\SUn}{SU}
\DeclareMathOperator{\supp}{supp}
\DeclareMathOperator{\Sup}{sup}
\DeclareMathOperator{\Sym}{Sym}
\DeclareMathOperator{\Tam}{Tam}
\DeclareMathOperator{\tors}{tors}
\DeclareMathOperator{\tr}{tr}
\DeclareMathOperator{\un}{un}
\DeclareMathOperator{\Un}{U}
\DeclareMathOperator{\val}{val}
\DeclareMathOperator{\vol}{vol}

\DeclareMathOperator{\Sets}{S \mkern1.04mu e \mkern1.04mu t \mkern1.04mu s}
    \newcommand{\cSets}{\scalebox{1.02}{\contour{black}{$\Sets$}}}
    
\DeclareMathOperator{\Groups}{G \mkern1.04mu r \mkern1.04mu o \mkern1.04mu u \mkern1.04mu p \mkern1.04mu s}
    \newcommand{\cGroups}{\scalebox{1.02}{\contour{black}{$\Groups$}}}

\DeclareMathOperator{\TTop}{T \mkern1.04mu o \mkern1.04mu p}
    \newcommand{\cTop}{\scalebox{1.02}{\contour{black}{$\TTop$}}}

\DeclareMathOperator{\Htp}{H \mkern1.04mu t \mkern1.04mu p}
    \newcommand{\cHtp}{\scalebox{1.02}{\contour{black}{$\Htp$}}}

\DeclareMathOperator{\Mod}{M \mkern1.04mu o \mkern1.04mu d}
    \newcommand{\cMod}{\scalebox{1.02}{\contour{black}{$\Mod$}}}

\DeclareMathOperator{\Ab}{A \mkern1.04mu b}
    \newcommand{\cAb}{\scalebox{1.02}{\contour{black}{$\Ab$}}}

\DeclareMathOperator{\Rings}{R \mkern1.04mu i \mkern1.04mu n \mkern1.04mu g \mkern1.04mu s}
    \newcommand{\cRings}{\scalebox{1.02}{\contour{black}{$\Rings$}}}

\DeclareMathOperator{\ComRings}{C \mkern1.04mu o \mkern1.04mu m \mkern1.04mu R \mkern1.04mu i \mkern1.04mu n \mkern1.04mu g \mkern1.04mu s}
    \newcommand{\cComRings}{\scalebox{1.05}{\contour{black}{$\ComRings$}}}

\DeclareMathOperator{\hHom}{H \mkern1.04mu o \mkern1.04mu m}
    \newcommand{\cHom}{\scalebox{1.02}{\contour{black}{$\hHom$}}}

         %  \item $\cGroups$
          %  \item $\cTop$
          %  \item $\cHtp$
          %  \item $\cMod$




\renewcommand{\k}{\kappa}
\newcommand{\Ff}{F_{f}}
\newcommand{\ts}{\,^{t}\!}


%Mathcal

\newcommand{\cA}{\mathcal{A}}
\newcommand{\cB}{\mathcal{B}}
\newcommand{\cC}{\mathcal{C}}
\newcommand{\cD}{\mathcal{D}}
\newcommand{\cE}{\mathcal{E}}
\newcommand{\cF}{\mathcal{F}}
\newcommand{\cG}{\mathcal{G}}
\newcommand{\cH}{\mathcal{H}}
\newcommand{\cI}{\mathcal{I}}
\newcommand{\cJ}{\mathcal{J}}
\newcommand{\cK}{\mathcal{K}}
\newcommand{\cL}{\mathcal{L}}
\newcommand{\cM}{\mathcal{M}}
\newcommand{\cN}{\mathcal{N}}
\newcommand{\cO}{\mathcal{O}}
\newcommand{\cP}{\mathcal{P}}
\newcommand{\cQ}{\mathcal{Q}}
\newcommand{\cR}{\mathcal{R}}
\newcommand{\cS}{\mathcal{S}}
\newcommand{\cT}{\mathcal{T}}
\newcommand{\cU}{\mathcal{U}}
\newcommand{\cV}{\mathcal{V}}
\newcommand{\cW}{\mathcal{W}}
\newcommand{\cX}{\mathcal{X}}
\newcommand{\cY}{\mathcal{Y}}
\newcommand{\cZ}{\mathcal{Z}}


%mathfrak (missing \fi)

\newcommand{\fa}{\mathfrak{a}}
\newcommand{\fA}{\mathfrak{A}}
\newcommand{\fb}{\mathfrak{b}}
\newcommand{\fB}{\mathfrak{B}}
\newcommand{\fc}{\mathfrak{c}}
\newcommand{\fC}{\mathfrak{C}}
\newcommand{\fd}{\mathfrak{d}}
\newcommand{\fD}{\mathfrak{D}}
\newcommand{\fe}{\mathfrak{e}}
\newcommand{\fE}{\mathfrak{E}}
\newcommand{\ff}{\mathfrak{f}}
\newcommand{\fF}{\mathfrak{F}}
\newcommand{\fg}{\mathfrak{g}}
\newcommand{\fG}{\mathfrak{G}}
\newcommand{\fh}{\mathfrak{h}}
\newcommand{\fH}{\mathfrak{H}}
\newcommand{\fI}{\mathfrak{I}}
\newcommand{\fj}{\mathfrak{j}}
\newcommand{\fJ}{\mathfrak{J}}
\newcommand{\fk}{\mathfrak{k}}
\newcommand{\fK}{\mathfrak{K}}
\newcommand{\fl}{\mathfrak{l}}
\newcommand{\fL}{\mathfrak{L}}
\newcommand{\fm}{\mathfrak{m}}
\newcommand{\fM}{\mathfrak{M}}
\newcommand{\fn}{\mathfrak{n}}
\newcommand{\fN}{\mathfrak{N}}
\newcommand{\fo}{\mathfrak{o}}
\newcommand{\fO}{\mathfrak{O}}
\newcommand{\fp}{\mathfrak{p}}
\newcommand{\fP}{\mathfrak{P}}
\newcommand{\fq}{\mathfrak{q}}
\newcommand{\fQ}{\mathfrak{Q}}
\newcommand{\fr}{\mathfrak{r}}
\newcommand{\fR}{\mathfrak{R}}
\newcommand{\fs}{\mathfrak{s}}
\newcommand{\fS}{\mathfrak{S}}
\newcommand{\ft}{\mathfrak{t}}
\newcommand{\fT}{\mathfrak{T}}
\newcommand{\fu}{\mathfrak{u}}
\newcommand{\fU}{\mathfrak{U}}
\newcommand{\fv}{\mathfrak{v}}
\newcommand{\fV}{\mathfrak{V}}
\newcommand{\fw}{\mathfrak{w}}
\newcommand{\fW}{\mathfrak{W}}
\newcommand{\fx}{\mathfrak{x}}
\newcommand{\fX}{\mathfrak{X}}
\newcommand{\fy}{\mathfrak{y}}
\newcommand{\fY}{\mathfrak{Y}}
\newcommand{\fz}{\mathfrak{z}}
\newcommand{\fZ}{\mathfrak{Z}}


%mathbf

\newcommand{\bfA}{\mathbf{A}}
\newcommand{\bfB}{\mathbf{B}}
\newcommand{\bfC}{\mathbf{C}}
\newcommand{\bfD}{\mathbf{D}}
\newcommand{\bfE}{\mathbf{E}}
\newcommand{\bfF}{\mathbf{F}}
\newcommand{\bfG}{\mathbf{G}}
\newcommand{\bfH}{\mathbf{H}}
\newcommand{\bfI}{\mathbf{I}}
\newcommand{\bfJ}{\mathbf{J}}
\newcommand{\bfK}{\mathbf{K}}
\newcommand{\bfL}{\mathbf{L}}
\newcommand{\bfM}{\mathbf{M}}
\newcommand{\bfN}{\mathbf{N}}
\newcommand{\bfO}{\mathbf{O}}
\newcommand{\bfP}{\mathbf{P}}
\newcommand{\bfQ}{\mathbf{Q}}
\newcommand{\bfR}{\mathbf{R}}
\newcommand{\bfS}{\mathbf{S}}
\newcommand{\bfT}{\mathbf{T}}
\newcommand{\bfU}{\mathbf{U}}
\newcommand{\bfV}{\mathbf{V}}
\newcommand{\bfW}{\mathbf{W}}
\newcommand{\bfX}{\mathbf{X}}
\newcommand{\bfY}{\mathbf{Y}}
\newcommand{\bfZ}{\mathbf{Z}}

\newcommand{\bfa}{\mathbf{a}}
\newcommand{\bfb}{\mathbf{b}}
\newcommand{\bfc}{\mathbf{c}}
\newcommand{\bfd}{\mathbf{d}}
\newcommand{\bfe}{\mathbf{e}}
\newcommand{\bff}{\mathbf{f}}
\newcommand{\bfg}{\mathbf{g}}
\newcommand{\bfh}{\mathbf{h}}
\newcommand{\bfi}{\mathbf{i}}
\newcommand{\bfj}{\mathbf{j}}
\newcommand{\bfk}{\mathbf{k}}
\newcommand{\bfl}{\mathbf{l}}
\newcommand{\bfm}{\mathbf{m}}
\newcommand{\bfn}{\mathbf{n}}
\newcommand{\bfo}{\mathbf{o}}
\newcommand{\bfp}{\mathbf{p}}
\newcommand{\bfq}{\mathbf{q}}
\newcommand{\bfr}{\mathbf{r}}
\newcommand{\bfs}{\mathbf{s}}
\newcommand{\bft}{\mathbf{t}}
\newcommand{\bfu}{\mathbf{u}}
\newcommand{\bfv}{\mathbf{v}}
\newcommand{\bfw}{\mathbf{w}}
\newcommand{\bfx}{\mathbf{x}}
\newcommand{\bfy}{\mathbf{y}}
\newcommand{\bfz}{\mathbf{z}}

%blackboard bold

\newcommand{\bbA}{\mathbb{A}}
\newcommand{\bbB}{\mathbb{B}}
\newcommand{\bbC}{\mathbb{C}}
\newcommand{\bbD}{\mathbb{D}}
\newcommand{\bbE}{\mathbb{E}}
\newcommand{\bbF}{\mathbb{F}}
\newcommand{\bbG}{\mathbb{G}}
\newcommand{\bbH}{\mathbb{H}}
\newcommand{\bbI}{\mathbb{I}}
\newcommand{\bbJ}{\mathbb{J}}
\newcommand{\bbK}{\mathbb{K}}
\newcommand{\bbL}{\mathbb{L}}
\newcommand{\bbM}{\mathbb{M}}
\newcommand{\bbN}{\mathbb{N}}
\newcommand{\bbO}{\mathbb{O}}
\newcommand{\bbP}{\mathbb{P}}
\newcommand{\bbQ}{\mathbb{Q}}
\newcommand{\bbR}{\mathbb{R}}
\newcommand{\bbS}{\mathbb{S}}
\newcommand{\bbT}{\mathbb{T}}
\newcommand{\bbU}{\mathbb{U}}
\newcommand{\bbV}{\mathbb{V}}
\newcommand{\bbW}{\mathbb{W}}
\newcommand{\bbX}{\mathbb{X}}
\newcommand{\bbY}{\mathbb{Y}}
\newcommand{\bbZ}{\mathbb{Z}}

\newcommand{\bmat}{\left( \begin{matrix}}
\newcommand{\emat}{\end{matrix} \right)}

\newcommand{\pmat}{\left( \begin{smallmatrix}}
\newcommand{\epmat}{\end{smallmatrix} \right)}

\newcommand{\lat}{\mathscr{L}}
\newcommand{\mat}[4]{\begin{pmatrix}{#1}&{#2}\\{#3}&{#4}\end{pmatrix}}
\newcommand{\ov}[1]{\overline{#1}}
\newcommand{\res}[1]{\underset{#1}{\RES}\,}
\newcommand{\up}{\upsilon}

\newcommand{\tac}{\textasteriskcentered}

%mahesh macros
\newcommand{\tm}{\textrm}

%Comments
\newcommand{\com}[1]{\vspace{5 mm}\par \noindent
\marginpar{\textsc{Comment}} \framebox{\begin{minipage}[c]{0.95
\textwidth} \tt #1 \end{minipage}}\vspace{5 mm}\par}

\newcommand{\Bmu}{\mbox{$\raisebox{-0.59ex}
  {$l$}\hspace{-0.18em}\mu\hspace{-0.88em}\raisebox{-0.98ex}{\scalebox{2}
  {$\color{white}.$}}\hspace{-0.416em}\raisebox{+0.88ex}
  {$\color{white}.$}\hspace{0.46em}$}{}}  %need graphicx and xcolor. this produces blackboard bold mu 

\newcommand{\hooktwoheadrightarrow}{%
  \hookrightarrow\mathrel{\mspace{-15mu}}\rightarrow
}

\makeatletter
\newcommand{\xhooktwoheadrightarrow}[2][]{%
  \lhook\joinrel
  \ext@arrow 0359\rightarrowfill@ {#1}{#2}%
  \mathrel{\mspace{-15mu}}\rightarrow
}
\makeatother

\renewcommand{\geq}{\geqslant}
    \renewcommand{\leq}{\leqslant}
    
    \newcommand{\bone}{\mathbf{1}}
    \newcommand{\sign}{\mathrm{sign}}
    \newcommand{\eps}{\varepsilon}
    \newcommand{\textui}[1]{\uline{\textit{#1}}}
    
    %\newcommand{\ov}{\overline}
    %\newcommand{\un}{\underline}
    \newcommand{\fin}{\mathrm{fin}}
    
    \newcommand{\chnum}{\titleformat
    {\chapter} % command
    [display] % shape
    {\centering} % format
    {\Huge \color{black} \shadowbox{\thechapter}} % label
    {-0.5em} % sep (space between the number and title)
    {\LARGE \color{black} \underline} % before-code
    }
    
    \newcommand{\chunnum}{\titleformat
    {\chapter} % command
    [display] % shape
    {} % format
    {} % label
    {0em} % sep
    { \begin{flushright} \begin{tabular}{r}  \Huge \color{black}
    } % before-code
    [
    \end{tabular} \end{flushright} \normalsize
    ] % after-code
    }

\newcommand{\littletaller}{\mathchoice{\vphantom{\big|}}{}{}{}}
\newcommand\restr[2]{{% we make the whole thing an ordinary symbol
  \left.\kern-\nulldelimiterspace % automatically resize the bar with \right
  #1 % the function
  \littletaller % pretend it's a little taller at normal size
  \right|_{#2} % this is the delimiter
  }}

\newcommand{\mtext}[1]{\hspace{6pt}\text{#1}\hspace{6pt}}

%This adds a "front cover" page.
%{\thispagestyle{empty}
%\vspace*{\fill}
%\begin{tabular}{l}
%\begin{tabular}{l}
%\includegraphics[scale=0.24]{oxy-logo.png}
%\end{tabular} \\
%\begin{tabular}{l}
%\Large \color{black} Module Theory, Linear Algebra, and Homological Algebra \\ \Large \color{black} Gianluca Crescenzo
%\end{tabular}
%\end{tabular}
%\newpage

\begin{document}
\begin{center}
{\large Math 397 \\[0.1in]Homework 4 \\[0.1in]}
{Name:} {\underline{Gianluca Crescenzo\hspace*{2in}}}\\[0.15in]
\end{center}
\vspace{4pt}
%%%%%%%%%%%%%%%%%%%%%%%%%%%%%%%%%%%%%%%%%%%%%%%%%%%%%%%%%%%%%
    \begin{exercise}
        Let $X$ be a metric space. Show that $X$ is second countable if and only if $X$ is separable. Conclude that if $X$ is a separable metric space, then every open set is the union of countably many open balls.
    \end{exercise}
        {\color{red} \begin{proof}
            Let $\{U_n\}_{n = 1}^\infty$ be a countable base for $X$. Let $x \in X$ and $\epsilon > 0$. Then $x \in U(x,\epsilon) \subseteq X$. We can find $U_n \in \{U_n\}_{n = 1}^\infty$ with $x \in U_n \subseteq U(x,\epsilon)$. So for any $a_n \in U_n$, we have $a_n \in U(x,\epsilon)$, giving $d(x,a_n) < \epsilon$. Thus $\{a_n\}_{n = 1}^\infty$ is dense; i.e., $X$ is separable.

            Let $\{a_n\}_{n = 1}^\infty$ be a countable dense subset. Claim: $\cB = \{U(a_n,\frac{1}{m}) \mid n,m \geq 1\}$ is a base. Let $U \in \tau_X$ and $x \in U$. Since $U$ is open, there exists $\epsilon > 0$ such that $U(x,\epsilon) \subseteq U$. Moreover, we can find $m \geq 1$ with $\epsilon > \frac{1}{m}$. Since $\{a_n\}_{n = 1}^\infty$ is dense, we can find $a_j \in \{a_n\}_{n = 1}^\infty$ such that $d(x,a_j) < \frac{1}{2m}$. Let $y \in U(a_j,\frac{1}{2m})$. Then:
                \begin{equation*}
                \begin{split}
                    d(x,y)
                    & \leq d(x,a_j) + d(a_j,y) \\
                    & < \frac{1}{2m} + \frac{1}{2m} \\
                    & = \frac{1}{m} \\
                    & < \epsilon.
                \end{split}
                \end{equation*}
            So $y \in U(x,\epsilon)$. Thus $x \in U(a_j,\frac{1}{2m}) \subseteq U(x,\epsilon) \subseteq U$, establishing $\cB$ as a base.x
        \end{proof}}
%%%%%%%%%%%%%%%%%%%%%%%%%%%%%%%%%%%%%%%%%%%%%%%%%%%%%%%%%%%%%
    \begin{exercise}
        Let $(X,d)$ be a metric space, $(x_n)_n$ a sequence in $X$, and $x \in X$. Show the following are equivalent:
            \begin{enumerate}[label = (\arabic*),itemsep=1pt,topsep=3pt]
                \item $(x_n)_n \rightarrow x$ in $X$;
                \item $(d(x_n,x))_n \rightarrow 0$ in $\bfR$;
                \item $(\forall V \in \cN_x)(\exists N \in \bfN):(\forall n \in \bfN)(n \geq N \implies x_n \in V)$.
            \end{enumerate}
    \end{exercise}
        {\color{red} \begin{proof}
            (1) $\Leftrightarrow$ (2) Let $\epsilon > 0$. Find $N$ large so for $n \geq N$ we have $d(x_n,x) < \epsilon$. This is equivalent to $|d(x_n,x) - 0| < \epsilon$, whence $(d(x_n,x))_n \rightarrow 0$. The other direction is identical.

            $(1) \Rightarrow (3)$ Let $V \in \cN_x$. Then there exists $\epsilon > 0$ so $U(x,\epsilon) \subseteq V$. Since $(x_n)_n \rightarrow x$, find $N$ large so for $n \geq N$ we have $d(x_n,x) < \epsilon$. Thus $x_n \in U(x,\epsilon) \subseteq V$.

            $(3) \Rightarrow (1)$ Let $\epsilon > 0$. Find $N$ large so $n \geq N$ implies $x_n \in U(x, \epsilon) \in \cN_x$. Then $d(x_n,x) < \epsilon$, giving $(x_n)_n \rightarrow x$.
        \end{proof}}
%%%%%%%%%%%%%%%%%%%%%%%%%%%%%%%%%%%%%%%%%%%%%%%%%%%%%%%%%%%%%
    \iffalse
    \newpage
    \begin{exercise}
        Let $X$ be a metric space and $x_0 \in X$. Show that a sequence $(x_n)_n$ converges to $x_0$ in $X$ if and only if every subsequence $(x_{n_k})_k$ admits a subsequence $(x_{n_{k_j}})_j$ with $(x_{n_{k_j}})_j \rightarrow x_0$.
    \end{exercise}
    \fi
%%%%%%%%%%%%%%%%%%%%%%%%%%%%%%%%%%%%%%%%%%%%%%%%%%%%%%%%%%%%%
    \addtocounter{exercise}{1}
    \begin{exercise}
        Let $\{(X_k,d_k)\}_{k \geq 1}$ be a family of metric spaces. Assume that for every $k \geq 1$ we have $d_k(x,y) \leq 1$ for all $x,y \in X_k$. Let:
            \begin{equation*}
            \begin{split}
                X&:=\prod_{k \geq 1}X_k \\
                d(f,g) &:= \sum_{k = 1}^\infty 2^{-k}d_k(f(k),g(k)).
            \end{split}
            \end{equation*}
        Show that a sequence $(f_n)_n$ converges to $f$ in $(X,d)$ if and only if $(f_n(k))_n \xrightarrow{d_k}f(k)$ for every $k \geq 1$.
    \end{exercise}
        {\color{red} \begin{proof}
            Let $(f_n)_n \xrightarrow{d} f$. Fix $k\geq 1$. We have:
                \begin{equation*}
                \begin{split}
                    0 \leq 2^{-k}d_k(f_n(k),f(k)) \leq d(f_n,f).
                \end{split}
                \end{equation*}
            Since $(d(f_n,f))_n \rightarrow 0$, multiplying $2^{-k}$ on all sides and applying the squeeze theorem yields $(d_k(f_n(k),f(k)))_n \rightarrow 0$. Whence $(f_n(k))_n \xrightarrow{d_k} f(k)$ for every $k \geq 1$.

            Now suppose $(f_n(k))_n \xrightarrow{d_k}f(k)$ for every $k \geq 1$. Then $(d_k(f_n(k),f(k)))_n \xrightarrow{d_k} 0$ for every $k \geq 1$. Find $K$ large so that:
                \begin{equation*}
                \begin{split}
                    \sum_{k > K}2^{-k} < \frac{\epsilon}{2}.
                \end{split}
                \end{equation*}
            Find $N_1,N_2,...,N_K$ sufficiently large so that for $n \geq N_i$ we have $d_i(f_n(i),f(i)) < \frac{\epsilon}{2}$. For $n \geq \max_{i = 1}^K N_i$ observe that:
                \begin{equation*}
                \begin{split}
                    d(f_n,f)
                    & = \sum_{k = 1}^\infty 2^{-k}d_k(f_n(k),f(k)) \\
                    & = \sum_{k = 1}^K 2^{-k}d_k(f_n(k),f(k)) + \sum_{k > K} 2^{-k}d_k(f_n(k),f(k)) \\
                    & \leq \sum_{k = 1}^K 2^{-k}d_k(f_n(k),f(k)) + \sum_{k > K} 2^{-k} \\
                    & < \sum_{k = 1}2^{-k}\frac{\epsilon}{2} + \frac{\epsilon}{2} \\
                    & \leq \frac{\epsilon}{2} + \frac{\epsilon}{2} \\
                    & = \epsilon.
                \end{split}
                \end{equation*}
            Since $(d(f_n,f))_n \rightarrow 0$, we have $(f_n)_n \xrightarrow{d} f$.
        \end{proof}}
%%%%%%%%%%%%%%%%%%%%%%%%%%%%%%%%%%%%%%%%%%%%%%%%%%%%%%%%%%%%%
    \begin{exercise}
        Let $V$ be a normed space. Show that the vector operations:
            \begin{equation*}
            \begin{split}
                a:V \times V &\rightarrow V; \h9 a(v,w) = v + w; \\
                \mu:F \times V &\rightarrow V; \h9 \mu(\alpha,w) = \alpha w
            \end{split}
            \end{equation*}
        are continuous.
    \end{exercise}
        {\color{red} \begin{proof}
            Let $((v_n,w_n))_n$ be a sequence in $V \times V$ converging to $(v_0,w_0) $. Then $(v_n)_n \rightarrow v_0$ and $(w_n)_n \rightarrow w_0$. Observe that:
                \begin{equation*}
                \begin{split}
                    (a(v_n,w_n))_n 
                    & = (v_n + w_n)_n \\
                    & = (v_n)_n + (w_n)_n \\
                    & \xrightarrow{n\rightarrow \infty} v_0 + w_0 \\
                    & = a(v_0,w_0).
                \end{split}
                \end{equation*}
            Thus $a$ is continuous at $(v_0,w_0)$. Since this point was arbitrary, $a$ is continuous.

            Now let $((\alpha_n,v_n))_n$ be a sequence in $F \times V$ converging to $(\alpha_0,v_0)$. Then $(\alpha_n)_n \rightarrow \alpha_0$ and $(v_n)_n \rightarrow v_0$. Observe that:
                \begin{equation*}
                \begin{split}
                    (\mu(\alpha_n,v_n))_n
                    & = (a_n v_n)_n \\
                    & = (a_n)_n (v_n)_n \\
                    & \xrightarrow{n \rightarrow \infty} \alpha_0 v_0 \\
                    & = \mu(\alpha_0,v_0).
                \end{split}
                \end{equation*}
            Thus $\mu$ is continuous at $(\alpha_0,v_0)$. Since this point was arbitrary, $\mu$ is continuous.
        \end{proof}}
%%%%%%%%%%%%%%%%%%%%%%%%%%%%%%%%%%%%%%%%%%%%%%%%%%%%%%%%%%%%%
    \addtocounter{exercise}{1}
    \iffalse
    \begin{exercise}
        Let $(X,d)$ be a metric space, $f,g:X \rightarrow F$ continuous maps, and $a \in F$. Show that $f+g$, $fg$, and $\alpha f$ are continuous.
    \end{exercise}
        {\color{red} \begin{proof}
            
        \end{proof}}
    \fi
%%%%%%%%%%%%%%%%%%%%%%%%%%%%%%%%%%%%%%%%%%%%%%%%%%%%%%%%%%%%%
    \begin{exercise}
        Consider the two metrics on $(0,\infty)$:
            \begin{equation*}
            \begin{split}
                d(s,t) := |s-t| \\
                \rho(s,t) := \left| \frac{1}{s} - \frac{1}{t} \right|
            \end{split}
            \end{equation*}
        Show that $d$ and $\rho$ are topologically equivalent. Are they uniformly equivalent?
    \end{exercise}
        {\color{red} \begin{proof}
            Let $x,x_0 \in X$. We will first show that $\id:(X,d) \rightarrow (X,\rho)$ is continuous. Note that:
                \begin{equation*}
                \begin{split}
                    \rho(\id(x),\id(x_0)) 
                    & = \left| \frac{1}{x} - \frac{1}{x_0} \right| \\
                    & = \frac{1}{x \cdot x_0} \left|x - x_0\right| \\
                    & = \frac{1}{x \cdot x_0} d(x,x_0).
                \end{split}
                \end{equation*}
            Thus $\id$ is Lipschitz. We will now show that $\id^{-1}:(X,\rho) \rightarrow (X,d)$ is continuous. Let $(x_n)_n$ be a sequence in $(X,\rho)$ such that $(x_n)_n \xrightarrow{\rho} x_0$. Then $(\rho(x_n,x_0))_n \rightarrow 0$, which is equivalent to $\left( d \left( \frac{1}{x_n}, \frac{1}{x_0} \right) \right)_n \rightarrow 0$. Since $\left( \frac{1}{x_n} \right)_n $ is a sequence of non-zero numbers converging to a non-zero limit $\frac{1}{x_0}$, the sequence of reciprocals $(x_n)_n$ will converge to $x_0$; i.e, $(\id^{-1}(x_n))_n \xrightarrow{d} \id^{-1}(x_0)$. This establishes $\id^{-1}$ as continuous, giving that $d$ and $\rho$ are topologically equivalent. 

            Let $\epsilon_0 =1$. Consider the sequences $\left( \frac{1}{n} \right)_n$ and $\left( \frac{1}{n+1} \right)_n$ in $(X,d)$. We have $\left( d \left( \frac{1}{n}, \frac{1}{n+1} \right) \right)_n \rightarrow 0$ and $\rho \left( \frac{1}{n}, \frac{1}{n+1} \right) \geq \epsilon_0$. Whence $\id:(X,d) \rightarrow (X,\rho)$ is not uniformly continuous; i.e., $d$ and $\rho$ are not uniformly equivalent.
        \end{proof}}
%%%%%%%%%%%%%%%%%%%%%%%%%%%%%%%%%%%%%%%%%%%%%%%%%%%%%%%%%%%%%
    \begin{exercise}
        Let $h:X \rightarrow Y$ be a homeomorphism of metric spaces. Show that the map:
            \begin{equation*}
            \begin{split}
                T_h: \left( C(Y), \lnorm \cdot \rnorm_u \right) \rightarrow \left( C(X), \lnorm \cdot \rnorm_u \right); \h9 T_h(f) = f \circ h
            \end{split}
            \end{equation*}
        is an isometric isomorphism of normed spaces, that is, $T_h$ is linear, bijective, and isometric.
    \end{exercise}
        {\color{red} \begin{proof}
            
        \end{proof}}
%%%%%%%%%%%%%%%%%%%%%%%%%%%%%%%%%%%%%%%%%%%%%%%%%%%%%%%%%%%%%
    \begin{exercise}
        
    \end{exercise}
        {\color{red} \begin{proof}
            Since $\lnorm T \rnorm_\text{op} \leq 1$, we have $\sup_{v \in B_V} \lnorm T(v) \rnorm_W \leq \sup_{v \in B_V}\lnorm v \rnorm_V$. So $\lnorm T(v) \rnorm_W \leq \lnorm v \rnorm_V$ for all $v \in V$. Since $\lnorm T^{-1} \rnorm_\text{op} \leq 1$, we have $\sup_{w \in B_W} \lnorm T^{-1}(w) \rnorm_V \leq \sup_{w \in B_W}\lnorm w \rnorm_W$. So $\lnorm T^{-1}(w) \rnorm_V \leq \lnorm w \rnorm_W$ for all $w \in W$. Since $T$ is a bijection, given $x \in X$ take $w = T(v)$. Then $\lnorm v \rnorm_V \leq \lnorm T(v) \rnorm_W$ for all $v \in V$. By antisymmetry, we have $\lnorm T(v) \rnorm_W = \lnorm v \rnorm_V$. Thus $T$ is an isometry.
        \end{proof}}
%%%%%%%%%%%%%%%%%%%%%%%%%%%%%%%%%%%%%%%%%%%%%%%%%%%%%%%%%%%%%
\end{document}