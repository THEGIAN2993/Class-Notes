\chapter{Metric Spaces}

\section{Introduction}
    \begin{definition}
        Let $X$ be a nonempty set. A \textit{metric} on $X$ is a map:
            \begin{equation*}
            \begin{split}
                d:X \times X \rightarrow \bfR^+
            \end{split}
            \end{equation*}
        satisfying for all $x,y,z \in X$:
            \begin{enumerate}[label = (\arabic*)]
                \item $d(x,y) = d(y,x)$;
                \item $d(x,z) \leq d(x,y) + d(y,z)$;
                \item $d(x,x) = 0$;
                \item If $d(x,y) = 0$, then $x=y$.
            \end{enumerate}
        If $d$ only satisfies (1), (2), and (3), then $d$ is called a \textit{semi-metric}. We call the pair $(X,d)$ a \textit{metric space} (or \textit{semi-metric space}).
    \end{definition}

    \begin{definition}
        Two metrics $d,\rho$ on $X$ are called \textit{equivalent} if there exists $c_1,c_2 \geq 0$ such that, for all $x,y \in X$:
            \begin{equation*}
            \begin{split}
                d(x,y) &\leq c_1 \rho(x,y); \\
                \rho(x,y) &\leq c_2 d(x,y).
            \end{split}
            \end{equation*}
    \end{definition}